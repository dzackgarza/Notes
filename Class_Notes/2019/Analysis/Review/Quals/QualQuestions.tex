\RequirePackage[l2tabu, orthodox]{nag}

%\documentclass[]{article}
\documentclass[11pt]{scrartcl}
\usepackage[usename, dvipsnames]{xcolor}
\usepackage[pdfencoding=auto]{hyperref}
\usepackage[msc-links]{amsrefs}
\usepackage{cleveref} % use \cref{}, automatically deduces theorem, proposition, etc
\usepackage[mathletters]{ucs}
\usepackage[utf8]{inputenc}
\usepackage[T1]{fontenc}
\usepackage{datetime}

\usepackage{array}
\usepackage{mathtools}
\usepackage{amsmath, amsthm, amssymb, amsfonts, amsxtra, amscd, thmtools}
\let\proof\relax
\let\endproof\relax

% Boxes around theorem environments.
\usepackage[many]{tcolorbox}

\usepackage{color}
%\usepackage{unicode-math}
\usepackage{newunicodechar}
\newunicodechar{ε}{\varepsilon}
\newunicodechar{δ}{\delta}
\newunicodechar{µ}{\mu}
\newunicodechar{→}{\to}
\newunicodechar{≤}{\leq}
\newunicodechar{∈}{\in}
\newunicodechar{⊆}{\subseteq}
\newunicodechar{Λ}{\Lambda}
\newunicodechar{∞}{\infty}
\newunicodechar{×}{\times}
\everymath{\displaystyle}



\usepackage{microtype}
\usepackage[pdfencoding=auto]{hyperref}
\usepackage{bookmark}
\usepackage{booktabs}
\usepackage{todonotes}
\usepackage[msc-links]{amsrefs}
\usepackage{cleveref} % use \cref{}, automatically deduces theorem, proposition, etc
\usepackage{csquotes}
\usepackage{longtable}
\usepackage{tabularx}
\usepackage{bbm}
% Creating multiple types of index
\usepackage{imakeidx}

% Remove indentation for new paragraphs
\usepackage{parskip}
% But leave space before amsthm environments
\makeatletter
\def\thm@space@setup{%
  \thm@preskip=2em
  \thm@postskip=2em
}
\makeatother


\usepackage{stmaryrd}
\usepackage{adjustbox}
\usepackage{centernot}
% \centernot\whatever


% Better indicator function
\usepackage{bbm}
\newcommand{\indic}[1]{\mathbbm{1} \left[ {#1} \right] }

% Highlight quote
\usepackage{environ}
\definecolor{camel}{rgb}{0.76, 0.6, 0.42}
\definecolor{babyblue}{rgb}{0.54, 0.81, 0.94}
\definecolor{block-gray}{gray}{0.85}
\NewEnviron{myblock}
{\colorbox{block-gray}{%
\parbox{\dimexpr\linewidth-2\fboxsep\relax}{%
\small\addtolength{\leftskip}{10mm}
\addtolength{\rightskip}{10mm}
\BODY}}
}
\renewcommand{\quote}{\myblock}
\renewcommand{\endquote}{\endmyblock}

% Nice math font that journals use
%\usepackage[lite]{mtpro2}
%\usepackage{mathrsfs}
%\usepackage{mathptmx}
\usepackage{lmodern}
%\usepackage[sc]{mathpazo}

% Theorem Styles
\usepackage[framemethod=tikz]{mdframed}

\theoremstyle{definition}
\newtheorem{exercise}{Exercise}[section]
\newtheorem{solution}{Solution}

% Theorem Style
\newtheoremstyle{theorem}% name
  {0em}%         Space above, empty = `usual value'
  {1em}%         Space below
  {\normalfont}% Body font
  {\parindent}%         Indent amount (empty = no indent, \parindent = para indent)
  {\bfseries}% Thm head font
  {.}%        Punctuation after thm head
  {\newline}% Space after thm head: \newline = linebreak
  {\thmname{#1}\thmnumber{ #2}\thmnote{\itshape{(#3)}}}%
\theoremstyle{theorem}
\tcolorboxenvironment{theorem}{
  boxrule=0pt,
  boxsep=0pt,
  breakable,
  enhanced jigsaw,
  fonttitle={\large\bfseries},
  opacityback=0.8,
  colframe=cyan,
  borderline west={4pt}{0pt}{orange},
  attach title to upper={}
}
\newtheorem{theorem}{Theorem}[section]

% Proposition Style
\tcolorboxenvironment{proposition}{
  boxrule=1pt,
  boxsep=0pt,
  breakable,
  enhanced jigsaw,
  opacityback=0.0,
  colframe=cyan
}
\newtheorem{proposition}[theorem]{Proposition}
\tcolorboxenvironment{lemma}{
  boxrule=1pt,
  boxsep=0pt,
  breakable,
  enhanced jigsaw,
  opacityback=0.2,
  colframe=cyan
}
\newtheorem{lemma}[theorem]{Lemma}
% Claim
\tcolorboxenvironment{claim}{
  boxrule=1pt,
  boxsep=0pt,
  breakable,
  enhanced jigsaw,
  opacityback=0.2,
  colframe=cyan
}
\newtheorem{claim}[theorem]{Claim}


% Corollary
\tcolorboxenvironment{corollary}{
  colback=cyan,
  boxrule=1pt,
  boxsep=0pt,
  breakable,
  enhanced jigsaw,
  opacityback=0.1,
  colframe=cyan
}
\newtheorem{corollary}[theorem]{Corollary}

% Proof Style
\newtheoremstyle{proof}% name
  {0em}%         Space above, empty = `usual value'
  {2em}%         Space below
  {\normalfont}% Body font
  {\parindent}%         Indent amount (empty = no indent, \parindent = para indent)
  {\itshape}% Thm head font
  {.}%        Punctuation after thm head
  {\newline}% Space after thm head: \newline = linebreak
  {\thmname{#1} \thmnote{\itshape{(#3)}}}%         Thm head spec
\theoremstyle{proof}
\tcolorboxenvironment{proof}{
  colback=camel,
  opacityfill=0.25,
  boxrule=1pt,
  boxsep=0pt,
  breakable,
  enhanced jigsaw
}
\newtheorem*{pf}{Proof}
\newenvironment{proof}
{\pushQED{$\qed$}\pf}
{\par\popQED\endpf}

% Definition Style
\newtheoremstyle{definition}% name
  {0em}%         Space above, empty = `usual value'
  {2em}%         Space below
  {\normalfont}% Body font
  {\parindent}%         Indent amount (empty = no indent, \parindent = para indent)
  {\bfseries}% Thm head font
  {.}%        Punctuation after thm head
  {\newline}% Space after thm head: \newline = linebreak
  {}%         Thm head spec
\theoremstyle{definition}
\tcolorboxenvironment{definition}{
  colback=babyblue,
  boxrule=0pt,
  boxsep=0pt,
  opacityfill=0.45,
  breakable,
  enhanced jigsaw,
  borderline west={4pt}{0pt}{blue},
  colbacktitle={babyblue},
  coltitle={black},
  fonttitle={\large\bfseries},
  attach title to upper={},
}
\newtheorem{definition}{Definition}[theorem]

% Break Environment
\makeatletter
\newtheoremstyle{break}% name
  {}%         Space above, empty = `usual value'
  {2em}%         Space below
  {
    \addtolength{\@totalleftmargin}{2.5em}
    \addtolength{\linewidth}{-2.5em}
    \parshape 1 2.5em \linewidth
  }% Body font
  {}%         Indent amount (empty = no indent, \parindent = para indent)
  {\bfseries}% Thm head font
  {.}%        Punctuation after thm head
  {\newline}% Space after thm head: \newline = linebreak
  {}%         Thm head spec
\makeatother

\theoremstyle{break}
\newtheorem{example}{Example}[section]

% Problem Style
\newtheoremstyle{problem} % name
  {0em}                   % Space above, empty = `usual value'
  {2em}                   % Space below
  {\normalfont}           % Body font
  {\parindent}            % Indent amount (empty = no indent, \parindent = para indent)
  {\itshape}              % Thm head font
  {}                      % Punctuation after thm head
  {\newline}              % Space after thm head: \newline = linebreak
  {\thmnote{\itshape{(#3)}}}     % Thm head spec
\theoremstyle{problem}
\tcolorboxenvironment{problem}{
  boxrule=1pt,
  boxsep=0pt,
  breakable,
  enhanced jigsaw,
  opacityback=0.0,
  colframe=cyan
}
\newtheorem{problem}{Problem}


%Pagination stuff.
\setlength{\topmargin}{-.3 in}
\setlength{\oddsidemargin}{0in}
\setlength{\evensidemargin}{0in}
\setlength{\textheight}{9.in}
\setlength{\textwidth}{6.5in}
% \pagestyle{empty} %removes page numbers.

% Inkscape figures from Vim
\usepackage{import}
\usepackage{pdfpages}
\usepackage{transparent}

\newcommand{\incfig}[1]{%
    \def\svgwidth{\columnwidth}
    \import{./figures/}{#1.pdf_tex}
}
%\pdfsuppresswarningpagegroup=1

% Pandoc-specific fixes
\providecommand{\tightlist}{%
  \setlength{\itemsep}{0pt}\setlength{\parskip}{0pt}}

% Tikz and Graphics
\usepackage{amscd}
\usepackage{tikz}
\usetikzlibrary{arrows, arrows.meta, cd, fadings, patterns, calc, decorations.markings, matrix, positioning}
\tikzfading[name=fade out, inner color=transparent!0, outer color=transparent!100]
\usepackage{pgfplots}
\pgfplotsset{compat=1.16}
\usepackage[inline]{asymptote}
\usepackage{tikz-layers}

%\usepackage{nath}
%\delimgrowth=1
\DeclarePairedDelimiter\qty{(}{)}

% Major Macros
\usepackage{graphicx}
\usepackage{float}
\DeclareFontFamily{U}{mathx}{\hyphenchar\font45}
\DeclareFontShape{U}{mathx}{m}{n}{
      <5> <6> <7> <8> <9> <10>
      <10.95> <12> <14.4> <17.28> <20.74> <24.88>
      mathx10
      }{}
\DeclareSymbolFont{mathx}{U}{mathx}{m}{n}
\DeclareMathSymbol{\bigtimes}{1}{mathx}{"91}

% Wide tikz equations
\newsavebox{\wideeqbox}
\newenvironment{wideeq}
  {\begin{displaymath}\begin{lrbox}{\wideeqbox}$\displaystyle}
  {$\end{lrbox}\makebox[0pt]{\usebox{\wideeqbox}}\end{displaymath}}



% Fancy chapter headers and footers
\usepackage{fancyhdr}

\pagestyle{fancy}
\fancyhf{}
\fancyhead[LE,RO]{\title}
\fancyhead[RE,LO]{\rightmark}
\fancyfoot[CE,CO]{\leftmark}
\fancyfoot[LE,RO]{\thepage}

\renewcommand{\headrulewidth}{2pt}
\renewcommand{\footrulewidth}{1pt}

% List of Theorems Attempt
\usepackage{etoolbox}
\makeatletter
\patchcmd\thmtlo@chaptervspacehack
  {\addtocontents{loe}{\protect\addvspace{10\p@}}}
  {\addtocontents{loe}{\protect\thmlopatch@endchapter\protect\thmlopatch@chapter{\thechapter}}}
  {}{}
\AtEndDocument{\addtocontents{loe}{\protect\thmlopatch@endchapter}}
\long\def\thmlopatch@chapter#1#2\thmlopatch@endchapter{%
  \setbox\z@=\vbox{#2}%
  \ifdim\ht\z@>\z@
    \hbox{\bfseries\chaptername\ #1}\nobreak
    #2
    \addvspace{10\p@}
  \fi
}
\def\thmlopatch@endchapter{}

\makeatother
\renewcommand{\thmtformatoptarg}[1]{ -- #1}
%\renewcommand{\listtheoremname}{List of definitions}

\newcommand{\ext}{\operatorname{Ext}}
\newcommand{\Ext}{\operatorname{Ext}}
\def\Endo{\operatorname{End}}
\def\Ind{\operatorname{Ind}}
\def\ind{\operatorname{Ind}}
\def\coind{\operatorname{Coind}}
\def\Res{\operatorname{Res}}
\def\Hol{\operatorname{Hol}}
\def\res{\operatorname{Res}}
\def\endo{\operatorname{End}}
\def\ind{\operatorname{Ind}}
\renewcommand{\AA}[0]{{\mathbb{A}}}
\DeclareMathOperator{\Exists}{\exists}
\DeclareMathOperator{\Forall}{\forall}
\newcommand{\Af}[0]{{\mathbb{A}}}
\newcommand{\CC}[0]{{\mathbb{C}}}
\newcommand{\CP}[0]{{\mathbb{CP}}}
\newcommand{\DD}[0]{{\mathbb{D}}}
\newcommand{\FF}[0]{{\mathbb{F}}}
\newcommand{\GF}[0]{{\mathbb{GF}}}
\newcommand{\GG}[0]{{\mathbb{G}}}
\newcommand{\HH}[0]{{\mathbb{H}}}
\newcommand{\HP}[0]{{\mathbb{HP}}}
\newcommand{\KK}[0]{{\mathbb{K}}}
\newcommand{\kk}[0]{{\Bbbk}}
\newcommand{\bbm}[0]{{\mathbb{M}}}
\newcommand{\NN}[0]{{\mathbb{N}}}
\newcommand{\OP}[0]{{\mathbb{OP}}}
\newcommand{\PP}[0]{{\mathbb{P}}}
\newcommand{\QQ}[0]{{\mathbb{Q}}}
\newcommand{\RP}[0]{{\mathbb{RP}}}
\newcommand{\RR}[0]{{\mathbb{R}}}
\newcommand{\SpSp}[0]{{\mathbb{S}}}
\renewcommand{\SS}[0]{{\mathbb{S}}}
\newcommand{\TT}[0]{{\mathbb{T}}}
\newcommand{\ZZ}[0]{{\mathbb{Z}}}
\newcommand{\ZnZ}[0]{\mathbb{Z}/n\mathbb{Z}}
\newcommand{\ZpZ}[0]{\mathbb{Z}/p\mathbb{Z}}
\newcommand{\Qp}[0]{\mathbb{Q}_{(p)}}
\newcommand{\Zp}[0]{\mathbb{Z}_{(p)}}
\newcommand{\Arg}[0]{\mathrm{Arg}}
\newcommand{\PGL}[0]{\mathrm{PGL}}
\newcommand{\GL}[0]{\mathrm{GL}}
\newcommand{\Gl}[0]{\mathrm{GL}}
\newcommand{\gl}[0]{\mathrm{GL}}
\newcommand{\mat}[0]{\mathrm{Mat}}
\newcommand{\Mat}[0]{\mathrm{Mat}}
\newcommand{\Rat}[0]{\mathrm{Rat}}
\newcommand{\Perv}[0]{\mathrm{Perv}}
\newcommand{\Gal}[0]{\mathrm{Gal}}
\newcommand{\Hilb}[0]{\mathrm{Hilb}}
\newcommand{\Quot}[0]{\mathrm{Quot}}
\newcommand{\Art}[0]{\mathrm{Art}}
\newcommand{\red}[0]{\mathrm{red}}
\newcommand{\alg}[0]{\mathrm{alg}}
\newcommand{\Pic}[0]{{\mathrm{Pic}~}}
\newcommand{\lcm}[0]{\mathrm{lcm}}
\newcommand{\maps}[0]{\mathrm{Maps}}
\newcommand{\maxspec}[0]{{\mathrm{maxSpec}~}}
\newcommand{\Tr}[0]{\mathrm{Tr}}
\newcommand{\adj}[0]{\mathrm{adj}}
\newcommand{\ad}[0]{\mathrm{ad}~}
\newcommand{\ann}[0]{\mathrm{Ann}}
\newcommand{\Ann}[0]{\mathrm{Ann}}
\newcommand{\arcsec}[0]{\mathrm{arcsec}}
\newcommand{\ch}[0]{\mathrm{char}~}
\newcommand{\Sp}[0]{{\mathrm{Sp}}}
\newcommand{\syl}[0]{{\mathrm{Syl}}}
\newcommand{\txand}[0]{{\text{ and }}}
\newcommand{\codim}[0]{\mathrm{codim}}
\newcommand{\txor}[0]{{\text{ or }}}
\newcommand{\txt}[1]{{\text{ {#1} }}}
\newcommand{\Gr}[0]{{\text{Gr}}}
\newcommand{\Aut}[0]{{\mathrm{Aut}}}
\newcommand{\aut}[0]{\mathrm{Aut}}
\newcommand{\Inn}[0]{{\mathrm{Inn}}}
\newcommand{\Out}[0]{{\mathrm{Out}}}
\newcommand{\mltext}[1]{\left\{\begin{array}{c}#1\end{array}\right\}}
\newcommand{\Fun}[0]{{\text{Fun}}}
\newcommand{\SL}[0]{{\text{SL}}}
\newcommand{\PSL}[0]{{\text{PSL}}}
\newcommand{\SO}[0]{{\text{SO}}}
\newcommand{\SU}[0]{{\text{SU}}}
\newcommand{\SP}[0]{{\text{SP}}}
\newcommand{\per}[0]{{\text{Per}}}
\newcommand{\loc}[0]{{\text{loc}}}
\newcommand{\Top}[0]{{\text{Top}}}
\newcommand{\Sch}[0]{{\text{Sch}}}
\newcommand{\sch}[0]{{\text{Sch}}}
\newcommand{\Set}[0]{{\text{Set}}}
\newcommand{\Sets}[0]{{\text{Set}}}
\newcommand{\Grp}[0]{{\text{Grp}}}
\newcommand{\Groups}[0]{{\text{Groups}}}
\newcommand{\Homeo}[0]{{\text{Homeo}}}
\newcommand{\Diffeo}[0]{{\text{Diffeo}}}
\newcommand{\MCG}[0]{{\text{MCG}}}
\newcommand{\set}[0]{{\text{Set}}}
\newcommand{\Tor}[0]{\text{Tor}}
\newcommand{\sets}[0]{{\text{Set}}}
\newcommand{\Sm}[0]{{\text{Sm}_k}}
\newcommand{\orr}[0]{{\text{ or }}}
\newcommand{\annd}[0]{{\text{ and }}}
\newcommand{\bung}[0]{\text{Bun}_G}
\newcommand{\const}[0]{{\text{const.}}}
\newcommand{\disc}[0]{{\text{disc}}}
\newcommand{\op}[0]{^\text{op}}
\newcommand{\id}[0]{\text{id}}
\newcommand{\im}[1]{\mathrm{im}({#1})}
\newcommand{\pt}[0]{{\{\text{pt}\}}}
\newcommand{\sep}[0]{^\text{sep}}
% \newcommand{\st}[0]{~{\text{s.t.}}~}
\newcommand{\tors}[0]{{\text{tors}}}
\newcommand{\tor}[0]{\text{Tor}}
\newcommand{\height}[0]{\text{ht}}
\newcommand{\cpt}[0]{\text{compact}}
\newcommand{\abs}[1]{{\left\lvert {#1} \right\rvert}}
\newcommand{\stack}[1]{\mathclap{\substack{ #1 }}} 
\newcommand{\qtext}[1]{{\quad \text{#1} \quad}}
\newcommand{\qst}[0]{{\quad \text{such that} \quad}}
\newcommand{\actsonl}[0]{\curvearrowleft}
\newcommand{\actson}[0]{\curvearrowright}
\newcommand{\bd}[0]{{\del}}
\newcommand{\bigast}[0]{{\mathop{\Large \ast}}}
\newcommand{\coker}[0]{\operatorname{coker}}
\newcommand{\cok}[0]{\operatorname{coker}}
\newcommand{\conjugate}[1]{{\overline{{#1}}}}
\newcommand{\converges}[1]{\overset{#1}}
\newcommand{\correspond}[1]{\theset{\substack{#1}}}
\newcommand{\cross}[0]{\times}
\newcommand{\by}[0]{\times}
\newcommand{\dash}[0]{{\hbox{-}}}
\newcommand{\dd}[2]{{\frac{\partial #1}{\partial #2}\,}}
\newcommand{\definedas}[0]{\coloneqq}
\newcommand{\da}[0]{\coloneqq}
\newcommand{\del}[0]{{\partial}}
\newcommand{\directlim}[0]{\varinjlim}
\newcommand{\disjoint}[0]{{\coprod}}
\newcommand{\divides}[0]{{~\Bigm|~}}
\newcommand{\dual}[0]{^\vee}
\newcommand{\sm}[0]{\setminus}
\newcommand{\smz}[0]{\setminus\theset{0}}
\newcommand{\eps}[0]{\varepsilon}
\newcommand{\equalsbecause}[1] {\stackrel{\mathclap{\scriptscriptstyle{#1}}}{=}}
\newcommand{\floor}[1]{{\left\lfloor #1 \right\rfloor}}
\DeclarePairedDelimiter{\ceil}{\lceil}{\rceil}
\newcommand{\from}[0]{\leftarrow}
\newcommand{\tofrom}[0]{\leftrightarrows}
\newcommand{\up}[0]{\uparrow}
\newcommand{\generators}[1]{\left\langle{#1}\right\rangle}
\newcommand{\gs}[1]{\left\langle{#1}\right\rangle}
\newcommand{\homotopic}[0]{\simeq}
\newcommand{\injectivelim}[0]{\varinjlim}
\newcommand{\injects}[0]{\hookrightarrow}
\newcommand{\inner}[2]{{\left\langle {#1},~{#2} \right\rangle}}
\newcommand{\union}[0]{\cup}
\newcommand{\Union}[0]{\bigcup}
\newcommand{\intersect}[0]{\cap}
\newcommand{\Intersect}[0]{\bigcap}
\newcommand{\into}[0]{\to}
\newcommand{\inverselim}[0]{\varprojlim}
\newcommand{\inv}[0]{^{-1}}
\newcommand{\mfa}[0]{{\mathfrak{a}}}
\newcommand{\mfb}[0]{{\mathfrak{b}}}
\newcommand{\mfc}[0]{{\mathfrak{c}}}
\newcommand{\mff}[0]{{\mathfrak{f}}}
\newcommand{\mfi}[0]{{\mathfrak{I}}}
\newcommand{\mfm}[0]{{\mathfrak{m}}}
\newcommand{\mfn}[0]{{\mathfrak{n}}}
\newcommand{\mfp}[0]{{\mathfrak{p}}}
\newcommand{\mfq}[0]{{\mathfrak{q}}}
\newcommand{\mfr}[0]{{\mathfrak{r}}}
\newcommand{\lieb}[0]{{\mathfrak{b}}}
\newcommand{\liegl}[0]{{\mathfrak{gl}}}
\newcommand{\lieg}[0]{{\mathfrak{g}}}
\newcommand{\lieh}[0]{{\mathfrak{h}}}
\newcommand{\lien}[0]{{\mathfrak{n}}}
\newcommand{\liesl}[0]{{\mathfrak{sl}}}
\newcommand{\lieso}[0]{{\mathfrak{so}}}
\newcommand{\liesp}[0]{{\mathfrak{sp}}}
\newcommand{\lieu}[0]{{\mathfrak{u}}}
\newcommand{\nilrad}[0]{{\mathfrak{N}}}
\newcommand{\jacobsonrad}[0]{{\mathfrak{J}}}
\newcommand{\mm}[0]{{\mathfrak{m}}}
\newcommand{\pr}[0]{{\mathfrak{p}}}
\newcommand{\mapsvia}[1]{\xrightarrow{#1}}
\newcommand{\kx}[1]{k[x_1, \cdots, x_{#1}]}
\newcommand{\MM}[0]{{\mathcal{M}}}
\newcommand{\OO}[0]{{\mathcal{O}}}
\newcommand{\imaginarypart}[1]{{\mathcal{Im}({#1})}}
\newcommand{\mca}[0]{{\mathcal{A}}}
\newcommand{\mcb}[0]{{\mathcal{B}}}
\newcommand{\mcc}[0]{{\mathcal{C}}}
\newcommand{\mcd}[0]{{\mathcal{D}}}
\newcommand{\mce}[0]{{\mathcal{E}}}
\newcommand{\mcf}[0]{{\mathcal{F}}}
\newcommand{\mcg}[0]{{\mathcal{G}}}
\newcommand{\mch}[0]{{\mathcal{H}}}
\newcommand{\mci}[0]{{\mathcal{I}}}
\newcommand{\mcj}[0]{{\mathcal{J}}}
\newcommand{\mck}[0]{{\mathcal{K}}}
\newcommand{\mcl}[0]{{\mathcal{L}}}
\newcommand{\mcm}[0]{{\mathcal{M}}}
\newcommand{\mcp}[0]{{\mathcal{P}}}
\newcommand{\mcs}[0]{{\mathcal{S}}}
\newcommand{\mct}[0]{{\mathcal{T}}}
\newcommand{\mcu}[0]{{\mathcal{U}}}
\newcommand{\mcv}[0]{{\mathcal{V}}}
\newcommand{\mcx}[0]{{\mathcal{X}}}
\newcommand{\mcz}[0]{{\mathcal{Z}}}
\newcommand{\cl}[0]{\mathrm{cl}}
\newcommand{\trdeg}[0]{\mathrm{trdeg}}
\newcommand{\dist}[0]{\mathrm{dist}}
\newcommand{\Dist}[0]{\mathrm{Dist}}
\newcommand{\crit}[0]{\mathrm{crit}}
\newcommand{\diam}[0]{{\mathrm{diam}}}
\newcommand{\gal}[0]{\mathrm{Gal}}
\newcommand{\diff}[0]{\mathrm{Diff}}
\newcommand{\diag}[0]{\mathrm{diag}}
\newcommand{\soc}[0]{\mathrm{Soc}\,}
\newcommand{\hd}[0]{\mathrm{Head}\,}
\newcommand{\grad}[0]{\mathrm{grad}~}
\newcommand{\hilb}[0]{\mathrm{Hilb}}
\newcommand{\minpoly}[0]{{\mathrm{minpoly}}}
\newcommand{\Hom}[0]{{\mathrm{Hom}}}
\newcommand{\Map}[0]{{\mathrm{Map}}}
\newcommand{\multinomial}[1]{\left(\!\!{#1}\!\!\right)}
\newcommand{\nil}[0]{{\mathrm{nil}}}
\newcommand{\normalneq}{\mathrel{\reflectbox{$\trianglerightneq$}}}
\newcommand{\normal}[0]{{~\trianglelefteq~}}
\newcommand{\norm}[1]{{\left\lVert {#1} \right\rVert}}
\newcommand{\pnorm}[2]{{\left\lVert {#1} \right\rVert}_{#2}}
\newcommand{\notdivides}[0]{\nmid}
\newcommand{\onto}[0]{\twoheadhthtarrow}
\newcommand{\ord}[0]{{\mathrm{Ord}}}
\newcommand{\pic}[0]{{\mathrm{Pic}~}}
\newcommand{\projectivelim}[0]{\varprojlim}
\newcommand{\rad}[0]{{\mathrm{rad}~}}
\newcommand{\ralg}[0]{\mathrm{R-alg}}
\newcommand{\kalg}[0]{k\dash\mathrm{alg}}
\newcommand{\rank}[0]{\operatorname{rank}}
\newcommand{\realpart}[1]{{\mathcal{Re}({#1})}}
\newcommand{\Log}[0]{\mathrm{Log}}
\newcommand{\reg}[0]{\mathrm{Reg}}
\newcommand{\restrictionof}[2]{{\left.{#1}\right|_{#2}}}
\newcommand{\ro}[2]{{\left.{#1}\right|_{#2}}}
\newcommand{\rk}[0]{{\mathrm{rank}}}
\newcommand{\evalfrom}[0]{\Big|}
\newcommand{\rmod}[0]{{R\dash\mathrm{mod}}}
\newcommand{\Mod}[0]{{\mathrm{Mod}}}
\newcommand{\rotate}[2]{{\style{display: inline-block; transform: rotate(#1deg)}{#2}}}
\newcommand{\selfmap}[0]{{\circlearrowleft}}
\newcommand{\semidirect}[0]{\rtimes}
\newcommand{\sgn}[0]{\mathrm{sgn}}
\newcommand{\sign}[0]{\mathrm{sign}}
\newcommand{\spanof}[0]{{\mathrm{span}}}
\newcommand{\spec}[0]{\mathrm{Spec}\,}
\newcommand{\mspec}[0]{\mathrm{mSpec}~}
\newcommand{\stab}[0]{{\mathrm{Stab}}}
\newcommand{\stirlingfirst}[2]{\genfrac{[}{]}{0pt}{}{#1}{#2}}
\newcommand{\stirling}[2]{\genfrac\{\}{0pt}{}{#1}{#2}}
\newcommand{\strike}[1]{{\enclose{horizontalstrike}{#1}}}
\newcommand{\suchthat}[0]{{~\mathrel{\Big|}~}}
\newcommand{\st}[0]{{~\mathrel{\Big|}~}}
\newcommand{\supp}[0]{{\mathrm{supp}}}
\newcommand{\surjects}[0]{\twoheadrightarrow}
\newcommand{\sym}[0]{\mathrm{Sym}}
\newcommand{\tensor}[0]{\otimes}
\newcommand{\connectsum}[0]{\mathop{\Large \#}}
\newcommand{\theset}[1]{\left\{{#1}\right\}}
\newcommand{\ts}[1]{\left\{{#1}\right\}}
\newcommand{\gens}[1]{\left\langle{#1}\right\rangle}
\newcommand{\thevector}[1]{{\left[ {#1} \right]}}
\newcommand{\tv}[1]{{\left[ {#1} \right]}}
\newcommand{\too}[1]{{\xrightarrow{#1}}}
\newcommand{\transverse}[0]{\pitchfork}
\newcommand{\trianglerightneq}{\mathrel{\ooalign{\raisebox{-0.5ex}{\reflectbox{\rotatebox{90}{$\nshortmid$}}}\cr$\triangleright$\cr}\mkern-3mu}}
\newcommand{\tr}[0]{\mathrm{Tr}}
\newcommand{\uniformlyconverges}[0]{\rightrightarrows}
\newcommand{\covers}[0]{\rightrightarrows}
\newcommand{\units}[0]{^{\times}}
\newcommand{\nonzero}[0]{^{\bullet}}
\newcommand{\wait}[0]{{\,\cdot\,}}
\newcommand{\wt}[0]{{\mathrm{wt}}}
\renewcommand{\bar}[1]{\mkern 1.5mu\overline{\mkern-1.5mu#1\mkern-1.5mu}\mkern 1.5mu}
\renewcommand{\div}[0]{\mathrm{Div}}
\newcommand{\Div}[0]{\mathrm{Div}}
\renewcommand{\hat}[1]{\widehat{#1}}
\renewcommand{\mid}[0]{\mathrel{\Big|}}
\renewcommand{\qed}[0]{\hfill\blacksquare}
\renewcommand{\too}[0]{\longrightarrow}
\renewcommand{\vector}[1]{\mathbf{#1}}
\let\oldexp\exp
\renewcommand{\exp}[1]{\oldexp\qty{#1}}
\let\oldperp\perp
\renewcommand{\perp}[0]{^\oldperp}
\newcommand*\dif{\mathop{}\!\mathrm{d}}
\newcommand{\ddt}{\tfrac{\dif}{\dif t}}
\newcommand{\ddx}{\tfrac{\dif}{\dif x}}

\DeclareMathOperator{\righttriplearrows} {{\; \tikz{ \foreach \y in {0, 0.1, 0.2} { \draw [-stealth] (0, \y) -- +(0.5, 0);}} \; }}


%%%%%%%%%%%%%%%%%%%%%%%%%%%%%%%%%%%%%%%%%%
\newtheoremstyle{break}% name
  {}%         Space above, empty = `usual value'
  {2em}%         Space below
  {\normalfont}% Body font
  {}%         Indent amount (empty = no indent, \parindent = para indent)
  {\bfseries}% Thm head font
  {.}%        Punctuation after thm head
  {\newline}% Space after thm head: \newline = linebreak
  {}%         Thm head spec
  
\theoremstyle{break}
\newtheorem{problem}{Problem}
\theoremstyle{definition}
\newtheorem{solution}{Solution}

\newcommand{\RR}[0]{{\mathbb{R}}}
\newcommand{\CC}[0]{{\mathbb{C}}}
\newcommand{\QQ}[0]{{\mathbb{Q}}}
\newcommand{\NN}[0]{{\mathbb{N}}}
\newcommand{\ZZ}[0]{{\mathbb{Z}}}

\newcommand{\abs}[1]{{\left\lvert {#1} \right\rvert}}
\newcommand{\inv}[0]{^{-1}}

\newcommand{\suchthat}[0]{{~\mid- ~}}
\newcommand{\theset}[1]{\left\{{#1}\right\}}
\newcommand{\norm}[1]{{\left\lVert {#1} \right\rVert}}
\newcommand{\wait}[0]{{\,\cdot\,}}
\newcommand{\inner}[2]{{\left\langle {#1},~{#2} \right\rangle}}

\everymath{\displaystyle}
%%%%%%%%%%%%%%%%%%%%%%%%%%%%%%%%%%%%%%%%%%
\synctex=1

\usepackage{environ}
\NewEnviron{killcontents}{}
\let\proof\killcontents
\let\endproof\endkillcontents

\title{UGA Real Analysis Qualifying Exams}
\date{}

\begin{document}
\maketitle
\tableofcontents 
\newpage

\hypertarget{fall-2019}{%
\section{Fall 2019}\label{fall-2019}}

\hypertarget{section}{%
\subsection{1.}\label{section}}

Let \(\{a_n\}_{n=1}^\infty\) be a sequence of real numbers.

\begin{enumerate}
\def\labelenumi{\alph{enumi}.}
\item
  Prove that if \(\displaystyle\lim_{n\to\infty} a_n = 0\), then
  \[
  \lim _{n \rightarrow \infty} \frac{a_{1}+\cdots+a_{n}}{n}=0
  \]
\item
  Prove that if \(\displaystyle\sum_{n=1}^{\infty} \frac{a_{n}}{n}\)
  converges, then \[
  \lim _{n \rightarrow \infty} \frac{a_{1}+\cdots+a_{n}}{n}=0
  \]
\end{enumerate}

\begin{proof}

\end{proof}
%%%%%%%%%%%%%%%%%%%%%%%%%%%%%

\hypertarget{section-1}{%
\subsection{2.}\label{section-1}}

Prove that \[
\left|\frac{d^{n}}{d x^{n}} \frac{\sin x}{x}\right| \leq \frac{1}{n}
\]

for all \(x \neq 0\) and positive integers \(n\).

\begin{quote}
Hint: Consider \(\displaystyle\int_0^1 \cos(tx) dt\)
\end{quote}

\begin{proof}

\end{proof}
%%%%%%%%%%%%%%%%%%%%%%%%%%%%%

\hypertarget{section-2}{%
\subsection{3.}\label{section-2}}

Let \(( X, \mathcal B, \mu )\) be a measure space with \(\mu(X) = 1\) and
\(\{B_n\}_{n=1}^\infty\) be a sequence of \(\mathcal B\)-measurable
subsets of \(X\), and \[
B \coloneqq \theset{x\in X \mid x\in B_n \text{ for infinitely many } n}.
\]

\begin{enumerate}
\def\labelenumi{\alph{enumi}.}
\item
  Argue that \(B\) is also a \(\mathcal{B} -\)measurable subset of
  \(X\).
\item
  Prove that if \(\sum_{n=1}^\infty \mu(B_n) < \infty\) then
  \(\mu(B)= 0\).
\item
  Prove that if \(\sum_{n=1}^\infty \mu(B_n) = \infty\) \textbf{and} the
  sequence of set complements \(\theset{B_n^c}_{n=1}^\infty\) satisfies
  \[
  \mu\left(\bigcap_{n=k}^{K} B_{n}^{c}\right)=\prod_{n=k}^{K}\left(1-\mu\left(B_{n}\right)\right)
  \] for all positive integers \(k\) and \(K\) with \(k < K\), then
  \(\mu(B) = 1\).
\end{enumerate}

\begin{quote}
Hint: Use the fact that \(1 - x \leq e^{-x}\) for all \(x\).
\end{quote}

\begin{proof}

\end{proof}
%%%%%%%%%%%%%%%%%%%%%%%%%%%%%

\hypertarget{section-3}{%
\subsection{4.}\label{section-3}}

Let \(\{u_n\}_{n=1}^\infty\) be an orthonormal sequence in a Hilbert space
\(\mathcal{H}\).

\begin{enumerate}
\def\labelenumi{\alph{enumi}.}
\item
  Prove that for every \(x \in \mathcal H\) one has \[
  \displaystyle\sum_{n=1}^{\infty}\left|\left\langle x, u_{n}\right\rangle\right|^{2} \leq\|x\|^{2}
  \]
\item
  Prove that for any sequence \(\{a_n\}_{n=1}^\infty \in \ell^2(\NN)\)
  there exists an element \(x\in\mathcal H\) such that \[
    a_n = \inner{x}{u_n} \text{ for all } n\in \NN
    \] and \[
    \norm{x}^2 = \sum_{n=1}^{\infty}\left|\left\langle x, u_{n}\right\rangle\right|^{2}
    \]
\end{enumerate}

\begin{proof}

\end{proof}
%%%%%%%%%%%%%%%%%%%%%%%%%%%%%

\hypertarget{section-4}{%
\subsection{5.}\label{section-4}}

\begin{enumerate}
\def\labelenumi{\alph{enumi}.}
\item
  Show that if \(f\) is continuous with compact support on \(\RR\), then
  \[
  \lim _{y \rightarrow 0} \int_{\mathbb{R}}|f(x-y)-f(x)| d x=0
  \]
\item
  Let \(f\in L^1(\RR)\) and for each \(h > 0\) let \[
  \mathcal{A}_{h} f(x):=\frac{1}{2 h} \int_{|y| \leq h} f(x-y) d y
  \]
\end{enumerate}

\begin{enumerate}
\def\labelenumi{\roman{enumi}.}
\setcounter{enumi}{0}
\tightlist
\item
  Prove that \(\left\|\mathcal{A}_{h} f\right\|_{1} \leq\|f\|_{1}\) for
  all \(h > 0\).
\item
  Prove that \(\mathcal{A}_h f \to f\) in \(L^1(\RR)\) as \(h \to 0^+\).
\end{enumerate}

\begin{proof}

\end{proof}
%%%%%%%%%%%%%%%%%%%%%%%%%%%%%

\hypertarget{spring-2019}{%
\section{Spring 2019}\label{spring-2019}}

\hypertarget{section}{%
\subsection{1}\label{section}}

Let \(C([0, 1])\) denote the space of all continuous real-valued
functions on \([0, 1]\).

\begin{enumerate}
\def\labelenumi{\alph{enumi}.}
\tightlist
\item
  Prove that \(C([0, 1])\) is complete under the uniform norm
  \(\norm{f}_u := \displaystyle\sup_{x\in [0,1]} |f (x)|\).
\item
  Prove that \(C([0, 1])\) is not complete under the \(L^1-\)norm
  \(\norm{f}_1 = \displaystyle\int_0^1 |f (x)| ~dx\).
\end{enumerate}

\hypertarget{section-1}{%
\subsection{2}\label{section-1}}

Let \(\mathcal B\) denote the set of all Borel subsets of \(\RR\) and
\(\mu : \mathcal B \to [0, \infty)\) denote a finite Borel measure on
\(\RR\).

\begin{enumerate}
\def\labelenumi{\alph{enumi}.}
\tightlist
\item
  Prove that if \(\{F_k\}\) is a sequence of Borel sets for which
  \(F_k \supseteq F_{k+1}\) for all \(k\), then \[
    \lim _{k \rightarrow \infty} \mu\left(F_{k}\right)=\mu\left(\bigcap_{k=1}^{\infty} F_{k}\right)
    \]
\item
  Suppose \(\mu\) has the property that \(\mu(E) = 0\) for every
  \(E \in \mathcal B\) with Lebesgue measure \(m(E) = 0\). Prove that
  for every \(\varepsilon > 0\) there exists \(\delta > 0\) so that if
  \(E \in \mathcal B\) with \(m(E) < \delta\), then \(\mu(E) < \varepsilon\).
\end{enumerate}

\hypertarget{section-2}{%
\subsection{3}\label{section-2}}

Let \(\{f_k\}\) be any sequence of functions in \(L^2([0, 1])\)
satisfying \(\norm{f_k}_2 \leq M\) for all \(k \in \NN\).

Prove that if \(f_k \to f\) almost everywhere, then \(f \in L^2([0, 1])\)
with \(\norm{f}_2 \leq M\) and \[
\lim _{k \rightarrow \infty} \int_{0}^{1} f_{k}(x) dx = \int_{0}^{1} f(x) d x
\]

\begin{quote}
Hint: Try using Fatou's Lemma to show that \(\norm{f}_2 \leq M\) and then
try applying Egorov's Theorem.
\end{quote}

\hypertarget{section-3}{%
\subsection{4}\label{section-3}}

Let \(f\) be a non-negative function on \(\RR^n\) and
\(\mathcal A = \{(x, t) \in \RR^n \times \RR : 0 \leq t \leq f (x)\}\).

Prove the validity of the following two statements:

\begin{enumerate}
\def\labelenumi{\alph{enumi}.}
\item
  \(f\) is a Lebesgue measurable function on \(\RR^n \iff \mathcal A\)
  is a Lebesgue measurable subset of \(\RR^{n+1}\)
\item
  If \(f\) is a Lebesgue measurable function on \(\RR^n\), then \[
    m(\mathcal{A})=\int_{\mathbb{R}^{n}} f(x) d x=\int_{0}^{\infty} m\left(\left\{x \in \mathbb{R}^{n}: f(x) \geq t\right\}\right) d t
    \]
\end{enumerate}

\hypertarget{section-4}{%
\subsection{5}\label{section-4}}

\begin{enumerate}
\def\labelenumi{\alph{enumi}.}
\item
  Show that \(L^2([0, 1]) \subseteq L^1([0, 1])\) and argue that \(L^2([0, 1])\)
  in fact forms a dense subset of \(L^1([0, 1])\).
\item
  Let \(\Lambda\) be a continuous linear functional on \(L^1([0, 1])\).

  Prove the Riesz Representation Theorem for \(L^1([0, 1])\) by
  following the steps below:

  \begin{enumerate}
  \def\labelenumii{\roman{enumii}.}
  \tightlist
  \item
    Establish the existence of a function \(g \in L^2([0, 1])\) which
    represents \(\Lambda\) in the sense that \[
      \Lambda(f ) = \int_0^1 f(x) \overline{g(x)} dx \text{ for all } f \in L^2([0, 1]).
    \]
  \end{enumerate}

  \begin{quote}
  Hint: You may use, without proof, the Riesz Representation Theorem for
  \(L^2([0, 1])\).
  \end{quote}

  \begin{enumerate}
  \def\labelenumii{\roman{enumii}.}
  \setcounter{enumii}{1}
  \tightlist
  \item
    Argue that the \(g\) obtained above must in fact belong to
    \(L^\infty([0, 1])\) and represent \(\Lambda\) in the sense that \[
    \Lambda(f)=\int_{0}^{1} f(x) \overline{g(x)} d x \quad \text { for all } f \in L^{1}([0,1])
    \] with \[
    \|g\|_{L^{\infty}([0,1])}=\|\Lambda\|_{L^{1}([0,1])^*}
    \]
  \end{enumerate}
\end{enumerate}

\hypertarget{fall-2018}{%
\section{Fall 2018}\label{fall-2018}}

\hypertarget{section}{%
\subsection{1}\label{section}}

Let \(f(x) = \frac 1 x\). Show that \(f\) is uniformly continuous on
\((1, \infty)\) but not on \((0,\infty)\).

\hypertarget{section-1}{%
\subsection{2}\label{section-1}}

Let \(E\subset \RR\) be a Lebesgue measurable set. Show that there is a
Borel set \(B \subset E\) such that \(m(E\setminus B) = 0\).

\hypertarget{section-2}{%
\subsection{3}\label{section-2}}

Suppose \(f(x)\) and \(xf(x)\) are integrable on \(\RR\). Define \(F\)
by \[
F(t):=\int_{-\infty}^{\infty} f(x) \cos (x t) d x
\] Show that \[
F^{\prime}(t)=-\int_{-\infty}^{\infty} x f(x) \sin (x t) d x.
\]

\hypertarget{section-3}{%
\subsection{4}\label{section-3}}

Let \(f\in L^1([0, 1])\). Prove that \[
\lim _{n \rightarrow \infty} \int_{0}^{1} f(x)|\sin n x| d x=\frac{2}{\pi} \int_{0}^{1} f(x) d x
\]

\begin{quote}
Hint: Begin with the case that \(f\) is the characteristic function of
an interval.
\end{quote}

\hypertarget{section-4}{%
\subsection{5}\label{section-4}}

Let \(f \geq 0\) be a measurable function on \(\RR\). Show that \[
\int_{\mathbb{R}} f=\int_{0}^{\infty} m(\{x: f(x)>t\}) d t
\]

\hypertarget{section-5}{%
\subsection{6}\label{section-5}}

Compute the following limit and justify your calculations: \[
\lim _{n \rightarrow \infty} \int_{1}^{n} \frac{d x}{\left(1+\frac{x}{n}\right)^{n} \sqrt[n]{x}}
\]

\hypertarget{spring-2018}{%
\section{Spring 2018}\label{spring-2018}}

\hypertarget{section}{%
\subsection{1}\label{section}}

Define \[
E:=\left\{x \in \mathbb{R}:\left|x-\frac{p}{q}\right|<q^{-3} \text { for infinitely many } p, q \in \mathbb{N}\right\}.
\]

Prove that \(m(E) = 0\).

\hypertarget{section-1}{%
\subsection{2}\label{section-1}}

Let \[
f_{n}(x):=\frac{x}{1+x^{n}}, \quad x \geq 0.
\]

\begin{enumerate}
\def\labelenumi{\alph{enumi}.}
\item
  Show that this sequence converges pointwise and find its limit. Is the
  convergence uniform on \([0, \infty)\)?
\item
  Compute \[
  \lim _{n \rightarrow \infty} \int_{0}^{\infty} f_{n}(x) d x
  \]
\end{enumerate}

\hypertarget{section-2}{%
\subsection{3}\label{section-2}}

Let \(f\) be a non-negative measurable function on \([0, 1]\).

Show that \[
\lim _{p \rightarrow \infty}\left(\int_{[0,1]} f(x)^{p} d x\right)^{\frac{1}{p}}=\|f\|_{\infty}.
\]

\hypertarget{section-3}{%
\subsection{4}\label{section-3}}

Let \(f\in L^2([0, 1])\) and suppose \[
\int_{[0,1]} f(x) x^{n} d x=0 \text { for all integers } n \geq 0.
\] Show that \(f = 0\) almost everywhere.

\hypertarget{section-4}{%
\subsection{5}\label{section-4}}

Suppose that

\begin{itemize}
\tightlist
\item
  \(f_n, f \in L^1\),
\item
  \(f_n \to f\) almost everywhere, and
\item
  \(\int\left|f_{n}\right| \rightarrow \int|f|\).
\end{itemize}

Show that \(\int f_{n} \rightarrow \int f\)

\hypertarget{fall-2017}{%
\section{Fall 2017}\label{fall-2017}}

\hypertarget{section}{%
\subsection{1}\label{section}}

Let \[
f(x) = \sum_{n=0}^{\infty} \frac{x^{n}}{n !}.
\]

Describe the intervals on which \(f\) does and does not converge
uniformly.

\hypertarget{section-1}{%
\subsection{2}\label{section-1}}

Let \(f(x) = x^2\) and \(E \subset [0, \infty) \coloneqq \RR^+\).

\begin{enumerate}
\def\labelenumi{\arabic{enumi}.}
\item
  Show that \[
  m^*(E) = 0 \iff m^*(f(E)) = 0.
  \]
\item
  Deduce that the map
\end{enumerate}

\begin{align*}
\phi: \mathcal{L}(\RR^+) &\to \mathcal{L}(\RR^+) \\
E &\mapsto f(E)
\end{align*} is a bijection from the class of Lebesgue measurable sets
of \([0, \infty)\) to itself.

\hypertarget{section-2}{%
\subsection{3}\label{section-2}}

Let \[
S = \mathrm{span}_\CC\theset{\chi_{(a, b)} \mid a, b \in \RR},
\] the complex linear span of characteristic functions of intervals of
the form \((a, b)\).

Show that for every \(f\in L^1(\RR)\), there exists a sequence of
functions \(\theset{f_n} \subset S\) such that \[
\lim _{n \rightarrow \infty}\left\|f_{n}-f\right\|_{1}=0
\]

\hypertarget{section-3}{%
\subsection{4}\label{section-3}}

Let \[
f_{n}(x)=n x(1-x)^{n}, \quad n \in \mathbb{N}.
\]

\begin{enumerate}
\def\labelenumi{\arabic{enumi}.}
\item
  Show that \(f_n \to 0\) pointwise but not uniformly on \([0, 1]\).

  \begin{quote}
  Hint: Consider the maximum of \(f_n\).
  \end{quote}
\item
  \[
    \lim _{n \rightarrow \infty} \int_{0}^{1} n(1-x)^{n} \sin(x)~dx=0
  \]
\end{enumerate}

\hypertarget{section-4}{%
\subsection{5}\label{section-4}}

Let \(\phi\) be a compactly supported smooth function that vanishes
outside of an interval \([-N, N]\) such that
\(\int_{\mathrm{R}} \phi(x) d x=1\).

For \(f\in L^1(\RR)\), define \[
K_{j}(x):=j \phi(j x), \quad \quad f \ast K_{j}(x):=\int_{\mathbb{R}} f(x-y) K_{j}(y)~dy
\] and prove the following:

\begin{enumerate}
\def\labelenumi{\arabic{enumi}.}
\item
  Each \(f\ast K_j\) is smooth and compactly supported.
\item
  \[
  \lim _{j \rightarrow \infty}\left\|f * K_{j}-f\right\|_{1}=0
  \]
\end{enumerate}

\begin{quote}
Hint: \[
\lim _{y \rightarrow 0} \int_{\mathbb{R}}|f(x-y)-f(x)| d y=0
\]
\end{quote}

\hypertarget{section-5}{%
\subsection{6}\label{section-5}}

Let \(X\) be a complete metric space and define a norm \[
\|f\|:=\max \{|f(x)|: x \in X\}.
\]

Show that \((C^0(\RR), \norm{\wait} )\) (the space of continuous
functions \(f: X\to \RR\)) is complete.

\hypertarget{spring-2017}{%
\section{Spring 2017}\label{spring-2017}}

\hypertarget{section}{%
\subsection{1}\label{section}}

Let \(K\) be the set of numbers in \([0, 1]\) whose decimal expansions
do not use the digit \(4\).

\begin{quote}
We use the convention that when a decimal number ends with 4 but all
other digits are different from 4, we replace the digit \(4\) with
\(399\cdots\). For example, \(0.8754 = 0.8753999\cdots\).
\end{quote}

Show that \(K\) is a compact, nowhere dense set without isolated points,
and find the Lebesgue measure \(m(K)\).

\hypertarget{section-1}{%
\subsection{2}\label{section-1}}

\begin{enumerate}
\def\labelenumi{\alph{enumi}.}
\item
  Let \(\mu\) be a measure on a measurable space \((X, \mathcal M)\) and
  \(f\) a positive measurable function.

  Define a measure \(\lambda\) by \[
  \lambda(E):=\int_{E} f ~d \mu, \quad E \in \mathcal{M}
  \]

  Show that for \(g\) any positive measurable function, \[
  \int_{X} g ~d \lambda=\int_{X} f g ~d \mu
  \]
\item
  Let \(E \subset \RR\) be a measurable set such that \[
  \int_{E} x^{2} ~d m=0.
  \] Show that \(m(E) = 0\).
\end{enumerate}

\hypertarget{section-2}{%
\subsection{3}\label{section-2}}

Let \[
f_{n}(x)=a e^{-n a x}-b e^{-n b x} \quad \text{ where } 0 < a < b.
\]

Show that

\begin{enumerate}
\def\labelenumi{\alph{enumi}.}
\tightlist
\item
  \(\sum_{n=1}^{\infty}\left|f_{n}\right| \text { is not in } L^{1}([0, \infty), m)\)
\end{enumerate}

\begin{quote}
Hint: \(f_n(x)\) has a root \(x_n\).
\end{quote}

\begin{enumerate}
\def\labelenumi{\alph{enumi}.}
\setcounter{enumi}{1}
\tightlist
\item
  \[
  \sum_{n=1}^{\infty} f_{n} \text { is in } L^{1}([0, \infty), m) 
  \quad \text { and } \quad 
  \int_{0}^{\infty} \sum_{n=1}^{\infty} f_{n}(x) ~d m=\ln \frac{b}{a}
  \]
\end{enumerate}

\hypertarget{section-3}{%
\subsection{4}\label{section-3}}

Let \(f(x, y)\) on \([-1, 1]^2\) be defined by \[
f(x, y) = \begin{cases}
\frac{x y}{\left(x^{2}+y^{2}\right)^{2}} & (x, y) \neq (0, 0) \\
0 & (x, y) = (0, 0)
\end{cases}
\] Determine if \(f\) is integrable.

\hypertarget{section-4}{%
\subsection{5}\label{section-4}}

Let \(f, g \in L^2(\RR)\). Prove that the formula \[
h(x):=\int_{-\infty}^{\infty} f(t) g(x-t) d t
\] defines a uniformly continuous function \(h\) on \(\RR\).

\hypertarget{section-5}{%
\subsection{5}\label{section-5}}

Show that the space \(C^1([a, b])\) is a Banach space when equipped with
the norm \[
\|f\|:=\sup _{x \in[a, b]}|f(x)|+\sup _{x \in[a, b]}\left|f^{\prime}(x)\right|.
\]

\hypertarget{fall-2016-neil-ish}{%
\section{Fall 2016 (Neil-ish)}\label{fall-2016-neil-ish}}

\hypertarget{section}{%
\subsection{1}\label{section}}

Define \[
f(x) = \sum_{n=1}^{\infty} \frac{1}{n^{x}}.
\]

Show that \(f\) converges to a differentiable function on
\((1, \infty)\) and that \[
f'(x)  =\sum_{n=1}^{\infty}\left(\frac{1}{n^{x}}\right)^{\prime}.
\]

\begin{quote}
Hint: \[
\left(\frac{1}{n^{x}}\right)^{\prime}=-\frac{1}{n^{x}} \ln n
\]
\end{quote}

\hypertarget{section-1}{%
\subsection{2}\label{section-1}}

Let \(f, g: [a, b] \to \RR\) be measurable with \[
\int_{a}^{b} f(x) ~d x=\int_{a}^{b} g(x) ~d x.
\]

Show that either

\begin{enumerate}
\def\labelenumi{\arabic{enumi}.}
\tightlist
\item
  \(f(x) = g(x)\) almost everywhere, or
\item
  There exists a measurable set \(E \subset [a, b]\) such that \[
  \int_{E} f(x) ~d x>\int_{E} g(x) ~d x
  \]
\end{enumerate}

\hypertarget{section-2}{%
\subsection{3}\label{section-2}}

Let \(f\in L^1(\RR)\). Show that \[
\lim _{x \rightarrow 0} \int_{\mathbb{R}}|f(y-x)-f(y)| d y=0
\]

\hypertarget{section-3}{%
\subsection{4}\label{section-3}}

Let \(( X, \mathcal{M}, \mu )\) be a measure space and suppose
\(\theset{E_n} \subset \mathcal M\) satisfies \[
\lim _{n \rightarrow \infty} \mu\left(X \backslash E_{n}\right)=0.
\]

Define \[
G \coloneqq \theset{x\in X : x \in E_n \text{ for only finitely many  } n}.
\]

Show that \(G \in \mathcal M\) and \(\mu(G) = 0\).

\hypertarget{section-4}{%
\subsection{5}\label{section-4}}

Let \(\phi\in L^\infty(\RR)\). Show that the following limit exists and
satisfies the equality \[
\lim _{n \rightarrow \infty}\left(\int_{\mathbb{R}} \frac{|\phi(x)|^{n}}{1+x^{2}} d x\right)^{\frac{1}{n}} = \norm{\phi}_\infty.
\]

\hypertarget{section-5}{%
\subsection{6}\label{section-5}}

Let \(f, g \in L^2(\RR)\). Show that \[
\lim _{n \rightarrow \infty} \int_{\mathbb{R}} f(x) g(x+n) d x=0
\]

\hypertarget{spring-2016-neil-ish}{%
\section{Spring 2016 (Neil-ish)}\label{spring-2016-neil-ish}}

\hypertarget{section}{%
\subsection{1}\label{section}}

For \(n\in \NN\), define \[
e_{n}=\left(1+\frac{1}{n}\right)^{n} 
\quad \text { and } \quad 
E_{n}=\left(1+\frac{1}{n}\right)^{n+1}
\]

Show that \(e_n < E_n\), and prove Bernoulli's inequality: \[
(1+x)^{n} \geq 1+n x \text { for }-1<x<\infty \text { and } n \in \mathbb{N}
\]

Use this to show the following:

\begin{enumerate}
\def\labelenumi{\arabic{enumi}.}
\tightlist
\item
  The sequence \(e_n\) is increasing.
\item
  The sequence \(E_n\) is decreasing.
\item
  \(2 < e_n < E_n < 4\).
\item
  \(\lim _{n \rightarrow \infty} e_{n}=\lim _{n \rightarrow \infty} E_{n}\).
\end{enumerate}

\hypertarget{section-1}{%
\subsection{2}\label{section-1}}

Let \(0 < \lambda < 1\) and construct a Cantor set \(C_\lambda\) by
successively removing middle intervals of length \(\lambda\).

Prove that \(m(C_\lambda) = 0\).

\hypertarget{section-2}{%
\subsection{3}\label{section-2}}

Let \(f\) be Lebesgue measurable on \(\RR\) and \(E \subset \RR\) be
measurable such that \[
0<A=\int_{E} f(x) d x<\infty.
\]

Show that for every \(0 < t < 1\), there exists a measurable set
\(E_t \subset E\) such that \[
\int_{E_{t}} f(x) d x=t A.
\]

\hypertarget{section-3}{%
\subsection{4}\label{section-3}}

Let \(E \subset \RR\) be measurable with \(m(E) < \infty\). Define \[
f(x)=m(E \cap(E+x)).
\]

Show that

\begin{enumerate}
\def\labelenumi{\arabic{enumi}.}
\tightlist
\item
  \(f\in L^1(\RR)\).
\item
  \(f\) is uniformly continuous.
\item
  \(\lim _{|x| \rightarrow \infty} f(x)=0\)
\end{enumerate}

\begin{quote}
Hint: \[
\chi_{E \cap(E+x)}(y)=\chi_{E}(y) \chi_{E}(y-x)
\]
\end{quote}

\hypertarget{section-4}{%
\subsection{5}\label{section-4}}

Let \((X, \mathcal M, \mu)\) be a measure space. For \(f\in L^1(\mu)\)
and \(\lambda > 0\), define \[
\phi(\lambda)=\mu(\{x \in X | f(x)>\lambda\}) 
\quad \text { and } \quad 
\psi(\lambda)=\mu(\{x \in X | f(x)<-\lambda\})
\]

Show that \(\phi, \psi\) are Borel measurable and \[
\int_{X}|f| ~d \mu=\int_{0}^{\infty}[\phi(\lambda)+\psi(\lambda)] ~d \lambda
\]

\hypertarget{section-5}{%
\subsection{6}\label{section-5}}

Without using the Riesz Representation Theorem, compute \[
\sup \left\{\left|\int_{0}^{1} f(x) e^{x} d x\right| \backepsilon f \in L^{2}([0,1], m) \text{ and } \|f\|_{2} \leq 1\right\}
\]

\hypertarget{fall-2015}{%
\section{Fall 2015}\label{fall-2015}}

\hypertarget{section}{%
\subsection{1}\label{section}}

Define \[
f(x)=c_{0}+c_{1} x^{1}+c_{2} x^{2}+\ldots+c_{n} x^{n} \text { with } n \text { even and } c_{n}>0.
\]

Show that there is a number \(x_m\) such that \(f(x_m) \leq f(x)\) for
all \(x\in \RR\).

\hypertarget{section-1}{%
\subsection{2}\label{section-1}}

Let \(f: \RR \to \RR\) be Lebesgue measurable.

\begin{enumerate}
\def\labelenumi{\arabic{enumi}.}
\tightlist
\item
  Show that there is a sequence of simple functions \(s_n(x)\) such that
  \(s_n(x) \to f(x)\) for all \(x\in \RR\).
\item
  Show that there is a Borel measurable function \(g\) such that
  \(g = f\) almost everywhere.
\end{enumerate}

\hypertarget{section-2}{%
\subsection{3}\label{section-2}}

Compute the following limit: \[
\lim _{n \rightarrow \infty} \int_{1}^{n} \frac{n e^{-x}}{1+n x^{2}} ~\sin \left(\frac x n\right) ~d x
\]

\hypertarget{section-3}{%
\subsection{4}\label{section-3}}

Let \(f: [1, \infty) \to \RR\) such that \(f(1) = 1\) and \[
f^{\prime}(x)= \frac{1} {x^{2}+f(x)^{2}}
\]

Show that the following limit exists and satisfies the equality \[
\lim _{x \rightarrow \infty} f(x) \leq 1 + \frac \pi 4
\]

\hypertarget{section-4}{%
\subsection{5}\label{section-4}}

Let \(f, g \in L^1(\RR)\) be Borel measurable.

\begin{enumerate}
\def\labelenumi{\arabic{enumi}.}
\tightlist
\item
  Show that
\end{enumerate}

\begin{itemize}
\tightlist
\item
  The function \[F(x, y) \coloneqq f(x-y) g(y)\] is Borel measurable on
  \(\RR^2\), and
\item
  For almost every \(y\in \RR\), \[F_y(x) \coloneqq f(x-y)g(y)\] is
  integrable with respect to \(y\).
\end{itemize}

\begin{enumerate}
\def\labelenumi{\arabic{enumi}.}
\setcounter{enumi}{1}
\tightlist
\item
  Show that \(f\ast g \in L^1(\RR)\) and \[
  \|f * g\|_{1} \leq\|f\|_{1}\|g\|_{1}
  \]
\end{enumerate}

\hypertarget{section-5}{%
\subsection{6}\label{section-5}}

Let \(f: [0, 1] \to \RR\) be continuous. Show that \[
\sup \left\{\|f g\|_{1} \suchthat g \in L^{1}[0,1],~~ \|g\|_{1} \leq 1\right\}=\|f\|_{\infty}
\]

\hypertarget{spring-2015}{%
\section{Spring 2015}\label{spring-2015}}

\hypertarget{section}{%
\subsection{1}\label{section}}

Let \((X, d)\) and \((Y, \rho)\) be metric spaces, \(f: X\to Y\), and
\(x_0 \in X\).

Prove that the following statements are equivalent:

\begin{enumerate}
\def\labelenumi{\arabic{enumi}.}
\tightlist
\item
  For every \(\varepsilon > 0 \quad \exists \delta > 0\) such that
  \(\rho( f(x), f(x_0) ) < \varepsilon\) whenever
  \(d(x, x_0) < \delta\).
\item
  The sequence \(\theset{f(x_n)}_{n=1}^\infty \to f(x_0)\) for every
  sequence \(\theset{x_n} \to x_0\) in \(X\).
\end{enumerate}

\hypertarget{section-1}{%
\subsection{2}\label{section-1}}

Let \(f: \RR \to \CC\) be continuous with period 1. Prove that \[
\lim _{N \rightarrow \infty} \frac{1}{N} \sum_{n=1}^{N} f(n \alpha)=\int_{0}^{1} f(t) d t \quad \forall \alpha \in \RR\setminus\QQ.
\]

\begin{quote}
Hint: show this first for the functions \(f(t) = e^{2\pi i k t}\) for
\(k\in \ZZ\).
\end{quote}

\hypertarget{section-2}{%
\subsection{3}\label{section-2}}

Let \(\mu\) be a finite Borel measure on \(\RR\) and \(E \subset \RR\)
Borel. Prove that the following statements are equivalent:

\begin{enumerate}
\def\labelenumi{\arabic{enumi}.}
\tightlist
\item
  \(\forall \varepsilon > 0\) there exists \(G\) open and \(F\) closed
  such that \[
  F \subseteq E \subseteq G \quad \text{and} \quad \mu(G\setminus F) < \varepsilon.
  \]
\item
  There exists a \(V \in G_\delta\) and \(H \in F_\sigma\) such that \[
  H \subseteq E \subseteq V \quad \text{and}\quad \mu(V\setminus H) = 0
  \]
\end{enumerate}

\hypertarget{section-3}{%
\subsection{4}\label{section-3}}

Define \[
f(x, y):=\left\{\begin{array}{ll}{\frac{x^{1 / 3}}{(1+x y)^{3 / 2}}} & {\text { if } 0 \leq x \leq y} \\ {0} & {\text { otherwise }}\end{array}\right.
\]

Carefully show that \(f \in L^1(\RR^2)\).

\hypertarget{section-4}{%
\subsection{5}\label{section-4}}

Let \(\mathcal H\) be a Hilbert space.

\begin{enumerate}
\def\labelenumi{\arabic{enumi}.}
\tightlist
\item
  Let \(x\in \mathcal H\) and \(\theset{u_n}_{n=1}^N\) be an orthonormal
  set. Prove that the best approximation to \(x\) in \(\mathcal H\) by
  an element in \(\mathrm{span}_\CC\theset{u_n}\) is given by \[
    \hat x \coloneqq \sum_{n=1}^N \inner{x}{u_n}u_n.
    \]
\item
  Conclude that finite dimensional subspaces of \(\mathcal H\) are
  always closed.
\end{enumerate}

\hypertarget{section-5}{%
\subsection{6}\label{section-5}}

Let \(f \in L^1(\RR)\) and \(g\) be a bounded measurable function on
\(\RR\).

\begin{enumerate}
\def\labelenumi{\arabic{enumi}.}
\tightlist
\item
  Show that the convolution \(f\ast g\) is well-defined, bounded, and
  uniformly continuous on \(\RR\).
\item
  Prove that one further assumes that \(g \in C^1(\RR)\) with bounded
  derivative, then \(f\ast g \in C^1(\RR)\) and \[
  \frac{d}{d x}(f * g)=f *\left(\frac{d}{d x} g\right)
  \]
\end{enumerate}

\hypertarget{fall-2014}{%
\section{Fall 2014}\label{fall-2014}}

\hypertarget{section}{%
\subsection{1}\label{section}}

Let \(\theset{f_n}\) be a sequence of continuous functions such that
\(\sum f_n\) converges uniformly.

Prove that \(\sum f_n\) is also continuous.

\hypertarget{section-1}{%
\subsection{2}\label{section-1}}

Let \(I\) be an index set and \(\alpha: I \to (0, \infty)\).

\begin{enumerate}
\def\labelenumi{\arabic{enumi}.}
\item
  Show that \[
  \sum_{i \in I} a(i):=\sup _{\substack{ J \subset I \\ J \text { finite }}} \sum_{i \in J} a(i)<\infty \implies I \text{ is countable.}
  \]
\item
  Suppose \(I = \QQ\) and \(\sum_{q \in \mathbb{Q}} a(q)<\infty\).
  Define \[
    f(x):=\sum_{\substack{q \in \mathbb{Q}\\ q \leq x}} a(q).
    \] Show that \(f\) is continuous at \(x \iff x\not\in \QQ\).
\end{enumerate}

\hypertarget{section-2}{%
\subsection{3}\label{section-2}}

Let \(f\in L^1(\RR)\). Show that \[
\forall\varepsilon > 0 ~~\exists \delta > 0 \text{ such that } m(E) < \delta \implies \int_{E}|f(x)| d x<\varepsilon
\]

\hypertarget{section-3}{%
\subsection{4}\label{section-3}}

Let \(g\in L^\infty([0, 1])\) Prove that \[
\int_{[0,1]} f(x) g(x) d x=0 \quad\text{for all continuous } f:[0, 1] \to \RR \implies g(x) = 0 \text{ almost everywhere. }
\]

\hypertarget{section-4}{%
\subsection{5}\label{section-4}}

\begin{enumerate}
\def\labelenumi{\arabic{enumi}.}
\item
  Let \(f \in C_c^0(\RR^n)\), and show \[
  \lim _{t \rightarrow 0} \int_{\mathbb{R}^{n}}|f(x+t)-f(x)| d x=0.
  \]
\item
  Extend the above result to \(f\in L^1(\RR^n)\) and show that \[
  f\in L^1(\RR^n),~ g\in L^\infty(\RR^n) \implies f \ast g \text{ is bounded and uniformly continuous. }
  \]
\end{enumerate}

\hypertarget{section-5}{%
\subsection{6}\label{section-5}}

Let \(1 \leq p,q \leq \infty\) be conjugate exponents, and show that \[
f \in L^p(\RR^n) \implies \|f\|_{p}=\sup _{\|g\|_{q}=1}\left|\int f(x) g(x) d x\right|
\]

\hypertarget{spring-2014}{%
\section{Spring 2014}\label{spring-2014}}

\hypertarget{section}{%
\subsection{1}\label{section}}

\begin{enumerate}
\def\labelenumi{\arabic{enumi}.}
\item
  Give an example of a continuous \(f\in L^1(\RR)\) such that
  \(f(x) \not\to 0\) as\(\abs x \to \infty\).
\item
  Show that if \(f\) is \emph{uniformly} continuous, then \[
  \lim_{\abs x \to \infty} f(x) = 0.
  \]
\end{enumerate}

\hypertarget{section-1}{%
\subsection{2}\label{section-1}}

Let \(\theset{a_n}\) be a sequence of real numbers such that \[
\theset{b_n} \in \ell^2(\NN) \implies \sum a_n b_n < \infty.
\] Show that \(\sum a_n^2 < \infty\).

\begin{quote}
Note: Assume \(a_n, b_n\) are all non-negative.
\end{quote}

\hypertarget{section-2}{%
\subsection{3}\label{section-2}}

Let \(f: \RR \to \RR\) and suppose \[
\forall x\in \RR,\quad f(x) \geq \limsup _{y \rightarrow x} f(y)
\] Prove that \(f\) is Borel measurable.

\hypertarget{section-3}{%
\subsection{4}\label{section-3}}

Let \((X, \mathcal M, \mu)\) be a measure space and suppose \(f\) is a
measurable function on \(X\). Show that \[
\lim _{n \rightarrow \infty} \int_{X} f^{n} ~d \mu =
\begin{cases}
\infty & \text{or} \\
\mu(f\inv(1)),
\end{cases}
\] and characterize the collection of functions of each type.

\hypertarget{section-4}{%
\subsection{5}\label{section-4}}

Let \(f, g \in L^1([0, 1])\) and for all \(x\in [0, 1]\) define \[
F(x):= \int_{0}^{x} f(y) d y \quad \text { and } \quad G(x):= \int_{0}^{x} g(y) d y.
\]

Prove that \[
\int_{0}^{1} F(x) g(x) d x=F(1) G(1)-\int_{0}^{1} f(x) G(x) d x
\]

\end{document}
