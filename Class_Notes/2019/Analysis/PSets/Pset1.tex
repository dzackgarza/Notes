\subsection{Exercises}

\begin{problem}\hfill\\
  Let $C$ denote the Cantor set.
\begin{enumerate}
  \item Show that $C$ contains point that is not an endpoint of one of the removed intervals.
  \item Show that $C$ is nowhere dense, meager, and has measure zero.
  \item Show that $C$ is uncountable.
\end{enumerate}
\end{problem}

\begin{solution}\hfill\\
  \begin{enumerate}
    \item First we will characterize the endpoints of the removed intervals. Let $C_n$ be the $n$Th stage of the deletion process that is used to define the Cantor set; then what remains is a union of intervals:
      $$
      C_n = [0, \frac 1 {3^n}] \union [\frac 2 {2^n}, \frac 3 {3^n}] \union \cdots \union[\frac {3^n-1} n, 1],
      $$
    and so the endpoints are precisely the numbers of the form $\frac{k}{3^n}$ where $0 \leq k \leq 3^n$. Moreover, any endpoint appearing in $C_n$ is never removed in any later step, and so all endpoints remaining in $C$ are of this form where we allow $0\leq n < \infty$.

    Thus, our goal is to produce a number $x\in [0,1]$ such that $x \neq \frac k {3^n}$ for any $k$ or $n$, but also satisfies $x\in C$. So we will need a general characterization of all of the points in $C$.

    \vspace{1em}\hrule

    Lemma: If $x\in C$, then one can find a ternary expansion for which all of the digits are either $0$ or $2$, i.e.
    $$
    x = \sum_{k=1}^\infty a_k 3^{-k} \quad \text{where } a_k \in \theset{0, 2}.
    $$

    Proof: By induction on the index $k$ in $a_k$, first consider note that if $x\in C$ then $x \in C_1 = [0, 1] \setminus [\frac 1 3, \frac 2 3]= [0, \frac 1 3] \union [\frac 2 3, 1]$. So if $x\in C_1$, then $x\not\in(\frac 1 3, \frac 2 3)$. But note that $a_1$ is computed in the following way:
    $$
    a_1 = \begin{cases}
      0 & 0 \leq x < \frac 1 3, \\
      1 & \frac 1 3 \leq x < \frac 2 3, \\
      2 & \frac 2 3 \leq x < 1.
    \end{cases}
    $$

    Since the interval $(\frac 1 3, \frac 2 3)$ is deleted in $C_1$, we find that $a_1 = 1 \iff x = \frac 1 3$. In this case, however, we claim that we can find a ternary expansion of $x$ that does not contain a $1$.  We first write
    $$
    x = \frac 1 3 = \sum_{k=1}^\infty a_k 3^{-k} \quad \text{where } a_1 = 1, a_{k>1} = 0,
    $$

    and then define
    $$
    x' = \sum_{k=1}^\infty b_k 3^{-k} \quad\text{where } b_1 = 0, b_{k>1} = 2.
    $$

    The claim now is that $x=x'$, which follows from the fact that this is a geometric sum that can be written in closed form:
    \begin{align*}
      x' &= \sum_{k=2}^\infty (2) 3^{-k} \\
      & = \left( \sum_{k=0}^\infty (2)3^{-k} \right) - 2 - 2(3\inv) \\
      & = 2\left( \sum_{k=0}^\infty 3^{-k} \right) - 2 - 2(3\inv) \\
      &= 2 (\frac 1 {1 - \frac 1 3}) - 2 - 2(3\inv) \\
      &= 2 \left(\frac 3 2\right) - 2 - 2(3\inv) \\
      &= 1 - \frac 2 3 \\
      &= \frac 1 3 = x.
    \end{align*}

    In short, we have $\frac 1 3 = (0.1)_3 = (0.222\cdots)_3$ as ternary expansions, and a similar proof shows that such an expansion without 1s can be found for any endpoint.

    For the inductive step, consider $a_n$: the claim is that if $a_n = 1$, then $x \not \in C_{n+1}$ -- that is, it is contained in one of the intervals deleted at the $n+1$st stage. Writing the deleted interval at this stage as $(a,b)$, we find that $a_n = 1$ if and only if $x \in [a, b)$. Since $x\in C$, the only way $a_n$ can be 1 is if $x$ was in fact the endpoint $a$ (since no previous digit was a 1, by hypothesis). However, as shown above, every such endpoint has a ternary expansion containing no 1s. $\qed$

    \vspace{1em}\hrule

    Therefore, if we can produce an $x$ that satisfies $x \neq \frac{k}{3^n}$ for any $k, n$ \textbf{and} $x$ has no 1s in its ternary expansion, we will have an $x\in C$ that is not an endpoint.

    So take
    $$
    x = (0.\overline{02})_3 = (0.020202\cdots)_3.
    $$

    This evidently has no 1s in its ternary expansion, and if we sum the corresponding geometric series, we find $x = \frac 1 4$. This is not of the form $\frac k {3^n}$ for any $k, n$, and thus fulfills both conditions.

  \item We first show that $C$ is nowhere dense by showing that the interior of its closure is empty, i.e. $(\overline C)^\circ = \emptyset$.

    To do so, we note that $C$ is itself closed and so $C = \overline C$. To see why this is, consider $C^c$; we'll show that it is open. By construction, $C_1^c$ is the open interval $(\frac 1 3, \frac 2 3)$ that is deleted, and similarly $C_n^c$ is the finite union of the open intervals that are deleted at the $n$th stage. But then
    $$
    C^c = \left(\intersect C_n \right)^c = \union C_n^c
    $$
    is an infinite union of open sets, which is also open. So $C$ is closed.

    It is also the case that $C$ has empty interior, so $C^\circ = \emptyset$. Towards a contradiction, suppose $x\in C$ is an interior point; then there is some neighborhood $N_\varepsilon(x) \subset C$. Since we are on the real line, we can write this as an interval $(x-\varepsilon, x+\varepsilon)$, which has length $2\varepsilon > 0$. Moreover, we have the containment
    $$
    (x-\varepsilon, x+ \varepsilon) \subset C \subset C_n
    $$
    for every $n$.

    Claim: The length of $C_n$ is $(\frac 2 3)^n$ where we define $C_0 = [0, 1]$. Letting $L_n$ be the length of $C_n$, one easy way to see that this is the case is to note that $L_n$ satisfies the recurrence relation
    $$
    L_{n+1} = \frac2 3 L_n,
    $$
    since an interval of length $\frac 1 3 L_n$ is removed at each stage. With the initial conditions $L_0 = 1$, it can be checked that $L_n = \left( \frac 2 3 \right)^n$ solves this relation.

    Now, since $x\in C = \intersect C_n$, it is in every $C_n$. So we can choose $n$ large enough such that
    $$
    \left( \frac 2 3 \right)^n \leq 2\varepsilon.
    $$

    Letting $\mu(X)$ denote the length of an interval, we always have $C \subseteq C_n$ and so $\mu(C) \leq \mu(C_n)$.

    Using the subadditivity of measures, we now have
    \begin{align*}
    (x-\varepsilon, x+ \varepsilon) \subset C \subset C_n \\
    \implies \mu(x-\varepsilon, x+ \varepsilon) \leq \mu(C) \leq \mu(C_n) \\
    \implies 2\varepsilon \leq \left( \frac 2 3 \right)^n,
    \end{align*}

    a contradiction. So $C$ has no interior points.

    But this means that
    $$
    (\overline C)^\circ = C^\circ = \emptyset,
    $$
    and so $C$ is nowhere dense.

    To see that $\mu(C) = 0$, we can use the fact that for any sets, measures are additive over disjoint sets and we have
    $$
    \mu(A) + \mu(X \setminus A) = \mu(X) \implies \mu(X \setminus A) = \mu(X) - \mu(A).
    $$

    Here we will take $X = [0,1]$, so $\mu(X) = 1$, and $A = C$ the Cantor set.

    By tracing through the construction of the Cantor set, letting $B_n$ be the length of the interval that is removed at each stage, we can deduce
    \begin{align*}
      B_1 &= \frac 1 3 \\
      B_2 &= \frac 2 9 \\
      \cdots \\
      B_n &= \frac {2^n}{3^{n+1}}
    .\end{align*}

    We can identify $B_n = \mu(C_n^c)$, and using the fact that $C_n^c \intersect C_{>n}^c = \emptyset$ and the fact that measures are additive over disjoint sets, we can compute

    \[
    \begin{align*}
      \mu(C) &= 1 - \mu(C^c) \\
             &= 1 - \mu((\intersect_{n=0}^\infty C_n)^c) \\
             &= 1 - \mu(\disjoint_{n=0}^\infty C_n^c) \\
             &= 1 - \sum_{n=0}^\infty \mu(C_n^c) \\
             &= 1 - \sum_{n=0}^\infty \frac{2^n}{3^{-n}} \\
             &= 1 - \frac 1 3 \sum_{n=0} \left( \frac 2 3 \right)^n \\
             &= 1 - \frac 1 3 \left( \frac 1 {1-\frac 2 3} \right) \\
             &= 1 - \frac 1 3 (3)  = 0,
    \end{align*}
    \]

    which is what we wanted to show. $\qed$

  \item Let $y\in [0,1]$ be arbitrary, we will construct an element $x\in C$ such that $y = f(x)$. We first note that every number has a binary expansion, and we can write
    $$
    y = \sum_{k=1}^\infty y_k 2^{-k} \quad\text{where } y_k \in \theset{0, 1}.
    $$

    Now we construct
    $$
    x = \sum_{k=1}^\infty a_k 3^{-k}\quad\text{where } a_k = 2y_k \implies a_k \in \theset{0, 2}.
    $$

    By the characterization given in part (1), we see that $x\in C$ because it has no 1s in its ternary expansion. Moreover, under $f$, we have $a_k \mapsto \frac 1 2 a_k = \frac 1 2 (2 a_k) = a_k$, and so $f(x) = y$ by construction.

    This shows that $C$ surjects onto $[0,1]$, and in particular, $\# C \geq \#[0, 1]$ holds for the cardinalities of these sets. Since $[0,1]$ is uncountable (say, by Cantor's diagonalization argument), this shows that $C$ is uncountable.
  \end{enumerate}
\end{solution}

\begin{problem}\hfill
  \begin{enumerate}
    \item Show that $X$ is $G_\delta$ iff $X^c$ is $F_\sigma$.
    \item Show that $X$ closed $\implies X$ is $G_\delta$ and $X$ open $\implies X$ is $F_\sigma$.
    \item Give an example of an $F_\sigma$ set that is not $G_\delta$, and a set that is neither.
  \end{enumerate}
\end{problem}


\begin{solution}\hfill
\begin{enumerate}
  \item To show the forward direction, suppose $X$ is a $F_\sigma$, so $X = \union_{i\in\NN} A_i$ with each $A_i$ an closed set. By definition, each $A_i^c$ is open, and we have
    $$
    X^c = \left( \union_{i\in \NN}A_i \right)^c = \intersect_{i\in\NN} A_i^c,
    $$
    which exhibits $X^c$ as a countable intersection of closed sets, making it an $G_\delta$.

    The reverse direction proceeds analogously: supposing $X^c$ is $G_\delta$, we can write $X^c = \intersect_{i\in\NN} B_i$ with each $B_i$ closed, where $B_i^c$ is open by definition, and
    $$
    X = (X^c)^c = (\intersect B_i)^c = \union B_i^c
    $$
\end{enumerate}
\end{solution}
