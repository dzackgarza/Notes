\subsection{Exercises}

\begin{problem}\hfill\\
  Let $C$ denote the Cantor set.
\begin{enumerate}
  \item Show that $C$ contains point that is not an endpoint of one of the removed intervals.
  \item Show that $C$ is nowhere dense, meager, and has measure zero.
  \item Show that $C$ is uncountable.
\end{enumerate}
\end{problem}

\begin{solution}\hfill\\
  \begin{enumerate}
    \item First we will characterize the endpoints of the removed intervals. Let $C_n$ be the $n$th stage of the deletion process that is used to define the Cantor set; then what remains is a union of intervals:
      $$
      C_n = [0, \frac 1 {3^n}] \union [\frac 2 {2^n}, \frac 3 {3^n}] \union \cdots \union[\frac {3^n-1} n, 1],
      $$
    and so the endpoints are precisely the numbers of the form $\frac{k}{3^n}$ where $0 \leq k \leq 3^n$. Moreover, any endpoint appearing in $C_n$ is never removed in any later step, and so all endpoints remaining in $C$ are of this form where we allow $0\leq n < \infty$.

    Thus, our goal is to produce a number $x\in [0,1]$ such that $x \neq \frac k {3^n}$ for any $k$ or $n$, but also satisfies $x\in C$.

    Claim: If $x\in C$, then one can find a ternary expansion for which all of the digits are either $0$ or $2$, i.e.
    $$
    x = \sum_{k=1}^\infty a_k 3^{-k} \quad \text{where } a_k \in \theset{0, 2}.
    $$

    Proof: By induction on the index $n$ in $C_n$, first consider $C_1 = [0, 1] \setminus [\frac 1 3, \frac 2 3]= [0, \frac 1 3] \union [\frac 2 3, 1]$. So if $x\in C_1$, then $x\not\in[\frac 1 3, \frac 2 3]$. But note that $a_1$ is computed in the following way:
    $$
    a_1 = \begin{cases}
      0 & 0 \leq x < \frac 1 3, \\
      1 & \frac 1 3 \leq x < \frac 2 3, \\
      2 & \frac 2 3 \leq x < 1.
    \end{cases}
    $$

    Since the interval $(\frac 1 3, \frac 2 3)$ is deleted in $C_1$, we find that $a_1 = 1$ iff $x = \frac 1 3$.In this case, however, we can write
    $$
    x = \frac 1 3 = \sum_{k=1}^\infty a_k 3^{-k} \quad \text{where } a_1 = 1, a_{k>1} = 0,
    $$

    which can alternatively be written
    $$
    x = \sum_{k=1}^\infty b_k 3^{-k} \quad\text{where } b_1 = 0, b_{k>1} = 2
    $$

    which follows from the fact that this is a geometric sum that can be written in closed form,
    $$

    $$

  \end{enumerate}
\end{solution}
