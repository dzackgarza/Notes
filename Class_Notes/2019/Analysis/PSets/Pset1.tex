\subsection{Exercises}

\begin{problem}\hfill\\
  Let $C$ denote the Cantor set.
\begin{enumerate}
  \item Show that $C$ contains point that is not an endpoint of one of the removed intervals.
  \item Show that $C$ is nowhere dense, meager, and has measure zero.
  \item Show that $C$ is uncountable.
\end{enumerate}
\end{problem}

\begin{solution}\hfill\\
  \begin{enumerate}
    \item First we will characterize the endpoints of the removed intervals. Let $C_n$ be the $n$th stage of the deletion process that is used to define the Cantor set; then what remains is a union of intervals:
      $$
      C_n = [0, \frac 1 {3^n}] \union [\frac 2 {2^n}, \frac 3 {3^n}] \union \cdots \union[\frac {3^n-1} n, 1],
      $$
    and so the endpoints are precisely the numbers of the form $\frac{k}{3^n}$ where $0 \leq k \leq 3^n$. Moreover, any endpoint appearing in $C_n$ is never removed in any later step, and so all endpoints remaining in $C$ are of this form where we allow $0\leq n < \infty$.

    Thus, our goal is to produce a number $x\in [0,1]$ such that $x \neq \frac k {3^n}$ for any $k$ or $n$, but also satisfies $x\in C$. So we will need a general characterization of all of the points in $C$.

    \\ \hrule\\

    Lemma: If $x\in C$, then one can find a ternary expansion for which all of the digits are either $0$ or $2$, i.e.
    $$
    x = \sum_{k=1}^\infty a_k 3^{-k} \quad \text{where } a_k \in \theset{0, 2}.
    $$

    Proof: By induction on the index $k$ in $a_k$, first consider note that if $x\in C$ then $x \in C_1 = [0, 1] \setminus [\frac 1 3, \frac 2 3]= [0, \frac 1 3] \union [\frac 2 3, 1]$. So if $x\in C_1$, then $x\not\in(\frac 1 3, \frac 2 3)$. But note that $a_1$ is computed in the following way:
    $$
    a_1 = \begin{cases}
      0 & 0 \leq x < \frac 1 3, \\
      1 & \frac 1 3 \leq x < \frac 2 3, \\
      2 & \frac 2 3 \leq x < 1.
    \end{cases}
    $$

    Since the interval $(\frac 1 3, \frac 2 3)$ is deleted in $C_1$, we find that $a_1 = 1 \iff x = \frac 1 3$. In this case, however, we claim that we can find a ternary expansion of $x$ that does not contain a $1$.  We first write
    $$
    x = \frac 1 3 = \sum_{k=1}^\infty a_k 3^{-k} \quad \text{where } a_1 = 1, a_{k>1} = 0,
    $$

    and then define
    $$
    x' = \sum_{k=1}^\infty b_k 3^{-k} \quad\text{where } b_1 = 0, b_{k>1} = 2.
    $$

    The claim now is that $x=x'$, which follows from the fact that this is a geometric sum that can be written in closed form:
    \begin{align*}
      x' &= \sum_{k=2}^\infty (2) 3^{-k} \\
      & = \left( \sum_{k=0}^\infty (2)3^{-k} \right) - 2 - 2(3\inv) \\
      & = 2\left( \sum_{k=0}^\infty 3^{-k} \right) - 2 - 2(3\inv) \\
      &= 2 (\frac 1 {1 - \frac 1 3}) - 2 - 2(3\inv) \\
      &= 2 \left(\frac 3 2\right) - 2 - 2(3\inv) \\
      &= 1 - \frac 2 3 \\
      &= \frac 1 3 = x.
    \end{align*}

    In short, we have $\frac 1 3 = (0.1)_3 = (0.222\cdots)_3$ as ternary expansions, and a similar proof shows that such an expansion without 1s can be found for any endpoint.

    For the inductive step, consider $a_n$: the claim is that if $a_n = 1$, then $x \not \in C_{n+1}$ -- that is, it is contained in one of the intervals deleted at the $n+1$st stage. Writing the deleted interval at this stage as $(a,b)$, we find that $a_n = 1$ if and only if $x \in [a, b)$. Since $x\in C$, the only way $a_n$ can be 1 is if $x$ was in fact the endpoint $a$. However, as shown above, every such endpoint has a ternary expansion containing no 1s. $\qed$


  \end{enumerate}
\end{solution}
