\RequirePackage[l2tabu, orthodox]{nag}

%\documentclass[]{article}
\documentclass[11pt]{scrartcl}
\usepackage[usename, dvipsnames]{xcolor}
\usepackage[pdfencoding=auto]{hyperref}
\usepackage[msc-links]{amsrefs}
\usepackage{cleveref} % use \cref{}, automatically deduces theorem, proposition, etc
\usepackage[mathletters]{ucs}
\usepackage[utf8]{inputenc}
\usepackage[T1]{fontenc}
\usepackage{datetime}

\usepackage{array}
\usepackage{mathtools}
\usepackage{amsmath, amsthm, amssymb, amsfonts, amsxtra, amscd, thmtools}
\let\proof\relax
\let\endproof\relax

% Boxes around theorem environments.
\usepackage[many]{tcolorbox}

\usepackage{color}
%\usepackage{unicode-math}
\usepackage{newunicodechar}
\newunicodechar{ε}{\varepsilon}
\newunicodechar{δ}{\delta}
\newunicodechar{µ}{\mu}
\newunicodechar{→}{\to}
\newunicodechar{≤}{\leq}
\newunicodechar{∈}{\in}
\newunicodechar{⊆}{\subseteq}
\newunicodechar{Λ}{\Lambda}
\newunicodechar{∞}{\infty}
\newunicodechar{×}{\times}
\everymath{\displaystyle}



\usepackage{microtype}
\usepackage[pdfencoding=auto]{hyperref}
\usepackage{bookmark}
\usepackage{booktabs}
\usepackage{todonotes}
\usepackage[msc-links]{amsrefs}
\usepackage{cleveref} % use \cref{}, automatically deduces theorem, proposition, etc
\usepackage{csquotes}
\usepackage{longtable}
\usepackage{tabularx}
\usepackage{bbm}
% Creating multiple types of index
\usepackage{imakeidx}

% Remove indentation for new paragraphs
\usepackage{parskip}
% But leave space before amsthm environments
\makeatletter
\def\thm@space@setup{%
  \thm@preskip=2em
  \thm@postskip=2em
}
\makeatother


\usepackage{stmaryrd}
\usepackage{adjustbox}
\usepackage{centernot}
% \centernot\whatever


% Better indicator function
\usepackage{bbm}
\newcommand{\indic}[1]{\mathbbm{1} \left[ {#1} \right] }

% Highlight quote
\usepackage{environ}
\definecolor{camel}{rgb}{0.76, 0.6, 0.42}
\definecolor{babyblue}{rgb}{0.54, 0.81, 0.94}
\definecolor{block-gray}{gray}{0.85}
\NewEnviron{myblock}
{\colorbox{block-gray}{%
\parbox{\dimexpr\linewidth-2\fboxsep\relax}{%
\small\addtolength{\leftskip}{10mm}
\addtolength{\rightskip}{10mm}
\BODY}}
}
\renewcommand{\quote}{\myblock}
\renewcommand{\endquote}{\endmyblock}

% Nice math font that journals use
%\usepackage[lite]{mtpro2}
%\usepackage{mathrsfs}
%\usepackage{mathptmx}
\usepackage{lmodern}
%\usepackage[sc]{mathpazo}

% Theorem Styles
\usepackage[framemethod=tikz]{mdframed}

\theoremstyle{definition}
\newtheorem{exercise}{Exercise}[section]
\newtheorem{solution}{Solution}

% Theorem Style
\newtheoremstyle{theorem}% name
  {0em}%         Space above, empty = `usual value'
  {1em}%         Space below
  {\normalfont}% Body font
  {\parindent}%         Indent amount (empty = no indent, \parindent = para indent)
  {\bfseries}% Thm head font
  {.}%        Punctuation after thm head
  {\newline}% Space after thm head: \newline = linebreak
  {\thmname{#1}\thmnumber{ #2}\thmnote{\itshape{(#3)}}}%
\theoremstyle{theorem}
\tcolorboxenvironment{theorem}{
  boxrule=0pt,
  boxsep=0pt,
  breakable,
  enhanced jigsaw,
  fonttitle={\large\bfseries},
  opacityback=0.8,
  colframe=cyan,
  borderline west={4pt}{0pt}{orange},
  attach title to upper={}
}
\newtheorem{theorem}{Theorem}[section]

% Proposition Style
\tcolorboxenvironment{proposition}{
  boxrule=1pt,
  boxsep=0pt,
  breakable,
  enhanced jigsaw,
  opacityback=0.0,
  colframe=cyan
}
\newtheorem{proposition}[theorem]{Proposition}
\tcolorboxenvironment{lemma}{
  boxrule=1pt,
  boxsep=0pt,
  breakable,
  enhanced jigsaw,
  opacityback=0.2,
  colframe=cyan
}
\newtheorem{lemma}[theorem]{Lemma}
% Claim
\tcolorboxenvironment{claim}{
  boxrule=1pt,
  boxsep=0pt,
  breakable,
  enhanced jigsaw,
  opacityback=0.2,
  colframe=cyan
}
\newtheorem{claim}[theorem]{Claim}


% Corollary
\tcolorboxenvironment{corollary}{
  colback=cyan,
  boxrule=1pt,
  boxsep=0pt,
  breakable,
  enhanced jigsaw,
  opacityback=0.1,
  colframe=cyan
}
\newtheorem{corollary}[theorem]{Corollary}

% Proof Style
\newtheoremstyle{proof}% name
  {0em}%         Space above, empty = `usual value'
  {2em}%         Space below
  {\normalfont}% Body font
  {\parindent}%         Indent amount (empty = no indent, \parindent = para indent)
  {\itshape}% Thm head font
  {.}%        Punctuation after thm head
  {\newline}% Space after thm head: \newline = linebreak
  {\thmname{#1} \thmnote{\itshape{(#3)}}}%         Thm head spec
\theoremstyle{proof}
\tcolorboxenvironment{proof}{
  colback=camel,
  opacityfill=0.25,
  boxrule=1pt,
  boxsep=0pt,
  breakable,
  enhanced jigsaw
}
\newtheorem*{pf}{Proof}
\newenvironment{proof}
{\pushQED{$\qed$}\pf}
{\par\popQED\endpf}

% Definition Style
\newtheoremstyle{definition}% name
  {0em}%         Space above, empty = `usual value'
  {2em}%         Space below
  {\normalfont}% Body font
  {\parindent}%         Indent amount (empty = no indent, \parindent = para indent)
  {\bfseries}% Thm head font
  {.}%        Punctuation after thm head
  {\newline}% Space after thm head: \newline = linebreak
  {}%         Thm head spec
\theoremstyle{definition}
\tcolorboxenvironment{definition}{
  colback=babyblue,
  boxrule=0pt,
  boxsep=0pt,
  opacityfill=0.45,
  breakable,
  enhanced jigsaw,
  borderline west={4pt}{0pt}{blue},
  colbacktitle={babyblue},
  coltitle={black},
  fonttitle={\large\bfseries},
  attach title to upper={},
}
\newtheorem{definition}{Definition}[theorem]

% Break Environment
\makeatletter
\newtheoremstyle{break}% name
  {}%         Space above, empty = `usual value'
  {2em}%         Space below
  {
    \addtolength{\@totalleftmargin}{2.5em}
    \addtolength{\linewidth}{-2.5em}
    \parshape 1 2.5em \linewidth
  }% Body font
  {}%         Indent amount (empty = no indent, \parindent = para indent)
  {\bfseries}% Thm head font
  {.}%        Punctuation after thm head
  {\newline}% Space after thm head: \newline = linebreak
  {}%         Thm head spec
\makeatother

\theoremstyle{break}
\newtheorem{example}{Example}[section]

% Problem Style
\newtheoremstyle{problem} % name
  {0em}                   % Space above, empty = `usual value'
  {2em}                   % Space below
  {\normalfont}           % Body font
  {\parindent}            % Indent amount (empty = no indent, \parindent = para indent)
  {\itshape}              % Thm head font
  {}                      % Punctuation after thm head
  {\newline}              % Space after thm head: \newline = linebreak
  {\thmnote{\itshape{(#3)}}}     % Thm head spec
\theoremstyle{problem}
\tcolorboxenvironment{problem}{
  boxrule=1pt,
  boxsep=0pt,
  breakable,
  enhanced jigsaw,
  opacityback=0.0,
  colframe=cyan
}
\newtheorem{problem}{Problem}


%Pagination stuff.
\setlength{\topmargin}{-.3 in}
\setlength{\oddsidemargin}{0in}
\setlength{\evensidemargin}{0in}
\setlength{\textheight}{9.in}
\setlength{\textwidth}{6.5in}
% \pagestyle{empty} %removes page numbers.

% Inkscape figures from Vim
\usepackage{import}
\usepackage{pdfpages}
\usepackage{transparent}

\newcommand{\incfig}[1]{%
    \def\svgwidth{\columnwidth}
    \import{./figures/}{#1.pdf_tex}
}
%\pdfsuppresswarningpagegroup=1

% Pandoc-specific fixes
\providecommand{\tightlist}{%
  \setlength{\itemsep}{0pt}\setlength{\parskip}{0pt}}

% Tikz and Graphics
\usepackage{amscd}
\usepackage{tikz}
\usetikzlibrary{arrows, arrows.meta, cd, fadings, patterns, calc, decorations.markings, matrix, positioning}
\tikzfading[name=fade out, inner color=transparent!0, outer color=transparent!100]
\usepackage{pgfplots}
\pgfplotsset{compat=1.16}
\usepackage[inline]{asymptote}
\usepackage{tikz-layers}

%\usepackage{nath}
%\delimgrowth=1
\DeclarePairedDelimiter\qty{(}{)}

% Major Macros
\usepackage{graphicx}
\usepackage{float}
\DeclareFontFamily{U}{mathx}{\hyphenchar\font45}
\DeclareFontShape{U}{mathx}{m}{n}{
      <5> <6> <7> <8> <9> <10>
      <10.95> <12> <14.4> <17.28> <20.74> <24.88>
      mathx10
      }{}
\DeclareSymbolFont{mathx}{U}{mathx}{m}{n}
\DeclareMathSymbol{\bigtimes}{1}{mathx}{"91}

% Wide tikz equations
\newsavebox{\wideeqbox}
\newenvironment{wideeq}
  {\begin{displaymath}\begin{lrbox}{\wideeqbox}$\displaystyle}
  {$\end{lrbox}\makebox[0pt]{\usebox{\wideeqbox}}\end{displaymath}}



% Fancy chapter headers and footers
\usepackage{fancyhdr}

\pagestyle{fancy}
\fancyhf{}
\fancyhead[LE,RO]{\title}
\fancyhead[RE,LO]{\rightmark}
\fancyfoot[CE,CO]{\leftmark}
\fancyfoot[LE,RO]{\thepage}

\renewcommand{\headrulewidth}{2pt}
\renewcommand{\footrulewidth}{1pt}

% List of Theorems Attempt
\usepackage{etoolbox}
\makeatletter
\patchcmd\thmtlo@chaptervspacehack
  {\addtocontents{loe}{\protect\addvspace{10\p@}}}
  {\addtocontents{loe}{\protect\thmlopatch@endchapter\protect\thmlopatch@chapter{\thechapter}}}
  {}{}
\AtEndDocument{\addtocontents{loe}{\protect\thmlopatch@endchapter}}
\long\def\thmlopatch@chapter#1#2\thmlopatch@endchapter{%
  \setbox\z@=\vbox{#2}%
  \ifdim\ht\z@>\z@
    \hbox{\bfseries\chaptername\ #1}\nobreak
    #2
    \addvspace{10\p@}
  \fi
}
\def\thmlopatch@endchapter{}

\makeatother
\renewcommand{\thmtformatoptarg}[1]{ -- #1}
%\renewcommand{\listtheoremname}{List of definitions}

\newcommand{\ext}{\operatorname{Ext}}
\newcommand{\Ext}{\operatorname{Ext}}
\def\Endo{\operatorname{End}}
\def\Ind{\operatorname{Ind}}
\def\ind{\operatorname{Ind}}
\def\coind{\operatorname{Coind}}
\def\Res{\operatorname{Res}}
\def\Hol{\operatorname{Hol}}
\def\res{\operatorname{Res}}
\def\endo{\operatorname{End}}
\def\ind{\operatorname{Ind}}
\renewcommand{\AA}[0]{{\mathbb{A}}}
\DeclareMathOperator{\Exists}{\exists}
\DeclareMathOperator{\Forall}{\forall}
\newcommand{\Af}[0]{{\mathbb{A}}}
\newcommand{\CC}[0]{{\mathbb{C}}}
\newcommand{\CP}[0]{{\mathbb{CP}}}
\newcommand{\DD}[0]{{\mathbb{D}}}
\newcommand{\FF}[0]{{\mathbb{F}}}
\newcommand{\GF}[0]{{\mathbb{GF}}}
\newcommand{\GG}[0]{{\mathbb{G}}}
\newcommand{\HH}[0]{{\mathbb{H}}}
\newcommand{\HP}[0]{{\mathbb{HP}}}
\newcommand{\KK}[0]{{\mathbb{K}}}
\newcommand{\kk}[0]{{\Bbbk}}
\newcommand{\bbm}[0]{{\mathbb{M}}}
\newcommand{\NN}[0]{{\mathbb{N}}}
\newcommand{\OP}[0]{{\mathbb{OP}}}
\newcommand{\PP}[0]{{\mathbb{P}}}
\newcommand{\QQ}[0]{{\mathbb{Q}}}
\newcommand{\RP}[0]{{\mathbb{RP}}}
\newcommand{\RR}[0]{{\mathbb{R}}}
\newcommand{\SpSp}[0]{{\mathbb{S}}}
\renewcommand{\SS}[0]{{\mathbb{S}}}
\newcommand{\TT}[0]{{\mathbb{T}}}
\newcommand{\ZZ}[0]{{\mathbb{Z}}}
\newcommand{\ZnZ}[0]{\mathbb{Z}/n\mathbb{Z}}
\newcommand{\ZpZ}[0]{\mathbb{Z}/p\mathbb{Z}}
\newcommand{\Qp}[0]{\mathbb{Q}_{(p)}}
\newcommand{\Zp}[0]{\mathbb{Z}_{(p)}}
\newcommand{\Arg}[0]{\mathrm{Arg}}
\newcommand{\PGL}[0]{\mathrm{PGL}}
\newcommand{\GL}[0]{\mathrm{GL}}
\newcommand{\Gl}[0]{\mathrm{GL}}
\newcommand{\gl}[0]{\mathrm{GL}}
\newcommand{\mat}[0]{\mathrm{Mat}}
\newcommand{\Mat}[0]{\mathrm{Mat}}
\newcommand{\Rat}[0]{\mathrm{Rat}}
\newcommand{\Perv}[0]{\mathrm{Perv}}
\newcommand{\Gal}[0]{\mathrm{Gal}}
\newcommand{\Hilb}[0]{\mathrm{Hilb}}
\newcommand{\Quot}[0]{\mathrm{Quot}}
\newcommand{\Art}[0]{\mathrm{Art}}
\newcommand{\red}[0]{\mathrm{red}}
\newcommand{\alg}[0]{\mathrm{alg}}
\newcommand{\Pic}[0]{{\mathrm{Pic}~}}
\newcommand{\lcm}[0]{\mathrm{lcm}}
\newcommand{\maps}[0]{\mathrm{Maps}}
\newcommand{\maxspec}[0]{{\mathrm{maxSpec}~}}
\newcommand{\Tr}[0]{\mathrm{Tr}}
\newcommand{\adj}[0]{\mathrm{adj}}
\newcommand{\ad}[0]{\mathrm{ad}~}
\newcommand{\ann}[0]{\mathrm{Ann}}
\newcommand{\Ann}[0]{\mathrm{Ann}}
\newcommand{\arcsec}[0]{\mathrm{arcsec}}
\newcommand{\ch}[0]{\mathrm{char}~}
\newcommand{\Sp}[0]{{\mathrm{Sp}}}
\newcommand{\syl}[0]{{\mathrm{Syl}}}
\newcommand{\txand}[0]{{\text{ and }}}
\newcommand{\codim}[0]{\mathrm{codim}}
\newcommand{\txor}[0]{{\text{ or }}}
\newcommand{\txt}[1]{{\text{ {#1} }}}
\newcommand{\Gr}[0]{{\text{Gr}}}
\newcommand{\Aut}[0]{{\mathrm{Aut}}}
\newcommand{\aut}[0]{\mathrm{Aut}}
\newcommand{\Inn}[0]{{\mathrm{Inn}}}
\newcommand{\Out}[0]{{\mathrm{Out}}}
\newcommand{\mltext}[1]{\left\{\begin{array}{c}#1\end{array}\right\}}
\newcommand{\Fun}[0]{{\text{Fun}}}
\newcommand{\SL}[0]{{\text{SL}}}
\newcommand{\PSL}[0]{{\text{PSL}}}
\newcommand{\SO}[0]{{\text{SO}}}
\newcommand{\SU}[0]{{\text{SU}}}
\newcommand{\SP}[0]{{\text{SP}}}
\newcommand{\per}[0]{{\text{Per}}}
\newcommand{\loc}[0]{{\text{loc}}}
\newcommand{\Top}[0]{{\text{Top}}}
\newcommand{\Sch}[0]{{\text{Sch}}}
\newcommand{\sch}[0]{{\text{Sch}}}
\newcommand{\Set}[0]{{\text{Set}}}
\newcommand{\Sets}[0]{{\text{Set}}}
\newcommand{\Grp}[0]{{\text{Grp}}}
\newcommand{\Groups}[0]{{\text{Groups}}}
\newcommand{\Homeo}[0]{{\text{Homeo}}}
\newcommand{\Diffeo}[0]{{\text{Diffeo}}}
\newcommand{\MCG}[0]{{\text{MCG}}}
\newcommand{\set}[0]{{\text{Set}}}
\newcommand{\Tor}[0]{\text{Tor}}
\newcommand{\sets}[0]{{\text{Set}}}
\newcommand{\Sm}[0]{{\text{Sm}_k}}
\newcommand{\orr}[0]{{\text{ or }}}
\newcommand{\annd}[0]{{\text{ and }}}
\newcommand{\bung}[0]{\text{Bun}_G}
\newcommand{\const}[0]{{\text{const.}}}
\newcommand{\disc}[0]{{\text{disc}}}
\newcommand{\op}[0]{^\text{op}}
\newcommand{\id}[0]{\text{id}}
\newcommand{\im}[1]{\mathrm{im}({#1})}
\newcommand{\pt}[0]{{\{\text{pt}\}}}
\newcommand{\sep}[0]{^\text{sep}}
% \newcommand{\st}[0]{~{\text{s.t.}}~}
\newcommand{\tors}[0]{{\text{tors}}}
\newcommand{\tor}[0]{\text{Tor}}
\newcommand{\height}[0]{\text{ht}}
\newcommand{\cpt}[0]{\text{compact}}
\newcommand{\abs}[1]{{\left\lvert {#1} \right\rvert}}
\newcommand{\stack}[1]{\mathclap{\substack{ #1 }}} 
\newcommand{\qtext}[1]{{\quad \text{#1} \quad}}
\newcommand{\qst}[0]{{\quad \text{such that} \quad}}
\newcommand{\actsonl}[0]{\curvearrowleft}
\newcommand{\actson}[0]{\curvearrowright}
\newcommand{\bd}[0]{{\del}}
\newcommand{\bigast}[0]{{\mathop{\Large \ast}}}
\newcommand{\coker}[0]{\operatorname{coker}}
\newcommand{\cok}[0]{\operatorname{coker}}
\newcommand{\conjugate}[1]{{\overline{{#1}}}}
\newcommand{\converges}[1]{\overset{#1}}
\newcommand{\correspond}[1]{\theset{\substack{#1}}}
\newcommand{\cross}[0]{\times}
\newcommand{\by}[0]{\times}
\newcommand{\dash}[0]{{\hbox{-}}}
\newcommand{\dd}[2]{{\frac{\partial #1}{\partial #2}\,}}
\newcommand{\definedas}[0]{\coloneqq}
\newcommand{\da}[0]{\coloneqq}
\newcommand{\del}[0]{{\partial}}
\newcommand{\directlim}[0]{\varinjlim}
\newcommand{\disjoint}[0]{{\coprod}}
\newcommand{\divides}[0]{{~\Bigm|~}}
\newcommand{\dual}[0]{^\vee}
\newcommand{\sm}[0]{\setminus}
\newcommand{\smz}[0]{\setminus\theset{0}}
\newcommand{\eps}[0]{\varepsilon}
\newcommand{\equalsbecause}[1] {\stackrel{\mathclap{\scriptscriptstyle{#1}}}{=}}
\newcommand{\floor}[1]{{\left\lfloor #1 \right\rfloor}}
\DeclarePairedDelimiter{\ceil}{\lceil}{\rceil}
\newcommand{\from}[0]{\leftarrow}
\newcommand{\tofrom}[0]{\leftrightarrows}
\newcommand{\up}[0]{\uparrow}
\newcommand{\generators}[1]{\left\langle{#1}\right\rangle}
\newcommand{\gs}[1]{\left\langle{#1}\right\rangle}
\newcommand{\homotopic}[0]{\simeq}
\newcommand{\injectivelim}[0]{\varinjlim}
\newcommand{\injects}[0]{\hookrightarrow}
\newcommand{\inner}[2]{{\left\langle {#1},~{#2} \right\rangle}}
\newcommand{\union}[0]{\cup}
\newcommand{\Union}[0]{\bigcup}
\newcommand{\intersect}[0]{\cap}
\newcommand{\Intersect}[0]{\bigcap}
\newcommand{\into}[0]{\to}
\newcommand{\inverselim}[0]{\varprojlim}
\newcommand{\inv}[0]{^{-1}}
\newcommand{\mfa}[0]{{\mathfrak{a}}}
\newcommand{\mfb}[0]{{\mathfrak{b}}}
\newcommand{\mfc}[0]{{\mathfrak{c}}}
\newcommand{\mff}[0]{{\mathfrak{f}}}
\newcommand{\mfi}[0]{{\mathfrak{I}}}
\newcommand{\mfm}[0]{{\mathfrak{m}}}
\newcommand{\mfn}[0]{{\mathfrak{n}}}
\newcommand{\mfp}[0]{{\mathfrak{p}}}
\newcommand{\mfq}[0]{{\mathfrak{q}}}
\newcommand{\mfr}[0]{{\mathfrak{r}}}
\newcommand{\lieb}[0]{{\mathfrak{b}}}
\newcommand{\liegl}[0]{{\mathfrak{gl}}}
\newcommand{\lieg}[0]{{\mathfrak{g}}}
\newcommand{\lieh}[0]{{\mathfrak{h}}}
\newcommand{\lien}[0]{{\mathfrak{n}}}
\newcommand{\liesl}[0]{{\mathfrak{sl}}}
\newcommand{\lieso}[0]{{\mathfrak{so}}}
\newcommand{\liesp}[0]{{\mathfrak{sp}}}
\newcommand{\lieu}[0]{{\mathfrak{u}}}
\newcommand{\nilrad}[0]{{\mathfrak{N}}}
\newcommand{\jacobsonrad}[0]{{\mathfrak{J}}}
\newcommand{\mm}[0]{{\mathfrak{m}}}
\newcommand{\pr}[0]{{\mathfrak{p}}}
\newcommand{\mapsvia}[1]{\xrightarrow{#1}}
\newcommand{\kx}[1]{k[x_1, \cdots, x_{#1}]}
\newcommand{\MM}[0]{{\mathcal{M}}}
\newcommand{\OO}[0]{{\mathcal{O}}}
\newcommand{\imaginarypart}[1]{{\mathcal{Im}({#1})}}
\newcommand{\mca}[0]{{\mathcal{A}}}
\newcommand{\mcb}[0]{{\mathcal{B}}}
\newcommand{\mcc}[0]{{\mathcal{C}}}
\newcommand{\mcd}[0]{{\mathcal{D}}}
\newcommand{\mce}[0]{{\mathcal{E}}}
\newcommand{\mcf}[0]{{\mathcal{F}}}
\newcommand{\mcg}[0]{{\mathcal{G}}}
\newcommand{\mch}[0]{{\mathcal{H}}}
\newcommand{\mci}[0]{{\mathcal{I}}}
\newcommand{\mcj}[0]{{\mathcal{J}}}
\newcommand{\mck}[0]{{\mathcal{K}}}
\newcommand{\mcl}[0]{{\mathcal{L}}}
\newcommand{\mcm}[0]{{\mathcal{M}}}
\newcommand{\mcp}[0]{{\mathcal{P}}}
\newcommand{\mcs}[0]{{\mathcal{S}}}
\newcommand{\mct}[0]{{\mathcal{T}}}
\newcommand{\mcu}[0]{{\mathcal{U}}}
\newcommand{\mcv}[0]{{\mathcal{V}}}
\newcommand{\mcx}[0]{{\mathcal{X}}}
\newcommand{\mcz}[0]{{\mathcal{Z}}}
\newcommand{\cl}[0]{\mathrm{cl}}
\newcommand{\trdeg}[0]{\mathrm{trdeg}}
\newcommand{\dist}[0]{\mathrm{dist}}
\newcommand{\Dist}[0]{\mathrm{Dist}}
\newcommand{\crit}[0]{\mathrm{crit}}
\newcommand{\diam}[0]{{\mathrm{diam}}}
\newcommand{\gal}[0]{\mathrm{Gal}}
\newcommand{\diff}[0]{\mathrm{Diff}}
\newcommand{\diag}[0]{\mathrm{diag}}
\newcommand{\soc}[0]{\mathrm{Soc}\,}
\newcommand{\hd}[0]{\mathrm{Head}\,}
\newcommand{\grad}[0]{\mathrm{grad}~}
\newcommand{\hilb}[0]{\mathrm{Hilb}}
\newcommand{\minpoly}[0]{{\mathrm{minpoly}}}
\newcommand{\Hom}[0]{{\mathrm{Hom}}}
\newcommand{\Map}[0]{{\mathrm{Map}}}
\newcommand{\multinomial}[1]{\left(\!\!{#1}\!\!\right)}
\newcommand{\nil}[0]{{\mathrm{nil}}}
\newcommand{\normalneq}{\mathrel{\reflectbox{$\trianglerightneq$}}}
\newcommand{\normal}[0]{{~\trianglelefteq~}}
\newcommand{\norm}[1]{{\left\lVert {#1} \right\rVert}}
\newcommand{\pnorm}[2]{{\left\lVert {#1} \right\rVert}_{#2}}
\newcommand{\notdivides}[0]{\nmid}
\newcommand{\onto}[0]{\twoheadhthtarrow}
\newcommand{\ord}[0]{{\mathrm{Ord}}}
\newcommand{\pic}[0]{{\mathrm{Pic}~}}
\newcommand{\projectivelim}[0]{\varprojlim}
\newcommand{\rad}[0]{{\mathrm{rad}~}}
\newcommand{\ralg}[0]{\mathrm{R-alg}}
\newcommand{\kalg}[0]{k\dash\mathrm{alg}}
\newcommand{\rank}[0]{\operatorname{rank}}
\newcommand{\realpart}[1]{{\mathcal{Re}({#1})}}
\newcommand{\Log}[0]{\mathrm{Log}}
\newcommand{\reg}[0]{\mathrm{Reg}}
\newcommand{\restrictionof}[2]{{\left.{#1}\right|_{#2}}}
\newcommand{\ro}[2]{{\left.{#1}\right|_{#2}}}
\newcommand{\rk}[0]{{\mathrm{rank}}}
\newcommand{\evalfrom}[0]{\Big|}
\newcommand{\rmod}[0]{{R\dash\mathrm{mod}}}
\newcommand{\Mod}[0]{{\mathrm{Mod}}}
\newcommand{\rotate}[2]{{\style{display: inline-block; transform: rotate(#1deg)}{#2}}}
\newcommand{\selfmap}[0]{{\circlearrowleft}}
\newcommand{\semidirect}[0]{\rtimes}
\newcommand{\sgn}[0]{\mathrm{sgn}}
\newcommand{\sign}[0]{\mathrm{sign}}
\newcommand{\spanof}[0]{{\mathrm{span}}}
\newcommand{\spec}[0]{\mathrm{Spec}\,}
\newcommand{\mspec}[0]{\mathrm{mSpec}~}
\newcommand{\stab}[0]{{\mathrm{Stab}}}
\newcommand{\stirlingfirst}[2]{\genfrac{[}{]}{0pt}{}{#1}{#2}}
\newcommand{\stirling}[2]{\genfrac\{\}{0pt}{}{#1}{#2}}
\newcommand{\strike}[1]{{\enclose{horizontalstrike}{#1}}}
\newcommand{\suchthat}[0]{{~\mathrel{\Big|}~}}
\newcommand{\st}[0]{{~\mathrel{\Big|}~}}
\newcommand{\supp}[0]{{\mathrm{supp}}}
\newcommand{\surjects}[0]{\twoheadrightarrow}
\newcommand{\sym}[0]{\mathrm{Sym}}
\newcommand{\tensor}[0]{\otimes}
\newcommand{\connectsum}[0]{\mathop{\Large \#}}
\newcommand{\theset}[1]{\left\{{#1}\right\}}
\newcommand{\ts}[1]{\left\{{#1}\right\}}
\newcommand{\gens}[1]{\left\langle{#1}\right\rangle}
\newcommand{\thevector}[1]{{\left[ {#1} \right]}}
\newcommand{\tv}[1]{{\left[ {#1} \right]}}
\newcommand{\too}[1]{{\xrightarrow{#1}}}
\newcommand{\transverse}[0]{\pitchfork}
\newcommand{\trianglerightneq}{\mathrel{\ooalign{\raisebox{-0.5ex}{\reflectbox{\rotatebox{90}{$\nshortmid$}}}\cr$\triangleright$\cr}\mkern-3mu}}
\newcommand{\tr}[0]{\mathrm{Tr}}
\newcommand{\uniformlyconverges}[0]{\rightrightarrows}
\newcommand{\covers}[0]{\rightrightarrows}
\newcommand{\units}[0]{^{\times}}
\newcommand{\nonzero}[0]{^{\bullet}}
\newcommand{\wait}[0]{{\,\cdot\,}}
\newcommand{\wt}[0]{{\mathrm{wt}}}
\renewcommand{\bar}[1]{\mkern 1.5mu\overline{\mkern-1.5mu#1\mkern-1.5mu}\mkern 1.5mu}
\renewcommand{\div}[0]{\mathrm{Div}}
\newcommand{\Div}[0]{\mathrm{Div}}
\renewcommand{\hat}[1]{\widehat{#1}}
\renewcommand{\mid}[0]{\mathrel{\Big|}}
\renewcommand{\qed}[0]{\hfill\blacksquare}
\renewcommand{\too}[0]{\longrightarrow}
\renewcommand{\vector}[1]{\mathbf{#1}}
\let\oldexp\exp
\renewcommand{\exp}[1]{\oldexp\qty{#1}}
\let\oldperp\perp
\renewcommand{\perp}[0]{^\oldperp}
\newcommand*\dif{\mathop{}\!\mathrm{d}}
\newcommand{\ddt}{\tfrac{\dif}{\dif t}}
\newcommand{\ddx}{\tfrac{\dif}{\dif x}}

\DeclareMathOperator{\righttriplearrows} {{\; \tikz{ \foreach \y in {0, 0.1, 0.2} { \draw [-stealth] (0, \y) -- +(0.5, 0);}} \; }}


\let\Begin\begin
\let\End\end
\newcommand\wrapenv[1]{#1}

\makeatletter
\def\ScaleWidthIfNeeded{%
 \ifdim\Gin@nat@width>\linewidth
    \linewidth
  \else
    \Gin@nat@width
  \fi
}
\def\ScaleHeightIfNeeded{%
  \ifdim\Gin@nat@height>0.9\textheight
    0.9\textheight
  \else
    \Gin@nat@width
  \fi
}
\makeatother

\setkeys{Gin}{width=\ScaleWidthIfNeeded,height=\ScaleHeightIfNeeded,keepaspectratio}%

\title{
\textbf{
    Algebra
  }
  }
\author{D. Zack Garza}
\date{\today}

\begin{document}

\maketitle
% \todo{Insert title and subtitle.}
\tableofcontents


\begin{center}\rule{0.5\linewidth}{\linethickness}\end{center}

\hypertarget{major-theorems}{%
\section{Major Theorems}\label{major-theorems}}

\wrapenv{\Begin{theorem}[Cauchy]}

For any prime \(p\) dividing the order of \(G\), there is an element
\(x\) of order \(p\) (and thus a subgroup \(H = \generators{x}\)).
\wrapenv{\End{theorem}}

\wrapenv{\Begin{theorem}[Lagrange]}

If \(H \leq G\) is a subgroup, then \(\abs{H} \divides \abs{G}\).
\wrapenv{\End{theorem}}

\wrapenv{\Begin{theorem}[Sylow 1]}

If \(\abs{G} = n = \prod p_{i}^{a_{i}}\) as a prime factorization, then
\(G\) has subgroups of order \(p_{i}^{a_{i}}\) for every \(i\).
Moreover, this holds for any \(1 \leq r \leq a_{i}\).
\wrapenv{\End{theorem}}

\wrapenv{\Begin{theorem}[Classification of finitely generated abelian groups]}

If \(G\) is a finitely generated abelian group, then
\(G \cong F \oplus T\), where \(F\) is free abelian and \(T\) is a
torsion group. If \(\abs T = n\), then
\(T \cong \bigoplus \ZZ_{p_{i}^{\alpha_{i}}}\) where
\(n = \prod p_{i}^{\alpha_{i}}\) is some factorization of \(n\) with the
\(p_{i}\) \textbf{not necessarily distinct}. \wrapenv{\End{theorem}}

\wrapenv{\Begin{theorem}}

Conjugacy classes partition \(G\) \begin{align*}
|G|=|Z(G)| + \sum_{\text{One representative in each orbit}} |C_{G}\left(g_{i}\right) |
= \sum_{asdsa} [G: C(g_{i}) ]
.\end{align*} \wrapenv{\End{theorem}}

Some nice lemmas:

\begin{itemize}
\tightlist
\item
  Every subgroup of a cyclic group is itself cyclic.
\end{itemize}

\hypertarget{lecture-1-thu-15-aug-2019}{%
\section{Lecture 1 (Thu 15 Aug 2019)}\label{lecture-1-thu-15-aug-2019}}

\begin{quote}
We'll be using Hungerford's Algebra text.Show that a finite abelian
group that is not cyclic contains a subgroup which is isomorphic
\end{quote}

\hypertarget{definitions}{%
\subsection{Definitions}\label{definitions}}

The following definitions will be useful to know by heart:

\begin{itemize}
\tightlist
\item
  The order of a group
\item
  Cartesian product
\item
  Relations
\item
  Equivalence relation
\item
  Partition
\item
  Binary operation
\item
  Group
\item
  Isomorphism
\item
  Abelian group
\item
  Cyclic group
\item
  Subgroup
\item
  Greatest common divisor
\item
  Least common multiple
\item
  Permutation
\item
  Transposition
\item
  Orbit
\item
  Cycle
\item
  The symmetric group \(S^{n}\)
\item
  The alternating group \(A_{n}\)
\item
  Even and odd permutations
\item
  Cosets
\item
  Index
\item
  The direct product of groups
\item
  Homomorphism
\item
  Image of a function
\item
  Inverse image of a function
\item
  Kernel
\item
  Normal subgroup
\item
  Factor group
\item
  Simple group
\end{itemize}

Here is a rough outline of the course:

\begin{itemize}
\tightlist
\item
  Group Theory

  \begin{itemize}
  \tightlist
  \item
    Groups acting on sets
  \item
    Sylow theorems and applications
  \item
    Classification
  \item
    Free and free abelian groups
  \item
    Solvable and simple groups
  \item
    Normal series
  \end{itemize}
\item
  Galois Theory

  \begin{itemize}
  \tightlist
  \item
    Field extensions
  \item
    Splitting fields
  \item
    Separability
  \item
    Finite fields
  \item
    Cyclotomic extensions
  \item
    Galois groups
  \item
    Solvability by radicals
  \end{itemize}
\item
  Module theory

  \begin{itemize}
  \tightlist
  \item
    Free modules
  \item
    Homomorphisms
  \item
    Projective and injective modules
  \item
    Finitely generated modules over a PID
  \end{itemize}
\item
  Linear Algebra

  \begin{itemize}
  \tightlist
  \item
    Matrices and linear transformations
  \item
    Rank and determinants
  \item
    Canonical forms
  \item
    Characteristic polynomials
  \item
    Eigenvalues and eigenvectors
  \end{itemize}
\end{itemize}

\hypertarget{preliminaries}{%
\subsection{Preliminaries}\label{preliminaries}}

\wrapenv{\Begin{definition}}

A \textbf{group} is an ordered pair \((G, \wait: G\cross G \to G)\)
where \(G\) is a set and \(\wait\) is a binary operation, which
satisfies the following axioms:

\begin{itemize}
\tightlist
\item
  Associativity: \((g_1 g_2)g_3 = g_1(g_2 g_3)\),
\item
  Identity: \(\exists e\in G \suchthat ge = eg = g\),
\item
  Inverses: \(g\in G \implies \exists h\in G \suchthat gh = gh = e\).
\end{itemize}

\wrapenv{\End{definition}}

\wrapenv{\wrapenv{\Begin{example}}}
\hfill

\begin{itemize}
\tightlist
\item
  \((\ZZ, +)\)
\item
  \((\QQ, +)\)
\item
  \((\QQ\units, \times)\)
\item
  \((\RR\units, \times)\)
\item
  (\(\GL(n, \RR), \times) = \theset{A \in \mathrm{Mat}_n \suchthat \det(A) \neq 0}\)
\item
  \((S_n, \circ)\)
\end{itemize}

\wrapenv{\wrapenv{\End{example}}}

\wrapenv{\Begin{definition}}

A subset \(S \subseteq G\) is a \textbf{subgroup} of \(G\) iff

\begin{enumerate}
\def\labelenumi{\arabic{enumi}.}
\tightlist
\item
  \(s_1, s_2 \in S \implies s_1 s_2 \in S\)
\item
  \(e\in S\)
\item
  \(s\in S \implies s\inv \in S\)
\end{enumerate}

\wrapenv{\End{definition}}

We denote such a subgroup \(S \leq G\).

Examples of subgroups:

\begin{itemize}
\tightlist
\item
  \((\ZZ, +) \leq (\QQ, +)\)
\item
  \(\SL(n, \RR) \leq \GL(n, \RR)\), where
  \(\SL(n, \RR) = \theset{A\in \GL(n, \RR) \suchthat \det(A) = 1}\)
\end{itemize}

\hypertarget{cyclic-groups}{%
\subsection{Cyclic Groups}\label{cyclic-groups}}

\wrapenv{\Begin{definition}}

A group \(G\) is \textbf{cyclic} iff \(G\) is generated by a single
element. \wrapenv{\End{definition}}

\wrapenv{\Begin{exercise}}

Show
\(\generators{g} = \theset{g^n \suchthat n\in\ZZ} \cong \intersect \theset{H \leq G \suchthat g \in H}\).
\wrapenv{\End{exercise}}

\wrapenv{\Begin{theorem}}

Let \(G\) be a cyclic group, so \(G \generators{g}\).

\begin{itemize}
\tightlist
\item
  If \(\abs{G} = \infty\), then \(G \cong \ZZ\).
\item
  If \(\abs{G} = n < \infty\), then \(G \cong \ZZ_n\).
\end{itemize}

\wrapenv{\End{theorem}}

\wrapenv{\Begin{definition}}

Let \(H \leq G\), and define a \textbf{right coset of \(G\)} by
\(aH = \theset{ah \suchthat H \in H}\). A similar definition can be made
for \textbf{left cosets}. \wrapenv{\End{definition}}

Then \(aH = bH \iff b\inv a \in G\) and \(Ha = Hb \iff ab\inv \in H\).

Some facts:

\begin{itemize}
\tightlist
\item
  Cosets partition \(H\),
  i.e.~\(b\not\in H \implies aH \intersect bH = \theset{e}\).
\item
  \(\abs{H} = \abs{aH} = \abs{Ha}\) for all \(a\in G\).
\end{itemize}

\wrapenv{\Begin{theorem}[Lagrange]}

If \(G\) is a finite group and \(H \leq G\), then
\(\abs{H} \divides \abs{G}\). \wrapenv{\End{theorem}}

\wrapenv{\Begin{definition}}

A subgroup \(N \leq G\) is \textbf{normal} iff \(gN = Ng\) for all
\(g\in G\), or equivalently \(gNg\inv \subseteq N\). I denote this
\(N \normal G\).

\wrapenv{\End{definition}}

When \(N \normal G\), the set of left/right cosets of \(N\) themselves
have a group structure. So we define
\begin{align*}G/N = \theset{gN \suchthat g\in G} \text{ where } (g_1 N)(g_2 N) = (g_1 g_2) N.
\end{align*}

Given \(H, K \leq G\), define
\(HK = \theset{hk \suchthat h\in H, k\in K}\). We have a general
formula,
\begin{align*}
\abs{HK} = \frac{\abs H \abs K}{\abs{H \intersect K}}.
\end{align*}

\hypertarget{homomorphisms}{%
\subsection{Homomorphisms}\label{homomorphisms}}

\wrapenv{\Begin{definition}}

Let \(G,G'\) be groups, then \(\varphi: G \to G'\) is a
\textbf{homomorphism} if \(\varphi(ab) = \varphi(a) \varphi(b)\).
\wrapenv{\End{definition}}

\wrapenv{\Begin{example}}

\begin{itemize}
\tightlist
\item
  \(\exp: (\RR, +) \to (\RR^{> 0}, \wait)\) where
  \(\exp(a+b) = e^{a+b} = e^a e^b = \exp(a) \exp(b)\).
\item
  \(\det: (\GL(n, \RR), \times) \to (\RR\units, \times)\) where
  \(\det(AB) = \det(A) \det(B)\).
\item
  Let \(N \normal G\) and \(\varphi G \to G/N\) given by
  \(\varphi(g) = gN\).
\item
  Let \(\varphi: \ZZ \to \ZZ_n\) where \(\phi(g) = [g] = g \mod n\)
  where \(\ZZ_n \cong \ZZ/n\ZZ\)
\end{itemize}

\wrapenv{\End{example}}

\wrapenv{\Begin{definition}}

Let \(\varphi: G \to G'\). Then \(\varphi\) is a \textbf{monomorphism}
iff it is injective, an \textbf{epimorphism} iff it is surjective, and
an \textbf{isomorphism} iff it is bijective. \wrapenv{\End{definition}}

\hypertarget{direct-products}{%
\subsection{Direct Products}\label{direct-products}}

Let \(G_1, G_2\) be groups, then define
\begin{align*}
G_1 \cross G_2 = \theset{(g_1, g_2) \suchthat g_1 \in G, g_2 \in G_2} \text{ where } (g_1, g_2)(h_1, h_2) = (g_1 h_1, g_2 ,h_2).
\end{align*}

We have the formula \(\abs{G_1 \cross G_2} = \abs{G_1} \abs{G_2}\).

\hypertarget{finitely-generated-abelian-groups}{%
\subsection{Finitely Generated Abelian
Groups}\label{finitely-generated-abelian-groups}}

\wrapenv{\Begin{definition}}

We say a group is \textbf{abelian} if \(G\) is commutative,
i.e.~\(g_1, g_2 \in G \implies g_1 g_2 = g_2 g_1\).
\wrapenv{\End{definition}}

\wrapenv{\Begin{definition}}

A group is \textbf{finitely generated} if there exist
\(\theset{g_1, g_2, \cdots g_n} \subseteq G\) such that
\(G = \generators{g_1, g_2, \cdots g_n}\). \wrapenv{\End{definition}}

This generalizes the notion of a cyclic group, where we can simply
intersect all of the subgroups that contain the \(g_i\) to define it.

We know what cyclic groups look like -- they are all isomorphic to
\(\ZZ\) or \(\ZZ_n\). So now we'd like a structure theorem for abelian
finitely generated groups.

\wrapenv{\Begin{theorem}}

Let \(G\) be a finitely generated abelian group. Then
\begin{align*}G \cong \ZZ^r \times \displaystyle\prod_{i=1}^s \ZZ_{p_i^{\alpha _i}}\end{align*}
for some finite \(r,s \in \NN\) and \(p_i\) are (not necessarily
distinct) primes. \wrapenv{\End{theorem}}

\wrapenv{\Begin{example}}

Let \(G\) be a finite abelian group of order 4. Then \(G \cong \ZZ_4\)
or \(\ZZ_2^2\), which are not isomorphic because every element in
\(\ZZ_2^2\) has order 2 where \(\ZZ_4\) contains an element of order 4.
\wrapenv{\End{example}}

\hypertarget{fundamental-homomorphism-theorem}{%
\subsection{Fundamental Homomorphism
Theorem}\label{fundamental-homomorphism-theorem}}

Let \(\varphi: G \to G'\) be a group homomorphism and define
\(\ker \varphi = \theset{g\in G \suchthat \varphi(g) = e'}\).

\hypertarget{the-first-homomorphism-theorem}{%
\subsubsection{The First Homomorphism
Theorem}\label{the-first-homomorphism-theorem}}

\wrapenv{\Begin{theorem}}

There exists a map \(\varphi': G/\ker \varphi \to G'\) such that the
following diagram commutes:
\begin{align*}
\begin{center}
\begin{tikzcd}
G \arrow[dd, "\eta"'] \arrow[rr, "\varphi", dotted] &  & G' \\
&  &    \\
G/\ker \varphi \arrow[rruu, "\varphi'"]             &  &
\end{tikzcd}
\end{center}
\end{align*}

That is, \(\varphi = \varphi' \circ \eta\), and \(\varphi'\) is an
isomorphism onto its image, so \(G/\ker \varphi = \im \varphi\). This
map is given by \(\varphi'(g(\ker \varphi)) = \varphi(g)\).
\wrapenv{\End{theorem}}

\wrapenv{\Begin{exercise}}

Check that \(\varphi\) is well-defined. \wrapenv{\End{exercise}}

\hypertarget{the-second-theorem}{%
\subsubsection{The Second Theorem}\label{the-second-theorem}}

\wrapenv{\Begin{theorem}}

Let \(K, N \leq G\) where \(N \normal G\). Then
\begin{align*}
\frac K {N \intersect K} \cong \frac {NK} N
\end{align*} \wrapenv{\End{theorem}}

\wrapenv{\Begin{proof}}

Define a map \(K \mapsvia{\varphi} NK/N\) by \(\varphi(k) = kN\). You
can show that \(\varphi\) is onto by looking at \(\ker \varphi\); note
that \(kN = \varphi(k) = N \iff k \in N\), and so
\(\ker \varphi = N \intersect K\). \wrapenv{\End{proof}}

\hypertarget{lecture-2}{%
\section{Lecture 2}\label{lecture-2}}

%\listoftodos

\bibliography{/home/zack/Notes/library.bib}

\end{document}
