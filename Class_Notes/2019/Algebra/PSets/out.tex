\hypertarget{exercises}{%
\section{Exercises}\label{exercises}}

If \(\sigma = (i_1 i_2 \cdots i_r) \in S_n\) and \(\tau \in S_n\), then
show that
\(\tau\sigma\tau^{-1} = (\tau(i_1) \tau(i_2) \cdots \tau(i_r))\).

Show that \(S_n \cong \left\langle (12), (123\cdots n)\right\rangle\)
and also that \(S_n \cong \left\langle (12), (23\cdots n)\right\rangle\)

Let \(G\) be a finite abelian group that is not cyclic. Show that \(G\)
contains a subgroup isomorphic to \(\mathbb{Z}_p \oplus \mathbb{Z}_p\)
for some prime \(p\).

Determine all abelian groups of order \(n\) for \(n\leq 20\).

Let \(G\) be a group and \(A \trianglelefteq G\) be a normal abelian
subgroup. Show that \(G/A\) acts on \(A\) by conjugation and construct a
homomorphism \(\varphi: G/A \to \mathrm{Aut}(A)\).

Let \(Z(G)\) be the center of \(G\). Show that if \(G/Z(G)\) is cyclic,
then \(G\) is abelian.

\emph{Note that Hungerford uses the notation \(C(G)\) for the center.}

Let \(G\) be a finite group and \(H \trianglelefteq G\) a normal
subgroup of order \(p^k\). Show that \(H\) is contained in every Sylow
\(p\)-subgroup of \(G\).

Let \(\left| G \right| = p^n q\) for some primes \(p > q\). Show that
\(G\) contains a unique normal subgroup of index \(q\).

\hypertarget{qual-problems}{%
\section{Qual Problems}\label{qual-problems}}

Let \(G\) be a finite group and \(p\) a prime number. Let \(X_p\) be the
set of Sylow-\(p\) subgroups of \(G\) and \(n_p\) be the cardinality of
\(X_p\). Let \(\mathrm{Sym}(X)\) be the permutation group on the set
\(X_p\).

\begin{enumerate}
\item
  Construct a homomorphism \(\rho: G \to \mathrm{Sym}(X_p)\) with image
  a transitive subgroup (i.e. with a single orbit).
\item
  Deduce that \(G\) is simple and the order of \(G\) divides \(n_p!\).
\item
  Show that for any \(1\leq a \leq 4\) and any prime power \(p^k\), no
  group of order \(ap^k\) is simple.
\end{enumerate}

Let \(G\) be a finite group and \(H<G\) a subgroup. Let \(n_H\) be the
number of subgroups of \(G\) that are conjugate to \(H\). Show that
\(n_H\) divides the order of \(G\).

Let \(G=S_5\), the symmetric group on 5 elements. Identify all conjugacy
classes of elements in \(G\), provide a representative from each class,
and prove that this list is complete.
