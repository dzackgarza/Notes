\subsection{Exercises}

\begin{problem}[Hungerford 1.6.3]
If $\sigma = (i_1 i_2 \cdots i_r) \in S_n$ and $\tau \in S_n$, then show that $\tau\sigma\tau^{-1} = (\tau(i_1) \tau(i_2) \cdots \tau(i_r))$.
\end{problem}

\begin{solution}
  Let $$\sigma = (s_1 s_2 \cdots s_r) \in S_n$$ in cycle notation, and $\tau \in S_n$ be arbitrary. Define $t_j = \tau(s_j)$; we would then like to show that $$(t_1, t_2, \cdots t_r) \coloneqq (\tau(s_1) \tau(s_2) \cdots \tau(s_r)) = \tau\sigma\tau\inv$$

  To this end, it suffices to show that $t_i$ maps to $t_{i+1 \mod r}$, under $\tau\sigma\tau\inv$, which is to say
  $$
  \tau\sigma\tau\inv(t_i) =
  \begin{cases}
    t_{i+1} & i+1 \leq r, \\
    t_1 & i = r
  \end{cases}.
  $$

  Bearing this in mind, we will immediately suppress notation and take all indices $\mod r$ for the rest of this problem.

  The following then follows simply by definitions:
  \begin{align*}
  \tau\sigma\tau\inv(t_i) &= \tau\sigma(s_i) \\
                          &= \tau(s_{i+1}) \\
                          &= t_{i+1}.
  .\end{align*}

\end{solution}

\begin{problem}[Hungerford 1.6.4]
\label{prob:1.2}
Show that $S_n \cong \left\langle (12), (123\cdots n)\right\rangle$ and also that $S_n \cong \left\langle (12), (23\cdots n)\right\rangle$
\end{problem}

\begin{problem}[Hungerford 2.2.1]
\label{prob:1.3}
Let $G$ be a finite abelian group that is not cyclic. Show that $G$ contains a subgroup isomorphic to $\mathbb{Z}_p \oplus \mathbb{Z}_p$ for some prime $p$.
\end{problem}

\begin{problem}[Hungerford 2.2.12.b]
\label{prob:1.4}
Determine all abelian groups of order $n$ for $n\leq 20$.
\end{problem}

\begin{problem}[Hungerford 2.4.1]
\label{prob:1.5}
Let $G$ be a group and $A \trianglelefteq G$ be a normal abelian subgroup. Show that $G/A$ acts on $A$ by conjugation and construct a homomorphism $\varphi: G/A \to \mathrm{Aut}(A)$.
\end{problem}

\begin{problem}[Hungerford 2.4.9]
\label{prob:1.6}
Let $Z(G)$ be the center of $G$. Show that if $G/Z(G)$ is cyclic, then $G$ is abelian.

\textit{Note that Hungerford uses the notation $C(G)$ for the center.}
\end{problem}

\begin{problem}[Hungerford 2.5.6]
\label{prob:1.7}
Let $G$ be a finite group and $H \trianglelefteq G$ a normal subgroup of order $p^k$. Show that $H$ is contained in every Sylow $p$-subgroup of $G$.
\end{problem}

\begin{problem}[Hungerford 2.5.9]
\label{prob:1.8}
Let $\left| G \right| = p^n q$ for some primes $p > q$. Show that $G$ contains a unique normal subgroup of index $q$.
\end{problem}

\subsection{Qual Problems}

\begin{problem}
\label{prob:1.9}
Let $G$ be a finite group and $p$ a prime number. Let $X_p$ be the set of Sylow-$p$ subgroups of $G$ and $n_p$ be the cardinality of $X_p$. Let $\mathrm{Sym}(X)$ be the permutation group on the set $X_p$.
    \begin{enumerate}
        \item Construct a homomorphism $\rho: G \to \mathrm{Sym}(X_p)$ with image a transitive subgroup (i.e. with a single orbit).
        \item Deduce that $G$ is simple and the order of $G$ divides $n_p!$.
        \item Show that for any $1\leq a \leq 4$ and any prime power $p^k$, no group of order $ap^k$ is simple.
    \end{enumerate}
\end{problem}

\begin{problem}
\label{prob:1.10}
Let $G$ be a finite group and $H<G$ a subgroup. Let $n_H$ be the number of subgroups of $G$ that are conjugate to $H$. Show that $n_H$ divides the order of $G$.
\end{problem}

\begin{problem}
\label{prob:1.11}
Let $G=S_5$, the symmetric group on 5 elements. Identify all conjugacy classes of elements in $G$, provide a representative from each class, and prove that this list is complete.
\end{problem}
