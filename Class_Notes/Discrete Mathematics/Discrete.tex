\documentclass[a4paper,10pt]{report}
\usepackage[utf8]{inputenc}
\usepackage{amssymb}
\usepackage{amsthm}
\usepackage{amsmath}
\usepackage{graphicx}
\usepackage[bookmarksopen, linktocpage]{hyperref}
\setlength{\parindent}{0pt}

\newcommand{\norm}[1]{\lVert#1\rVert} 		%Double vertical bars for norms
\newcommand{\ip}[2]{\langle#1,#2\rangle}	%Inner product
\newcommand{\vb}[1]{\mathbf{#1}}		%Bold vector

\makeatletter
\renewcommand*\env@matrix[1][*\c@MaxMatrixCols c]{%
  \hskip -\arraycolsep
  \let\@ifnextchar\new@ifnextchar
  \array{#1}}
\makeatother
% Augmented matrices. Use like this:
% \begin{pmatrix}[cc|c]

\begin{document}
\title{Discrete Mathematics, Spring 2014 Notes}
\author{Zack Garza}
\maketitle
\tableofcontents

\chapter{Recurrence Relations}
\section{Applications}
Definition: If the $n$th term a sequence, $a_n$, is a function of a previous term of the sequence, we call the sequence $a_n + a_{n-1} + a_{n-2}\cdots +a_0$ a \textbf{recurrence relation}.

Definition: If a sequence $s$ satisfies a recurrence relation, it is called a \textit{solution}. A recurrence relation along with initial conditions uniquely determines a solution sequence.


Examples:
\begin{enumerate}
	\item
		The Fibonacci Sequence ${f_n}$ satisfies the recurrence relation
		\begin{gather*}
			f_n = f_{n-1} + f_{n-2} \\
			\text{where } \\
			f_1 = 1 \text{ and } f_2 = 1.
		\end{gather*}
	\item
		The Tower of Hanoi: The goal is to move all the discs to peg 3. Let $H_n$ denote the number of moves needed to solve a tower with $n$ discs. Determine a recurrence relation for the sequence $\{H_n\}$.

		\textit{Solution:}
		Starting with $n$ discs on peg 1, move $n-1$ discs to peg 2. While we don't know how many moves this will take, we do know that it will be equal to $H_{n-1}$ moves.

		Use one move to move the largest disc to peg 3, which adds 1 move.

		The remaining $n-1$ discs can then be moved to peg 3 in $H_{n-1}$ moves.
		%?%

		This gives the recurrent relation
		\begin{gather*}
			H_n = 2H_{n-1} + 1 \\
			\text{where } H_1 = 1.
		\end{gather*}

		This can be solved iteratively:
		\begin{align*}
			H_n &= &2   &H_{n-1} 			&+ 1 			\\
				&= &2   &(2H_{n-2} + 1) 	&+ 1 			\\
				&= &2^2 &H_{n-2} 			&+ 2 + 1 		\\
				&= &2^2 &(2H_{n-3} + 1) 	&+ 2 + 1 		\\
				&= &2^3 &H_{n-3}			&+ 2^2 + 2 + 1
		\end{align*}

		Generalizing the above pattern, we see that
		\begin{align*}
			H_n &= 2^{n-1}H_1 + 2^{n-2} + 2^{n-3}+\cdots+2+1							\\
				&= 2^{n-1} + 2^{n-2} + 2^{n-3}+\cdots+2+1 &\text{ (because $H_1$ = 1)} 	\\
				&= \sum_{i=0}^{n-1} 2^i \\
				&= 2^n - 1 &\text{(Sum of a geometric series)}  %Listed in Theorem 1, Section 2.4
		\end{align*}
\end{enumerate}

\section{Explicit Solutions}
Definition: Given a recurrence relation of the form
\begin{gather}\label{definition:recurrence_relation}
	a_n = c_1a_{n-1} + c_2a_{n-2} + \cdots + c_ka_{n-k} \\
	\text{where } c_1, c_2, \cdots c_k \in \mathbb{R} \text{ and } c_k \neq 0, \notag
\end{gather}
we refer to $a_n$ as a \textit{linear homogeneous relation of degree k with constant coefficients}.

These terms are defined as follows:
\begin{enumerate}
	\item \textbf{Linear: }
		The right hand side of ~\ref{definition:recurrence_relation} is the sum of the previous terms in $a_n$ multiplied by a function of $n$. \\
		For example, $a_n = a_{n-1} + a_{n-2}^2$ is not linear.
	\item \textbf{Homogeneous: }
		No terms occur that are not multiples of the $a_j$s. 				\\ %?
		For example, $H_n = 2H_{n-1} + 1$ is not homogeneous.
	\item \textbf{Constant Coefficients}
		The coefficients do not depend on $n$.								\\
		For example, $B_n = nB_{n-1}$ does not have constant coefficients.
	\item \textbf{Degree: }
		The degree is the number of previous terms that determine $a_n$.	\\
		For example, the degree of $f_n$ is 2, and the degree of $a_n = a_{n-5}$ is 5.
\end{enumerate}

\subsection{Solving}
\subsubsection{The Method of Characteristic Equations}
	Generally, we look for solutions of the form $a_n = r^n$, where $r$ is some constant.

	Given a sequence
	\begin{gather*}
		a_n = c_1a_{n-1} + c_2a_{n-2} + \cdots +c_ka_{n-k},
	\end{gather*}

	$a_n = r^n$ is a solution if and only if
	\begin{gather*}\label{eq:rec_soln}
		r^n = c_1r^{n-1} + c_2r^{n-2} + \cdots + c_kr^{n-k}.
	\end{gather*}

	Dividing both sides of~\ref{eq:rec_soln} by $r^{n-k}$ yields
	\begin{gather*}
		r_k = c_1r^{k-1} + c_2r^{k-2} + \cdots + c_{k-1}r + c_k.
	\end{gather*}

	Rearranging terms shows that
	\begin{gather}
		r_k - c_1r^{k-1} - c_2r^{k-2} - \cdots - c_k = 0,
	\end{gather}

	which means that in order for $a_n = r^n$ to be a solution to the recurrence relation, $r$ must be a solution to the equation above. We refer to this as the \textbf{characteristic equation} of the recurrence relation, and the solutions to this equation are the \textbf{characteristic roots} of the recurrence relation. \\ \\

	Exercise: Let $c_1,c_2\in\mathbb{R}$. Suppose $r^2 - c_1r - c_2 = 0$ has two distinct roots, $r_1$ and $r_2$. Show for the recurrence relation
	\begin{gather*}
		a_n = c_1a_{n-1} + c_2a_{n-2},
	\end{gather*}

	a solution is given by
	\begin{gather*}
		a_n = \alpha r_1^n + \alpha_2 r_2^n, n\in\mathbb{N},
	\end{gather*}
	where $\alpha$ and $\alpha_2$ are constants.(Note: This is a actually a theorem!)

	\textit{Hints: }
		\begin{enumerate}
			\item
				Show that \textbf{if} $r_1$ and $r_2$ are roots to the characteristic equation and $\alpha$ and $\alpha_2$ are constants, \textbf{then} the sequence $\{a_n\}$ given by $a_n = \alpha r_1^n + \alpha_2 r_2^n$ is a solution to the recurrence relation.
			\item
				Show that \textbf{if} the sequence $\{a_n\}$ is a solution, \textbf{then} $a_n$ must equal $\alpha r_1^n + \alpha_2 r_2^n$ for some $\alpha$ and $\alpha_2$.
		\end{enumerate}

	\textit{Solution: }
	\begin{enumerate}
		\item
			Showing that the sequence $a_n = \alpha r_1^n + \alpha_2 r_2^n$ is a solution to the recurrence relation:

			Since it is assumed that $r_1$ and $r_2$ are roots of $r^2 - c_1r - c_2 = 0$,
			\begin{align*}
				r_1^2 &= c_1r_1 + c_2 \\
				r_2^2 &= c_1r_2 + c_2.
			\end{align*}

			Supposing that this is a solution for any term in $a_n$, this shows that
			\begin{align*}
				a_{n-1} = \alpha r_1^{n-1} + \alpha_2 r_2^{n-2} \\
				a_{n-2} = \alpha r_1^{n-1} + \alpha_2 r_2^{n-2}
			\end{align*}

			Substituting this into the original recurrence relation:
			\begin{align*}
				c_1a_{n-1} + c_2a_{n-2} &= c_1(\alpha r_1^{n-1} + \alpha_2 r_2^{n-2}) + c_2(\alpha r_1^{n-2} + \alpha_2 r_2^{n-2}) \\
				&= \alpha r_1^{n-2}(c_1r_1 + c_2) +\alpha_2 r_2^{n-2}(c_1r_2 + c_2) \\
				&= \alpha r_1^{n-2}(r_1^2) +\alpha_2 r_2^{n-2}(r_2^2) \\
				&= \alpha r_1^n + \alpha_2 r_2^n \\
				&= a_n
			\end{align*}

		\item
			Showing that every solution to $a_n$ is of the form $a_n = \alpha r_1^{n} +\alpha r_2^{n}$:

			Suppose $\{a_n\}$ is a solution, and the recurrence relation has the initial conditions $a_0 = C_0$ and $a_1 = C_1$. It remains to show that $\exists \alpha,\alpha_2$ that satisfy these same initial conditions.

			\begin{gather*}
				a_0 = C_0 \and a_0 = \alpha r_{1}^0 + \alpha_2 r_{2}^0 \\
				a_1 = C_1 \and \alpha r_1^{1} + \alpha_2 r_2^{1}
			\end{gather*}

			This gives the following system of linear equations:
			\begin{align*}
				C_0 &= \alpha + \alpha_2 \\
				C_1 &= \alpha r_1 + \alpha_2 r_2
			\end{align*}

			Which has unique solutions in terms of $C_1$ and $C_2$ for $r_1 \neq r_2$.
	\end{enumerate}

	So what's the punchline? This means that given a linear homogeneous recurrence relation with constant coefficients, the solution can be uniquely expressed as
	\begin{gather}
		a_n = \alpha r_1^n + \alpha_2 r_2^n,
	\end{gather}
	where $r_1$ and $r_2$ are the two distinct roots of the characteristic equation, and the initial conditions will determine the constant coefficients.

	Exercise: Use this result to derive an explicit expression for the Fibonacci sequence.

	What if there is only one root? We simply force orthogonality, and the solutions will take the form of
	\begin{gather}
		a_n = \alpha r^n + \alpha_2 nr^n
	\end{gather}

	In fact, this can be generalized -- as long as the characteristic equation has $k$ distinct roots, the solution will be of the form
	\begin{gather}
		a_n = \alpha_1 r_1^n + \alpha_2 r_2^n + \cdots + \alpha_3 r_k^n
	\end{gather}

	This can be generalized even further, in the case that the characteristic equation as multiple roots with multiplicity $m$. The general solution
	%TODO: Finish

\section{Generating Functions}
Generating functions usually occur in the form of power series -- however, we only regard these series formally. The sequence is extracted from the coefficients of the series, where the variable $x$ only plays the role of a placeholder. We consider these series to be a ring of algebraic objects, defined as \textit{formal power series}, for which the values of $x$ is merely a formal symbol. This means that it is nonsensical to evaluate such a series at any given point - the symbol $x$ is much like a clothesline, for which the power mark the positions of the ``clothespins'' on which the sequence is ``pinned''. This also means that we are not concerned with questions of convergence.

In order to find a generating function for which the $n$th coefficient is equal to the $n$th term in a sequence, we require that the recurrence relation is equipped with a clear range of values. ``Don't settle for a recurrence that has an unqualified free variable''.

General technique:
\begin{enumerate}
	\item
		Define the generating function you will look for.
		For example, for a sequence $a_o, a_1, a_2...$, a natural choice would be $A(x) = \sum_{j\geq0}a_jx^j$.
	\item
		Take the original recurrence relation, and multiply both sides by $x^n$
	\item
		Sum both sides \textit{over all the values of $n$ for which the relation is valid.}
	\item
		Try to express the result in terms of $A(x)$.
	\item
		Optional: To obtain an explicit formula for the $n$th term, attempt to expand $A(x)$. May require partial fraction decomposition, and definitely requires knowledge of series representations and equivalents.
\end{enumerate}



\chapter{Relations}
\section{Functions as Relations}
Definition: For two sets, say $A$ and $B$, a \textbf{binary relation from $A$ to $B$} is a subset of $A \times B$. In other words, it is a set $R$ of ordered pairs where the first element comes from $A$, and the second element comes from $B$. When an ordered pair $(a,b)$ is a member of $R$, we say $a$ is \textbf{related} to $b$ by $R$. Relations among elements of more than two sets are referred to as \textbf{$n$-ary } relations. \\ \\

Definition: If we have a relation from $A$ to itself, we called this a \textit{a relation on $A$}, which is a subset of $A \times A$. \\ \\

Definition: A relation $R$ on a set $A$ is called \textit{reflexive} if $(a,a) \in R ~\forall a \in A$. In other words, $\forall a((a,a)\in R)$. This means that a relation is reflexive if every element of $A$ is related to itself. \\ \\

Definition: A relation $R$ on a set $A$ is \textit{symmetric} if

$\forall a,b\in A, [(a,b)\in R \Rightarrow (b,a)\in R$]. \\ \\

Definition: A relation $R$ on a set $A$ is \textit{antisymmetric} if

$\forall a,b\in A, [((a,b)\in R \land (b,a)\in R) \Rightarrow a=b$].\\ \\

Definition: A relation $R$ on a set $A$ is \textit{transitive} if

$\forall a,b,c\in A, [(a,b)\in R \land (b,c)\in R \Rightarrow (a,c)\in R].$ \\ \\

Theorem: A relation $R$ on a set $A$ is transitive $\iff R^n\subseteq R ~\forall n\in \mathbb{N}.$

Definition: Let $R$ be a relation from sets $A$ to $B$, and $S$ be a relation from $B$ to $C$. The \textbf{composite} of the relations $R$ and $S$ is the relation, for which membership is defined as follows:
$\forall a\in A, \forall c\in C, \exists b\in B [(a,b)\in R \land (b,c)\in S]$. The notation $R^n$ for such a relation denotes the function composed with itself $n$ times. This relation is denoted $S\circ R$.
\\ \\

Definition: If a relation $R$ on a set $A$ is reflexive, symmetric, and transitive, it is said to be an \textbf{equivalence relation}. If $R$ is an equivalence relation on $A$, we denote the \textbf{equivalence class} of $R$ as $[a]_R = \left\{s | (a,s) \in R\right\}$ and we call any $b \in [a]_R$ a \textbf{representative} of the equivalence class.

Definition: A relation $R$ on a set $S$ is called a \textbf{partial ordering} or\textbf{partial order} if it is reflexive, antisymmetric, and transitive. A set $S$ together with a partial ordering is called a \textbf{poset} and is denoted by $(S,R)$. The symbol $\preccurlyeq$ is introduced to discuss how elements with posets are ordered. \\ \\

Definition: Elements $a$ and $b$ of a poset $(S, \preccurlyeq)$ are said to be \textbf{comparable} if either $a\preccurlyeq b$ or $b\succcurlyeq a$. When neither condition applies, $a$ and $b$ are said to be \textbf{incomparable}. \\ \\

Definition: A poset $(S, \preccurlyeq)$ is a \textbf{well-ordered set} if the operation $\preccurlyeq$ is a total ordering and every nonempty subset of $S$ has a ``least'' element.

\chapter{Misc}
Definition: \textbf{P vs. NP}

\textbf{P: The class of polynomial-time problems}

A decision problem in in P if it can be solved by a deterministic Turing machine in polynomial time in terms of the size of its input. That is, there exists a Turing machine $T$ that solves the decision, and a polynomial $p(n)$ such that $\forall n\in \mathbb{N}$, $T$ halts after no more than $p(n)$ transitions.

\textbf{NP: The class of nondeterministic polynomial-time problems}
A decision problem in in NP if it can be solved by a nondeterministic Turing machine in polynomial time in terms of its input. \\ \\

Definition: A \textit{partition of a set $S$} is a collection of nonempty, pairwise disjoint sets whose union is $S$. This is also referred to as an \textit{equivalence relation} on $S$. The sets into which $S$ is partitioned are called \textit{classes} of the partition. \\

\indent Related note: Given a set $[n] = \left\{1,2,\cdots , n\right\}$, the number of partitions of $[n]$ into $k$ classes is given by $\left\{\frac{n}{k}\right\}$, and is referred to as the Stirling number of the second kind.

Fact: The Cauchy product rule
\begin{gather}
	\sum_n a_nx^n \sum_n b_nx^n = \sum_n a_k b_{n-k}.
\end{gather}

\end{document}
