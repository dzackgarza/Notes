\RequirePackage[l2tabu, orthodox]{nag}

%\documentclass[]{article}
\documentclass[11pt]{scrartcl}
\usepackage[usename, dvipsnames]{xcolor}
\usepackage[pdfencoding=auto]{hyperref}
\usepackage[msc-links]{amsrefs}
\usepackage{cleveref} % use \cref{}, automatically deduces theorem, proposition, etc
\usepackage[mathletters]{ucs}
\usepackage[utf8]{inputenc}
\usepackage[T1]{fontenc}
\usepackage{datetime}

\usepackage{array}
\usepackage{mathtools}
\usepackage{amsmath, amsthm, amssymb, amsfonts, amsxtra, amscd, thmtools}
\let\proof\relax
\let\endproof\relax

% Boxes around theorem environments.
\usepackage[many]{tcolorbox}

\usepackage{color}
%\usepackage{unicode-math}
\usepackage{newunicodechar}
\newunicodechar{ε}{\varepsilon}
\newunicodechar{δ}{\delta}
\newunicodechar{µ}{\mu}
\newunicodechar{→}{\to}
\newunicodechar{≤}{\leq}
\newunicodechar{∈}{\in}
\newunicodechar{⊆}{\subseteq}
\newunicodechar{Λ}{\Lambda}
\newunicodechar{∞}{\infty}
\newunicodechar{×}{\times}
\everymath{\displaystyle}



\usepackage{microtype}
\usepackage[pdfencoding=auto]{hyperref}
\usepackage{bookmark}
\usepackage{booktabs}
\usepackage{todonotes}
\usepackage[msc-links]{amsrefs}
\usepackage{cleveref} % use \cref{}, automatically deduces theorem, proposition, etc
\usepackage{csquotes}
\usepackage{longtable}
\usepackage{tabularx}
\usepackage{bbm}
% Creating multiple types of index
\usepackage{imakeidx}

% Remove indentation for new paragraphs
\usepackage{parskip}
% But leave space before amsthm environments
\makeatletter
\def\thm@space@setup{%
  \thm@preskip=2em
  \thm@postskip=2em
}
\makeatother


\usepackage{stmaryrd}
\usepackage{adjustbox}
\usepackage{centernot}
% \centernot\whatever


% Better indicator function
\usepackage{bbm}
\newcommand{\indic}[1]{\mathbbm{1} \left[ {#1} \right] }

% Highlight quote
\usepackage{environ}
\definecolor{camel}{rgb}{0.76, 0.6, 0.42}
\definecolor{babyblue}{rgb}{0.54, 0.81, 0.94}
\definecolor{block-gray}{gray}{0.85}
\NewEnviron{myblock}
{\colorbox{block-gray}{%
\parbox{\dimexpr\linewidth-2\fboxsep\relax}{%
\small\addtolength{\leftskip}{10mm}
\addtolength{\rightskip}{10mm}
\BODY}}
}
\renewcommand{\quote}{\myblock}
\renewcommand{\endquote}{\endmyblock}

% Nice math font that journals use
%\usepackage[lite]{mtpro2}
%\usepackage{mathrsfs}
%\usepackage{mathptmx}
\usepackage{lmodern}
%\usepackage[sc]{mathpazo}

% Theorem Styles
\usepackage[framemethod=tikz]{mdframed}

\theoremstyle{definition}
\newtheorem{exercise}{Exercise}[section]
\newtheorem{solution}{Solution}

% Theorem Style
\newtheoremstyle{theorem}% name
  {0em}%         Space above, empty = `usual value'
  {1em}%         Space below
  {\normalfont}% Body font
  {\parindent}%         Indent amount (empty = no indent, \parindent = para indent)
  {\bfseries}% Thm head font
  {.}%        Punctuation after thm head
  {\newline}% Space after thm head: \newline = linebreak
  {\thmname{#1}\thmnumber{ #2}\thmnote{\itshape{(#3)}}}%
\theoremstyle{theorem}
\tcolorboxenvironment{theorem}{
  boxrule=0pt,
  boxsep=0pt,
  breakable,
  enhanced jigsaw,
  fonttitle={\large\bfseries},
  opacityback=0.8,
  colframe=cyan,
  borderline west={4pt}{0pt}{orange},
  attach title to upper={}
}
\newtheorem{theorem}{Theorem}[section]

% Proposition Style
\tcolorboxenvironment{proposition}{
  boxrule=1pt,
  boxsep=0pt,
  breakable,
  enhanced jigsaw,
  opacityback=0.0,
  colframe=cyan
}
\newtheorem{proposition}[theorem]{Proposition}
\tcolorboxenvironment{lemma}{
  boxrule=1pt,
  boxsep=0pt,
  breakable,
  enhanced jigsaw,
  opacityback=0.2,
  colframe=cyan
}
\newtheorem{lemma}[theorem]{Lemma}
% Claim
\tcolorboxenvironment{claim}{
  boxrule=1pt,
  boxsep=0pt,
  breakable,
  enhanced jigsaw,
  opacityback=0.2,
  colframe=cyan
}
\newtheorem{claim}[theorem]{Claim}


% Corollary
\tcolorboxenvironment{corollary}{
  colback=cyan,
  boxrule=1pt,
  boxsep=0pt,
  breakable,
  enhanced jigsaw,
  opacityback=0.1,
  colframe=cyan
}
\newtheorem{corollary}[theorem]{Corollary}

% Proof Style
\newtheoremstyle{proof}% name
  {0em}%         Space above, empty = `usual value'
  {2em}%         Space below
  {\normalfont}% Body font
  {\parindent}%         Indent amount (empty = no indent, \parindent = para indent)
  {\itshape}% Thm head font
  {.}%        Punctuation after thm head
  {\newline}% Space after thm head: \newline = linebreak
  {\thmname{#1} \thmnote{\itshape{(#3)}}}%         Thm head spec
\theoremstyle{proof}
\tcolorboxenvironment{proof}{
  colback=camel,
  opacityfill=0.25,
  boxrule=1pt,
  boxsep=0pt,
  breakable,
  enhanced jigsaw
}
\newtheorem*{pf}{Proof}
\newenvironment{proof}
{\pushQED{$\qed$}\pf}
{\par\popQED\endpf}

% Definition Style
\newtheoremstyle{definition}% name
  {0em}%         Space above, empty = `usual value'
  {2em}%         Space below
  {\normalfont}% Body font
  {\parindent}%         Indent amount (empty = no indent, \parindent = para indent)
  {\bfseries}% Thm head font
  {.}%        Punctuation after thm head
  {\newline}% Space after thm head: \newline = linebreak
  {}%         Thm head spec
\theoremstyle{definition}
\tcolorboxenvironment{definition}{
  colback=babyblue,
  boxrule=0pt,
  boxsep=0pt,
  opacityfill=0.45,
  breakable,
  enhanced jigsaw,
  borderline west={4pt}{0pt}{blue},
  colbacktitle={babyblue},
  coltitle={black},
  fonttitle={\large\bfseries},
  attach title to upper={},
}
\newtheorem{definition}{Definition}[theorem]

% Break Environment
\makeatletter
\newtheoremstyle{break}% name
  {}%         Space above, empty = `usual value'
  {2em}%         Space below
  {
    \addtolength{\@totalleftmargin}{2.5em}
    \addtolength{\linewidth}{-2.5em}
    \parshape 1 2.5em \linewidth
  }% Body font
  {}%         Indent amount (empty = no indent, \parindent = para indent)
  {\bfseries}% Thm head font
  {.}%        Punctuation after thm head
  {\newline}% Space after thm head: \newline = linebreak
  {}%         Thm head spec
\makeatother

\theoremstyle{break}
\newtheorem{example}{Example}[section]

% Problem Style
\newtheoremstyle{problem} % name
  {0em}                   % Space above, empty = `usual value'
  {2em}                   % Space below
  {\normalfont}           % Body font
  {\parindent}            % Indent amount (empty = no indent, \parindent = para indent)
  {\itshape}              % Thm head font
  {}                      % Punctuation after thm head
  {\newline}              % Space after thm head: \newline = linebreak
  {\thmnote{\itshape{(#3)}}}     % Thm head spec
\theoremstyle{problem}
\tcolorboxenvironment{problem}{
  boxrule=1pt,
  boxsep=0pt,
  breakable,
  enhanced jigsaw,
  opacityback=0.0,
  colframe=cyan
}
\newtheorem{problem}{Problem}


%Pagination stuff.
\setlength{\topmargin}{-.3 in}
\setlength{\oddsidemargin}{0in}
\setlength{\evensidemargin}{0in}
\setlength{\textheight}{9.in}
\setlength{\textwidth}{6.5in}
% \pagestyle{empty} %removes page numbers.

% Inkscape figures from Vim
\usepackage{import}
\usepackage{pdfpages}
\usepackage{transparent}

\newcommand{\incfig}[1]{%
    \def\svgwidth{\columnwidth}
    \import{./figures/}{#1.pdf_tex}
}
%\pdfsuppresswarningpagegroup=1

% Pandoc-specific fixes
\providecommand{\tightlist}{%
  \setlength{\itemsep}{0pt}\setlength{\parskip}{0pt}}

% Tikz and Graphics
\usepackage{amscd}
\usepackage{tikz}
\usetikzlibrary{arrows, arrows.meta, cd, fadings, patterns, calc, decorations.markings, matrix, positioning}
\tikzfading[name=fade out, inner color=transparent!0, outer color=transparent!100]
\usepackage{pgfplots}
\pgfplotsset{compat=1.16}
\usepackage[inline]{asymptote}
\usepackage{tikz-layers}

%\usepackage{nath}
%\delimgrowth=1
\DeclarePairedDelimiter\qty{(}{)}

% Major Macros
\usepackage{graphicx}
\usepackage{float}
\DeclareFontFamily{U}{mathx}{\hyphenchar\font45}
\DeclareFontShape{U}{mathx}{m}{n}{
      <5> <6> <7> <8> <9> <10>
      <10.95> <12> <14.4> <17.28> <20.74> <24.88>
      mathx10
      }{}
\DeclareSymbolFont{mathx}{U}{mathx}{m}{n}
\DeclareMathSymbol{\bigtimes}{1}{mathx}{"91}

% Wide tikz equations
\newsavebox{\wideeqbox}
\newenvironment{wideeq}
  {\begin{displaymath}\begin{lrbox}{\wideeqbox}$\displaystyle}
  {$\end{lrbox}\makebox[0pt]{\usebox{\wideeqbox}}\end{displaymath}}



% Fancy chapter headers and footers
\usepackage{fancyhdr}

\pagestyle{fancy}
\fancyhf{}
\fancyhead[LE,RO]{\title}
\fancyhead[RE,LO]{\rightmark}
\fancyfoot[CE,CO]{\leftmark}
\fancyfoot[LE,RO]{\thepage}

\renewcommand{\headrulewidth}{2pt}
\renewcommand{\footrulewidth}{1pt}

% List of Theorems Attempt
\usepackage{etoolbox}
\makeatletter
\patchcmd\thmtlo@chaptervspacehack
  {\addtocontents{loe}{\protect\addvspace{10\p@}}}
  {\addtocontents{loe}{\protect\thmlopatch@endchapter\protect\thmlopatch@chapter{\thechapter}}}
  {}{}
\AtEndDocument{\addtocontents{loe}{\protect\thmlopatch@endchapter}}
\long\def\thmlopatch@chapter#1#2\thmlopatch@endchapter{%
  \setbox\z@=\vbox{#2}%
  \ifdim\ht\z@>\z@
    \hbox{\bfseries\chaptername\ #1}\nobreak
    #2
    \addvspace{10\p@}
  \fi
}
\def\thmlopatch@endchapter{}

\makeatother
\renewcommand{\thmtformatoptarg}[1]{ -- #1}
%\renewcommand{\listtheoremname}{List of definitions}

\newcommand{\ext}{\operatorname{Ext}}
\newcommand{\Ext}{\operatorname{Ext}}
\def\Endo{\operatorname{End}}
\def\Ind{\operatorname{Ind}}
\def\ind{\operatorname{Ind}}
\def\coind{\operatorname{Coind}}
\def\Res{\operatorname{Res}}
\def\Hol{\operatorname{Hol}}
\def\res{\operatorname{Res}}
\def\endo{\operatorname{End}}
\def\ind{\operatorname{Ind}}
\renewcommand{\AA}[0]{{\mathbb{A}}}
\DeclareMathOperator{\Exists}{\exists}
\DeclareMathOperator{\Forall}{\forall}
\newcommand{\Af}[0]{{\mathbb{A}}}
\newcommand{\CC}[0]{{\mathbb{C}}}
\newcommand{\CP}[0]{{\mathbb{CP}}}
\newcommand{\DD}[0]{{\mathbb{D}}}
\newcommand{\FF}[0]{{\mathbb{F}}}
\newcommand{\GF}[0]{{\mathbb{GF}}}
\newcommand{\GG}[0]{{\mathbb{G}}}
\newcommand{\HH}[0]{{\mathbb{H}}}
\newcommand{\HP}[0]{{\mathbb{HP}}}
\newcommand{\KK}[0]{{\mathbb{K}}}
\newcommand{\kk}[0]{{\Bbbk}}
\newcommand{\bbm}[0]{{\mathbb{M}}}
\newcommand{\NN}[0]{{\mathbb{N}}}
\newcommand{\OP}[0]{{\mathbb{OP}}}
\newcommand{\PP}[0]{{\mathbb{P}}}
\newcommand{\QQ}[0]{{\mathbb{Q}}}
\newcommand{\RP}[0]{{\mathbb{RP}}}
\newcommand{\RR}[0]{{\mathbb{R}}}
\newcommand{\SpSp}[0]{{\mathbb{S}}}
\renewcommand{\SS}[0]{{\mathbb{S}}}
\newcommand{\TT}[0]{{\mathbb{T}}}
\newcommand{\ZZ}[0]{{\mathbb{Z}}}
\newcommand{\ZnZ}[0]{\mathbb{Z}/n\mathbb{Z}}
\newcommand{\ZpZ}[0]{\mathbb{Z}/p\mathbb{Z}}
\newcommand{\Qp}[0]{\mathbb{Q}_{(p)}}
\newcommand{\Zp}[0]{\mathbb{Z}_{(p)}}
\newcommand{\Arg}[0]{\mathrm{Arg}}
\newcommand{\PGL}[0]{\mathrm{PGL}}
\newcommand{\GL}[0]{\mathrm{GL}}
\newcommand{\Gl}[0]{\mathrm{GL}}
\newcommand{\gl}[0]{\mathrm{GL}}
\newcommand{\mat}[0]{\mathrm{Mat}}
\newcommand{\Mat}[0]{\mathrm{Mat}}
\newcommand{\Rat}[0]{\mathrm{Rat}}
\newcommand{\Perv}[0]{\mathrm{Perv}}
\newcommand{\Gal}[0]{\mathrm{Gal}}
\newcommand{\Hilb}[0]{\mathrm{Hilb}}
\newcommand{\Quot}[0]{\mathrm{Quot}}
\newcommand{\Art}[0]{\mathrm{Art}}
\newcommand{\red}[0]{\mathrm{red}}
\newcommand{\alg}[0]{\mathrm{alg}}
\newcommand{\Pic}[0]{{\mathrm{Pic}~}}
\newcommand{\lcm}[0]{\mathrm{lcm}}
\newcommand{\maps}[0]{\mathrm{Maps}}
\newcommand{\maxspec}[0]{{\mathrm{maxSpec}~}}
\newcommand{\Tr}[0]{\mathrm{Tr}}
\newcommand{\adj}[0]{\mathrm{adj}}
\newcommand{\ad}[0]{\mathrm{ad}~}
\newcommand{\ann}[0]{\mathrm{Ann}}
\newcommand{\Ann}[0]{\mathrm{Ann}}
\newcommand{\arcsec}[0]{\mathrm{arcsec}}
\newcommand{\ch}[0]{\mathrm{char}~}
\newcommand{\Sp}[0]{{\mathrm{Sp}}}
\newcommand{\syl}[0]{{\mathrm{Syl}}}
\newcommand{\txand}[0]{{\text{ and }}}
\newcommand{\codim}[0]{\mathrm{codim}}
\newcommand{\txor}[0]{{\text{ or }}}
\newcommand{\txt}[1]{{\text{ {#1} }}}
\newcommand{\Gr}[0]{{\text{Gr}}}
\newcommand{\Aut}[0]{{\mathrm{Aut}}}
\newcommand{\aut}[0]{\mathrm{Aut}}
\newcommand{\Inn}[0]{{\mathrm{Inn}}}
\newcommand{\Out}[0]{{\mathrm{Out}}}
\newcommand{\mltext}[1]{\left\{\begin{array}{c}#1\end{array}\right\}}
\newcommand{\Fun}[0]{{\text{Fun}}}
\newcommand{\SL}[0]{{\text{SL}}}
\newcommand{\PSL}[0]{{\text{PSL}}}
\newcommand{\SO}[0]{{\text{SO}}}
\newcommand{\SU}[0]{{\text{SU}}}
\newcommand{\SP}[0]{{\text{SP}}}
\newcommand{\per}[0]{{\text{Per}}}
\newcommand{\loc}[0]{{\text{loc}}}
\newcommand{\Top}[0]{{\text{Top}}}
\newcommand{\Sch}[0]{{\text{Sch}}}
\newcommand{\sch}[0]{{\text{Sch}}}
\newcommand{\Set}[0]{{\text{Set}}}
\newcommand{\Sets}[0]{{\text{Set}}}
\newcommand{\Grp}[0]{{\text{Grp}}}
\newcommand{\Groups}[0]{{\text{Groups}}}
\newcommand{\Homeo}[0]{{\text{Homeo}}}
\newcommand{\Diffeo}[0]{{\text{Diffeo}}}
\newcommand{\MCG}[0]{{\text{MCG}}}
\newcommand{\set}[0]{{\text{Set}}}
\newcommand{\Tor}[0]{\text{Tor}}
\newcommand{\sets}[0]{{\text{Set}}}
\newcommand{\Sm}[0]{{\text{Sm}_k}}
\newcommand{\orr}[0]{{\text{ or }}}
\newcommand{\annd}[0]{{\text{ and }}}
\newcommand{\bung}[0]{\text{Bun}_G}
\newcommand{\const}[0]{{\text{const.}}}
\newcommand{\disc}[0]{{\text{disc}}}
\newcommand{\op}[0]{^\text{op}}
\newcommand{\id}[0]{\text{id}}
\newcommand{\im}[1]{\mathrm{im}({#1})}
\newcommand{\pt}[0]{{\{\text{pt}\}}}
\newcommand{\sep}[0]{^\text{sep}}
% \newcommand{\st}[0]{~{\text{s.t.}}~}
\newcommand{\tors}[0]{{\text{tors}}}
\newcommand{\tor}[0]{\text{Tor}}
\newcommand{\height}[0]{\text{ht}}
\newcommand{\cpt}[0]{\text{compact}}
\newcommand{\abs}[1]{{\left\lvert {#1} \right\rvert}}
\newcommand{\stack}[1]{\mathclap{\substack{ #1 }}} 
\newcommand{\qtext}[1]{{\quad \text{#1} \quad}}
\newcommand{\qst}[0]{{\quad \text{such that} \quad}}
\newcommand{\actsonl}[0]{\curvearrowleft}
\newcommand{\actson}[0]{\curvearrowright}
\newcommand{\bd}[0]{{\del}}
\newcommand{\bigast}[0]{{\mathop{\Large \ast}}}
\newcommand{\coker}[0]{\operatorname{coker}}
\newcommand{\cok}[0]{\operatorname{coker}}
\newcommand{\conjugate}[1]{{\overline{{#1}}}}
\newcommand{\converges}[1]{\overset{#1}}
\newcommand{\correspond}[1]{\theset{\substack{#1}}}
\newcommand{\cross}[0]{\times}
\newcommand{\by}[0]{\times}
\newcommand{\dash}[0]{{\hbox{-}}}
\newcommand{\dd}[2]{{\frac{\partial #1}{\partial #2}\,}}
\newcommand{\definedas}[0]{\coloneqq}
\newcommand{\da}[0]{\coloneqq}
\newcommand{\del}[0]{{\partial}}
\newcommand{\directlim}[0]{\varinjlim}
\newcommand{\disjoint}[0]{{\coprod}}
\newcommand{\divides}[0]{{~\Bigm|~}}
\newcommand{\dual}[0]{^\vee}
\newcommand{\sm}[0]{\setminus}
\newcommand{\smz}[0]{\setminus\theset{0}}
\newcommand{\eps}[0]{\varepsilon}
\newcommand{\equalsbecause}[1] {\stackrel{\mathclap{\scriptscriptstyle{#1}}}{=}}
\newcommand{\floor}[1]{{\left\lfloor #1 \right\rfloor}}
\DeclarePairedDelimiter{\ceil}{\lceil}{\rceil}
\newcommand{\from}[0]{\leftarrow}
\newcommand{\tofrom}[0]{\leftrightarrows}
\newcommand{\up}[0]{\uparrow}
\newcommand{\generators}[1]{\left\langle{#1}\right\rangle}
\newcommand{\gs}[1]{\left\langle{#1}\right\rangle}
\newcommand{\homotopic}[0]{\simeq}
\newcommand{\injectivelim}[0]{\varinjlim}
\newcommand{\injects}[0]{\hookrightarrow}
\newcommand{\inner}[2]{{\left\langle {#1},~{#2} \right\rangle}}
\newcommand{\union}[0]{\cup}
\newcommand{\Union}[0]{\bigcup}
\newcommand{\intersect}[0]{\cap}
\newcommand{\Intersect}[0]{\bigcap}
\newcommand{\into}[0]{\to}
\newcommand{\inverselim}[0]{\varprojlim}
\newcommand{\inv}[0]{^{-1}}
\newcommand{\mfa}[0]{{\mathfrak{a}}}
\newcommand{\mfb}[0]{{\mathfrak{b}}}
\newcommand{\mfc}[0]{{\mathfrak{c}}}
\newcommand{\mff}[0]{{\mathfrak{f}}}
\newcommand{\mfi}[0]{{\mathfrak{I}}}
\newcommand{\mfm}[0]{{\mathfrak{m}}}
\newcommand{\mfn}[0]{{\mathfrak{n}}}
\newcommand{\mfp}[0]{{\mathfrak{p}}}
\newcommand{\mfq}[0]{{\mathfrak{q}}}
\newcommand{\mfr}[0]{{\mathfrak{r}}}
\newcommand{\lieb}[0]{{\mathfrak{b}}}
\newcommand{\liegl}[0]{{\mathfrak{gl}}}
\newcommand{\lieg}[0]{{\mathfrak{g}}}
\newcommand{\lieh}[0]{{\mathfrak{h}}}
\newcommand{\lien}[0]{{\mathfrak{n}}}
\newcommand{\liesl}[0]{{\mathfrak{sl}}}
\newcommand{\lieso}[0]{{\mathfrak{so}}}
\newcommand{\liesp}[0]{{\mathfrak{sp}}}
\newcommand{\lieu}[0]{{\mathfrak{u}}}
\newcommand{\nilrad}[0]{{\mathfrak{N}}}
\newcommand{\jacobsonrad}[0]{{\mathfrak{J}}}
\newcommand{\mm}[0]{{\mathfrak{m}}}
\newcommand{\pr}[0]{{\mathfrak{p}}}
\newcommand{\mapsvia}[1]{\xrightarrow{#1}}
\newcommand{\kx}[1]{k[x_1, \cdots, x_{#1}]}
\newcommand{\MM}[0]{{\mathcal{M}}}
\newcommand{\OO}[0]{{\mathcal{O}}}
\newcommand{\imaginarypart}[1]{{\mathcal{Im}({#1})}}
\newcommand{\mca}[0]{{\mathcal{A}}}
\newcommand{\mcb}[0]{{\mathcal{B}}}
\newcommand{\mcc}[0]{{\mathcal{C}}}
\newcommand{\mcd}[0]{{\mathcal{D}}}
\newcommand{\mce}[0]{{\mathcal{E}}}
\newcommand{\mcf}[0]{{\mathcal{F}}}
\newcommand{\mcg}[0]{{\mathcal{G}}}
\newcommand{\mch}[0]{{\mathcal{H}}}
\newcommand{\mci}[0]{{\mathcal{I}}}
\newcommand{\mcj}[0]{{\mathcal{J}}}
\newcommand{\mck}[0]{{\mathcal{K}}}
\newcommand{\mcl}[0]{{\mathcal{L}}}
\newcommand{\mcm}[0]{{\mathcal{M}}}
\newcommand{\mcp}[0]{{\mathcal{P}}}
\newcommand{\mcs}[0]{{\mathcal{S}}}
\newcommand{\mct}[0]{{\mathcal{T}}}
\newcommand{\mcu}[0]{{\mathcal{U}}}
\newcommand{\mcv}[0]{{\mathcal{V}}}
\newcommand{\mcx}[0]{{\mathcal{X}}}
\newcommand{\mcz}[0]{{\mathcal{Z}}}
\newcommand{\cl}[0]{\mathrm{cl}}
\newcommand{\trdeg}[0]{\mathrm{trdeg}}
\newcommand{\dist}[0]{\mathrm{dist}}
\newcommand{\Dist}[0]{\mathrm{Dist}}
\newcommand{\crit}[0]{\mathrm{crit}}
\newcommand{\diam}[0]{{\mathrm{diam}}}
\newcommand{\gal}[0]{\mathrm{Gal}}
\newcommand{\diff}[0]{\mathrm{Diff}}
\newcommand{\diag}[0]{\mathrm{diag}}
\newcommand{\soc}[0]{\mathrm{Soc}\,}
\newcommand{\hd}[0]{\mathrm{Head}\,}
\newcommand{\grad}[0]{\mathrm{grad}~}
\newcommand{\hilb}[0]{\mathrm{Hilb}}
\newcommand{\minpoly}[0]{{\mathrm{minpoly}}}
\newcommand{\Hom}[0]{{\mathrm{Hom}}}
\newcommand{\Map}[0]{{\mathrm{Map}}}
\newcommand{\multinomial}[1]{\left(\!\!{#1}\!\!\right)}
\newcommand{\nil}[0]{{\mathrm{nil}}}
\newcommand{\normalneq}{\mathrel{\reflectbox{$\trianglerightneq$}}}
\newcommand{\normal}[0]{{~\trianglelefteq~}}
\newcommand{\norm}[1]{{\left\lVert {#1} \right\rVert}}
\newcommand{\pnorm}[2]{{\left\lVert {#1} \right\rVert}_{#2}}
\newcommand{\notdivides}[0]{\nmid}
\newcommand{\onto}[0]{\twoheadhthtarrow}
\newcommand{\ord}[0]{{\mathrm{Ord}}}
\newcommand{\pic}[0]{{\mathrm{Pic}~}}
\newcommand{\projectivelim}[0]{\varprojlim}
\newcommand{\rad}[0]{{\mathrm{rad}~}}
\newcommand{\ralg}[0]{\mathrm{R-alg}}
\newcommand{\kalg}[0]{k\dash\mathrm{alg}}
\newcommand{\rank}[0]{\operatorname{rank}}
\newcommand{\realpart}[1]{{\mathcal{Re}({#1})}}
\newcommand{\Log}[0]{\mathrm{Log}}
\newcommand{\reg}[0]{\mathrm{Reg}}
\newcommand{\restrictionof}[2]{{\left.{#1}\right|_{#2}}}
\newcommand{\ro}[2]{{\left.{#1}\right|_{#2}}}
\newcommand{\rk}[0]{{\mathrm{rank}}}
\newcommand{\evalfrom}[0]{\Big|}
\newcommand{\rmod}[0]{{R\dash\mathrm{mod}}}
\newcommand{\Mod}[0]{{\mathrm{Mod}}}
\newcommand{\rotate}[2]{{\style{display: inline-block; transform: rotate(#1deg)}{#2}}}
\newcommand{\selfmap}[0]{{\circlearrowleft}}
\newcommand{\semidirect}[0]{\rtimes}
\newcommand{\sgn}[0]{\mathrm{sgn}}
\newcommand{\sign}[0]{\mathrm{sign}}
\newcommand{\spanof}[0]{{\mathrm{span}}}
\newcommand{\spec}[0]{\mathrm{Spec}\,}
\newcommand{\mspec}[0]{\mathrm{mSpec}~}
\newcommand{\stab}[0]{{\mathrm{Stab}}}
\newcommand{\stirlingfirst}[2]{\genfrac{[}{]}{0pt}{}{#1}{#2}}
\newcommand{\stirling}[2]{\genfrac\{\}{0pt}{}{#1}{#2}}
\newcommand{\strike}[1]{{\enclose{horizontalstrike}{#1}}}
\newcommand{\suchthat}[0]{{~\mathrel{\Big|}~}}
\newcommand{\st}[0]{{~\mathrel{\Big|}~}}
\newcommand{\supp}[0]{{\mathrm{supp}}}
\newcommand{\surjects}[0]{\twoheadrightarrow}
\newcommand{\sym}[0]{\mathrm{Sym}}
\newcommand{\tensor}[0]{\otimes}
\newcommand{\connectsum}[0]{\mathop{\Large \#}}
\newcommand{\theset}[1]{\left\{{#1}\right\}}
\newcommand{\ts}[1]{\left\{{#1}\right\}}
\newcommand{\gens}[1]{\left\langle{#1}\right\rangle}
\newcommand{\thevector}[1]{{\left[ {#1} \right]}}
\newcommand{\tv}[1]{{\left[ {#1} \right]}}
\newcommand{\too}[1]{{\xrightarrow{#1}}}
\newcommand{\transverse}[0]{\pitchfork}
\newcommand{\trianglerightneq}{\mathrel{\ooalign{\raisebox{-0.5ex}{\reflectbox{\rotatebox{90}{$\nshortmid$}}}\cr$\triangleright$\cr}\mkern-3mu}}
\newcommand{\tr}[0]{\mathrm{Tr}}
\newcommand{\uniformlyconverges}[0]{\rightrightarrows}
\newcommand{\covers}[0]{\rightrightarrows}
\newcommand{\units}[0]{^{\times}}
\newcommand{\nonzero}[0]{^{\bullet}}
\newcommand{\wait}[0]{{\,\cdot\,}}
\newcommand{\wt}[0]{{\mathrm{wt}}}
\renewcommand{\bar}[1]{\mkern 1.5mu\overline{\mkern-1.5mu#1\mkern-1.5mu}\mkern 1.5mu}
\renewcommand{\div}[0]{\mathrm{Div}}
\newcommand{\Div}[0]{\mathrm{Div}}
\renewcommand{\hat}[1]{\widehat{#1}}
\renewcommand{\mid}[0]{\mathrel{\Big|}}
\renewcommand{\qed}[0]{\hfill\blacksquare}
\renewcommand{\too}[0]{\longrightarrow}
\renewcommand{\vector}[1]{\mathbf{#1}}
\let\oldexp\exp
\renewcommand{\exp}[1]{\oldexp\qty{#1}}
\let\oldperp\perp
\renewcommand{\perp}[0]{^\oldperp}
\newcommand*\dif{\mathop{}\!\mathrm{d}}
\newcommand{\ddt}{\tfrac{\dif}{\dif t}}
\newcommand{\ddx}{\tfrac{\dif}{\dif x}}

\DeclareMathOperator{\righttriplearrows} {{\; \tikz{ \foreach \y in {0, 0.1, 0.2} { \draw [-stealth] (0, \y) -- +(0.5, 0);}} \; }}




\let\Begin\begin
\let\End\end
\newcommand\wrapenv[1]{#1}

\makeatletter
\def\ScaleWidthIfNeeded{%
 \ifdim\Gin@nat@width>\linewidth
    \linewidth
  \else
    \Gin@nat@width
  \fi
}
\def\ScaleHeightIfNeeded{%
  \ifdim\Gin@nat@height>0.9\textheight
    0.9\textheight
  \else
    \Gin@nat@width
  \fi
}
\makeatother

\setkeys{Gin}{width=\ScaleWidthIfNeeded,height=\ScaleHeightIfNeeded,keepaspectratio}%

\title{
\textbf{
  Algebraic Geometry Problems 
  }
  }







\begin{document}

\date{}
\author{D. Zack Garza}
\maketitle

\tableofcontents
\newpage


\begin{quote}
Source:
\href{https://www.mathematik.uni-kl.de/~gathmann/class/alggeom-2019/alggeom-2019-c1.pdf}{Section
1 of Gathmann}
\end{quote}

\hypertarget{problem-set-1}{%
\section{Problem Set 1}\label{problem-set-1}}

\begin{exercise}[Gathmann 1.19]

Prove that every affine variety \(X\subset \AA^n/k\) consisting of only
finitely many points can be written as the zero locus of \(n\)
polynomials.

\begin{quote}
Hint: Use interpolation. It is useful to assume at first that all points
in \(X\) have different \(x_1\dash\)coordinates.
\end{quote}

\end{exercise}

\begin{solution}

Let
\(X = \ts{\vector p_1, \cdots, \vector p_d} =\ts{\vector p_j}_{j=1}^d\),
where each \(\vector p_j\in \AA^n\) can be written in coordinates
\begin{align*}\vector p_j \da \thevector{p_j^1, p_j^2, \cdots, p_j^n}.\end{align*}

\begin{remark}

Proof idea: for some fixed \(k\) with \(2\leq k \leq n\), consider the
pairs \((p_j^1, p_j^k) \in \AA^2\). Letting \(j\) range over
\(1\leq j \leq d\) yields \(d\) points of the form \((x, y) \in \AA^2\),
so construct an interpolating polynomial such that \(f(x) = y\) for each
tuple. Then \(f(x) - y\) vanishes at every such tuple.

\hfill\break

Doing this for each \(k\) (keeping the first coordinate always of the
form \(p_j^1\) and letting the second coordinate vary) yields \(n-1\)
polynomials in \(k[x_1, x_k] \subseteq \kx{n}\), then adding in the
polynomial \(p(x) = \prod_j (x-p_j^1)\) yields a system the vanishes
precisely on \(\ts{\vector p_j}\).

\end{remark}

\begin{claim}

Without loss of generality, we can assume all of the first components
\(\ts{p_j^1}_{j=1}^d\) are distinct.

\end{claim}

\todo[inline]{Todo: follows from "rotation of axes"?}

We will use the following fact:

\begin{theorem}[Lagrange]

Given a set of \(d\) points \(\ts{(x_i, y_i)}_{i=1}^d\) with all \(x_i\)
distinct, there exists a unique polynomial of degree \(d\) in
\(f \in k[x]\) such that \(\tilde f(x_i) = y_i\) for every \(i\).

This can be explicitly given by
\begin{align*}  
\tilde f(x) = \sum_{i=1}^d y_i \qty{\prod_{\substack{0\leq m \leq d \\ m\neq i}} \qty{x - x_m \over x_i - x_m }}
.\end{align*}

Equivalently, there is a polynomial \(f\) defined by
\(f(x_i) = \tilde f(x_i) - y_i\) of degree \(d\) whose roots are
precisely the \(x_i\).

\end{theorem}

\vspace{2em}

Using this theorem, we define a system of \(n\) polynomials in the
following way:

\begin{itemize}
\item
  Define \(f_1 \in k[x_1] \subseteq k[x_1, \cdots, x_n]\) by
  \begin{align*}f_1(x) = \prod_{i=1}^d \qty{x - p_i^1}.\end{align*} Then
  the roots of \(f_1\) are precisely the first components of the points
  \(p\).

  \hfill\break
\item
  Define \(f_2 \in k[x_1, x_2] \subseteq k[x_1, \cdots, x_n]\) by
  considering the ordered pairs
  \begin{align*}\ts{(x_1, x_2) = (p_j^1, p_j^2)},\end{align*} then
  taking the unique Lagrange interpolating polynomial \(\tilde f_2\)
  satisfying \(\tilde f_2(p_j^1) = p_j^2\) for all \(1\leq j \leq d\).
  Then set \(f_2 \da \tilde f_2(x_1) - x_2 \in k[x_1, x_2]\).
\end{itemize}

\hfill\break

\begin{itemize}
\item
  Define \(f_3 \in k[x_1, x_3] \subseteq k[x_1, \cdots, x_n]\) by
  considering the ordered pairs
  \begin{align*}\ts{(x_1, x_3) = (p_j^1, p_j^3)},\end{align*} then
  taking the unique Lagrange interpolating polynomial \(\tilde f_3\)
  satisfying \(\tilde f_2(p_j^1) = p_j^3\) for all \(1\leq j \leq d\).
  Then set \(f_3 \da \tilde f_3(x_1) - x_3 \in k[x_1, x_3]\).
\item
  \(\cdots\)
\end{itemize}

\vspace{2em}

Continuing in this way up to \(f_n \in k[x_1, x_n]\) yields a system of
\(n\) polynomials.

\hfill\break

\begin{proposition}

\(V(f_1, \cdots, f_n) = X\).

\end{proposition}

\begin{proof}

\begin{claim}

\(X\subseteq V(f_i)\):

\end{claim}

This is essentially by construction. Letting \(p_j\in X\) be arbitrary,
we find that
\begin{align*}  
f_1(p_j) 
= \prod_{i=1}^d \qty{p_j^1 - p_i^1}
= (p_j^1 - p_j^1) \prod_{\substack{i\leq d \\ i\neq j}} \qty{p_j^1 - p_i^1} = 0
.\end{align*}

Similarly, for \(2\leq k \leq n\),
\begin{align*}  
f_k(p_j) = \tilde f_k(p_j^1) - p_j^k = 0 
,\end{align*} which follows from the fact that
\(\tilde f_k(p_j^1) = p_j^k\) for every \(k\) and every \(j\) by the
construction of \(\tilde f_k\).

\begin{claim}

\(X^c \subseteq V(f_i)^c\):

\end{claim}

This follows from the fact the polynomials \(f\) given by Lagrange
interpolation are unique, and thus the roots of \(\tilde f\) are unique.
But if some other point was in \(V(f_i)\), then one of its coordinates
would be another root of some \(\tilde f\).

\end{proof}

\end{solution}

\begin{exercise}[Gathmann 1.21]

Determine \(\sqrt{I}\) for
\begin{align*}  
I\da \gens{x_1^3 - x_2^6,\, x_1 x_2 - x_2^3} \normal \CC[x_1, x_2]
.\end{align*}

\end{exercise}

\begin{solution}

For notational purposes, let \(\mathcal{I}, \mathcal{V}\) denote the
maps in Hilbert's Nullstellensatz, we then have
\begin{align*}(\mathcal{I} \circ \mathcal{V})(I) = \sqrt{I}.\end{align*}

So we consider \(\mathcal{V}(I) \subseteq \AA^2/\CC\), the vanishing
locus of these two polynomials, which yields the system
\begin{align*}  
\begin{cases}
x^3 - y^6 & = 0 \\
xy - y^3 & = 0.
\end{cases}
\end{align*} In the second equation, we have \((x- y^2)y = 0\), and
since \(\CC[x, y]\) is an integral domain, one term must be zero.

\begin{enumerate}
\def\labelenumi{\arabic{enumi}.}
\item
  If \(y=0\), then \(x^3 = 0 \implies x= 0\), and thus
  \((0, 0) \in \mathcal{V}(I)\), i.e.~the origin is contained in this
  vanishing locus.
\item
  Otherwise, if \(x-y^2 = 0\), then \(x=y^2\), with no further
  conditions coming from the first equation.
\end{enumerate}

Combining these conditions,
\begin{align*}P\da \ts{(t^2, t) \suchthat t\in \CC} \subset \mathcal{V}(I).\end{align*}

where \(I = \gens{x^3 - y^6, xy-y^3}\).

We have \(P = \mathcal{V}(I)\), and so taking the ideal generated by
\(P\) yields
\begin{align*}  
\qty{\mathcal{I} \circ \mathcal{V}} (I) = \mathcal{I}(P) = \gens{y-x^2} \in \CC[x ,y]
\end{align*}

and thus \(\sqrt{I} = \gens{y-x^2}\).

\end{solution}

\begin{exercise}[Gathmann 1.22]

Let \(X\subset \AA^3/k\) be the union of the three coordinate axes.
Compute generators for the ideal \(I(X)\) and show that it can not be
generated by fewer than 3 elements.

\end{exercise}

\begin{solution}

\textbf{Claim}:
\begin{align*}I(X) = \gens{x_2 x_3,\, x_1 x_3,\, x_1 x_2}.\end{align*}

We can write \(X = X_1 \union X_2 \union X_3\), where

\begin{itemize}
\tightlist
\item
  The \(x_1\dash\)axis is given by \(X_1 \da V(x_2 x_3)\)
  \(\implies I(X_1) = \gens{x_2 x_3}\),
\item
  The \(x_2\dash\)axis is given by \(X_2 \da V(x_1 x_3)\)
  \(\implies I(X_2) = \gens{x_1 x_3}\),
\item
  The \(x_3\dash\)axis is given by \(X_3 \da V(x_1 x_2)\)
  \(\implies I(X_3) = \gens{x_1 x_2}\).
\end{itemize}

Here we've used, for example, that
\begin{align*}I(V(x_2 x_3)) = \sqrt{\gens{x_2 x_3}} = \gens{x_2 x_3}\end{align*}
by applying the Nullstellensatz and noting that \(\gens{x_2x_3}\) is
radical since it is generated by a squarefree monomial.

We then have
\begin{align*}  
I(X) &= I(X_1 \union X_2 \union X_3) \\
&= I(X_1) \intersect I(X_2) \intersect I(X_3) \\
&= \sqrt{I(X_1) + I(X_2) + I(X_3)} \\
&= \sqrt{\gens{x_2, x_3} + \gens{x_1 x_3} + \gens{x_1 x_2}} \\
&= \sqrt{\gens{x_2x_3,\, x_1 x_3,\, x_1 x_2}} \hspace{8em}\text{since } \gens{a} + \gens{b} = \gens{a, b} \\
&= {\gens{x_2x_3,\, x_1 x_3,\, x_1 x_2}}
,\end{align*} where in the last equality we've again used the fact that
an ideal generated by squarefree monomials is radical.

\begin{claim}

\(I(X)\) can not be generated by 2 or fewer elements.

Let \(J\da I(X)\) and \(R\da k[x_1, x_2, x_3]\), and toward a
contradiction, suppose \(J = \gens{r, s}\). Define
\(\mfm \da \gens{x, y, z}\) and a quotient map
\begin{align*}\pi: J \to J/\mfm J\end{align*} and consider the images
\(\pi(r), \pi(s)\).

\hfill\break
Note that \(J/\mfm J\) is an \(R/\mfm\dash\)module, and since
\(R/\mfm \cong k\), \(J/\mfm J\) is in fact a \(k\dash\)vector space.
Since \(\pi(r), \pi(s)\) generate \(J/\mfm J\) as a \(k\dash\)module,
\begin{align*}\dim_k J/\mfm J \leq 2.\end{align*}

\hfill\break
But this is a contradiction, since we can produce 3 \(k\dash\)linearly
independent elements in \(J/\mfm J\): namely
\(\pi(x_1 x_2), \pi(x_1 x_3), \pi(x_2 x_3)\). Suppose there exist
\(\alpha_i\) such that
\begin{align*}  
\alpha_1 \pi(x_1 x_2) + \alpha_2 \pi(x_1 x_3) + \alpha_3 \pi(x_2 x_3) = 0 \in J/\mfm J \iff
\alpha_1 x_1 x_2 + \alpha_2 x_1 x_3 + \alpha_3 x_2 x_3 \in \mfm J
,\end{align*}

But we can then note that
\begin{align*}  
\mfm J = \gens{x_1, x_2. x_3}\gens{x_1 x_2, x_1 x_3, x_2 x_3} = 
\gens{x_1^2 x_2,\, x_1^2 x_3,\, x_1x_2 x_3, \cdots}
.\end{align*} can't contain any nonzero elements of degree \(d<3\), so
no such \(\alpha_i\) can exist and these elements are \(k\dash\)linearly
independent.

\end{claim}

\end{solution}

\begin{exercise}[Gathmann 1.23: Relative Nullstellensatz]

Let \(Y\subset \AA^n/k\) be an affine variety and define \(A(Y)\) by the
quotient
\begin{align*}  
\pi: k[x_1,\cdots, x_n] \to A(Y) \da k[x_1, \cdots, x_n]/I(Y)
.\end{align*}

\begin{enumerate}
\def\labelenumi{\alph{enumi}.}
\item
  Show that \(V_Y(J) = V(\pi^{-1}(J))\) for every \(J\normal A(Y)\).
\item
  Show that \(\pi^{-1} (I_Y(X)) = I(X)\) for every affine subvariety
  \(X\subseteq Y\).
\item
  Using the fact that \(I(V(J)) \subset \sqrt{J}\) for every
  \(J\normal k[x_1, \cdots, x_n]\), deduce that
  \(I_Y(V_Y(J)) \subset \sqrt{J}\) for every \(J\normal A(Y)\).
\end{enumerate}

Conclude that there is an inclusion-reversing bijection
\begin{align*}  
  \correspond{\text{Affine subvarieties}\\ \text{of } Y} \iff \correspond{\text{Radical ideals} \\ \text{in } A(Y)}
  .\end{align*}

\end{exercise}

\newpage

\begin{exercise}[Extra]

Let \(J \normal k[x_1, \cdots, x_n]\) be an ideal, and find a
counterexample to \(I(V(J)) =\sqrt{J}\) when \(k\) is not algebraically
closed.

\end{exercise}

\begin{solution}

Take \(J = \gens{x^2+1} \normal \RR[x]\), noting that \(J\) is
nontrivial and proper but \(\RR\) is not algebraically closed. Then
\(V(J) \subseteq \RR\) is empty, and thus \(I(V(J)) = I(\emptyset)\).

\begin{claim}

\(I(V(J)) = \RR[x]\).

Checking definitions, for any set \(X \subset \AA^n/k\) we have
\begin{align*}  
I(X) = \ts{f\in \RR[x] \st \forall x\in X,\, f(x)=0} \\
\end{align*} and so we vacuously have
\begin{align*}
I(\emptyset) = \ts{f\in \RR[x] \st \forall x\in \emptyset,\, f(x)=0}
= \ts{f\in \RR[x]} = \RR[x]
.\end{align*}

\end{claim}

\begin{claim}

\(\sqrt{J} \neq \RR[x]\).

This follows from the fact that maximal ideals are radical, and
\(\RR[x]/ J \cong \CC\) being a field implies that \(J\) is maximal. In
this case \(\sqrt{J} = J \neq \RR[x]\).

\hfill\break
That maximal ideals are radical follows from the fact that if
\(J\normal R\) is maximal, we have \(J \subset \sqrt{J} \subset R\)
which forces \(\sqrt{J} = J\) or \(\sqrt{J}=R\).

\hfill\break
But if \(\sqrt{J}=R\), then
\begin{align*}  
1\in \sqrt{J} \implies 1^n \in J \text{ for some }n \implies 1 \in J \implies J=R
,\end{align*} contradicting the assumption that \(J\) is maximal and
thus proper by definition.

\end{claim}

\end{solution}

\hypertarget{problem-set-2}{%
\section{Problem Set 2}\label{problem-set-2}}

\begin{exercise}[Gathmann 2.17]

Find the irreducible components of
\begin{align*}  
X = V(x - yz, xz - y^2) \subset \AA^3/\CC
.\end{align*}

\end{exercise}

\begin{solution}

Since \(x=yz\) for all points in \(X\), we have
\begin{align*}  
X &= V(x-yz, yz^2 - y^2) \\
&= V\qty{x-yz, y(z^2 - y) } \\
&= V(x-yz, y) \union V(x-yz, z^2-y) \\
&\da X_1 \union X_2
.\end{align*}

\begin{claim}

These two subvarieties are irreducible.

\end{claim}

It suffices to show that the \(A(X_i)\) are integral domains. We have
\begin{align*}  
A(X_1) \da \CC[x,y,z] / \gens{x-yz, y} \cong \CC[y,z]/\gens{y} \cong \CC[z]
,\end{align*} which is an integral domain since \(\CC\) is a field and
thus an integral domain, and
\begin{align*}  
A(X_2) \da \CC[x,y,z]/\gens{x-yz, z^2 - y} \cong \CC[y,z]/\gens{z^2-y} \cong \CC[y]
,\end{align*} which is an integral domain for the same reason.

\end{solution}

\begin{exercise}[Gathmann 2.18]

Let \(X\subset \AA^n\) be an arbitrary subset and show that
\begin{align*}  
V(I(X)) = \bar{X}
.\end{align*}

\end{exercise}

\begin{solution}

\hfill

\(\bar X \subseteq V(I(X))\):\\
We have \(X\subseteq V(I(X))\) and since \(V(J)\) is closed in the
Zariski topology for any ideal \(J \normal \kx{n}\) by definition,
\(V(I(X))\) is closed. Thus
\begin{align*}  
X\subseteq V(I(X)) \text{ and } V(I(X))\text{ closed } \implies \bar X \subseteq V(I(X))
,\end{align*} since \(\bar X\) is the intersection of all closed sets
containing \(X\).\\

\(V(I(X)) \subseteq \bar X\):\\
Noting that \(V(\wait), I(\wait)\) are individually order-reversing, we
find that \(V(I(\wait))\) is order-\emph{preserving} and thus
\begin{align*}  
X\subseteq \bar X \implies V(I(X)) \subseteq V(I(\bar X)) = \bar X
,\end{align*} where in the last equality we've used part (i) of the
Nullstellensatz: if \(X\) is an affine variety, then \(V(I(X)) = X\).
This applies here because \(\bar X\) is always closed, and the closed
sets in the Zariski topology are precisely the affine varieties.

\end{solution}

\begin{exercise}[Gathmann 2.21]

Let \(\ts{U_i}_{i\in I} \covers X\) be an open cover of a topological
space with \(U_i \intersect U_j \neq \emptyset\) for every \(i, j\).

\begin{enumerate}
\def\labelenumi{\alph{enumi}.}
\item
  Show that if \(U_i\) is connected for every \(i\) then \(X\) is
  connected.
\item
  Show that if \(U_i\) is irreducible for every \(i\) then \(X\) is
  irreducible.
\end{enumerate}

\end{exercise}

\begin{solution}[a]

Suppose toward a contradiction that \(X = X_1 \disjoint X_2\) with
\(X_i\) proper, disjoint, and open. Since \(\ts{U_i} \covers X\), for
each \(j\in I\) this would force one of \(U_j \subseteq X_1\) or
\(U_j \subseteq X_2\), since otherwise
\(U_j \intersect X_1 \intersect X_2\) would be nonempty.\\
So without loss of generality (relabeling if necessary), assume
\(U_j \in X_1\) for some fixed \(j\). But then for every \(i\neq j\), we
have \(U_i \intersect U_j\) nonempty by assumption, and so in fact
\(U_i \subseteq X_1\) for every \(i\in I\). But then
\(\union_{i\in I}U_i \subseteq X_1\), and since \(\ts{U_i}\) was a
cover, this forces \(X\subseteq X_1\) and thus \(X_2 = \emptyset\).

\end{solution}

\begin{solution}[b]

\begin{claim}

\(X\) is irreducible \(\iff\) any two open subsets intersect.

\end{claim}

This follows because otherwise, if \(U, V \subset X\) are open and
disjoint then \(X\sm U,\, X\sm V\) are proper and closed. But then we
can write \(X = \qty{X\sm U} \disjoint \qty{X\sm V}\) as a union of
proper closed subsets, forcing \(X\) to not be irreducible.\\

So it suffices to show that if \(U, V\subset X\) then \(U\intersect V\)
is nonempty. Since \(\ts{U_i} \covers X\), we can find a pair \(i, j\)
such that there is at least one point in \(U\intersect U_i\) and one
point in \(V \intersect U_j\).\\

But by assumption \(U_i\intersect U_j\) is nonempty, so both
\(U\intersect U_i\) and \(U_j \intersect U_i\) are open nonempty subsets
of \(U_i\). Since \(U_i\) was assumed irreducible, they must intersect,
so there exists a point
\begin{align*}  
x_0 \in \qty{U\intersect U_i} \intersect \qty{U_j \intersect U_i} = U\intersect \qty{U_i \intersect U_j} \da \tilde U
.\end{align*}

We can now similarly note that \(\tilde U \intersect V\) and
\(U_j \intersect V\) are nonempty open subsets of \(V\), and thus
intersect. So there is a point
\begin{align*}  
\tilde x_0 \in \qty{\tilde U \intersect V} \intersect \qty{U_j \intersect V} = \tilde U\intersect V = U\intersect V \intersect \qty{U_i \intersect U_j}
,\end{align*} and in particular \(\tilde x_0 \in U\intersect V\) as
desired.

\end{solution}

\begin{exercise}[Gathmann 2.22]

Let \(f:X\to Y\) be a continuous map of topological spaces.

\begin{enumerate}
\def\labelenumi{\alph{enumi}.}
\item
  Show that if \(X\) is connected then \(f(X)\) is connected.
\item
  Show that if \(X\) is irreducible then \(f(X)\) is irreducible.
\end{enumerate}

\end{exercise}

\begin{solution}[a]

Toward a contradiction, if \(f(X) = Y_1 \disjoint Y_2\) with
\(Y_1, Y_2\) nonempty and open in \(Y\), then
\begin{align*}f\inv(f(X)) \subseteq X\end{align*} on one hand, and
\begin{align*}f^{-1}(f(X)) = f^{-1}(Y_1) \disjoint f^{-1}(Y_2)\end{align*}
on the other. If \(f\) is continuous, the preimages \(f^{-1}(Y_i)\) are
open (and nonempty), so \(X\) contains a disconnected subset. However,
every subset of a connected set must be connected, so this contradicts
the connectedness of \(X\).

\end{solution}

\begin{solution}[b]

Suppose \(f(X) = Y_1 \union Y_2\) with \(Y_i\) proper closed subsets of
\(Y\). Then
\(f^{-1}(Y_1) \union f^{-1}(Y^2) = (f^{-1} \circ f)(X) \subseteq X\) are
closed in \(X\), since \(f\) is continuous. Since \(X\) is irreducible,
without loss of generality (by relabeling), this forces
\(X_1 = \emptyset\). But then \(f(X_1) = \emptyset\), forcing
\(f(X) = Y_2\).

\end{solution}

\begin{definition}[Ideal Quotient]

For two ideals \(J_1, J_2\normal R\), the \emph{ideal quotient} is
defined by
\begin{align*}  
J_1 : J_2 \da \ts{f\in R \st fJ_2 \subset J_1}
.\end{align*}

\end{definition}

\begin{exercise}[Gathmann 2.23]

Let \(X\) be an affine variety.

\begin{enumerate}
\def\labelenumi{\alph{enumi}.}
\item
  Show that if \(Y_1, Y_2 \subset X\) are subvarieties then
  \begin{align*}  
  I(\bar{Y_1\sm Y_2}) = I(Y_1): I(Y_2)
  .\end{align*}
\item
  If \(J_1, J_2 \normal A(X)\) are radical, then
  \begin{align*}  
  \bar{V(J_1) \sm V(J_2)} = V(J_1: J_2)
  .\end{align*}
\end{enumerate}

\end{exercise}

\begin{solution}

?

\end{solution}

\begin{exercise}[Gathmann 2.24]

Let \(X \subset \AA^n,\, Y\subset \AA^m\) be irreducible affine
varieties, and show that \(X\cross Y\subset \AA^{n+m}\) is irreducible.

\end{exercise}

\begin{solution}

That \(X\cross Y\) is again an affine variety follows from writing
\(X=V(I),\, Y=V(J)\), then \(X\cross Y = V(I+J)\) where
\(I+J\normal k[x_1, \cdots, x_n, y_1, \cdots, y_m]\). So let
\begin{align*}X\cross Y = U \union V\end{align*} with \(U, V\) proper
and closed, and let \(\pi_X, \pi_Y\) be the projections onto the
factors.

\begin{claim}

For each \(x\in X\), \(\pi^{-1}(x) \cong Y\) is contained in only one of
\(U\) or \(V\).

\end{claim}

Note that if this is true, we can write \(X = G_U \union G_V\) where
\begin{align*}  
G_U\da\ts{x\in X \st \pi_X^{-1}(x) \subseteq U}
\end{align*} are the points for which the entire fiber lies in \(U\),
and similarly \(G_V\) are those for which the fiber lies in \(V\). If we
can then show that \(G_U, G_V\) are closed, by irreducibility of \(X\)
this will force (wlog) \(G_V = \emptyset\) and \(X = G_U\). But then
\begin{align*}  
\pi_X^{-1}(X) = X\cross Y \text{ and }\pi_X^{-1}(G_U) = U  \implies X\cross Y = U
.\end{align*} which shows that \(X\cross Y\) is irreducible.

\begin{proof}[Every fiber is contained in one irreducible component]

For any fixed \(x\), we can write
\begin{align*}  
\pi_X^{-1}(x) = \qty{\pi_X\inv(x) \intersect U } \union \qty{\pi_X^{-1}(x) \intersect V}
.\end{align*}

Since points are closed in the Zariski topology and \(\pi_X\) is
continuous, each \(\pi_X^{-1}(x)\) is closed. and thus
\(\pi_X^{-1}(x)\intersect U\) is closed (and similarly for \(V\)).
Noting that \(\pi_X^{-1}(x) \cong \ts{x}\cross Y \cong Y\), where we've
assumed \(Y\) to be irreducible, we can conclude wlog that
\(\pi_X^{-1}(x) \intersect V = \emptyset\).

\end{proof}

\begin{proof}[$G_U, G_V$ are closed]

Wlog consider \(G_U \subseteq X\). Fixing any point \(y_0 \in Y\), we
have
\begin{align*}X\cong X_{y_0} \da X\cross \ts{y_0} \subseteq X\cross Y,\end{align*}
so we can identify \(G_U \subset X\) with \(G_U\subset X_{y_0}\) inside
a \(Y\dash\)fiber the product. But then
\begin{align*}G_U = X_{y_0} \intersect U \subseteq X\cross Y,\end{align*}
where \(U\) is closed in \(X\cross Y\) and thus closed in \(X_{y_0}\),
and \(X_{y_0}\) is trivially closed in itself. This exhibits \(G_U\) as
the intersection of two sets that are closed in \(X_{y_0} \cong X\).

\end{proof}

\end{solution}

\hypertarget{problem-set-3}{%
\section{Problem Set 3}\label{problem-set-3}}

\begin{exercise}[Gathmann 2.33]

Define
\begin{align*}  
X \da \ts{M \in \mat(2\times 3, k) \st \rk M \leq 1} \subseteq \AA^6/k
.\end{align*}

Show that \(X\) is an irreducible variety, and find its dimension.

\end{exercise}

\begin{solution}

We'll use the following fact from linear algebra:

\begin{definition}[Matrix Minor]

For an \(m\times n\) matrix, a \emph{minor of order} \(\ell\) is the
determinant of a \(\ell\times \ell\) submatrix obtained by deleting any
\(m-\ell\) rows and any \(n-\ell\) columns.

\end{definition}

\begin{theorem}[Rank is a Function of Minors]

If \(A\in \Mat(m \times n, k)\) is a matrix, then the rank of \(A\) is
equal to the order of largest nonzero minor.

\end{theorem}

Thus
\begin{align*}  
M_{ij} = 0 \text{ for all $\ell\times \ell$ minors } M_{ij} \iff \rk(M) < \ell
,\end{align*} following from the fact that if one takes
\(\ell = \min(m,n)\) and all \(\ell\times \ell\) minors vanish, then the
largest nonzero minor must be of size \(j\times j\) for
\(j\leq \ell -1\). But \(\det M_{ij}\) is a polynomial \(f_{ij}\) in its
entries, which means that \(X\) can be written as
\begin{align*}  
X = V\qty{\ts{f_{ij}}}
,\end{align*} which exhibits \(X\) as a variety. Thus
\begin{align*}  
M = 
\begin{bmatrix}
x & y & z \\
a & b & c
\end{bmatrix}
\implies
X = V\qty{\gens{xb-ya, yc-zb, xc-za}} \subset \AA^6 
.\end{align*}

\begin{claim}

The ideal above is prime, and so the coordinate ring \(A(X)\) is a
domain and thus \(X\) is irreducible.

\end{claim}

\begin{claim}

\(\dim (X) = 4\).

\end{claim}

Heuristic: there are three degrees of freedom in choosing the first row
\(x,y,z\). To enforce the rank 1 condition, the second row must be a
scalar multiple of the first, yielding one degree of freedom for the
scalar.

\begin{quote}
Note: I looked at this for a couple of hours, but I don't know how to
prove either of these statements with the tools we have so far!
\end{quote}

\end{solution}

\begin{exercise}[Gathmann 2.34]

Let \(X\) be a topological space, and show

\begin{enumerate}
\def\labelenumi{\alph{enumi}.}
\item
  If \(\ts{U_i}_{i\in I} \covers X\), then
  \(\dim X = \sup_{i\in I} \dim U_i\).
\item
  If \(X\) is an irreducible affine variety and \(U\subset X\) is a
  nonempty subset, then \(\dim X = \dim U\). Does this hold for any
  irreducible topological space?
\end{enumerate}

\end{exercise}

\begin{solution}

\begin{quote}
Strictly for notational convenience, we'll treat \(\ts{U_i}\) is if it
were a countable open cover.
\end{quote}

\textbf{Part a:} We first note that if \(U \subseteq V\), then
\(\dim U \leq \dim V\). If this were not the case, one could find a
chain \(\ts{I_j}\) of closed irreducible subsets of \(V\) of length
\(n>\dim U\). But then \(I'_j \da I_j \intersect U\) would again be a
closed irreducible set, yielding a chain of length \(n\) in \(U\). Thus
\(\dim X\geq \dim U_i\), and it remains true that
\(\dim X \geq \sup \dim U_i\), so it suffices to show that
\(\dim X \leq \sup \dim U_i\).\\

Set \(s \da \sup_i \dim U_i\) and \(n\da \dim X\), we want to show that
\(s\geq n\). Let \(\ts{I_j}_{j\leq n}\) be a maximal chain of length
\(n\) of closed irreducible subsets of \(X\), so we have
\begin{align*}  
\emptyset \subsetneq I_0 \subsetneq I_1 \subsetneq \cdots \subsetneq I_n \subseteq X
.\end{align*}

Since \(I_0\subset X\) and \(\ts{U_i}\) covers \(X\), we can find some
\(U_{0}\in \ts{U_i}\) such that \(I_0\intersect U_0\) is nonempty, since
otherwise there would be a point in
\(I_0 \cap \qty{X\setminus \cup_{i\in J} U_i} = \emptyset\). We can do
this for every \(I_j\), so define \(A_j \da I_j \intersect U_0\).\\

Each \(A_j\) is now closed in \(U_0\), and must remain irreducible,
since any decomposition of \(A_j\) would lift to a decomposition of
\(I_0\). To see that \(A_0 \subsetneq A_1\), i.e.~that the inclusions
are still proper, we can just note that
\begin{align*}  
x\in A_{i+1}\setminus A_i \iff x\in  
\qty{I_{i+1} \intersect U_0} \setminus 
\qty{I_{i} \intersect U_0} = \qty{I_1 \setminus I_2}\intersect U_0 = \emptyset
.\end{align*} But this exhibits a length \(n\) chain in \(U_0\), so
\(\dim U_0 \geq n\). Taking suprema, we have
\begin{align*}  
n \leq \dim U_0 \leq \sup_{i\in J} \dim U_i = s
.\end{align*}

\textbf{Part b}: The answer is \textbf{no}: we can produce a space \(X\)
with some \(\dim X\) and a subset \(U\) satisfying \(\dim U < \dim X\).

Define a space and a topology by
\begin{align*}  
X \da \ts{a, b} \qquad \tau \da\ts{\emptyset, X, \ts 1}
,\end{align*} Here \(\ts{b}\) is the only proper and closed subset,
since its complement is open, so \(X\) must be irreducible. We can find
an maximal ascending chain of length \(1\),
\begin{align*}  
\emptyset \subsetneq \ts{b} \subsetneq X
,\end{align*} and so \(\dim X = 1\). However, for \(U\da \ts{a}\), there
is only one possible maximal chain:
\begin{align*}  
\emptyset \subsetneq \ts{a} = X
,\end{align*} so \(\dim U = 0\).

\end{solution}

\begin{exercise}[Gathmann 2.36]

Prove the following:

\begin{enumerate}
\def\labelenumi{\alph{enumi}.}
\item
  Every noetherian topological space is compact. In particular, every
  open subset of an affine variety is compact in the Zariski topology.
\item
  A complex affine variety of dimension at least 1 is never compact in
  the classical topology.
\end{enumerate}

\end{exercise}

\begin{exercise}[Gathmann 2.40]

Let
\begin{align*}  
R = k[x_1, x_2, x_3, x_4] / \gens{x_1 x_4 - x_2 x_3} 
\end{align*} and show the following:

\begin{enumerate}
\def\labelenumi{\alph{enumi}.}
\item
  \(R\) is an integral domain of dimension 3.
\item
  \(x_1, \cdots, x_4\) are irreducible but not prime in \(R\), and thus
  \(R\) is not a UFD.
\item
  \(x_1 x_4\) and \(x_2 x_3\) are two decompositions of the same element
  in \(R\) which are nonassociate.
\item
  \(\gens{x_1, x_2}\) is a prime ideal of codimension 1 in \(R\) that is
  not principal.
\end{enumerate}

\end{exercise}

\begin{exercise}[Problem 5]

Consider a set \(U\) in the complement of \((0, 0) \in \AA^2\). Prove
that any regular function on \(U\) extends to a regular function on all
of \(\AA^2\).

\end{exercise}

\hypertarget{problem-set-4-tuesday-october-06}{%
\section{Problem Set 4 (Tuesday, October
06)}\label{problem-set-4-tuesday-october-06}}

\begin{problem}[Gathmann 3.20]

Let \(X\subset \AA^n\)be an affine variety and \(a\in X\). Show that
\begin{align*}  
\OO_{X, a} = \OO_{\AA^n, a} / I(X) \OO_{A^n,a}
,\end{align*} where \(I(X) \OO_{\AA^n, a}\) denotes the ideal in
\(\OO_{\AA^n, a}\) generated by all quotients \(f/1\) for \(f\in I(X)\).

\end{problem}

\begin{problem}[Gathmann 3.21]

Let \(a\in \RR\), and consider sheaves \(\mathcal{F}\) on \(\RR\) with
the standard topology:

\begin{enumerate}
\def\labelenumi{\arabic{enumi}.}
\tightlist
\item
  \(\mathcal{F} \da\) the sheaf of continuous functions
\item
  \(\mathcal{F} \da\) the sheaf of locally polynomial functions.
\end{enumerate}

For which is the stalk \(\mathcal{F}_a\) a local ring?

\begin{quote}
Recall that a local ring has precisely one maximal ideal.
\end{quote}

\end{problem}

\begin{problem}[Gathmann 3.22]

Let \(\phi, \psi \in \mathcal{F}(U)\) be two sections of some sheaf
\(\mathcal{F}\) on an open \(U\subseteq X\) and show that

\begin{enumerate}
\def\labelenumi{\alph{enumi}.}
\item
  If \(\phi, \psi\) agree on all stalks, so
  \(\bar{(U, \phi)} = \bar{(U, \psi)} \in \mathcal{F}_a\) for all
  \(a\in U\), then \(\phi\) and \(\psi\) are equal.
\item
  If \(\mathcal{F} \da \OO_X\) is the sheaf of regular functions on some
  irreducible affine variety \(X\), then if \(\psi = \phi\) on one stalk
  \(\mathcal{F}_a\), then \(\phi = \psi\) everywhere.
\item
  For a general sheaf \(\mathcal{F}\) on \(X\), (b) is false.
\end{enumerate}

\end{problem}

\begin{definition}[Stalk at a subspace]

Let \(Y\subset X\) be a nonempty and irreducible subspace of \(X\) a
topological space with a sheaf \(\mathcal{F}\) on \(X\). Then the stalk
of \(\mathcal{F}\) at \(Y\) is defined by the pairs \((U, \phi)\) such
that \(U\subset X\) is open, \(U\cap Y\) is nonempty, and
\(\phi \in \mathcal{F}(U)\), where we identify
\((U, \phi) \sim (U',\phi')\) iff there is a small enough open set such
that the restrictions agree.

\end{definition}

\begin{problem}[Gathmann 3.23: Geometry of a Certain Localization]

Let \(Y\subset X\) be a nonempty and irreducible subvariety of an affine
variety \(X\), and show that the stalk \(\OO_{X, Y}\) of \(\OO_X\) at
\(Y\) is a \(k\dash\)algebra which is isomorphic to the localization
\(A(X)_{I(Y)}\).

\end{problem}

\begin{problem}[Gathmann 3.24]

Let \(\mathcal{F}\) be a sheaf on \(X\) a topological space and
\(a\in X\). Show that the stalk \(\mathcal{F}_a\) is a \emph{local
object}, i.e.~if \(U\subset X\) is an open neighborhood of \(a\), then
\(\mathcal{F}_a\) is isomorphic to the stalk of
\(\ro{ \mathcal{F} }{U}\) at \(a\) on \(U\) viewed as a topological
space.

\end{problem}

\hypertarget{problem-set-5-monday-october-26}{%
\section{Problem Set 5 (Monday, October
26)}\label{problem-set-5-monday-october-26}}

\begin{problem}[Gathmann 4.13]

Let \(f:X\to Y\) be a morphism of affine varieties and
\(f^*: A(Y) \to A(X)\) the induced map on coordinate rings. Determine if
the following statements are true or false:

\begin{enumerate}
\def\labelenumi{\alph{enumi}.}
\item
  \(f\) is surjective \(\iff f^*\) is injective.
\item
  \(f\) is injective \(\iff f^*\) is surjective.
\item
  If \(f:\AA^1\to\AA^1\) is an isomorphism, then \(f\) is \emph{affine
  linear}, i.e.~\(f(x) = ax+b\) for some \(a, b\in k\).
\item
  If \(f:\AA^2\to\AA^2\) is an isomorphism, then \(f\) is \emph{affine
  linear}, i.e.~\(f(x) = Ax+b\) for some \(a \in \Mat(2\times 2, k)\)
  and \(b\in k^2\).
\end{enumerate}

\end{problem}

\begin{solution}

\hfill

\begin{enumerate}
\def\labelenumi{\alph{enumi}.}
\tightlist
\item
  \textbf{True}. This follows because if \(p, q\in A(Y)\), then
  \begin{align*}  
    f* p &= f^* q \\
    &\implies (p\circ f) = (q\circ f) && \text{by definition}\\
    &\implies p = q 
    ,\end{align*} where in the last implication we've used the fact that
  \(f\) is surjective iff \(f\) admits a right-inverse.
\end{enumerate}

\end{solution}

\begin{problem}[Gathmann 4.19]

Which of the following are isomorphic as ringed spaces over \(\CC\)?

\begin{enumerate}
\def\labelenumi{(\alph{enumi})}
\item
  \(\mathbb{A}^{1} \backslash\{1\}\)
\item
  \(V\left(x_{1}^{2}+x_{2}^{2}\right) \subset \mathbb{A}^{2}\)
\item
  \(V\left(x_{2}-x_{1}^{2}, x_{3}-x_{1}^{3}\right) \backslash\{0\} \subset \mathbb{A}^{3}\)
\item
  \(V\left(x_{1} x_{2}\right) \subset \mathbb{A}^{2}\)
\item
  \(V\left(x_{2}^{2}-x_{1}^{3}-x_{1}^{2}\right) \subset \mathbb{A}^{2}\)
\item
  \(V\left(x_{1}^{2}-x_{2}^{2}-1\right) \subset \mathbb{A}^{2}\)
\end{enumerate}

\end{problem}

\begin{problem}[Gathmann 5.7]

Show that

\begin{enumerate}
\def\labelenumi{\alph{enumi}.}
\item
  Every morphism \(f:\AA^1\smz \to \PP^1\) can be extended to a morphism
  \(\hat f: \AA^1 \to \PP^1\).
\item
  Not every morphism \(f:\AA^2\smz \to \PP^1\) can be extended to a
  morphism \(\hat f: \AA^2 \to \PP^1\).
\item
  Every morphism \(\PP^1\to \AA^1\) is constant.
\end{enumerate}

\end{problem}

\begin{problem}[Gathmann 5.8]

Show that

\begin{enumerate}
\def\labelenumi{\alph{enumi}.}
\item
  Every isomorphism \(f:\PP^1\to \PP^1\) is of the form
  \begin{align*}  
  f(x) = {ax+b \over cx+d} && a,b,c,d\in k
  .\end{align*} where \(x\) is an affine coordinate on
  \(\AA^1\subset \PP^1\).
\item
  Given three distinct points \(a_i \in \PP^1\) and three distinct
  points \(b_i \in \PP^1\), there is a unique isomorphism
  \(f:\PP^1 \to \PP^1\) such that \(f(a_i) = b_i\) for all \(i\).
\end{enumerate}

\end{problem}

\begin{proposition}[?]

There is a bijection
\begin{align*}
\begin{array}{c}
\{\text { morphisms } X \rightarrow Y\} \stackrel{1: 1}{\longleftrightarrow}\left\{K \text { -algebra homomorphisms } \mathscr{O}_{Y}(Y) \rightarrow \mathscr{O}_{X}(X)\right\} \\
f \longmapsto f^{*}
\end{array}
\end{align*}

\end{proposition}

\begin{problem}[Gathmann 5.9]

Does the above bijection hold if

\begin{enumerate}
\def\labelenumi{\alph{enumi}.}
\tightlist
\item
  \(X\) is an arbitrary prevariety but \(Y\) is still affine?
\item
  \(Y\) is an arbitrary prevariety but \(X\) is still affine?
\end{enumerate}

\end{problem}




\end{document}
