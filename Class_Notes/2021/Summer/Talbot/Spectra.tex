\input{"preamble.tex"}


\let\Begin\begin
\let\End\end
\newcommand\wrapenv[1]{#1}

\makeatletter
\def\ScaleWidthIfNeeded{%
 \ifdim\Gin@nat@width>\linewidth
    \linewidth
  \else
    \Gin@nat@width
  \fi
}
\def\ScaleHeightIfNeeded{%
  \ifdim\Gin@nat@height>0.9\textheight
    0.9\textheight
  \else
    \Gin@nat@width
  \fi
}
\makeatother

\setkeys{Gin}{width=\ScaleWidthIfNeeded,height=\ScaleHeightIfNeeded,keepaspectratio}%

\title{
\textbf{
    Title
  }
  }







\begin{document}

\date{}
\author{D. Zack Garza}
\maketitle


\newpage

% Note: addsec only in KomaScript
\addsec{Table of Contents}
\tableofcontents
\newpage

\hypertarget{thursday-june-03}{%
\section{Thursday, June 03}\label{thursday-june-03}}

\hypertarget{part-1}{%
\subsection{Part 1}\label{part-1}}

\begin{quote}
Jeremey Hanh, MIT: something abelian about commutative ring spectra See
Tablbot 2017 notes and Richter's survey article ``Commutative Ring
Spectra''.
\end{quote}

\begin{remark}

The 2017 Talbot was on structured ring spectra, i.e.~``brave new
algebra'', where we study \(E_n\dash\)ring spectra. Getting the subject
off the ground requires many definitions, care of May, EKMM, Lurie, and
more.

\end{remark}

\begin{remark}

The key objects we'll be considering:

\begin{itemize}
\tightlist
\item
  \(\SS\), the initial \(E_\infty\) ring spectrum
\item
  Thom \(E_\infty\)ring spectra: \(\MO, \MSO, \MSp, \MStr, \MU\), and
  the sphere fits into this pattern as framed bordism
\end{itemize}

We construct \emph{other} \(E_\infty\) rings primarily to study these
motivating examples.

\end{remark}

\begin{example}[?]

A ring spectrum that shows up naturally when ?: \(\MU\), for which
\begin{align*}
\pi_* \MU \cong \ZZ[x_1, x_2, \cdots]
&& \text{where } \abs{x_i} = 2i
\end{align*}
Localing at a prime \(p\) splits \(\MU_{(p)}\) into sum of suspensions
\(\BP\) where
\begin{align*}
\pi_* \BP = \ZZ_{(p)}[x_1, x_2, \cdots]
&&
\text{where } \abs{x_i} = 2p^i - 2
.\end{align*}
We can write
\begin{align*}
\BP = \MU / \gens{x_j \st j\neq p^i - 1}
,\end{align*}
so \(\BP\) has simpler homotopy, and \(\MU\) splits into a sum of copies
of \(\BP\), which are easier to work with because the homotopy is more
sparsely distributed. \(\BP\) is a bit of a ``designer'' spectrum, and
\(\MU\) is more geometric.

\end{example}

\begin{question}

Does this splitting preserve the face that \(\MU\) is an \(E_\infty\)
ring spectrum? I.e. is \(\BP\) an \(E_\infty\) ring spectrum?

\end{question}

\begin{answer}

No! See Lawson and Senger, who prove that \(\BP\) not an
\(E_{2p^2 + q}\dash\)ring spectrum. It turns out that understanding the
\(E_4\) structure is sufficient for many applications.

\end{answer}

\begin{theorem}[Basterra, Mandell]

\(\BP\) is an \(E_4\dash\)algebra retract of \(\MU\).

\end{theorem}

\begin{remark}

Some open questions:

\begin{enumerate}
\def\labelenumi{\arabic{enumi}.}
\tightlist
\item
  Is \(\BP\) an \(E_5\dash\MU\dash\)algebra?
\item
  The \(E_n\) operad acts on \(\BP\), so is \(\BP\) a
  \(\Disk_2\dash\)algebra? In particular, can one take factorization
  homology against unframed surfaces? This would correspond to a trivial
  \(S^1\dash\)action.
\end{enumerate}

\end{remark}

\begin{remark}

Studying other \(E_\infty\) rings naturally leads to problems in
obstruction theory. We study \(\SS\) via chromatic homotopy theory. The
basic strategy:

\begin{itemize}
\item
  Resolve \(\SS\) by other \(E_\infty\) rings, namely the
  \(K(n)\dash\)local sphere \(L_{K(n)} \SS\). Note that this is a
  Bousfield localization. This is useful precisely because of the
  chromatic convergence theorem, and one can build a tower whose
  associated graded are these local spheres.
\item
  Resolve \(L_{K(n)} \SS\) for a fixed \(n\) by the Hopkins-Miller
  \(\EO\dash\)theories, which are \(E_\infty\) rings of the form
  \(E_n^{hG}\) for \(G\) a finite group acting on the height \(n\)
  Morava \(E\dash\)theory \(E_n\). These \(\EO\) theories are supposed
  to be the basic building blocks of the \(K(n)\dash\)local spheres.
  (Check)
\end{itemize}

\end{remark}

\begin{question}[A big one]

Can one generally construct the \(\EO\) localizations above?

\end{question}

\begin{remark}

Proved for a fixed degree/height? Check. A recent triumph of obstruction
theory is that the building block \(\EO\) theories have been built. This
was a prominent topic in 2017 Talbot.

\end{remark}

\begin{remark}

How one builds \(\EO\) theories:

\begin{itemize}
\tightlist
\item
  Build \(E_n\) as a homotopy commutative ring using the Landweber exact
  functor theorem, which is not too difficult.
\item
  Promote the \(E_n\) to \(E_\infty\) rings using obstruction theory.

  \begin{itemize}
  \tightlist
  \item
    See Robinson, Goerss-Hopkins, Lurie, Pstragowki and VanKoughnett.
    These allow one to make \(G\dash\)actions by \(E_\infty\) ring maps.
  \item
    Rough sketch: build an \(E_\infty\) ring in \(\ho\Sp\) (the homotopy
    category of spectra), which is already a homotopy commutative ring.
    Then do this in the homotopy 2-category of \(\Sp\), then the
    homotopy 3-category of \(\Sp\), and so on. This uses that an
    \(\infty\dash\)category is a sequence of \(n\dash\)categories.
  \end{itemize}
\end{itemize}

\end{remark}

\begin{question}

Can one compute \(\pi_* \EO\) for \(\EO \da E_n^{hG}\) for various \(n\)
and \(G\)?

\end{question}

\begin{example}[?]

The key to the Kervaire invariant one question is computing
\(\pi_* E_4 ^{hC_8}\), and captures information about diffeomorphism
classes of exotic spheres.

\end{example}

\begin{observation}

In practice, these \(\EO\) theories (which are all \(K(n)\dash\)local)
seem to be \(K(n)\dash\)localizations of nice connective ring spectra.

\end{observation}

\begin{example}[?]

At the prime \(p=2\), \(E_1^{hC_2} = \KO\localize{2}\) is the
localization of \(\KO\) at 2, and turns out to be equal to
\(L_{K(1)}(k_0)\): It turns out that
\(E_2^{hC_{24}} = L_{K(2)} (\tmf)\), and so \(\ko\) and \(\tmf\) are
``connective \(E_\infty\) lifts'' of \(E_1^{hC_2}\) and
\(E_2^{hC_{24}}\).

\begin{align*}
E_1^{hC_2} = \KO\localize{2} &= L_{K(1)}(\ko) \\
E_2^{hC_{24}} &= L_{K(2)}(\tmf)
.\end{align*}

\end{example}

\begin{remark}

These lifts are closer to geometry than \(\EO\) theories, e.g.~there is
an \(E_\infty\) ring map due to Ando-Hopkins-Rezk
\begin{align*}
\MString \to \tmf
.\end{align*}

\begin{quote}
Find comments
\end{quote}

See how \(\MSpin\) splits as \todo[inline]{Missed}

so one might expect that \(\MString\) is

\todo[inline]{??? Missed}

\end{remark}

\begin{remark}

An observation due to Hi-Kriz and Hill-Hopkins-Ravenel: using sparsity,
the easiest way to compute \(\pi_* \EO\) is to compute \(\pi_* \eo\)
where \(\eo\) is a good connective lift of \(\EO\). Note that
\(\pi_* \tmf\) is finitely-generated in each degree, and it's useful to
work with something ``small'' in computations, for example if you're
trying to rule out differentials in spectral sequences. A nice way to
organize the computations of \(\pi_* \EO\) is to understand them via
lifts with better finiteness properties.

\end{remark}

\begin{question}

Can we make connective \(\eo\) theories highly structured with
\(L_{K(n)} \eo = \EO = E_n^{hG}\) for various and \(n\) and \(G\)?

\end{question}

\begin{quote}
See example in comment
\end{quote}

\begin{remark}

\(\tmf\) is a connective \(\EO\) theory, so how is \(\tmf\localize{2}\)
built? For full details, see the tmf book or Lurie's ``Elliptic
Cohomology''. A sketch:
\(\tmf\localize{2} = \tau_{\geq 0} L_2 \tmf\localize{2}\) where the
latter is built out of a finite resolution involving the following three
terms:

\begin{itemize}
\tightlist
\item
  \(L_{K(2)} \tmf\)
\item
  \(L_{K(1)} \tmf\)
\item
  \(L_{\QQ} \tmf\)
\end{itemize}

Note that \(L_2\) is the second stage of the chromatic tower. The basic
strategy: take the monochromatic pieces above, which are relatively easy
to make and work with, and find a way to glue them together.

\end{remark}

\begin{remark}

An idea: one can try to make \(\eo\) as a connective cover of some
\(L_n\dash\)local object. See Lawson, Berhrens-Lawson and \(\TAF\)
(topological automorphic forms). This worked very well for \(\tmf\), and
there is currently partial progress at height 3. We're not yet able to
construct a connective version of \(E_4^{hC_8}\), which was needed in
Kervaire Invariant One.

Note that all techniques used here seem to work equally well for
building \(E_\infty\) as \(E_n\) rings for finite \(n\).

\begin{quote}
See comments
\end{quote}

\end{remark}

\begin{remark}

An alternate idea that let Hill-Hopkins-Ravenel solve Kervaire Invariant
One, and recently developed by Beaudry-Hill-Shi-Zeng, constructs a
connective version of \(E_4^{hC_8}\). However, with this construction,
it's less clear how much structure there is on the object. They use the
following procedure:

\begin{itemize}
\item
  Put a \(C_8\) action on \(\MU^{\tensor 4}\) by viewing this as a norm
  \(\NN_{C_2}^{C_8} \MU_\RR\)

  \todo[inline]{Check $\RR$ notation..}

  The norm here gives a way of boosting a \(C_2\) action on one tensor
  factor to a \(C_8\) action on 4 tensor factors.
\item
  Quotient by some elements, possibly losing structure, to obtain a
  quotient \(Q\). The connective version of \(E_4^{hC_8}\) is
  \(Q^{C_8}\).

  \begin{itemize}
  \tightlist
  \item
    It'd be interesting to know how much structure is lost here!
  \end{itemize}
\end{itemize}

\end{remark}

\begin{question}

Some natural questions that arise here:

\begin{itemize}
\tightlist
\item
  What group actions (with various amounts of structure,
  e.g.~\(E_\infty\)) act on tensor powers \(\MU^{\tensor m}\)?
\item
  What structure exists on quotients of such tensor powers?
\end{itemize}

\end{question}

\begin{remark}

Note that these quotients can be destructive when it comes to
maintaining \(E_n\) ring structures. Answering these amounts to building
structured models for these connective spectra. Understanding these two
questions would allow computing fixed points of certain Morava
\(E\dash\)theories? Big question: do these tensor products admit
\(G\dash\)actions beyond those which come from norms?

\end{remark}

\begin{remark}

Getting at these \(\EO\) would be huge! The HHR construction is
spectacular but somehow only works at the prime \(2\) and for cyclic
groups??

\end{remark}

\hypertarget{part-2}{%
\subsection{Part 2}\label{part-2}}

\begin{remark}

Recall that \(\BP\) is an \(E_4\) ring spectrum with
\(\pi* \BP \cong \ZZ_{(p)}[x_1, x_2, \cdots]\). What structure exists on
quotients of \(\BP\)? Any progress here would lead to many natural next
questions, e.g.~adding in group actions. There has been recent progress,
some of which is ripe for generalization -- e.g.~how much can be made
equivariant?

\end{remark}

\begin{example}[?]

Take
\begin{align*}
\BP \gens{ n }\da \BP / \gens{ v_n , v_{n+1}, \cdots }
\implies 
\pi_*\BP \gens{ n } \cong \ZZ_{(p)}[x_1, x_2, \cdots, x_n]
.\end{align*}

\end{example}

\begin{theorem}[Baker-Jeanneret]

For any choice of indecomposable generators
\(v_{n+1}, v_{n+2}, \cdots\), \(\BP\gens{ n }\) is an \(E_1\dash \BP\)
algebra.

\end{theorem}

\begin{theorem}[H-Wilson]

There exists a specific choice of generators \(v_{n+1}, v_{n+2}\) is an
\(E_3\dash\BP\) algebra.

\end{theorem}

\begin{remark}

It'd be exciting to try to take this result and use it in the
equivariant setting. We'll try to discuss a bit how this theorem is
proved.

\end{remark}

\begin{proposition}[?]

If \(x\in \pi_{2\ell} \BP\) is any class in \(\pi_* \BP\), then
\(\BP / \gens{ x }\) is an \(E_1\dash\BP\) algebra.

\end{proposition}

\begin{remark}

This says you can freely mod out by any generator and still obtain an
\(E_1\) structure.

\end{remark}

\begin{proof}[?]

Let
\(S^0[a_{2\ell}] = S^0 \oplus S^{2\ell} \oplus S^{4\ell} \oplus \cdots\)
denote the free \(E_1\) ring on \(S^{2\ell}\). There is an \(E_1\) ring
map
\begin{align*}
\psi: S^0[a_{2\ell}] \to \BP
,\end{align*}
which hits \(x\), and
\begin{align*}
\BP / \gens{ x } = \BP \tensor_{S^0[a_{2\ell}] } S^0 
\end{align*}
where we use \(\eps: S^0[a_{2\ell}] \to S^0\). It suffices to prove the
following lemma:

\begin{lemma}[?]

\(S^0[a_{2\ell}]\) and \(\psi\) can be made \(E_2\).

\end{lemma}

\begin{proof}[of lemma, in the case $\ell=1$]

Consider
\begin{align*}
S^0[a_2] = S^0 \oplus S^2 \oplus S^4 \oplus \cdots = \Free_{E_1}(S^4)
.\end{align*}
This turns out to be equal to \(\Sigma_+^\infty \Loop S^3\) using ???
(see comment). Now use \(S^4 = \Loop\HP^\infty\), so this is
\(\Sigma_+^\infty \Omega^2 \HP^\infty\), which has an \(E_2\) ring
structure.

There is a filtration
\begin{align*}
S^2 \to S^4\cong \HP^1 \to \HP^2 \to \HP^3 \to \cdot \to \HP^\infty
,\end{align*}
which yields a filtration
\begin{align*}
\Suspendpinf \Loop^2 S^4 = \Suspendpinf \Loop^2 \HP^1 \to \Suspendpinf \Loop^2 \HP^2 \to \cdots
.\end{align*}

One can try to produce maps out of each filtered pieced:

\begin{center}
\begin{tikzcd}
    {\Sigma_+^\infty \Omega^2 S^4} &&&& \BP \\
    \\
    {\Sigma_+^\infty \Omega^2 \HP^2} \\
    \\
    {\Sigma_+^\infty \Omega^2 \HP^3} \\
    \vdots \\
    {\Sigma_+^\infty \Omega^2 S^3}
    \arrow[from=5-1, to=6-1]
    \arrow[from=6-1, to=7-1]
    \arrow[from=1-1, to=1-5]
    \arrow[from=1-1, to=3-1]
    \arrow[from=3-1, to=5-1]
    \arrow[curve={height=30pt}, dashed, from=3-1, to=1-5]
    \arrow[curve={height=24pt}, dashed, from=5-1, to=1-5]
    \arrow[curve={height=30pt}, dashed, from=7-1, to=1-5]
\end{tikzcd}
\end{center}

\begin{quote}
\href{https://q.uiver.app/?q=WzAsNixbMCwwLCJcXFNpZ21hXyteXFxpbmZ0eSBcXE9tZWdhXjIgU140Il0sWzAsMiwiXFxTaWdtYV8rXlxcaW5mdHkgXFxPbWVnYV4yIFxcSFBeMiJdLFswLDQsIlxcU2lnbWFfK15cXGluZnR5IFxcT21lZ2FeMiBcXEhQXjMiXSxbMCw2LCJcXFNpZ21hXyteXFxpbmZ0eSBcXE9tZWdhXjIgU14zIl0sWzAsNSwiXFx2ZG90cyJdLFs0LDAsIlxcQlAiXSxbMiw0XSxbNCwzXSxbMCw1XSxbMCwxXSxbMSwyXSxbMSw1LCIiLDIseyJjdXJ2ZSI6NSwic3R5bGUiOnsiYm9keSI6eyJuYW1lIjoiZGFzaGVkIn19fV0sWzIsNSwiIiwyLHsiY3VydmUiOjQsInN0eWxlIjp7ImJvZHkiOnsibmFtZSI6ImRhc2hlZCJ9fX1dLFszLDUsIiIsMix7ImN1cnZlIjo1LCJzdHlsZSI6eyJib2R5Ijp7Im5hbWUiOiJkYXNoZWQifX19XV0=}{Link
to Diagram}
\end{quote}

At each stage, the obstruction to lifting is a map out of a free \(E_2\)
algebra on an odd degree class in \(\pi_* \BP\), which is concentrated
in even degrees.

\begin{quote}
Find comment.
\end{quote}

\end{proof}

\end{proof}

\begin{remark}

So these are free as \(E_1\) rings, and are secretly \(E_2\) rings
(although not free \(E_2\) rings) which have a simple presentation that
makes them easy to map into objects with even-degree homotopy.

\begin{quote}
Find comment on first obstruction.
\end{quote}

\end{remark}

\begin{remark}

Proving a relativey easy proof that introduces a new technique used to
show that \(\BP \gens{ n }\) can be made \(E_3\).

\end{remark}

\begin{theorem}[?]

Connective \(K(n)\) exists as an \(E_1\dash\SS\) algebra.

\end{theorem}

\begin{remark}

We hve \(\pi_* K(n) = \FF_p[v_n]\) where \(\abs{v_n} = 2p^n - 2\) and
\(K(n) = \BP / \gens{ ? }\). There is a Postnikov tower:

\begin{center}
\begin{tikzcd}
    &&&& {k(n)} \\
    &&&& \vdots \\
    \\
    {v_n^2} & {\Sigma^{4p^n-4} \FF_p} &&& {\tau_{\leq 4p^n-4} k(n)} \\
    \\
    {v_n} & {\Sigma^{2p^n-2} \FF_p} &&& {\tau_{\leq 2p^n-2} k(n)} \\
    \\
    &&&& {\FF_p} && {\Sigma^{2p^n-2}\FF_p} && {\Sigma^{4p^n}}
    \arrow[from=1-5, to=2-5]
    \arrow[from=2-5, to=4-5]
    \arrow["{Q_n}", from=8-7, to=8-9]
    \arrow["{Q_n}", from=8-5, to=8-7]
    \arrow[from=6-5, to=8-5]
    \arrow[from=4-5, to=6-5]
    \arrow[from=4-2, to=4-5]
    \arrow[from=6-2, to=6-5]
\end{tikzcd}
\end{center}

\begin{quote}
\href{https://q.uiver.app/?q=WzAsMTEsWzEsMywiXFxTaWdtYV57NHBebi00fSBcXEZGX3AiXSxbMSw1LCJcXFNpZ21hXnsycF5uLTJ9IFxcRkZfcCJdLFs0LDUsIlxcdGF1X3tcXGxlcSAycF5uLTJ9IGsobikiXSxbNCwzLCJcXHRhdV97XFxsZXEgNHBebi00fSBrKG4pIl0sWzQsMCwiayhuKSJdLFs0LDEsIlxcdmRvdHMiXSxbNCw3LCJcXEZGX3AiXSxbNiw3LCJcXFNpZ21hXnsycF5uLTJ9XFxGRl9wIl0sWzgsNywiXFxTaWdtYV57NHBebn0iXSxbMCwzLCJ2X25eMiJdLFswLDUsInZfbiJdLFs0LDVdLFs1LDNdLFs3LDgsIlFfbiJdLFs2LDcsIlFfbiJdLFsyLDZdLFszLDJdLFswLDNdLFsxLDJdXQ==}{Link
to Diagram}
\end{quote}

\begin{itemize}
\item
  To build \(\tau_{2p^n-2} k(n)\), one just needs to identify
  \(Q_n \in \pi_0 \Hom(\FF_p, \Sigma^{2p^n - 1} \FF_p)\). So one needs
  to identify \(Q_n \in \pi_* \Hom(\FF_p, \FF_p)\), the
  \(\mathrm{mod} p\) Steenrod algebra.
\item
  To build \(\tau_{\leq 4p^n - 4}k(n)\), one needs to check that
  \(Q_n^2 = 0\) in the \(\mod p\) Steenrod algebra, which is an Adem
  relation.
\end{itemize}

Note that understanding \(\pi_* \Hom(\FF_p, \FF_p)\) as a group lets on
build \(\tau_{\leq 2p^n - 2}k(n)\)., but the next stage requires knowing
this is a \emph{ring} along with the Adem relation. Since
\(\Hom(\FF_p, \FF_p)\) is an \(E_1\) ring, and understanding this ring
structure would allow building \(k(n)\) completely as a spectrum. Here
\(\Hom(\FF_p, \FF_p)\) parameterizes all 2-stage Postnikov towers in the
sense that its homotopy groups record this data.

\end{remark}

\begin{remark}

How to build \(k(n)\) as an \(E_1\) ring instead of a spectrum:

\begin{itemize}
\tightlist
\item
  Write down the object parameterizing two-stage Postnikov towers in the
  category of \(E_1\) rings. This is well-known to be the
  \textbf{\(E_1\dash\) center} \(\mathcal{Z}_{E_1}(\FF_p)\), also known
  as \(\THC(\FF_p)\), the topological Hochschild cohomology of
  \(\FF_p\). This is known to be an \(E_2\) ring and if one understands
  its \(E_2\) structure well, one learns that \(E_2\) rings are more
  complicated than 2-stage Postnikov towers.
\end{itemize}

\end{remark}

\begin{remark}

Bokstedt proved that \(\pi_* \THC(\FF_p)\) is concentrated in even
degrees. Thus given any class \(x_{2\ell}\in \pi_{2\ell} \THC(\FF_p)\)
parameterizing some 2-stage \(E_1\) ring, by the previous theorem there
is an \(E_2\) ring map
\begin{align*}
S^0[a_{2\ell}] \to \THC(\FF_p)
.\end{align*}

\end{remark}




\end{document}
