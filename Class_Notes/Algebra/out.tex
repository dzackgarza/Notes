%\RequirePackage[l2tabu, orthodox]{nag}

%\documentclass[]{article}
\documentclass[11pt, a4paper, bibliography=totoc, parskip=full]{scrartcl}
% Remove indentation for new paragraphs
%\usepackage{parskip}
% But leave space before amsthm environments
\makeatletter
\def\thm@space@setup{%
  \thm@preskip=0em
  \thm@postskip=0emem
}
\makeatother
\setkomafont{subsection}{\normalfont\itshape}
\setkomafont{author}{\fontfamily{qzc}\selectfont\LARGE}
% Encoding
\usepackage[utf8]{inputenc}
\usepackage[T1]{fontenc}
\usepackage[usename, dvipsnames]{xcolor}
\usepackage{datetime}
\usepackage{biblatex}
\usepackage{spectralsequences}
% Fancy symbols
\usepackage{adforn}
\usepackage{pgfornament}
% Make figures stay where you put them!
\usepackage{float}
\floatplacement{figure}{H}

\usepackage[autooneside=false,automark]{scrlayer-scrpage}
%\clearscrheadfoot{}
\clearpairofpagestyles
\renewcommand*{\sectionmarkformat}{}
\KOMAoption{headwidth}{198mm:-9em} % 
\KOMAoption{headheight}{50mm} % 
\KOMAoption{footwidth}{159mm:0em} % 
\KOMAoption{headsepline}{1.1pt} % 
\KOMAoption{footsepline}{0.5pt} % 
\ihead[\clearscrheadfoot]{\mbox{
    \sffamily\bfseries\smash{%
    \setlength\fboxsep{0pt}\raisebox{-2pt}{
        \colorbox{gray!80}{\makebox[33.7mm]{\hfill%
        \ifnum\value{section}>0
            \textcolor{white}{\fontsize{18}{19}\selectfont
            \thesection }~
        \fi%
    \rule[-2.85pt]{2mm}{13mm}}}}}
}}
\chead{\hspace{33.7mm}\leftmark}
\cfoot*{\scriptsize \rightmark \hfill \pagemark}



% List of Theorems Attempt
\usepackage{etoolbox}
\makeatletter
\patchcmd\thmtlo@chaptervspacehack
  {\addtocontents{loe}{\protect\addvspace{10\p@}}}
  {\addtocontents{loe}{\protect\thmlopatch@endchapter\protect\thmlopatch@chapter{\thechapter}}}
  {}{}
\AtEndDocument{\addtocontents{loe}{\protect\thmlopatch@endchapter}}
\long\def\thmlopatch@chapter#1#2\thmlopatch@endchapter{%
  \setbox\z@=\vbox{#2}%
  \ifdim\ht\z@>\z@
    \hbox{\bfseries\chaptername\ #1}\nobreak
    #2
    \addvspace{10\p@}
  \fi
}
\def\thmlopatch@endchapter{}

\def\ll@tdefn{%
   \protect\thmtopatch@numbernametext
     \ifx\@empty\thmt@shortoptarg\else[\thmt@shortoptarg]\fi
     {\csname the\thmt@envname\endcsname}%
     {\thmt@thmname}%
}

\newcommand\thmtopatch@numbernametext[3][]{%
  #2 #3%
  \if\relax\detokenize{#1}\relax\else\space -- #1\fi
}

\makeatother
\renewcommand{\thmtformatoptarg}[1]{ -- #1}
%\renewcommand{\listtheoremname}{List of definitions}

%\usepackage{adjustbox}

% Custom section headings: redefining the KOMA-Script command \sectionlinesformat:
\newdimen\thesectionwdmax
\newdimen\sectiontitlewd
\AtBeginDocument{%
  \setbox0=\hbox{{\usekomafont{disposition}\usekomafont{section}\scalebox{2}{99}}}%
  \thesectionwdmax\wd0%
  \sectiontitlewd\dimexpr\textwidth-\thesectionwdmax-.75cm-1pt\relax
}

\renewcommand\sectionformat{\protect\scalebox{2}{\thesection}}
\addtokomafont{section}{\huge}



%\bibliography{/home/zack/Notes/library.bib}

\usepackage{stackrel}
\usepackage{array}
\usepackage{mathtools}
\usepackage{amsmath, amsthm, amssymb, amsfonts, amsxtra, amscd, thmtools, xpatch}
\usepackage{calligra, mathrsfs}
\usepackage{colonequals}
\let\proof\relax
\let\endproof\relax


\usepackage{color}
%\usepackage{unicode-math}
\usepackage{newunicodechar}
\newunicodechar{ε}{\varepsilon}
\newunicodechar{δ}{\delta}
\newunicodechar{≤}{\leq}
\newunicodechar{∈}{\in}
\newunicodechar{⊆}{\subseteq}
\newunicodechar{Λ}{\Lambda}
\newunicodechar{∞}{\infty}

\usepackage{centernot}
% \centernot\whatever

% Nice math font that journals use
%\usepackage[lite]{mtpro2}
%\usepackage{mathrsfs}
\usepackage{lmodern}

% Tikz 
\usepackage{tikz}
\usetikzlibrary{arrows.meta, cd, fadings, patterns, calc, matrix, positioning, decorations, decorations.pathreplacing, decorations.markings, shapes, backgrounds, fit, shapes.geometric, intersections}
\tikzfading[name=fade out, inner color=transparent!0, outer color=transparent!100]
\usepackage{tikz-layers}
\usepackage{ifthen}
\usepackage{xifthen}
\usepackage{tikz-3dplot}

% Major Macros
\usepackage{graphicx}

\DeclareFontFamily{U}{mathx}{\hyphenchar\font45}
\DeclareFontShape{U}{mathx}{m}{n}{
      <5> <6> <7> <8> <9> <10>
      <10.95> <12> <14.4> <17.28> <20.74> <24.88>
      mathx10
      }{}
\DeclareSymbolFont{mathx}{U}{mathx}{m}{n}
%\DeclareMathSymbol{\bigtimes}{1}{mathx}{"91}


% MnSymbol. Undefines lhook for some reason.
\let\originallhook\lhook
\usepackage{MnSymbol}% More dots
\let\lhook\originallhook
% \udots: SW to NE



% Optional arguments in macro definitions
\usepackage{xargs}
%\usepackage{stix2}
%\usepackage{libertineotf}

%\usepackage[msc-links]{amsrefs}
%\usepackage[style=authoryear]{biblatex}



\usepackage{svg}
\input{/home/zack/Notes/Latex/tikzmacros.tex}


\usepackage{cancel}


% Grow parentheses appropriately
% Better indicator function
\usepackage{bbm}
\newcommand{\indic}[1]{\mathbbm{1}\left[#1\right]}
\newcommand{\bbone}[0]{\mathbbm{1}}
% \mapsfrom
\usepackage{stmaryrd}
% Better delimiters
%\DeclarePairedDelimiter\qty{(}{)}
  \newcommand{\qty}[1]{\left( {#1} \right)}
\DeclarePairedDelimiter\intvl{[}{]}
% Big Asterisk
\newcommand{\Conv}{\mathop{\scalebox{1.5}{\raisebox{-0.2ex}{$\ast$}}}}%
% No more phi please
\let\origphi\phi
\let\phi\varphi
% Include emoji
\usepackage{apple_emoji}
\newcommand{\done}[0]{\scalebox{0.75}{✨}}
\newcommand{\work}[0]{\scalebox{0.75}{🚩}}
% Everything else
%\newcommand{\abs}[2]{{\left\lvert {#2} \right\rvert_{\text{#1}}}}
%\newcommand{\char}[0]{\text{char}}
%\newcommand{\hom}[0]{\text{Hom}}
%\newcommand{\suchthat}[0]{{~\backepsilon ~}}
%\newcommand{\vector}[1]{{\mathbf{ {#1} }}}
\def\Endo{\operatorname{End}}
\def\Ind{\operatorname{Ind}}
\def\Res{\operatorname{Res}}
\def\endo{\operatorname{End}}
\def\ind{\operatorname{Ind}}
\def\res{\operatorname{Res}}
\newcommand{\Af}[0]{{\mathbb{A}}}
\newcommand{\Arg}[0]{\mathrm{Arg}}
\newcommand{\Aut}[0]{{\text{Aut}}}
\newcommand{\CC}[0]{{\mathbb{C}}}
\newcommand{\CP}[0]{{\mathbb{CP}}}
\newcommand{\DD}[0]{{\mathbb{D}}}
\newcommand{\FF}[0]{{\mathbb{F}}}
\newcommand{\Fun}[0]{{\text{Fun}}}
\newcommand{\GF}[0]{{\mathbb{GF}}}
\newcommand{\GG}[0]{{\mathbb{G}}}
\newcommand{\GL}[0]{\mathrm{GL}}
\newcommand{\Gl}[0]{\mathrm{GL}}
\newcommand{\gl}[0]{\mathrm{GL}}
\newcommand{\Gal}[0]{\mathrm{Gal}}
\newcommand{\Gr}[0]{{\text{Gr}}}
\newcommand{\HH}[0]{{\mathbb{H}}}
\newcommand{\HP}[0]{{\mathbb{HP}}}
\newcommand{\Hilb}[0]{mathrm{Hilb}}
\newcommand{\KK}[0]{{\mathbb{K}}}
\newcommand{\MM}[0]{{\mathcal{M}}}
\newcommand{\NN}[0]{{\mathbb{N}}}
\newcommand{\OO}[0]{{\mathcal{O}}}
\newcommand{\OP}[0]{{\mathbb{OP}}}
\newcommand{\PP}[0]{{\mathbb{P}}}
\newcommand{\Pic}[0]{{\mathrm{Pic}~}}
\newcommand{\QQ}[0]{{\mathbb{Q}}}
\newcommand{\Qp}[0]{\mathbb{Q}_{(p)}}
\newcommand{\RP}[0]{{\mathbb{RP}}}
\newcommand{\RR}[0]{{\mathbb{R}}}
\newcommand{\SL}[0]{{\text{SL}}}
\newcommand{\SO}[0]{{\text{SO}}}
\newcommand{\SP}[0]{{\text{SP}}}
\newcommand{\Sch}[0]{{\text{Sch}}}
\newcommand{\Sets}[0]{{\text{Sets}}}
\newcommand{\Set}[0]{{\text{Sets}}}
\newcommand{\Sm}[0]{{\text{Sm}_k}}
\newcommand{\Sp}[0]{{\mathbb{S}}}
\newcommand{\TT}[0]{{\mathbb{T}}}
\newcommand{\Tor}[0]{\text{Tor}}
\newcommand{\Tr}[0]{\mathrm{Tr}}
\newcommand{\ZZ}[0]{{\mathbb{Z}}}
\newcommand{\ZnZ}[0]{\mathbb{Z}/n\mathbb{Z}}
\newcommand{\ZpZ}[0]{\mathbb{Z}/p\mathbb{Z}}
\newcommand{\Zp}[0]{\mathbb{Z}_{(p)}}
\newcommand{\abs}[1]{{\left\lvert {#1} \right\rvert}}
\newcommand{\actsonl}[0]{\curvearrowleft}
\newcommand{\actson}[0]{\curvearrowright}
\newcommand{\adj}[0]{\mathrm{adj}}
\newcommand{\ad}[0]{\mathrm{ad}~}
\newcommand{\annd}[0]{{\text{ and }}}
\newcommand{\ann}[0]{\mathrm{Ann}}
\newcommand{\arcsec}[0]{\mathrm{arcsec}}
\newcommand{\aut}[0]{\mathrm{Aut}}
\newcommand{\bd}[0]{{\del}}
\newcommand{\bigast}[0]{{\mathop{\Large \ast}}}
\newcommand{\bung}[0]{\text{Bun}_G}
\newcommand{\ch}[0]{\mathrm{char}~}
\newcommand{\coker}[0]{\operatorname{coker}}
\newcommand{\cok}[0]{\operatorname{coker}}
\newcommand{\conjugate}[1]{{\overline{{#1}}}}
\newcommand{\const}[0]{{\text{const.}}}
\newcommand{\converges}[1]{\overset{#1}}
\newcommand{\correspond}[1]{\theset{\substack{#1}}}
\newcommand{\covers}[0]{\twoheadrightarrow}
\newcommand{\crit}[0]{\mathrm{crit}}
\newcommand{\cross}[0]{\times}
\newcommand{\dash}[0]{{\hbox{-}}}
\newcommand{\dd}[2]{{\frac{\partial #1}{\partial #2}}}
\newcommand{\definedas}[0]{\coloneqq}
\newcommand{\del}[0]{{\partial}}
\newcommand{\diam}[0]{{\mathrm{diam}}}
\newcommand{\directlim}[0]{\varinjlim}
\newcommand{\disjoint}[0]{{\coprod}}
\newcommand{\divides}[0]{{~\Bigm|~}}
\newcommand{\dual}[0]{^\vee}
\newcommand{\eps}[0]{\varepsilon}
\newcommand{\equalsbecause}[1]{{\stackrel{\mbox{$\tiny{\text{ #1 }}$}}{=}}}
\newcommand{\ext}[0]{\text{Ext}}
\newcommand{\floor}[1]{{\left\lfloor #1 \right\rfloor}}
\newcommand{\from}[0]{\leftarrow}
\newcommand{\gal}[0]{\mathrm{Gal}}
\newcommand{\generators}[1]{\left\langle{#1}\right\rangle}
\newcommand{\grad}[0]{\mathrm{grad}}
\newcommand{\hilb}[0]{mathrm{Hilb}}
\newcommand{\homotopic}[0]{\simeq}
\newcommand{\id}[0]{\text{id}}
\newcommand{\imaginarypart}[1]{{\mathcal{Im}({#1})}}
\newcommand{\im}[0]{{\text{im}~}}
\newcommand{\indicator}[1]{{\unicode{x1D7D9}\left[#1\right]}}
\newcommand{\injectivelim}[0]{\varinjlim}
\newcommand{\injects}[0]{\hookrightarrow}
\newcommand{\inner}[2]{{\left\langle {#1},~{#2} \right\rangle}}
\newcommand{\intersect}[0]{\bigcap}
\newcommand{\into}[0]{\to}
\newcommand{\inverselim}[0]{\varprojlim}
\newcommand{\inv}[0]{^{-1}}
\newcommand{\lcm}[0]{\mathrm{lcm}}
\newcommand{\lieb}[0]{{\mathfrak{b}}}
\newcommand{\liegl}[0]{{\mathfrak{gl}}}
\newcommand{\lieg}[0]{{\mathfrak{g}}}
\newcommand{\lieh}[0]{{\mathfrak{h}}}
\newcommand{\lien}[0]{{\mathfrak{n}}}
\newcommand{\liesl}[0]{{\mathfrak{sl}}}
\newcommand{\lieso}[0]{{\mathfrak{so}}}
\newcommand{\liesp}[0]{{\mathfrak{sp}}}
\newcommand{\lieu}[0]{{\mathfrak{u}}}
\newcommand{\mapsvia}[1]{\xrightarrow{#1}}
\newcommand{\maps}[0]{\mathrm{Maps}}
\newcommand{\maxspec}[0]{{\mathrm{maxSpec}~}}
\newcommand{\mca}[0]{{\mathcal{A}}}
\newcommand{\mcb}[0]{{\mathcal{B}}}
\newcommand{\mcc}[0]{{\mathcal{C}}}
\newcommand{\mcd}[0]{{\mathcal{D}}}
\newcommand{\mce}[0]{{\mathcal{E}}}
\newcommand{\mcf}[0]{{\mathcal{F}}}
\newcommand{\mcg}[0]{{\mathcal{G}}}
\newcommand{\mch}[0]{{\mathcal{H}}}
\newcommand{\mci}[0]{{\mathcal{I}}}
\newcommand{\mcj}[0]{{\mathcal{J}}}
\newcommand{\mcl}[0]{{\mathcal{L}}}
\newcommand{\mcp}[0]{{\mathcal{P}}}
\newcommand{\mcs}[0]{{\mathcal{S}}}
\newcommand{\mcv}[0]{{\mathcal{V}}}
\newcommand{\mcx}[0]{{\mathcal{X}}}
\newcommand{\mcz}[0]{{\mathcal{Z}}}
\newcommand{\mfa}[0]{{\mathfrak{a}}}
\newcommand{\mfb}[0]{{\mathfrak{b}}}
\newcommand{\mfc}[0]{{\mathfrak{c}}}
\newcommand{\mfm}[0]{{\mathfrak{m}}}
\newcommand{\mfp}[0]{{\mathfrak{p}}}
\newcommand{\mfr}[0]{{\mathfrak{r}}}
\newcommand{\minpoly}[0]{{\mathrm{minpoly}}}
\newcommand{\mltext}[1]{\left\{\begin{array}{c}#1\end{array}\right\}}
\newcommand{\mm}[0]{{\mathfrak{m}}}
\newcommand{\multinomial}[1]{\left(\!\!{#1}\!\!\right)}
\newcommand{\nil}[0]{{\mathrm{nil}}}
\newcommand{\normalneq}{\mathrel{\reflectbox{$\trianglerightneq$}}}
\newcommand{\normal}[0]{{~\trianglelefteq~}}
\newcommand{\norm}[1]{{\left\lVert {#1} \right\rVert}}
\newcommand{\notdivides}[0]{\nmid}
\newcommand{\onto}[0]{\twoheadrightarrow}
\newcommand{\ord}[0]{{\mathrm{Ord}}}
\newcommand{\orr}[0]{{\text{ or }}}
\newcommand{\pic}[0]{{\mathrm{Pic}~}}
\newcommand{\projectivelim}[0]{\varprojlim}
\newcommand{\pr}[0]{{\mathfrak{p}}}
\newcommand{\pt}[0]{\{\text{pt}\}}
\newcommand{\rad}[0]{{\mathrm{rad}}}
\newcommand{\ralg}[0]{\mathrm{R-alg}}
\newcommand{\rank}[0]{\operatorname{rank}}
\newcommand{\realpart}[1]{{\mathcal{Re}({#1})}}
\newcommand{\reg}[0]{\mathrm{Reg}}
\newcommand{\restrictionof}[2]{{\left.{#1}\right|_{#2}}}
\newcommand{\rk}[0]{{\mathrm{rank}}}
\newcommand{\rmod}[0]{\mathrm{R-mod}}
\newcommand{\rotate}[2]{{\style{display: inline-block; transform: rotate(#1deg)}{#2}}}
\newcommand{\selfmap}[0]{{\circlearrowleft}}
\newcommand{\semidirect}[0]{\rtimes}
\newcommand{\sep}[0]{^\text{sep}}
\newcommand{\set}[0]{{\text{Sets}}}
\newcommand{\sgn}[0]{\mathrm{sgn}}
\newcommand{\sign}[0]{\mathrm{sign}}
\newcommand{\spanof}[0]{{\mathrm{span}}}
\newcommand{\spec}[0]{{\mathrm{Spec}}}
\newcommand{\stab}[0]{{\mathrm{Stab}}}
\newcommand{\stirlingfirst}[2]{\genfrac{[}{]}{0pt}{}{#1}{#2}}
\newcommand{\stirling}[2]{\genfrac\{\}{0pt}{}{#1}{#2}}
\newcommand{\strike}[1]{{\enclose{horizontalstrike}{#1}}}
\newcommand{\st}[0]{~{\text{s.t.}}~}
\newcommand{\suchthat}[0]{{~\mathrel{\Big|}~}}
\newcommand{\supp}[0]{{\mathrm{supp}}}
\newcommand{\surjects}[0]{\twoheadrightarrow}
\newcommand{\sym}[0]{\mathrm{Sym}}
\newcommand{\tensor}[0]{\otimes}
\newcommand{\theset}[1]{\left\{{#1}\right\}}
\newcommand{\thevector}[1]{{\left[ {#1} \right]}}
\newcommand{\too}[1]{{\xrightarrow{#1}}}
\newcommand{\tors}[0]{{\text{tors}}}
\newcommand{\tor}[0]{\text{Tor}}
\newcommand{\transverse}[0]{\pitchfork}
\newcommand{\trianglerightneq}{\mathrel{\ooalign{\raisebox{-0.5ex}{\reflectbox{\rotatebox{90}{$\nshortmid$}}}\cr$\triangleright$\cr}\mkern-3mu}}
\newcommand{\tr}[0]{\mathrm{Tr}}
\newcommand{\uniformlyconverges}[0]{\rightrightarrows}
\newcommand{\union}[0]{\bigcup}
\newcommand{\units}[0]{^{\times}}
\newcommand{\wait}[0]{{\,\cdot\,}}
\newcommand{\wt}[0]{{\mathrm{wt}}}
\renewcommand{\AA}[0]{{\mathbb{A}}}
\renewcommand{\bar}[1]{\overline{#1}}
\renewcommand{\div}[0]{\mathrm{Div}}
\renewcommand{\hat}[1]{\widehat{#1}}
\renewcommand{\mid}[0]{\mathrel{\Big|}}
\renewcommand{\qed}[0]{\hfill\blacksquare}
\renewcommand{\qty}[1]{{\left(  {#1} \right)}}
\renewcommand{\to}[0]{\longrightarrow}
\renewcommand{\vector}[1]{\mathbf{#1}}



\usepackage[noabbrev, capitalise, nameinlink]{cleveref} % use \cref{}, automatically deduces theorem, proposition, etc
\crefname{figure}{fig.}{figs.}
\Crefname{figure}{Fig.}{Figs.}
\crefname{equation}{equation}{eqns.}
\Crefname{equation}{Equation}{Eqns.}
\crefname{section}{section}{secs.}
\Crefname{section}{Section}{Secs.}


\let\Begin\begin
\let\End\end
\newcommand\wrapenv[1]{#1}

\makeatletter
\def\ScaleWidthIfNeeded{%
 \ifdim\Gin@nat@width>\linewidth
    \linewidth
  \else
    \Gin@nat@width
  \fi
}
\def\ScaleHeightIfNeeded{%
  \ifdim\Gin@nat@height>0.9\textheight
    0.9\textheight
  \else
    \Gin@nat@width
  \fi
}
\makeatother

\setkeys{Gin}{width=\ScaleWidthIfNeeded,height=\ScaleHeightIfNeeded,keepaspectratio}%

\title{
\textbf{
    Math 200A Homework Question Compendium
  }
  }
\author{D. Zack Garza}
\date{\today}

\begin{document}

\maketitle
% \todo{Insert title and subtitle.}
\tableofcontents


\hypertarget{one}{%
\section{One}\label{one}}

\begin{enumerate}
\def\labelenumi{\arabic{enumi}.}
\item
  \emph{Given}: \begin{align*}
  \forall x \in G, x^2 = e\end{align*}
   \emph{Show}: \begin{align*}
  G \in \mathbf{Ab}\end{align*}
\item
  \emph{Given}: \begin{align*}
  |G|<\infty, |G| = 0\mod 2\end{align*}
   \emph{Show}: \begin{align*}
  \exists g\in G \ni o(g) = 2\end{align*}
\item
  \emph{Given:} \begin{align*}
  G\in \mathbf{Ab}\end{align*}
   \emph{Show}: \begin{align*}
  T(G) \leq G\end{align*}
   (where \begin{align*}
  T(G) = \{ g\in G : |g| < \infty\}\end{align*}
\item
  \emph{Show}: Every finite group is finitely generated.

  \begin{itemize}
  \tightlist
  \item
    \emph{Show}: \begin{align*}
    \ZZ\end{align*}
     is finitely generated
  \item
    \emph{Show}: \begin{align*}
    H \leq (\QQ, +) \implies\end{align*}
     \begin{align*}
    H\end{align*}
     is cyclic
  \item
    \emph{Show}: \begin{align*}
    \QQ\end{align*}
     is not finitely generated
  \end{itemize}
\item
  \emph{Show}: \begin{align*}
  \QQ/\ZZ\end{align*}
   has, for each coset, exactly one representative in \begin{align*}
  [0, 1) \cap \QQ\end{align*}

  \begin{itemize}
  \tightlist
  \item
    \emph{Show}: Every element of \begin{align*}
    \QQ/\ZZ\end{align*}
     has finite order.
  \item
    \emph{Show}: There are elements in \begin{align*}
    \QQ/\ZZ\end{align*}
     of arbitrarily large order.
  \item
    \emph{Show}: \begin{align*}
    \QQ/\ZZ = T(\RR/\ZZ)\end{align*}
  \item
    \emph{Show}: \begin{align*}
    \QQ/\ZZ \cong \mathbb{C}^x\end{align*}
  \end{itemize}
\item
  \emph{Given}: \begin{align*}
  G/Z(G)\end{align*}
   is cyclic \emph{Show}: \begin{align*}
  G\end{align*}
   is abelian
\item
  \emph{Given}: \begin{align*}
  H \normal G, K \normal G,H \cap K = e\end{align*}
   \emph{Show}: \begin{align*}
  \forall h\in H, \forall k\in K, hk = kh\end{align*}
\item
  \emph{Given}: \begin{align*}
  |G| < \infty, \quad H \leq G, \quad N \normal G, (|H|, [G:N]) = 1\end{align*}
   \emph{Show}: \begin{align*}
  H \leq N\end{align*}
\item
  \emph{Given}: \begin{align*}
  |G| < \infty, N \normal G, (|N|, [G:N]) =1\end{align*}
   \emph{Show}: \begin{align*}
  N\end{align*}
   is the unique subgroup of order \begin{align*}
  |N|\end{align*}
\end{enumerate}

\hypertarget{two}{%
\section{Two}\label{two}}

\begin{enumerate}
\def\labelenumi{\arabic{enumi}.}
\item
  \emph{Given}: For every triplet in \begin{align*}
  G\end{align*}
  , two elements commute \emph{Show}: \begin{align*}
  G\end{align*}
   is abelian
\item
  \emph{Given}: \begin{align*}
  H_1, H_2, H_3 \leq G, G = H_1 \cup H_2\end{align*}
   \emph{Show}: \begin{align*}
  G=H_1 \vee G=H_2\end{align*}
\item
  \emph{Given}: \begin{align*}
  G=H_1 \cup H_2 \cup H_3, G\end{align*}
   finite \emph{Show}: \begin{align*}
  G=H_i \vee \forall i, [G:H_i] = 2\end{align*}
\item
  \emph{Show}: TFAE; \begin{align*}
  \text{clos}(H)\end{align*}
   is:

  \begin{itemize}
  \tightlist
  \item
    The smallest normal subgroup of \begin{align*}
    G\end{align*}
     containing \begin{align*}
    H\end{align*}
    .
  \item
    The subgroup generated by all conjugates of \begin{align*}
    H\end{align*}
    .
  \item
    \begin{align*}
    \bigcap_{H \leq N \normal G} N\end{align*}
  \item
    \begin{align*}
    \phi: G \rightarrow -\end{align*}
    , \begin{align*}
    \phi(H) = e\end{align*}
    , then \begin{align*}
    \phi\end{align*}
     factors through \begin{align*}
    G/\text{clos}(H)\end{align*}
  \end{itemize}
\item
  \emph{Given}: \begin{align*}
  H, K \normal HK \leq G\end{align*}
   \emph{Show}: \begin{align*}
   \frac{HK}{H\cap K} \cong \frac{HK}{H}\times \frac{HK}{K}\end{align*}
\item
  \emph{Given}: \begin{align*}
  H\leq G, N \normal G, H \in \text{Hall}(G)\end{align*}
   \emph{Show}: \begin{align*}
  H\cap N \in \text{Hall}(N) \wedge \frac{HN}{N} \in \text{Hall}(\frac{G}{N})\end{align*}
\item
  \emph{Given}: \begin{align*}
  |G| = n, G\end{align*}
   cyclic, \begin{align*}
  \sigma_i: G \rightarrow G \ni x \mapsto x^i\end{align*}

  \begin{itemize}
  \tightlist
  \item
    Show \begin{align*}
    \sigma_i \in End(G)\end{align*}
  \item
    Show \begin{align*}
    \sigma_i \in Aut(G)\end{align*}
     iff \begin{align*}
    (i, n) = 1\end{align*}
  \item
    \begin{align*}
    \sigma_i = \sigma_j\end{align*}
     iff \begin{align*}
    i=j\mod n\end{align*}
  \item
    \begin{align*}
    \tau \in Aut(G) \implies \exists i \ni \tau = \sigma_i\end{align*}
  \item
    \begin{align*}
    \sigma_i \circ \sigma_j = \sigma_{ij}\end{align*}
  \end{itemize}

  \begin{enumerate}
  \def\labelenumii{\arabic{enumii}.}
  \setcounter{enumii}{5}
  \tightlist
  \item
    The map \begin{align*}
    \psi: Z_n^\times \rightarrow Aut(G)\\ i \mapsto \sigma_i\end{align*}
     is an isomorphism.
  \end{enumerate}
\item
  \emph{Given}: \begin{align*}
  G\end{align*}
   is cyclic \emph{Show}: \begin{align*}
  Aut(G)\end{align*}
   is abelian of order \begin{align*}
  \phi(n)\end{align*}
\item
  \emph{Show}: \begin{align*}
  D_\infty \cong \langle a,b \mid b^2 = e, ba = a^{-1} b \rangle\end{align*}
\item
  \emph{Show}: \begin{align*}
  Q_8 \cong \langle a,b \mid a^2 = b^2, a^{-1}ba = b^{-1}\rangle\end{align*}
\item
  \emph{Show}: \begin{align*}
  \generators{x,y \mid xy^2 = y^3, yx^2 = x^3y} = \generators{e}\end{align*}
\end{enumerate}

\hypertarget{three}{%
\section{Three}\label{three}}

\begin{enumerate}
\def\labelenumi{\arabic{enumi}.}
\item
  \emph{Given}: \begin{align*}
  G \sim X\end{align*}
   transitively, \begin{align*}
  H \normal G\end{align*}

  \begin{itemize}
  \tightlist
  \item
    \emph{Show}: \begin{align*}
    H \sim X\end{align*}
    , but possibly not transitively
  \item
    \emph{Show}: \begin{align*}
    G\end{align*}
     acts transitively on \begin{align*}
    \theset{ \mathcal{O_h} : h\ \in H}\end{align*}
  \item
    \emph{Show}: \begin{align*}
    \forall i, j, |\mathcal{O}_{h_i}| = |\mathcal{O}_{h_j}|\end{align*}
  \item
    \emph{Given}: \begin{align*}
    x\in \mathcal{O}_h\end{align*}
     \emph{Show}: \begin{align*}
    |\mathcal{O}_h| = |H : H \cap G_x|\end{align*}
  \item
    \emph{Show}: \begin{align*}
    |\{\mathcal{O}_h\}_{h\in H}| = [G : HG_x]\end{align*}
  \end{itemize}
\item
  \emph{Given}: \begin{align*}
  \mathcal{K}\end{align*}
   a conjugacy class in \begin{align*}
  S_n\end{align*}
  , \begin{align*}
  \{\mathcal{O}_{s}:s\in S_n\}\end{align*}
   orbits of an \begin{align*}
  A_n\end{align*}
  -action on \begin{align*}
  S_n\end{align*}
   \emph{Show}: \begin{align*}
  \mathcal{K} = \mathcal{O}_s \vee \mathcal{K} = \mathcal{O}_{s_i} \cup \mathcal{O}_{s_j}\end{align*}
   \emph{Show}: Case 2 occurs iff \begin{align*}
  \{k_i\}\end{align*}
  , the cycle lengths in disjoint cycle form, are odd and distinct
\item
  i: \begin{align*}
  |G| < \infty, H < G\end{align*}

  \begin{itemize}
  \tightlist
  \item
    \emph{Show}: \begin{align*}
    \{ gHg^{-1} : g\in G\} = [G : N_G(H)]\end{align*}
  \item
    \emph{Show}: \begin{align*}
    G \neq \bigcup_{g\in G} gHg^{-1}\end{align*}
  \end{itemize}
\item
  Prove Cauchy's Theorem. Given \begin{align*}
  p\mid o(G) <\infty\end{align*}
   \begin{align*}

  X = \theset{ (a_{i})_{i=1}^p \in G^p \suchthat \prod_{i=1}^p a_i = e} \\
  asdsadas
  \end{align*}

  \begin{itemize}
  \tightlist
  \item
    \emph{Show}: \begin{align*}
    (a_1 a_2\cdots a_p) = e \implies (a_2 a_3 \cdots a_p a_1) = e\end{align*}
  \item
    \emph{Show}: \begin{align*}
    (Z_p, +) \sim X\end{align*}
     and \begin{align*}
    \bar 1 \sim (a_1 a_2 \cdots a_p) = (a_2 a_3 \cdots a_p a_1)\end{align*}
  \item
    \emph{Show}: \begin{align*}
    |X| = |G|^{p-1}\end{align*}
  \item
    \emph{Show}: \begin{align*}
    \{ \mathcal{O}_x : |\mathcal{O}_x| = 1 \} > 1\end{align*}
     and \begin{align*}
    \exists a \in G \ni a^p = e\end{align*}
  \end{itemize}
\item
  \emph{Given}: \begin{align*}
  G \sim X, \quad |G| < \infty , \quad 1 < |X| < \infty\end{align*}
\end{enumerate}

\begin{itemize}
\tightlist
\item
  \emph{Show}: \begin{align*}
  \exists g\ in G \ni \forall x\in X, g\sim x \neq x\end{align*}
\item
  \emph{Show}: This holds if \begin{align*}
  |G| = \infty\end{align*}
  , but not if \begin{align*}
  |X| = \infty\end{align*}
   as well.
\end{itemize}

\begin{enumerate}
\def\labelenumi{\arabic{enumi}.}
\setcounter{enumi}{5}
\item
  \emph{Given}: \begin{align*}
  H \leq G\end{align*}
  . \emph{Show}: \begin{align*}
  \text{core}(H)\end{align*}
   is

  \begin{itemize}
  \tightlist
  \item
    The largest \begin{align*}
    N \normal G, N \subseteq H\end{align*}
  \item
    Generated by all normal subgroups contained in \begin{align*}
    H\end{align*}
  \item
    Given by \begin{align*}
    \bigcap_{g\in G} gHg^{-1}\end{align*}
  \item
    The kernel of \begin{align*}
    G \sim \frac{G}{H} \ni x \sim gH = (xg)H\end{align*}
  \end{itemize}
\item
  \emph{Given}: \begin{align*}
  [H : G]= n < \infty\end{align*}

  \begin{itemize}
  \tightlist
  \item
    \emph{Show}: \begin{align*}
    [\text{core}(H) : G]\end{align*}
     divides \begin{align*}
    n!\end{align*}
  \item
    \emph{Show}: \begin{align*}
    G\end{align*}
     simple \begin{align*}
    \implies o(G) \mid n! \wedge |G| < \infty\end{align*}
  \end{itemize}
\item
  \emph{Given}: \begin{align*}
  A_n\end{align*}
   is simple for \begin{align*}
  n\geq 5\end{align*}
   \emph{Show}: \begin{align*}
  \not \exists H \in A_n \ni [H : A_n] < n\end{align*}
   \emph{Show}: \begin{align*}
  \exists H [H : A_n] = n\end{align*}
\item
  \emph{Given}: \begin{align*}
  r\end{align*}
   beads of \begin{align*}
  n\end{align*}
   colors \emph{Show}: How many distinct circular bracelets can be made.
\end{enumerate}

\hypertarget{four}{%
\section{Four}\label{four}}

\begin{enumerate}
\def\labelenumi{\arabic{enumi}.}
\item
  \emph{Given}: \begin{align*}
  H~\text{char}~G\end{align*}
   \emph{Show}: \begin{align*}
  H \normal G\end{align*}
\item
  \emph{Given}: \begin{align*}
  H ~\text{char}~ K \normal G\end{align*}
   \emph{Show}: \begin{align*}
  H \normal G\end{align*}
\item
  \emph{Given}: \begin{align*}
  K = \langle k \rangle \normal G\end{align*}
   \emph{Show}: \begin{align*}
  H \leq K \implies H \normal G\end{align*}
\item
  \emph{Show} \begin{align*}
  H \normal K \normal G \not\implies H \normal G\end{align*}
\item
  \emph{Given}: \begin{align*}
  P \leq H \leq K \leq G < \infty, P \in \text{Syl}_p(G)\end{align*}
   \emph{Show}: \begin{align*}
  P, H \normal K \implies P \normal K\end{align*}
\item
  \emph{Show}: \begin{align*}
  N_G(N_G(P)) = N_G(P)\end{align*}
\item
  \emph{Given}: \begin{align*}
  \sigma \in Aut(G)\end{align*}
   \emph{Show}: \begin{align*}
  \sigma Inn(G) \sigma^{-1} = Inn(G)\end{align*}
   iff \begin{align*}
  \forall g\in G, g^{-1}\sigma(g)\in Z(G)\end{align*}
\item
  \emph{Show}: \begin{align*}
  Inn(G) ~\text{char}~ Aut(G)\end{align*}
\item
  \emph{Given}: \begin{align*}
  H \subseteq G, P \in \text{Syl}_p(G)\end{align*}

  \begin{itemize}
  \tightlist
  \item
    \emph{Show}: \begin{align*}
    \exists g \in G \ni gPg^{-1} \in Syl_p(H)\end{align*}
  \item
    \emph{Given}: \begin{align*}
    H \normal G\end{align*}
     \emph{Show}: \begin{align*}
    P\cap H \in \text{Syl}_p(H)\end{align*}
  \item
    \emph{Given}: \begin{align*}
    P \normal G\end{align*}
     \emph{Show}: \begin{align*}
    P \cap H \in \text{Syl}_p(H)\end{align*}
     and \begin{align*}
    |\text{Syl}_p(H)| = 1\end{align*}
  \end{itemize}
\item
  \emph{Given}: \begin{align*}
  |G| = pqr, p < q < r\end{align*}
   \emph{Show}: \begin{align*}
  \exists P_i \in \text{Syl}_i(G) \normal G\end{align*}
\item
  \emph{Given}: \begin{align*}
  |G| =595\end{align*}
   \emph{Show}: All sylow subgroups are normal
\item
  \emph{Given}: \begin{align*}
  |G| = p(p+1)\end{align*}
   \emph{Show}: \begin{align*}
  \exists N \normal G\end{align*}
   where \begin{align*}
  |N| = p\end{align*}
   or \begin{align*}
  p+1\end{align*}
\end{enumerate}

\hypertarget{five}{%
\section{Five}\label{five}}

\begin{enumerate}
\def\labelenumi{\arabic{enumi}.}
\item
  Given: \begin{align*}
  G = H \semidirect_\psi K\end{align*}
   \begin{align*}
  \psi: K \rightarrow Aut(H) \\ k \mapsto \psi(k)\end{align*}
   \begin{align*}
  \theta \in Aut(H)\end{align*}
   \begin{align*}
  \rho: K \rightarrow K\end{align*}
   \begin{align*}
  \phi_\theta: Aut(H) \rightarrow Aut(H) \\ \rho \mapsto \theta \circ \rho \circ \theta^{-1}\end{align*}
   \begin{align*}
  \psi_2: K \rightarrow Aut(H) \\ k \mapsto (\phi_\theta \circ \psi)(k)\end{align*}
   \begin{align*}
  \psi_3: K \rightarrow Aut(H) \\ k \mapsto (\psi \circ \rho)(k)\end{align*}

  Show: \begin{align*}
  H \semidirect_\psi K \cong H \semidirect_{\psi_2} K \cong H \semidirect_{\psi_3} K\end{align*}
\item
  Classify groups of order \begin{align*}
  pq, p < q, p \mid q-1\end{align*}
\item
  Classify groups of order 20.
\item
  Classify groups of order 75.
\item
  Show: \begin{align*}
  |G| < 60 \implies G\end{align*}
   is not simple.
\item
  Show: \begin{align*}
  |G| < 60 \implies G\end{align*}
   is solvable
\item
  Given: \begin{align*}
  |G| < \infty\end{align*}
  , \begin{align*}
  H \leq G\end{align*}
   maximal \begin{align*}
  \implies [G:H] = p\end{align*}
  , a prime. Show: \begin{align*}
  |G|\end{align*}
   is solvable

  \begin{itemize}
  \tightlist
  \item
    Given: \begin{align*}
    P \in Syl_p(G) \wedge \exists H \ni N_G(P) \leq H \leq G\end{align*}
     Show: \begin{align*}
    [G:H] = 1 \mod p\end{align*}
  \item
    Given: \begin{align*}
    p \mid o(G)\end{align*}
    , the largest such prime Show: \begin{align*}
    \exists P \normal G \in Syl_p(G),\end{align*}
  \end{itemize}
\item
  \begin{align*}
  |G| < \infty\end{align*}

  \begin{itemize}
  \tightlist
  \item
    Given: \begin{align*}
    G\end{align*}
     is characteristically simple Show: \begin{align*}
    \exists H~(\text{simple})~ \ni G \cong H^n\end{align*}
    . Show: Whether or not the converse holds
  \item
    Given: \begin{align*}
    N \normal G\end{align*}
     minimal Show: \begin{align*}
    N\end{align*}
     is characteristically simple, \begin{align*}
    N \cong H^n\end{align*}
  \end{itemize}
\end{enumerate}

\hypertarget{six}{%
\section{Six}\label{six}}

\begin{enumerate}
\def\labelenumi{\arabic{enumi}.}
\item
  Given: \begin{align*}
  G\end{align*}
   is nilpotent Show: \begin{align*}
  H \leq G \implies H, G/H\end{align*}
   are nilpotent
\item
  Show: \begin{align*}
  G/Z(G)\end{align*}
   is nilpotent \begin{align*}
  \implies G\end{align*}
   is nilpotent
\item
  Given: \begin{align*}
  |G| < \infty\end{align*}
   Show: \begin{align*}
  |G|\end{align*}
   is nilpotent iff \begin{align*}
  a,b\in G, (a,b)=1 \implies ab=ba\end{align*}
\item
  Show: \begin{align*}
  D_{2n}\end{align*}
   is nilpotent iff \begin{align*}
  n = 2^{i}\end{align*}
\item
  Given: \begin{align*}
  |G| < \infty\end{align*}

  \begin{itemize}
  \tightlist
  \item
    Show \begin{align*}
    \Phi(G)~\text{char}~ G\end{align*}
  \item
    Show \begin{align*}
    \Phi(G)\end{align*}
     is nilpotent
  \item
    Given: \begin{align*}
    |P| = p^e\end{align*}
     Show: \begin{align*}
    P / \Phi(P)\end{align*}
     is an elementary abelian p-group Show:\begin{align*}
    N \normal P, P/N\end{align*}
     is elementary abelian \begin{align*}
    \implies \Phi(P) \subseteq N\end{align*}
  \end{itemize}
\item
  Given: \begin{align*}
  R\end{align*}
   a commutative ring, \begin{align*}
  x,y \in R\end{align*}
   nilpotent

  \begin{itemize}
  \tightlist
  \item
    Show: \begin{align*}
    x+y\end{align*}
     is nilpotent Show: \begin{align*}
    \{x\in R : x \text{ is nilpotent}\} \normal R\end{align*}
  \item
    Given: \begin{align*}
    u\in R^\times, x\in R\end{align*}
     nilpotent Show: \begin{align*}
    u+x\in R^\times\end{align*}
  \item
    Show: An counterexample to 1 when \begin{align*}
    R\end{align*}
     is noncommutative.
  \end{itemize}
\item
  Given: \begin{align*}
  R\end{align*}
   a commutative ring, \begin{align*}
  R[[x]]\end{align*}
   its formal power series

  \begin{itemize}
  \tightlist
  \item
    Show: \begin{align*}
    \sum_{i=0}^\infty a_i x^i \in R[[x]]^\times \iff a_0 \in R^\times\end{align*}
  \item
    Show: \begin{align*}
    R\end{align*}
     a domain \begin{align*}
    \implies R[[x]]\end{align*}
     a domain
  \item
    Given: \begin{align*}
    R\end{align*}
     a field Show: \begin{align*}
    I = \{ r \in R[[x]] : r_0 = 0\}\end{align*}
     is a maximal ideal of \begin{align*}
    R[[x]]\end{align*}
     Show: \begin{align*}
    I\end{align*}
     is the unique maximal ideal
  \end{itemize}
\item
  Given: \begin{align*}
  R\end{align*}
   a commutative ring, \begin{align*}
  G\end{align*}
   a finite group, \begin{align*}
  RG\end{align*}
   a group ring.

  \begin{itemize}
  \tightlist
  \item
    Given: \begin{align*}
    \mathcal{K} = \{ k_1, k_2, \cdots k_m\}\end{align*}
     a conjugacy class in \begin{align*}
    G\end{align*}
     Show: \begin{align*}
    K = \sum_{i=1}^m k_i \in RG \implies K \in Z(RG)\end{align*}
  \item
    Given: \begin{align*}
    \mathcal{K}_1\cdots \mathcal{K}_r\end{align*}
     distinct conjugacy classes in \begin{align*}
    G\end{align*}
    , \begin{align*}
    K_i = \sum_{j}k_j \ni k_j \in \mathcal{K}_i\end{align*}
     Show: \begin{align*}
    Z(RG) = \{\sum a_l K_l : \forall 1 \leq l \leq r, a_l \in R \}\end{align*}
     (All \begin{align*}
    R\end{align*}
    -linear combinations of the \begin{align*}
    \mathcal{K}_i\end{align*}
    )
  \end{itemize}
\item
  Given: \begin{align*}
  R\end{align*}
   a ring, \begin{align*}
  M_n(R)\end{align*}
   its matrix ring

  \begin{itemize}
  \tightlist
  \item
    Given: \begin{align*}
    I \normal R\end{align*}
     (two-sided) Show: \begin{align*}
    M_n(I) \normal M_n(R)\end{align*}
     Show: \begin{align*}
    \frac{M_n(R)}{M_n(I)} \cong M_n(\frac{R}{I})\end{align*}
  \item
    Show: \begin{align*}
    \forall I_M \normal M_n(R), I\end{align*}
     is of the form \begin{align*}
    M_n(I)\end{align*}
     for some \begin{align*}
    I \normal R\end{align*}
     Show: \begin{align*}
    R\end{align*}
     a division ring \begin{align*}
    \implies M_n(R)\end{align*}
     is a simple ring.
  \end{itemize}
\end{enumerate}

\hypertarget{seven}{%
\section{Seven}\label{seven}}

\hypertarget{eight}{%
\section{Eight}\label{eight}}

%\listoftodos

\bibliography{/home/zack/Notes/library.bib}

\end{document}
