\RequirePackage[l2tabu, orthodox]{nag}

%\documentclass[]{article}
\documentclass[11pt]{scrartcl}
\usepackage[usename, dvipsnames]{xcolor}
\usepackage[pdfencoding=auto]{hyperref}
\usepackage[msc-links]{amsrefs}
\usepackage{cleveref} % use \cref{}, automatically deduces theorem, proposition, etc
\usepackage[mathletters]{ucs}
\usepackage[utf8]{inputenc}
\usepackage[T1]{fontenc}
\usepackage{datetime}

\usepackage{array}
\usepackage{mathtools}
\usepackage{amsmath, amsthm, amssymb, amsfonts, amsxtra, amscd, thmtools}
\let\proof\relax
\let\endproof\relax

% Boxes around theorem environments.
\usepackage[many]{tcolorbox}

\usepackage{color}
%\usepackage{unicode-math}
\usepackage{newunicodechar}
\newunicodechar{ε}{\varepsilon}
\newunicodechar{δ}{\delta}
\newunicodechar{µ}{\mu}
\newunicodechar{→}{\to}
\newunicodechar{≤}{\leq}
\newunicodechar{∈}{\in}
\newunicodechar{⊆}{\subseteq}
\newunicodechar{Λ}{\Lambda}
\newunicodechar{∞}{\infty}
\newunicodechar{×}{\times}
\everymath{\displaystyle}



\usepackage{microtype}
\usepackage[pdfencoding=auto]{hyperref}
\usepackage{bookmark}
\usepackage{booktabs}
\usepackage{todonotes}
\usepackage[msc-links]{amsrefs}
\usepackage{cleveref} % use \cref{}, automatically deduces theorem, proposition, etc
\usepackage{csquotes}
\usepackage{longtable}
\usepackage{tabularx}
\usepackage{bbm}
% Creating multiple types of index
\usepackage{imakeidx}

% Remove indentation for new paragraphs
\usepackage{parskip}
% But leave space before amsthm environments
\makeatletter
\def\thm@space@setup{%
  \thm@preskip=2em
  \thm@postskip=2em
}
\makeatother


\usepackage{stmaryrd}
\usepackage{adjustbox}
\usepackage{centernot}
% \centernot\whatever


% Better indicator function
\usepackage{bbm}
\newcommand{\indic}[1]{\mathbbm{1} \left[ {#1} \right] }

% Highlight quote
\usepackage{environ}
\definecolor{camel}{rgb}{0.76, 0.6, 0.42}
\definecolor{babyblue}{rgb}{0.54, 0.81, 0.94}
\definecolor{block-gray}{gray}{0.85}
\NewEnviron{myblock}
{\colorbox{block-gray}{%
\parbox{\dimexpr\linewidth-2\fboxsep\relax}{%
\small\addtolength{\leftskip}{10mm}
\addtolength{\rightskip}{10mm}
\BODY}}
}
\renewcommand{\quote}{\myblock}
\renewcommand{\endquote}{\endmyblock}

% Nice math font that journals use
%\usepackage[lite]{mtpro2}
%\usepackage{mathrsfs}
%\usepackage{mathptmx}
\usepackage{lmodern}
%\usepackage[sc]{mathpazo}

% Theorem Styles
\usepackage[framemethod=tikz]{mdframed}

\theoremstyle{definition}
\newtheorem{exercise}{Exercise}[section]
\newtheorem{solution}{Solution}

% Theorem Style
\newtheoremstyle{theorem}% name
  {0em}%         Space above, empty = `usual value'
  {1em}%         Space below
  {\normalfont}% Body font
  {\parindent}%         Indent amount (empty = no indent, \parindent = para indent)
  {\bfseries}% Thm head font
  {.}%        Punctuation after thm head
  {\newline}% Space after thm head: \newline = linebreak
  {\thmname{#1}\thmnumber{ #2}\thmnote{\itshape{(#3)}}}%
\theoremstyle{theorem}
\tcolorboxenvironment{theorem}{
  boxrule=0pt,
  boxsep=0pt,
  breakable,
  enhanced jigsaw,
  fonttitle={\large\bfseries},
  opacityback=0.8,
  colframe=cyan,
  borderline west={4pt}{0pt}{orange},
  attach title to upper={}
}
\newtheorem{theorem}{Theorem}[section]

% Proposition Style
\tcolorboxenvironment{proposition}{
  boxrule=1pt,
  boxsep=0pt,
  breakable,
  enhanced jigsaw,
  opacityback=0.0,
  colframe=cyan
}
\newtheorem{proposition}[theorem]{Proposition}
\tcolorboxenvironment{lemma}{
  boxrule=1pt,
  boxsep=0pt,
  breakable,
  enhanced jigsaw,
  opacityback=0.2,
  colframe=cyan
}
\newtheorem{lemma}[theorem]{Lemma}
% Claim
\tcolorboxenvironment{claim}{
  boxrule=1pt,
  boxsep=0pt,
  breakable,
  enhanced jigsaw,
  opacityback=0.2,
  colframe=cyan
}
\newtheorem{claim}[theorem]{Claim}


% Corollary
\tcolorboxenvironment{corollary}{
  colback=cyan,
  boxrule=1pt,
  boxsep=0pt,
  breakable,
  enhanced jigsaw,
  opacityback=0.1,
  colframe=cyan
}
\newtheorem{corollary}[theorem]{Corollary}

% Proof Style
\newtheoremstyle{proof}% name
  {0em}%         Space above, empty = `usual value'
  {2em}%         Space below
  {\normalfont}% Body font
  {\parindent}%         Indent amount (empty = no indent, \parindent = para indent)
  {\itshape}% Thm head font
  {.}%        Punctuation after thm head
  {\newline}% Space after thm head: \newline = linebreak
  {\thmname{#1} \thmnote{\itshape{(#3)}}}%         Thm head spec
\theoremstyle{proof}
\tcolorboxenvironment{proof}{
  colback=camel,
  opacityfill=0.25,
  boxrule=1pt,
  boxsep=0pt,
  breakable,
  enhanced jigsaw
}
\newtheorem*{pf}{Proof}
\newenvironment{proof}
{\pushQED{$\qed$}\pf}
{\par\popQED\endpf}

% Definition Style
\newtheoremstyle{definition}% name
  {0em}%         Space above, empty = `usual value'
  {2em}%         Space below
  {\normalfont}% Body font
  {\parindent}%         Indent amount (empty = no indent, \parindent = para indent)
  {\bfseries}% Thm head font
  {.}%        Punctuation after thm head
  {\newline}% Space after thm head: \newline = linebreak
  {}%         Thm head spec
\theoremstyle{definition}
\tcolorboxenvironment{definition}{
  colback=babyblue,
  boxrule=0pt,
  boxsep=0pt,
  opacityfill=0.45,
  breakable,
  enhanced jigsaw,
  borderline west={4pt}{0pt}{blue},
  colbacktitle={babyblue},
  coltitle={black},
  fonttitle={\large\bfseries},
  attach title to upper={},
}
\newtheorem{definition}{Definition}[theorem]

% Break Environment
\makeatletter
\newtheoremstyle{break}% name
  {}%         Space above, empty = `usual value'
  {2em}%         Space below
  {
    \addtolength{\@totalleftmargin}{2.5em}
    \addtolength{\linewidth}{-2.5em}
    \parshape 1 2.5em \linewidth
  }% Body font
  {}%         Indent amount (empty = no indent, \parindent = para indent)
  {\bfseries}% Thm head font
  {.}%        Punctuation after thm head
  {\newline}% Space after thm head: \newline = linebreak
  {}%         Thm head spec
\makeatother

\theoremstyle{break}
\newtheorem{example}{Example}[section]

% Problem Style
\newtheoremstyle{problem} % name
  {0em}                   % Space above, empty = `usual value'
  {2em}                   % Space below
  {\normalfont}           % Body font
  {\parindent}            % Indent amount (empty = no indent, \parindent = para indent)
  {\itshape}              % Thm head font
  {}                      % Punctuation after thm head
  {\newline}              % Space after thm head: \newline = linebreak
  {\thmnote{\itshape{(#3)}}}     % Thm head spec
\theoremstyle{problem}
\tcolorboxenvironment{problem}{
  boxrule=1pt,
  boxsep=0pt,
  breakable,
  enhanced jigsaw,
  opacityback=0.0,
  colframe=cyan
}
\newtheorem{problem}{Problem}


%Pagination stuff.
\setlength{\topmargin}{-.3 in}
\setlength{\oddsidemargin}{0in}
\setlength{\evensidemargin}{0in}
\setlength{\textheight}{9.in}
\setlength{\textwidth}{6.5in}
% \pagestyle{empty} %removes page numbers.

% Inkscape figures from Vim
\usepackage{import}
\usepackage{pdfpages}
\usepackage{transparent}

\newcommand{\incfig}[1]{%
    \def\svgwidth{\columnwidth}
    \import{./figures/}{#1.pdf_tex}
}
%\pdfsuppresswarningpagegroup=1

% Pandoc-specific fixes
\providecommand{\tightlist}{%
  \setlength{\itemsep}{0pt}\setlength{\parskip}{0pt}}

% Tikz and Graphics
\usepackage{amscd}
\usepackage{tikz}
\usetikzlibrary{arrows, arrows.meta, cd, fadings, patterns, calc, decorations.markings, matrix, positioning}
\tikzfading[name=fade out, inner color=transparent!0, outer color=transparent!100]
\usepackage{pgfplots}
\pgfplotsset{compat=1.16}
\usepackage[inline]{asymptote}
\usepackage{tikz-layers}

%\usepackage{nath}
%\delimgrowth=1
\DeclarePairedDelimiter\qty{(}{)}

% Major Macros
\usepackage{graphicx}
\usepackage{float}
\DeclareFontFamily{U}{mathx}{\hyphenchar\font45}
\DeclareFontShape{U}{mathx}{m}{n}{
      <5> <6> <7> <8> <9> <10>
      <10.95> <12> <14.4> <17.28> <20.74> <24.88>
      mathx10
      }{}
\DeclareSymbolFont{mathx}{U}{mathx}{m}{n}
\DeclareMathSymbol{\bigtimes}{1}{mathx}{"91}

% Wide tikz equations
\newsavebox{\wideeqbox}
\newenvironment{wideeq}
  {\begin{displaymath}\begin{lrbox}{\wideeqbox}$\displaystyle}
  {$\end{lrbox}\makebox[0pt]{\usebox{\wideeqbox}}\end{displaymath}}



% Fancy chapter headers and footers
\usepackage{fancyhdr}

\pagestyle{fancy}
\fancyhf{}
\fancyhead[LE,RO]{\title}
\fancyhead[RE,LO]{\rightmark}
\fancyfoot[CE,CO]{\leftmark}
\fancyfoot[LE,RO]{\thepage}

\renewcommand{\headrulewidth}{2pt}
\renewcommand{\footrulewidth}{1pt}

% List of Theorems Attempt
\usepackage{etoolbox}
\makeatletter
\patchcmd\thmtlo@chaptervspacehack
  {\addtocontents{loe}{\protect\addvspace{10\p@}}}
  {\addtocontents{loe}{\protect\thmlopatch@endchapter\protect\thmlopatch@chapter{\thechapter}}}
  {}{}
\AtEndDocument{\addtocontents{loe}{\protect\thmlopatch@endchapter}}
\long\def\thmlopatch@chapter#1#2\thmlopatch@endchapter{%
  \setbox\z@=\vbox{#2}%
  \ifdim\ht\z@>\z@
    \hbox{\bfseries\chaptername\ #1}\nobreak
    #2
    \addvspace{10\p@}
  \fi
}
\def\thmlopatch@endchapter{}

\makeatother
\renewcommand{\thmtformatoptarg}[1]{ -- #1}
%\renewcommand{\listtheoremname}{List of definitions}

\newcommand{\ext}{\operatorname{Ext}}
\newcommand{\Ext}{\operatorname{Ext}}
\def\Endo{\operatorname{End}}
\def\Ind{\operatorname{Ind}}
\def\ind{\operatorname{Ind}}
\def\coind{\operatorname{Coind}}
\def\Res{\operatorname{Res}}
\def\Hol{\operatorname{Hol}}
\def\res{\operatorname{Res}}
\def\endo{\operatorname{End}}
\def\ind{\operatorname{Ind}}
\renewcommand{\AA}[0]{{\mathbb{A}}}
\DeclareMathOperator{\Exists}{\exists}
\DeclareMathOperator{\Forall}{\forall}
\newcommand{\Af}[0]{{\mathbb{A}}}
\newcommand{\CC}[0]{{\mathbb{C}}}
\newcommand{\CP}[0]{{\mathbb{CP}}}
\newcommand{\DD}[0]{{\mathbb{D}}}
\newcommand{\FF}[0]{{\mathbb{F}}}
\newcommand{\GF}[0]{{\mathbb{GF}}}
\newcommand{\GG}[0]{{\mathbb{G}}}
\newcommand{\HH}[0]{{\mathbb{H}}}
\newcommand{\HP}[0]{{\mathbb{HP}}}
\newcommand{\KK}[0]{{\mathbb{K}}}
\newcommand{\kk}[0]{{\Bbbk}}
\newcommand{\bbm}[0]{{\mathbb{M}}}
\newcommand{\NN}[0]{{\mathbb{N}}}
\newcommand{\OP}[0]{{\mathbb{OP}}}
\newcommand{\PP}[0]{{\mathbb{P}}}
\newcommand{\QQ}[0]{{\mathbb{Q}}}
\newcommand{\RP}[0]{{\mathbb{RP}}}
\newcommand{\RR}[0]{{\mathbb{R}}}
\newcommand{\SpSp}[0]{{\mathbb{S}}}
\renewcommand{\SS}[0]{{\mathbb{S}}}
\newcommand{\TT}[0]{{\mathbb{T}}}
\newcommand{\ZZ}[0]{{\mathbb{Z}}}
\newcommand{\ZnZ}[0]{\mathbb{Z}/n\mathbb{Z}}
\newcommand{\ZpZ}[0]{\mathbb{Z}/p\mathbb{Z}}
\newcommand{\Qp}[0]{\mathbb{Q}_{(p)}}
\newcommand{\Zp}[0]{\mathbb{Z}_{(p)}}
\newcommand{\Arg}[0]{\mathrm{Arg}}
\newcommand{\PGL}[0]{\mathrm{PGL}}
\newcommand{\GL}[0]{\mathrm{GL}}
\newcommand{\Gl}[0]{\mathrm{GL}}
\newcommand{\gl}[0]{\mathrm{GL}}
\newcommand{\mat}[0]{\mathrm{Mat}}
\newcommand{\Mat}[0]{\mathrm{Mat}}
\newcommand{\Rat}[0]{\mathrm{Rat}}
\newcommand{\Perv}[0]{\mathrm{Perv}}
\newcommand{\Gal}[0]{\mathrm{Gal}}
\newcommand{\Hilb}[0]{\mathrm{Hilb}}
\newcommand{\Quot}[0]{\mathrm{Quot}}
\newcommand{\Art}[0]{\mathrm{Art}}
\newcommand{\red}[0]{\mathrm{red}}
\newcommand{\alg}[0]{\mathrm{alg}}
\newcommand{\Pic}[0]{{\mathrm{Pic}~}}
\newcommand{\lcm}[0]{\mathrm{lcm}}
\newcommand{\maps}[0]{\mathrm{Maps}}
\newcommand{\maxspec}[0]{{\mathrm{maxSpec}~}}
\newcommand{\Tr}[0]{\mathrm{Tr}}
\newcommand{\adj}[0]{\mathrm{adj}}
\newcommand{\ad}[0]{\mathrm{ad}~}
\newcommand{\ann}[0]{\mathrm{Ann}}
\newcommand{\Ann}[0]{\mathrm{Ann}}
\newcommand{\arcsec}[0]{\mathrm{arcsec}}
\newcommand{\ch}[0]{\mathrm{char}~}
\newcommand{\Sp}[0]{{\mathrm{Sp}}}
\newcommand{\syl}[0]{{\mathrm{Syl}}}
\newcommand{\txand}[0]{{\text{ and }}}
\newcommand{\codim}[0]{\mathrm{codim}}
\newcommand{\txor}[0]{{\text{ or }}}
\newcommand{\txt}[1]{{\text{ {#1} }}}
\newcommand{\Gr}[0]{{\text{Gr}}}
\newcommand{\Aut}[0]{{\mathrm{Aut}}}
\newcommand{\aut}[0]{\mathrm{Aut}}
\newcommand{\Inn}[0]{{\mathrm{Inn}}}
\newcommand{\Out}[0]{{\mathrm{Out}}}
\newcommand{\mltext}[1]{\left\{\begin{array}{c}#1\end{array}\right\}}
\newcommand{\Fun}[0]{{\text{Fun}}}
\newcommand{\SL}[0]{{\text{SL}}}
\newcommand{\PSL}[0]{{\text{PSL}}}
\newcommand{\SO}[0]{{\text{SO}}}
\newcommand{\SU}[0]{{\text{SU}}}
\newcommand{\SP}[0]{{\text{SP}}}
\newcommand{\per}[0]{{\text{Per}}}
\newcommand{\loc}[0]{{\text{loc}}}
\newcommand{\Top}[0]{{\text{Top}}}
\newcommand{\Sch}[0]{{\text{Sch}}}
\newcommand{\sch}[0]{{\text{Sch}}}
\newcommand{\Set}[0]{{\text{Set}}}
\newcommand{\Sets}[0]{{\text{Set}}}
\newcommand{\Grp}[0]{{\text{Grp}}}
\newcommand{\Groups}[0]{{\text{Groups}}}
\newcommand{\Homeo}[0]{{\text{Homeo}}}
\newcommand{\Diffeo}[0]{{\text{Diffeo}}}
\newcommand{\MCG}[0]{{\text{MCG}}}
\newcommand{\set}[0]{{\text{Set}}}
\newcommand{\Tor}[0]{\text{Tor}}
\newcommand{\sets}[0]{{\text{Set}}}
\newcommand{\Sm}[0]{{\text{Sm}_k}}
\newcommand{\orr}[0]{{\text{ or }}}
\newcommand{\annd}[0]{{\text{ and }}}
\newcommand{\bung}[0]{\text{Bun}_G}
\newcommand{\const}[0]{{\text{const.}}}
\newcommand{\disc}[0]{{\text{disc}}}
\newcommand{\op}[0]{^\text{op}}
\newcommand{\id}[0]{\text{id}}
\newcommand{\im}[1]{\mathrm{im}({#1})}
\newcommand{\pt}[0]{{\{\text{pt}\}}}
\newcommand{\sep}[0]{^\text{sep}}
% \newcommand{\st}[0]{~{\text{s.t.}}~}
\newcommand{\tors}[0]{{\text{tors}}}
\newcommand{\tor}[0]{\text{Tor}}
\newcommand{\height}[0]{\text{ht}}
\newcommand{\cpt}[0]{\text{compact}}
\newcommand{\abs}[1]{{\left\lvert {#1} \right\rvert}}
\newcommand{\stack}[1]{\mathclap{\substack{ #1 }}} 
\newcommand{\qtext}[1]{{\quad \text{#1} \quad}}
\newcommand{\qst}[0]{{\quad \text{such that} \quad}}
\newcommand{\actsonl}[0]{\curvearrowleft}
\newcommand{\actson}[0]{\curvearrowright}
\newcommand{\bd}[0]{{\del}}
\newcommand{\bigast}[0]{{\mathop{\Large \ast}}}
\newcommand{\coker}[0]{\operatorname{coker}}
\newcommand{\cok}[0]{\operatorname{coker}}
\newcommand{\conjugate}[1]{{\overline{{#1}}}}
\newcommand{\converges}[1]{\overset{#1}}
\newcommand{\correspond}[1]{\theset{\substack{#1}}}
\newcommand{\cross}[0]{\times}
\newcommand{\by}[0]{\times}
\newcommand{\dash}[0]{{\hbox{-}}}
\newcommand{\dd}[2]{{\frac{\partial #1}{\partial #2}\,}}
\newcommand{\definedas}[0]{\coloneqq}
\newcommand{\da}[0]{\coloneqq}
\newcommand{\del}[0]{{\partial}}
\newcommand{\directlim}[0]{\varinjlim}
\newcommand{\disjoint}[0]{{\coprod}}
\newcommand{\divides}[0]{{~\Bigm|~}}
\newcommand{\dual}[0]{^\vee}
\newcommand{\sm}[0]{\setminus}
\newcommand{\smz}[0]{\setminus\theset{0}}
\newcommand{\eps}[0]{\varepsilon}
\newcommand{\equalsbecause}[1] {\stackrel{\mathclap{\scriptscriptstyle{#1}}}{=}}
\newcommand{\floor}[1]{{\left\lfloor #1 \right\rfloor}}
\DeclarePairedDelimiter{\ceil}{\lceil}{\rceil}
\newcommand{\from}[0]{\leftarrow}
\newcommand{\tofrom}[0]{\leftrightarrows}
\newcommand{\up}[0]{\uparrow}
\newcommand{\generators}[1]{\left\langle{#1}\right\rangle}
\newcommand{\gs}[1]{\left\langle{#1}\right\rangle}
\newcommand{\homotopic}[0]{\simeq}
\newcommand{\injectivelim}[0]{\varinjlim}
\newcommand{\injects}[0]{\hookrightarrow}
\newcommand{\inner}[2]{{\left\langle {#1},~{#2} \right\rangle}}
\newcommand{\union}[0]{\cup}
\newcommand{\Union}[0]{\bigcup}
\newcommand{\intersect}[0]{\cap}
\newcommand{\Intersect}[0]{\bigcap}
\newcommand{\into}[0]{\to}
\newcommand{\inverselim}[0]{\varprojlim}
\newcommand{\inv}[0]{^{-1}}
\newcommand{\mfa}[0]{{\mathfrak{a}}}
\newcommand{\mfb}[0]{{\mathfrak{b}}}
\newcommand{\mfc}[0]{{\mathfrak{c}}}
\newcommand{\mff}[0]{{\mathfrak{f}}}
\newcommand{\mfi}[0]{{\mathfrak{I}}}
\newcommand{\mfm}[0]{{\mathfrak{m}}}
\newcommand{\mfn}[0]{{\mathfrak{n}}}
\newcommand{\mfp}[0]{{\mathfrak{p}}}
\newcommand{\mfq}[0]{{\mathfrak{q}}}
\newcommand{\mfr}[0]{{\mathfrak{r}}}
\newcommand{\lieb}[0]{{\mathfrak{b}}}
\newcommand{\liegl}[0]{{\mathfrak{gl}}}
\newcommand{\lieg}[0]{{\mathfrak{g}}}
\newcommand{\lieh}[0]{{\mathfrak{h}}}
\newcommand{\lien}[0]{{\mathfrak{n}}}
\newcommand{\liesl}[0]{{\mathfrak{sl}}}
\newcommand{\lieso}[0]{{\mathfrak{so}}}
\newcommand{\liesp}[0]{{\mathfrak{sp}}}
\newcommand{\lieu}[0]{{\mathfrak{u}}}
\newcommand{\nilrad}[0]{{\mathfrak{N}}}
\newcommand{\jacobsonrad}[0]{{\mathfrak{J}}}
\newcommand{\mm}[0]{{\mathfrak{m}}}
\newcommand{\pr}[0]{{\mathfrak{p}}}
\newcommand{\mapsvia}[1]{\xrightarrow{#1}}
\newcommand{\kx}[1]{k[x_1, \cdots, x_{#1}]}
\newcommand{\MM}[0]{{\mathcal{M}}}
\newcommand{\OO}[0]{{\mathcal{O}}}
\newcommand{\imaginarypart}[1]{{\mathcal{Im}({#1})}}
\newcommand{\mca}[0]{{\mathcal{A}}}
\newcommand{\mcb}[0]{{\mathcal{B}}}
\newcommand{\mcc}[0]{{\mathcal{C}}}
\newcommand{\mcd}[0]{{\mathcal{D}}}
\newcommand{\mce}[0]{{\mathcal{E}}}
\newcommand{\mcf}[0]{{\mathcal{F}}}
\newcommand{\mcg}[0]{{\mathcal{G}}}
\newcommand{\mch}[0]{{\mathcal{H}}}
\newcommand{\mci}[0]{{\mathcal{I}}}
\newcommand{\mcj}[0]{{\mathcal{J}}}
\newcommand{\mck}[0]{{\mathcal{K}}}
\newcommand{\mcl}[0]{{\mathcal{L}}}
\newcommand{\mcm}[0]{{\mathcal{M}}}
\newcommand{\mcp}[0]{{\mathcal{P}}}
\newcommand{\mcs}[0]{{\mathcal{S}}}
\newcommand{\mct}[0]{{\mathcal{T}}}
\newcommand{\mcu}[0]{{\mathcal{U}}}
\newcommand{\mcv}[0]{{\mathcal{V}}}
\newcommand{\mcx}[0]{{\mathcal{X}}}
\newcommand{\mcz}[0]{{\mathcal{Z}}}
\newcommand{\cl}[0]{\mathrm{cl}}
\newcommand{\trdeg}[0]{\mathrm{trdeg}}
\newcommand{\dist}[0]{\mathrm{dist}}
\newcommand{\Dist}[0]{\mathrm{Dist}}
\newcommand{\crit}[0]{\mathrm{crit}}
\newcommand{\diam}[0]{{\mathrm{diam}}}
\newcommand{\gal}[0]{\mathrm{Gal}}
\newcommand{\diff}[0]{\mathrm{Diff}}
\newcommand{\diag}[0]{\mathrm{diag}}
\newcommand{\soc}[0]{\mathrm{Soc}\,}
\newcommand{\hd}[0]{\mathrm{Head}\,}
\newcommand{\grad}[0]{\mathrm{grad}~}
\newcommand{\hilb}[0]{\mathrm{Hilb}}
\newcommand{\minpoly}[0]{{\mathrm{minpoly}}}
\newcommand{\Hom}[0]{{\mathrm{Hom}}}
\newcommand{\Map}[0]{{\mathrm{Map}}}
\newcommand{\multinomial}[1]{\left(\!\!{#1}\!\!\right)}
\newcommand{\nil}[0]{{\mathrm{nil}}}
\newcommand{\normalneq}{\mathrel{\reflectbox{$\trianglerightneq$}}}
\newcommand{\normal}[0]{{~\trianglelefteq~}}
\newcommand{\norm}[1]{{\left\lVert {#1} \right\rVert}}
\newcommand{\pnorm}[2]{{\left\lVert {#1} \right\rVert}_{#2}}
\newcommand{\notdivides}[0]{\nmid}
\newcommand{\onto}[0]{\twoheadhthtarrow}
\newcommand{\ord}[0]{{\mathrm{Ord}}}
\newcommand{\pic}[0]{{\mathrm{Pic}~}}
\newcommand{\projectivelim}[0]{\varprojlim}
\newcommand{\rad}[0]{{\mathrm{rad}~}}
\newcommand{\ralg}[0]{\mathrm{R-alg}}
\newcommand{\kalg}[0]{k\dash\mathrm{alg}}
\newcommand{\rank}[0]{\operatorname{rank}}
\newcommand{\realpart}[1]{{\mathcal{Re}({#1})}}
\newcommand{\Log}[0]{\mathrm{Log}}
\newcommand{\reg}[0]{\mathrm{Reg}}
\newcommand{\restrictionof}[2]{{\left.{#1}\right|_{#2}}}
\newcommand{\ro}[2]{{\left.{#1}\right|_{#2}}}
\newcommand{\rk}[0]{{\mathrm{rank}}}
\newcommand{\evalfrom}[0]{\Big|}
\newcommand{\rmod}[0]{{R\dash\mathrm{mod}}}
\newcommand{\Mod}[0]{{\mathrm{Mod}}}
\newcommand{\rotate}[2]{{\style{display: inline-block; transform: rotate(#1deg)}{#2}}}
\newcommand{\selfmap}[0]{{\circlearrowleft}}
\newcommand{\semidirect}[0]{\rtimes}
\newcommand{\sgn}[0]{\mathrm{sgn}}
\newcommand{\sign}[0]{\mathrm{sign}}
\newcommand{\spanof}[0]{{\mathrm{span}}}
\newcommand{\spec}[0]{\mathrm{Spec}\,}
\newcommand{\mspec}[0]{\mathrm{mSpec}~}
\newcommand{\stab}[0]{{\mathrm{Stab}}}
\newcommand{\stirlingfirst}[2]{\genfrac{[}{]}{0pt}{}{#1}{#2}}
\newcommand{\stirling}[2]{\genfrac\{\}{0pt}{}{#1}{#2}}
\newcommand{\strike}[1]{{\enclose{horizontalstrike}{#1}}}
\newcommand{\suchthat}[0]{{~\mathrel{\Big|}~}}
\newcommand{\st}[0]{{~\mathrel{\Big|}~}}
\newcommand{\supp}[0]{{\mathrm{supp}}}
\newcommand{\surjects}[0]{\twoheadrightarrow}
\newcommand{\sym}[0]{\mathrm{Sym}}
\newcommand{\tensor}[0]{\otimes}
\newcommand{\connectsum}[0]{\mathop{\Large \#}}
\newcommand{\theset}[1]{\left\{{#1}\right\}}
\newcommand{\ts}[1]{\left\{{#1}\right\}}
\newcommand{\gens}[1]{\left\langle{#1}\right\rangle}
\newcommand{\thevector}[1]{{\left[ {#1} \right]}}
\newcommand{\tv}[1]{{\left[ {#1} \right]}}
\newcommand{\too}[1]{{\xrightarrow{#1}}}
\newcommand{\transverse}[0]{\pitchfork}
\newcommand{\trianglerightneq}{\mathrel{\ooalign{\raisebox{-0.5ex}{\reflectbox{\rotatebox{90}{$\nshortmid$}}}\cr$\triangleright$\cr}\mkern-3mu}}
\newcommand{\tr}[0]{\mathrm{Tr}}
\newcommand{\uniformlyconverges}[0]{\rightrightarrows}
\newcommand{\covers}[0]{\rightrightarrows}
\newcommand{\units}[0]{^{\times}}
\newcommand{\nonzero}[0]{^{\bullet}}
\newcommand{\wait}[0]{{\,\cdot\,}}
\newcommand{\wt}[0]{{\mathrm{wt}}}
\renewcommand{\bar}[1]{\mkern 1.5mu\overline{\mkern-1.5mu#1\mkern-1.5mu}\mkern 1.5mu}
\renewcommand{\div}[0]{\mathrm{Div}}
\newcommand{\Div}[0]{\mathrm{Div}}
\renewcommand{\hat}[1]{\widehat{#1}}
\renewcommand{\mid}[0]{\mathrel{\Big|}}
\renewcommand{\qed}[0]{\hfill\blacksquare}
\renewcommand{\too}[0]{\longrightarrow}
\renewcommand{\vector}[1]{\mathbf{#1}}
\let\oldexp\exp
\renewcommand{\exp}[1]{\oldexp\qty{#1}}
\let\oldperp\perp
\renewcommand{\perp}[0]{^\oldperp}
\newcommand*\dif{\mathop{}\!\mathrm{d}}
\newcommand{\ddt}{\tfrac{\dif}{\dif t}}
\newcommand{\ddx}{\tfrac{\dif}{\dif x}}

\DeclareMathOperator{\righttriplearrows} {{\; \tikz{ \foreach \y in {0, 0.1, 0.2} { \draw [-stealth] (0, \y) -- +(0.5, 0);}} \; }}


\let\Begin\begin
\let\End\end
\newcommand\wrapenv[1]{#1}

\makeatletter
\def\ScaleWidthIfNeeded{%
 \ifdim\Gin@nat@width>\linewidth
    \linewidth
  \else
    \Gin@nat@width
  \fi
}
\def\ScaleHeightIfNeeded{%
  \ifdim\Gin@nat@height>0.9\textheight
    0.9\textheight
  \else
    \Gin@nat@width
  \fi
}
\makeatother

\setkeys{Gin}{width=\ScaleWidthIfNeeded,height=\ScaleHeightIfNeeded,keepaspectratio}%

\title{
\textbf{
    Math 200A Homework Question Compendium
  }
  }
\author{D. Zack Garza}
\date{\today}

\begin{document}

\maketitle
% \todo{Insert title and subtitle.}
\tableofcontents


\hypertarget{one}{%
\section{One}\label{one}}

\begin{enumerate}
\def\labelenumi{\arabic{enumi}.}
\item
  \emph{Given}: \begin{align*}\forall x \in G, x^2 = e\end{align*}
  \emph{Show}: \begin{align*}G \in \mathbf{Ab}\end{align*}
\item
  \emph{Given}: \begin{align*}|G|<\infty, |G| = 0\mod 2\end{align*}
  \emph{Show}: \begin{align*}\exists g\in G \ni o(g) = 2\end{align*}
\item
  \emph{Given:} \begin{align*}G\in \mathbf{Ab}\end{align*} \emph{Show}:
  \begin{align*}T(G) \leq G\end{align*} (where
  \begin{align*}T(G) = \{ g\in G : |g| < \infty\}\end{align*}
\item
  \emph{Show}: Every finite group is finitely generated.

  \begin{itemize}
  \tightlist
  \item
    \emph{Show}: \begin{align*}\ZZ\end{align*} is finitely generated
  \item
    \emph{Show}: \begin{align*}H \leq (\QQ, +) \implies\end{align*}
    \begin{align*}H\end{align*} is cyclic
  \item
    \emph{Show}: \begin{align*}\QQ\end{align*} is not finitely generated
  \end{itemize}
\item
  \emph{Show}: \begin{align*}\QQ/\ZZ\end{align*} has, for each coset,
  exactly one representative in
  \begin{align*}[0, 1) \cap \QQ\end{align*}

  \begin{itemize}
  \tightlist
  \item
    \emph{Show}: Every element of \begin{align*}\QQ/\ZZ\end{align*} has
    finite order.
  \item
    \emph{Show}: There are elements in \begin{align*}\QQ/\ZZ\end{align*}
    of arbitrarily large order.
  \item
    \emph{Show}: \begin{align*}\QQ/\ZZ = T(\RR/\ZZ)\end{align*}
  \item
    \emph{Show}: \begin{align*}\QQ/\ZZ \cong \mathbb{C}^x\end{align*}
  \end{itemize}
\item
  \emph{Given}: \begin{align*}G/Z(G)\end{align*} is cyclic \emph{Show}:
  \begin{align*}G\end{align*} is abelian
\item
  \emph{Given}:
  \begin{align*}H \normal G, K \normal G,H \cap K = e\end{align*}
  \emph{Show}:
  \begin{align*}\forall h\in H, \forall k\in K, hk = kh\end{align*}
\item
  \emph{Given}:
  \begin{align*}|G| < \infty, \quad H \leq G, \quad N \normal G, (|H|, [G:N]) = 1\end{align*}
  \emph{Show}: \begin{align*}H \leq N\end{align*}
\item
  \emph{Given}:
  \begin{align*}|G| < \infty, N \normal G, (|N|, [G:N]) =1\end{align*}
  \emph{Show}: \begin{align*}N\end{align*} is the unique subgroup of
  order \begin{align*}|N|\end{align*}
\end{enumerate}

\hypertarget{two}{%
\section{Two}\label{two}}

\begin{enumerate}
\def\labelenumi{\arabic{enumi}.}
\item
  \emph{Given}: For every triplet in \begin{align*}G\end{align*}, two
  elements commute \emph{Show}: \begin{align*}G\end{align*} is abelian
\item
  \emph{Given}:
  \begin{align*}H_1, H_2, H_3 \leq G, G = H_1 \cup H_2\end{align*}
  \emph{Show}: \begin{align*}G=H_1 \vee G=H_2\end{align*}
\item
  \emph{Given}: \begin{align*}G=H_1 \cup H_2 \cup H_3, G\end{align*}
  finite \emph{Show}:
  \begin{align*}G=H_i \vee \forall i, [G:H_i] = 2\end{align*}
\item
  \emph{Show}: TFAE; \begin{align*}\text{clos}(H)\end{align*} is:

  \begin{itemize}
  \tightlist
  \item
    The smallest normal subgroup of \begin{align*}G\end{align*}
    containing \begin{align*}H\end{align*}.
  \item
    The subgroup generated by all conjugates of
    \begin{align*}H\end{align*}.
  \item
    \begin{align*}\bigcap_{H \leq N \normal G} N\end{align*}
  \item
    \begin{align*}\phi: G \rightarrow -\end{align*},
    \begin{align*}\phi(H) = e\end{align*}, then
    \begin{align*}\phi\end{align*} factors through
    \begin{align*}G/\text{clos}(H)\end{align*}
  \end{itemize}
\item
  \emph{Given}: \begin{align*}H, K \normal HK \leq G\end{align*}
  \emph{Show}:
  \begin{align*} \frac{HK}{H\cap K} \cong \frac{HK}{H}\times \frac{HK}{K}\end{align*}
\item
  \emph{Given}:
  \begin{align*}H\leq G, N \normal G, H \in \text{Hall}(G)\end{align*}
  \emph{Show}:
  \begin{align*}H\cap N \in \text{Hall}(N) \wedge \frac{HN}{N} \in \text{Hall}(\frac{G}{N})\end{align*}
\item
  \emph{Given}: \begin{align*}|G| = n, G\end{align*} cyclic,
  \begin{align*}\sigma_i: G \rightarrow G \ni x \mapsto x^i\end{align*}

  \begin{itemize}
  \tightlist
  \item
    Show \begin{align*}\sigma_i \in End(G)\end{align*}
  \item
    Show \begin{align*}\sigma_i \in Aut(G)\end{align*} iff
    \begin{align*}(i, n) = 1\end{align*}
  \item
    \begin{align*}\sigma_i = \sigma_j\end{align*} iff
    \begin{align*}i=j\mod n\end{align*}
  \item
    \begin{align*}\tau \in Aut(G) \implies \exists i \ni \tau = \sigma_i\end{align*}
  \item
    \begin{align*}\sigma_i \circ \sigma_j = \sigma_{ij}\end{align*}
  \end{itemize}

  \begin{enumerate}
  \def\labelenumii{\arabic{enumii}.}
  \setcounter{enumii}{5}
  \tightlist
  \item
    The map
    \begin{align*}\psi: Z_n^\times \rightarrow Aut(G)\\ i \mapsto \sigma_i\end{align*}
    is an isomorphism.
  \end{enumerate}
\item
  \emph{Given}: \begin{align*}G\end{align*} is cyclic \emph{Show}:
  \begin{align*}Aut(G)\end{align*} is abelian of order
  \begin{align*}\phi(n)\end{align*}
\item
  \emph{Show}:
  \begin{align*}D_\infty \cong \langle a,b \mid b^2 = e, ba = a^{-1} b \rangle\end{align*}
\item
  \emph{Show}:
  \begin{align*}Q_8 \cong \langle a,b \mid a^2 = b^2, a^{-1}ba = b^{-1}\rangle\end{align*}
\item
  \emph{Show}:
  \begin{align*}\generators{x,y \mid xy^2 = y^3, yx^2 = x^3y} = \generators{e}\end{align*}
\end{enumerate}

\hypertarget{three}{%
\section{Three}\label{three}}

\begin{enumerate}
\def\labelenumi{\arabic{enumi}.}
\item
  \emph{Given}: \begin{align*}G \sim X\end{align*} transitively,
  \begin{align*}H \normal G\end{align*}

  \begin{itemize}
  \tightlist
  \item
    \emph{Show}: \begin{align*}H \sim X\end{align*}, but possibly not
    transitively
  \item
    \emph{Show}: \begin{align*}G\end{align*} acts transitively on
    \begin{align*}\theset{ \mathcal{O_h} : h\ \in H}\end{align*}
  \item
    \emph{Show}:
    \begin{align*}\forall i, j, |\mathcal{O}_{h_i}| = |\mathcal{O}_{h_j}|\end{align*}
  \item
    \emph{Given}: \begin{align*}x\in \mathcal{O}_h\end{align*}
    \emph{Show}:
    \begin{align*}|\mathcal{O}_h| = |H : H \cap G_x|\end{align*}
  \item
    \emph{Show}:
    \begin{align*}|\{\mathcal{O}_h\}_{h\in H}| = [G : HG_x]\end{align*}
  \end{itemize}
\item
  \emph{Given}: \begin{align*}\mathcal{K}\end{align*} a conjugacy class
  in \begin{align*}S_n\end{align*},
  \begin{align*}\{\mathcal{O}_{s}:s\in S_n\}\end{align*} orbits of an
  \begin{align*}A_n\end{align*}-action on \begin{align*}S_n\end{align*}
  \emph{Show}:
  \begin{align*}\mathcal{K} = \mathcal{O}_s \vee \mathcal{K} = \mathcal{O}_{s_i} \cup \mathcal{O}_{s_j}\end{align*}
  \emph{Show}: Case 2 occurs iff \begin{align*}\{k_i\}\end{align*}, the
  cycle lengths in disjoint cycle form, are odd and distinct
\item
  i: \begin{align*}|G| < \infty, H < G\end{align*}

  \begin{itemize}
  \tightlist
  \item
    \emph{Show}:
    \begin{align*}\{ gHg^{-1} : g\in G\} = [G : N_G(H)]\end{align*}
  \item
    \emph{Show}:
    \begin{align*}G \neq \bigcup_{g\in G} gHg^{-1}\end{align*}
  \end{itemize}
\item
  Prove Cauchy's Theorem. Given
  \begin{align*}p\mid o(G) <\infty\end{align*} \begin{align*}
  X = \theset{ (a_{i})_{i=1}^p \in G^p \suchthat \prod_{i=1}^p a_i = e} \\
  asdsadas
  \end{align*}

  \begin{itemize}
  \tightlist
  \item
    \emph{Show}:
    \begin{align*}(a_1 a_2\cdots a_p) = e \implies (a_2 a_3 \cdots a_p a_1) = e\end{align*}
  \item
    \emph{Show}: \begin{align*}(Z_p, +) \sim X\end{align*} and
    \begin{align*}\bar 1 \sim (a_1 a_2 \cdots a_p) = (a_2 a_3 \cdots a_p a_1)\end{align*}
  \item
    \emph{Show}: \begin{align*}|X| = |G|^{p-1}\end{align*}
  \item
    \emph{Show}:
    \begin{align*}\{ \mathcal{O}_x : |\mathcal{O}_x| = 1 \} > 1\end{align*}
    and \begin{align*}\exists a \in G \ni a^p = e\end{align*}
  \end{itemize}
\item
  \emph{Given}:
  \begin{align*}G \sim X, \quad |G| < \infty , \quad 1 < |X| < \infty\end{align*}
\end{enumerate}

\begin{itemize}
\tightlist
\item
  \emph{Show}:
  \begin{align*}\exists g\ in G \ni \forall x\in X, g\sim x \neq x\end{align*}
\item
  \emph{Show}: This holds if \begin{align*}|G| = \infty\end{align*}, but
  not if \begin{align*}|X| = \infty\end{align*} as well.
\end{itemize}

\begin{enumerate}
\def\labelenumi{\arabic{enumi}.}
\setcounter{enumi}{5}
\item
  \emph{Given}: \begin{align*}H \leq G\end{align*}. \emph{Show}:
  \begin{align*}\text{core}(H)\end{align*} is

  \begin{itemize}
  \tightlist
  \item
    The largest \begin{align*}N \normal G, N \subseteq H\end{align*}
  \item
    Generated by all normal subgroups contained in
    \begin{align*}H\end{align*}
  \item
    Given by \begin{align*}\bigcap_{g\in G} gHg^{-1}\end{align*}
  \item
    The kernel of
    \begin{align*}G \sim \frac{G}{H} \ni x \sim gH = (xg)H\end{align*}
  \end{itemize}
\item
  \emph{Given}: \begin{align*}[H : G]= n < \infty\end{align*}

  \begin{itemize}
  \tightlist
  \item
    \emph{Show}: \begin{align*}[\text{core}(H) : G]\end{align*} divides
    \begin{align*}n!\end{align*}
  \item
    \emph{Show}: \begin{align*}G\end{align*} simple
    \begin{align*}\implies o(G) \mid n! \wedge |G| < \infty\end{align*}
  \end{itemize}
\item
  \emph{Given}: \begin{align*}A_n\end{align*} is simple for
  \begin{align*}n\geq 5\end{align*} \emph{Show}:
  \begin{align*}\not \exists H \in A_n \ni [H : A_n] < n\end{align*}
  \emph{Show}: \begin{align*}\exists H [H : A_n] = n\end{align*}
\item
  \emph{Given}: \begin{align*}r\end{align*} beads of
  \begin{align*}n\end{align*} colors \emph{Show}: How many distinct
  circular bracelets can be made.
\end{enumerate}

\hypertarget{four}{%
\section{Four}\label{four}}

\begin{enumerate}
\def\labelenumi{\arabic{enumi}.}
\item
  \emph{Given}: \begin{align*}H~\text{char}~G\end{align*} \emph{Show}:
  \begin{align*}H \normal G\end{align*}
\item
  \emph{Given}: \begin{align*}H ~\text{char}~ K \normal G\end{align*}
  \emph{Show}: \begin{align*}H \normal G\end{align*}
\item
  \emph{Given}:
  \begin{align*}K = \langle k \rangle \normal G\end{align*} \emph{Show}:
  \begin{align*}H \leq K \implies H \normal G\end{align*}
\item
  \emph{Show}
  \begin{align*}H \normal K \normal G \not\implies H \normal G\end{align*}
\item
  \emph{Given}:
  \begin{align*}P \leq H \leq K \leq G < \infty, P \in \text{Syl}_p(G)\end{align*}
  \emph{Show}:
  \begin{align*}P, H \normal K \implies P \normal K\end{align*}
\item
  \emph{Show}: \begin{align*}N_G(N_G(P)) = N_G(P)\end{align*}
\item
  \emph{Given}: \begin{align*}\sigma \in Aut(G)\end{align*} \emph{Show}:
  \begin{align*}\sigma Inn(G) \sigma^{-1} = Inn(G)\end{align*} iff
  \begin{align*}\forall g\in G, g^{-1}\sigma(g)\in Z(G)\end{align*}
\item
  \emph{Show}: \begin{align*}Inn(G) ~\text{char}~ Aut(G)\end{align*}
\item
  \emph{Given}:
  \begin{align*}H \subseteq G, P \in \text{Syl}_p(G)\end{align*}

  \begin{itemize}
  \tightlist
  \item
    \emph{Show}:
    \begin{align*}\exists g \in G \ni gPg^{-1} \in Syl_p(H)\end{align*}
  \item
    \emph{Given}: \begin{align*}H \normal G\end{align*} \emph{Show}:
    \begin{align*}P\cap H \in \text{Syl}_p(H)\end{align*}
  \item
    \emph{Given}: \begin{align*}P \normal G\end{align*} \emph{Show}:
    \begin{align*}P \cap H \in \text{Syl}_p(H)\end{align*} and
    \begin{align*}|\text{Syl}_p(H)| = 1\end{align*}
  \end{itemize}
\item
  \emph{Given}: \begin{align*}|G| = pqr, p < q < r\end{align*}
  \emph{Show}:
  \begin{align*}\exists P_i \in \text{Syl}_i(G) \normal G\end{align*}
\item
  \emph{Given}: \begin{align*}|G| =595\end{align*} \emph{Show}: All
  sylow subgroups are normal
\item
  \emph{Given}: \begin{align*}|G| = p(p+1)\end{align*} \emph{Show}:
  \begin{align*}\exists N \normal G\end{align*} where
  \begin{align*}|N| = p\end{align*} or \begin{align*}p+1\end{align*}
\end{enumerate}

\hypertarget{five}{%
\section{Five}\label{five}}

\begin{enumerate}
\def\labelenumi{\arabic{enumi}.}
\item
  Given: \begin{align*}G = H \semidirect_\psi K\end{align*}
  \begin{align*}\psi: K \rightarrow Aut(H) \\ k \mapsto \psi(k)\end{align*}
  \begin{align*}\theta \in Aut(H)\end{align*}
  \begin{align*}\rho: K \rightarrow K\end{align*}
  \begin{align*}\phi_\theta: Aut(H) \rightarrow Aut(H) \\ \rho \mapsto \theta \circ \rho \circ \theta^{-1}\end{align*}
  \begin{align*}\psi_2: K \rightarrow Aut(H) \\ k \mapsto (\phi_\theta \circ \psi)(k)\end{align*}
  \begin{align*}\psi_3: K \rightarrow Aut(H) \\ k \mapsto (\psi \circ \rho)(k)\end{align*}

  Show:
  \begin{align*}H \semidirect_\psi K \cong H \semidirect_{\psi_2} K \cong H \semidirect_{\psi_3} K\end{align*}
\item
  Classify groups of order
  \begin{align*}pq, p < q, p \mid q-1\end{align*}
\item
  Classify groups of order 20.
\item
  Classify groups of order 75.
\item
  Show: \begin{align*}|G| < 60 \implies G\end{align*} is not simple.
\item
  Show: \begin{align*}|G| < 60 \implies G\end{align*} is solvable
\item
  Given: \begin{align*}|G| < \infty\end{align*},
  \begin{align*}H \leq G\end{align*} maximal
  \begin{align*}\implies [G:H] = p\end{align*}, a prime. Show:
  \begin{align*}|G|\end{align*} is solvable

  \begin{itemize}
  \tightlist
  \item
    Given:
    \begin{align*}P \in Syl_p(G) \wedge \exists H \ni N_G(P) \leq H \leq G\end{align*}
    Show: \begin{align*}[G:H] = 1 \mod p\end{align*}
  \item
    Given: \begin{align*}p \mid o(G)\end{align*}, the largest such prime
    Show: \begin{align*}\exists P \normal G \in Syl_p(G),\end{align*}
  \end{itemize}
\item
  \begin{align*}|G| < \infty\end{align*}

  \begin{itemize}
  \tightlist
  \item
    Given: \begin{align*}G\end{align*} is characteristically simple
    Show:
    \begin{align*}\exists H~(\text{simple})~ \ni G \cong H^n\end{align*}.
    Show: Whether or not the converse holds
  \item
    Given: \begin{align*}N \normal G\end{align*} minimal Show:
    \begin{align*}N\end{align*} is characteristically simple,
    \begin{align*}N \cong H^n\end{align*}
  \end{itemize}
\end{enumerate}

\hypertarget{six}{%
\section{Six}\label{six}}

\begin{enumerate}
\def\labelenumi{\arabic{enumi}.}
\item
  Given: \begin{align*}G\end{align*} is nilpotent Show:
  \begin{align*}H \leq G \implies H, G/H\end{align*} are nilpotent
\item
  Show: \begin{align*}G/Z(G)\end{align*} is nilpotent
  \begin{align*}\implies G\end{align*} is nilpotent
\item
  Given: \begin{align*}|G| < \infty\end{align*} Show:
  \begin{align*}|G|\end{align*} is nilpotent iff
  \begin{align*}a,b\in G, (a,b)=1 \implies ab=ba\end{align*}
\item
  Show: \begin{align*}D_{2n}\end{align*} is nilpotent iff
  \begin{align*}n = 2^{i}\end{align*}
\item
  Given: \begin{align*}|G| < \infty\end{align*}

  \begin{itemize}
  \tightlist
  \item
    Show \begin{align*}\Phi(G)~\text{char}~ G\end{align*}
  \item
    Show \begin{align*}\Phi(G)\end{align*} is nilpotent
  \item
    Given: \begin{align*}|P| = p^e\end{align*} Show:
    \begin{align*}P / \Phi(P)\end{align*} is an elementary abelian
    p-group Show:\begin{align*}N \normal P, P/N\end{align*} is
    elementary abelian
    \begin{align*}\implies \Phi(P) \subseteq N\end{align*}
  \end{itemize}
\item
  Given: \begin{align*}R\end{align*} a commutative ring,
  \begin{align*}x,y \in R\end{align*} nilpotent

  \begin{itemize}
  \tightlist
  \item
    Show: \begin{align*}x+y\end{align*} is nilpotent Show:
    \begin{align*}\{x\in R : x \text{ is nilpotent}\} \normal R\end{align*}
  \item
    Given: \begin{align*}u\in R^\times, x\in R\end{align*} nilpotent
    Show: \begin{align*}u+x\in R^\times\end{align*}
  \item
    Show: An counterexample to 1 when \begin{align*}R\end{align*} is
    noncommutative.
  \end{itemize}
\item
  Given: \begin{align*}R\end{align*} a commutative ring,
  \begin{align*}R[[x]]\end{align*} its formal power series

  \begin{itemize}
  \tightlist
  \item
    Show:
    \begin{align*}\sum_{i=0}^\infty a_i x^i \in R[[x]]^\times \iff a_0 \in R^\times\end{align*}
  \item
    Show: \begin{align*}R\end{align*} a domain
    \begin{align*}\implies R[[x]]\end{align*} a domain
  \item
    Given: \begin{align*}R\end{align*} a field Show:
    \begin{align*}I = \{ r \in R[[x]] : r_0 = 0\}\end{align*} is a
    maximal ideal of \begin{align*}R[[x]]\end{align*} Show:
    \begin{align*}I\end{align*} is the unique maximal ideal
  \end{itemize}
\item
  Given: \begin{align*}R\end{align*} a commutative ring,
  \begin{align*}G\end{align*} a finite group,
  \begin{align*}RG\end{align*} a group ring.

  \begin{itemize}
  \tightlist
  \item
    Given:
    \begin{align*}\mathcal{K} = \{ k_1, k_2, \cdots k_m\}\end{align*} a
    conjugacy class in \begin{align*}G\end{align*} Show:
    \begin{align*}K = \sum_{i=1}^m k_i \in RG \implies K \in Z(RG)\end{align*}
  \item
    Given: \begin{align*}\mathcal{K}_1\cdots \mathcal{K}_r\end{align*}
    distinct conjugacy classes in \begin{align*}G\end{align*},
    \begin{align*}K_i = \sum_{j}k_j \ni k_j \in \mathcal{K}_i\end{align*}
    Show:
    \begin{align*}Z(RG) = \{\sum a_l K_l : \forall 1 \leq l \leq r, a_l \in R \}\end{align*}
    (All \begin{align*}R\end{align*}-linear combinations of the
    \begin{align*}\mathcal{K}_i\end{align*})
  \end{itemize}
\item
  Given: \begin{align*}R\end{align*} a ring,
  \begin{align*}M_n(R)\end{align*} its matrix ring

  \begin{itemize}
  \tightlist
  \item
    Given: \begin{align*}I \normal R\end{align*} (two-sided) Show:
    \begin{align*}M_n(I) \normal M_n(R)\end{align*} Show:
    \begin{align*}\frac{M_n(R)}{M_n(I)} \cong M_n(\frac{R}{I})\end{align*}
  \item
    Show: \begin{align*}\forall I_M \normal M_n(R), I\end{align*} is of
    the form \begin{align*}M_n(I)\end{align*} for some
    \begin{align*}I \normal R\end{align*} Show:
    \begin{align*}R\end{align*} a division ring
    \begin{align*}\implies M_n(R)\end{align*} is a simple ring.
  \end{itemize}
\end{enumerate}

\hypertarget{seven}{%
\section{Seven}\label{seven}}

\hypertarget{eight}{%
\section{Eight}\label{eight}}

%\listoftodos

\bibliography{/home/zack/Notes/library.bib}

\end{document}
