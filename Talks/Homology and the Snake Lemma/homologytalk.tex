\documentclass[12pt]{article}
\usepackage[utf8]{inputenc}
\usepackage{graphicx}
\usepackage{parskip}
\usepackage{amsfonts}
\usepackage[top=1.0in,bottom=1.0in]{geometry}


\usepackage{tikz}
\usetikzlibrary{calc}
\newcommand\HRule{\rule{\textwidth}{1pt}}
\newcommand*\diff{\mathop{}\!\mathrm{d}}
\newcommand*\Diff[1]{\mathop{}\!\mathrm{d^#1}}


\begin{document}
%\maketitle
\pagestyle{empty} 
\thispagestyle{empty}

\begin{tikzpicture}[remember picture, overlay]
  \draw[line width = 4pt] ($(current page.north west) + (0.5in,-0.5in)$) rectangle ($(current page.south east) + (-0.5in,0.5in)$);
\end{tikzpicture}

\centering
\huge
SUMS PRESENTS\\[0.4em]
\textbf{Lightning Talks}\\[0.4em]


\\[1.25em]

\large
\resizebox{\linewidth}{!}{$\displaystyle\int_{\mathbb{R}} e^{x^2} \diff{x} = \sqrt{\pi}$\hspace{0.3in}$\displaystyle\nabla \times E = -\frac{\partial \textbf{B}}{\partial t}$\hspace{0.3in}$\displaystyle \mathbb{R} / \mathbb{Z} \cong S^1$}\\[2em]
\Large
A lightning talk is a short presentation meant to briefly introduce a topic or concept to an audience. 
These talks are held monthly and are entirely student-run. Attendance is open to anyone, with any level of mathematical background.

We also hold giveaways for math-related books and provide light refreshments!


\Huge
\textit{This Month's Topic:}\\[0.4em]
\Large 
Homology and the Snake Lemma

\normalsize

Homology is an algebraic tool that has proved to be useful in many areas of Mathematics, and in recent years has even been used in applied settings such as large scale data analysis.

This talk is motivated by the movie "It's My Turn" from 1980, which features an infamous scene involving a result known as "The Snake Lemma." Surprisingly, this scene is mathematically precise, and is even referenced as a proof in a text by Weibel's "Introduction to Homological Algebra"!

The goal of this talk will be to provide a brief overview of some of the mathematics used in this scene, and ultimately to formulate what the Snake Lemma is and an example of how it can be used. In the process, we will also cover some of the basics of homology and its generalization in homological algebra.

\Large
\textbf{Monday, November 20th\\ 3PM -- 4PM, AP\&M 7421}\\[0.6em]

\begin{center}
\small
    FUNDED BY A.S.
\end{center}

\end{document}