\hypertarget{working-draft-integration-bee-integrals}{%
\section{Working Draft: Integration Bee
Integrals}\label{working-draft-integration-bee-integrals}}

For help with syntax, see
\href{https://math.meta.stackexchange.com/questions/5020/mathjax-basic-tutorial-and-quick-reference}{this
link on Math StackExchange}

Rules and Guidelines

This rules need to be read before the competition starts.

\begin{enumerate}
\def\labelenumi{\arabic{enumi}.}
\tightlist
\item
  Participants must correctly evaluate indefinite or definite, proper,
  single real variable integrals in the time allotted.
\item
  In each round, both participants will approach the board and an
  integral will be shown. The timer will then begin a 10 second
  countdown at their discretion, during which neither participant may
  write on the board. When the timer announces the end of the countdown,
  participants may begin writing their work and solutions on the board.
\item
  Each round will last at most 3 minutes. If no correct answer is
  arrived at by either contestant within this time, the same
  participants will start a new round with a new integral. If in the
  additional round, both of the contestants do not get correct answers,
  both of them will be eliminated of the bracket map.
\item
  At any time during the round, a participant may circle or box their
  final solution. When the circle/box is completed, both participants
  must stop writing. The clock will be stopped, while the judge(s)
  consider the solution.
\item
  If at any point a boxed or circled solution is deemed to be correct,
  the round is over and the corresponding participant advanced to the
  next round, while the other contestant is eliminated from the
  competition.
\item
  If a boxed or circled solution is found to be incorrect, the timer
  will provide a 5 second countdown, after which point the clock will be
  restarted and the contestants may resume writing.
\item
  A competitor may present no more than two solutions per round. Two
  incorrect solutions does not disqualify a competitor, however - if
  their opponent does not arrive at a correct solution within the
  allotted time, both competitors may participate in an additional
  round. If in the additional round, both of the contestants do not get
  correct answers, both of them will be eliminated of the bracket map.
\item
  In your answers, it is not necessary to include the arbitrary constant
  C in an indefinite integral, nor the absolute value sign around the
  argument of a logarithm.
\end{enumerate}

Here are some basic examples, add further ones in list format \#\# Yi
Fu's image - \(\int x^2 dx\) = \(2x\) - \(\int e^x dx\) = \(e^x\) -
\(\int \sin (x) dx\) = \(- \cos (x)\) - \(\int \frac{1}{x} dx\) =
\(\ln (x)\) - \(\int_{-\infty}^\infty e^{-x^2} dx\) = \(\sqrt{\pi}\) -
\(\int_0^1 \int_y^1 \sin (x^2) dx dy\) = \(\sin^2(\frac{1}{2})\) -
\(\int_{-\infty}^{\infty} x^3 e^{-x} dx\) = -
\(\int_2^\infty \frac{1}{x(\ln (x))^2} dx\) = \(\frac{1}{log(2)}\) -
\(\int \frac{x}{x^2-x+\frac{5}{6}}dx\) = \(\Im\)

\hypertarget{annies-file}{%
\subsection{Annie's file}\label{annies-file}}

\begin{itemize}
\tightlist
\item
  \(\displaystyle \int \frac{2 \arctan e^x}{e^{2x}} dx = -e^{2x}\arctan e^x - \displaystyle\frac{1}{e^x}-\arctan e^x + C\)
\item
  \(\displaystyle \int \frac{\sin^2(x)}{\cos^3(x)} = \frac{1}{2}\sec(x)\tan(x) - \frac{1}{2}\ln|\sec(x) - \tan(x)| dx\)
\item
  \(\displaystyle \int \frac{xe^x}{(e^x + 1)^2}dx = \frac{xe^x}{e^x+1} - \ln(e^x + 1) + C\)
\item
  \(\displaystyle \int \frac{\sin^2(x)}{\cos^3(x)}dx = \frac{1}{2} \tan(x)(\sin(x)\tan(x)+ \cos(x)) - \frac{1}{2} \ln|\sec(x) +\tan(x)| + C\)
\end{itemize}

\hypertarget{yijia-phone-upload}{%
\subsection{Yijia phone upload}\label{yijia-phone-upload}}

\begin{itemize}
\tightlist
\item
  \(\displaystyle \int \frac{1}{x(x-1)^2} = \ln(x)\) -
  \(\frac{1}{x-1} - \ln(x-1) + C\)
\item
  \(\displaystyle \int \frac{x + \sin(x)}{1 + \cos(x)}dx = x \tan \frac{x}{2} + C\)
\item
  \(\displaystyle \int \frac{1}{x^2}\cos(\frac{1}{x})dx = - \sin{\frac{1}{x}} + C\)
\item
  \(\displaystyle \int \frac{1}{x^2\sqrt{4x^2 -1}}dx = \frac{\sqrt{4x^2-1}}{x} +C\)
\item
  \(\displaystyle \int (2^x + x^2) dx = \frac{2^x}{\ln(2)} + \frac{1}{3}x^3 + C\)
\end{itemize}

\hypertarget{jonathans-teachers-stuff}{%
\subsection{Jonathan's teacher's stuff}\label{jonathans-teachers-stuff}}

\hypertarget{minor-tricks}{%
\subsubsection{Minor Tricks}\label{minor-tricks}}

\begin{itemize}
\tightlist
\item
  \(\displaystyle\int \frac{e^x}{e^x+1}dx\) = \(log({e^x+1})\)
\item
  \(\displaystyle\int \sin(2x) \left(\frac{\sin(x)+\cos(x)}{2} \right)dx\)=\(\frac{1}3({sin^3(x)-cos^3(x)})\)
\item
  \(\displaystyle\int (\sin(x) + \cos(x))^2dx\) =
  \(x-\frac{1}{2}cos(2x)\)
\item
  \(\displaystyle\int \frac{\sec^3(x) + e^{\sin(x)}}{\sec(x)}dx\) =
  \(\sec(x)(-e^{\sin(x)}\sin(\frac{x}{2})+\sin(x)+e^{\sin(x)}\cos^2(\frac{x}{2}))\)
\item
  \(\displaystyle\int \frac{\sin^3(x)}{\cos(x) - \cos^3(x)}dx\) =
  \(- \log(cos(x))\)
\item
  \(\displaystyle\int \ln(x^2-1)dx\)=\(-log(1 - x) + log(1 + x) + x (-2 + log(-1 + x^2))\)
\item
  \(\displaystyle\int \ln(x^2 + 6x + 5)dx\)=\(log(1 + x) + 5 log(5 + x) + x (-2 + log(5 + 6 x + x^2))\)
\item
  \(\displaystyle\int \sqrt{\frac{1+x}{1-x}}dx\)
\item
  \(\displaystyle\int \frac{1- \sqrt{x}}{1 + \sqrt{x}}dx\) =
  \(4\sqrt{x}- x - 4 log(1 + \sqrt{x})\)
\item
  \(\displaystyle\int \frac{1}{1-\sin(x)}dx\)=\(\frac{2sin(\frac{x}{2})}{cos(x/2) - sin(x/2)}\)
\item
  \(\displaystyle\int \frac{x}{\sqrt[3]{x+1}}dx\)=\(\frac{3}{10}(1 + x)^{2/3}(-3+2x)\)
\item
  \(\displaystyle\int \frac{1}{\sqrt{e^{2x}-1}}dx\) =
  \(\tan^{-1}(\sqrt{e^{2x}-1})\)
\item
  \(\displaystyle\int \frac{\sin^2(2x)}{1+\cos(2x)}dx\) =
  \(x-\cos(x)\sin(x)\)
\item
  \(\displaystyle\int \frac{1+e^x}{1-e^x}dx\) = \(x-2\ln(|e^x-1|)\)
\item
  \(\displaystyle\int \frac{1}{(x+1)\sqrt{x}}dx\) =
  \(2\tan^{-1}(\sqrt{x})\)
\item
  \(\displaystyle\int \frac{x^2 + 2x + 1}{x\sqrt{x^2-1}}dx\) =
  \(2\ln(|\sqrt{x^2-1}+x|)+\tan^{-1}(\sqrt{x^2-1})+\sqrt{x^2-1}\)
\item
  \(\displaystyle\int (\tan(x) + \cot(x))^2dx\) = \(\tan(x)-\cot(x)\)
\item
  \(\displaystyle\int x\sin^{-1}\left(\frac{1}{x}\right)dx\) =
  \(\frac{x^2\csc^{-1}(x)+\sqrt{x^2-1}}{2}\)
\item
  \(\displaystyle\int \frac{1}{x(x-1)^2}dx\) =
  \(\ln{x}-\frac{1}{x-1}-\ln(x-1)\)
\item
  \(\displaystyle\int \frac{x+\sin{x}}{1+\cos{x}}dx\) =
  \(x\tan{\frac{x}{2}}\)
\item
  \(\displaystyle\int \frac{1}{x^2}\cos{\frac{1}{x}}dx\) =
  \(-\sin{\frac{1}{x}}\)
\item
  \(\displaystyle\int \frac{1}{x^2\sqrt{4x^2-1}}dx\) =
  \(\frac{\sqrt{4x^2-1}}{x}\)
\item
  \(\displaystyle\int 2^x+x^2dx\) = \(\frac{2^x}{ln2}+\frac{x^3}{3}\)
\end{itemize}

\hypertarget{product-to-sum-formulas}{%
\subsubsection{Product to Sum formulas}\label{product-to-sum-formulas}}

\begin{itemize}
\tightlist
\item
  \(\displaystyle\int \sin(4x)\cos(3x)dx\)=\(\frac{-7cos(x)-cos(7x)}{14}\)
\item
  \(\displaystyle\int 4\cos(x)\cos(2x)\sin(3x)dx\)=\(\frac{-6cos(2x)-3cos(4x)-2cos(6x)}{12}\)
\item
  \(\displaystyle\int \cos\left(x+\frac{\pi}{4}\right) \cos\left(x- \frac{\pi}{4}\right)dx\)
  = \(\frac{sin(2x)}{4}\)
\end{itemize}

\hypertarget{rational-muxn-substitutions}{%
\subsubsection{\texorpdfstring{Rational \([\mu(x)]^n\)
substitutions}{Rational {[}\textbackslash{}mu(x){]}\^{}n substitutions}}\label{rational-muxn-substitutions}}

\begin{itemize}
\tightlist
\item
  \(\displaystyle\int \frac{1}{\sqrt{x}+2\sqrt[3]{x}}dx\) =
  \(2\sqrt{x}-6\sqrt[3]{x}+24\sqrt[6]{x}-48log(\sqrt[6]{x}+2)\)
\item
  \(\displaystyle\int \frac{1}{\sqrt{x} - \sqrt[3]{x}}dx\)=\(2\sqrt{x}+3\sqrt[3]{x}+6\sqrt[6]{x}+6log(1-\sqrt[6]{x})\)
\item
  \(\displaystyle\int \frac{1}{\sqrt{x} + \sqrt[4]{x}}dx\)=\(2\sqrt{x}-4\sqrt[4]{x}+4log(\sqrt[4]{x}+1)\)
\item
  \(\displaystyle\int \frac{\sqrt[3]{x}+1}{\sqrt[3]{x} - 1}dx\)=\(3x^{2/3}+x+6\sqrt[3]{x}+6log(1-\sqrt[3]{x})\)
\item
  \(\displaystyle\int \frac{x}{1-x^2 + \sqrt{1- x^2}}dx\)=\(-log(\sqrt{1-x^2}+1)\)
\end{itemize}

\hypertarget{weierstauss-substitution}{%
\subsubsection{Weierstauss
substitution}\label{weierstauss-substitution}}

\begin{itemize}
\tightlist
\item
  \(\displaystyle\int \frac{1}{1 + \sin(x) + \cos(x)}dx\)=\(log(sin(\frac{x}{2})+cos(\frac{x}{2}))-log(cos(\frac{x}{2}))\)
\item
  \(\displaystyle\int \frac{1-\sin(x)}{1 + \cos(x)}dx\)=\(tan(\frac{x}{2})+2log(cos(\frac{x}{2}))\)
\item
  \(\displaystyle\int \frac{1}{\sin(x) + \cos(x)}dx\)=\((-1-i)(-1)^{3/4}tanh^{-1}(\frac{tan(\frac{x}{2})-1}{\sqrt{2}})\)
  **way too hard!!! \#\#\# Interesting u-substitutions
\item
  \(\displaystyle\int \sin(x)\cos(\cos(x)) dx\)=\(-sin(cos(x))\)
\item
  \(\displaystyle\int e^{(x+e^x)} dx\)=\(e^{e^{x}}\)
\item
  \(\displaystyle\int e^{\sin^2(x)}\sin(2x)dx\)= \(e^{sin^2(x)}\)
\item
  \(\displaystyle\int \frac{\sin^3{\sqrt{x}}}{2\sqrt{x}}dx\)=\(\frac{cos(3\sqrt{x})-9cos(\sqrt{x})}{12}\)
\item
  \(\displaystyle\int \frac{\ln(x)}{x}dx\)=\(\frac{log^2(x)}{2}\)
\item
  \(\displaystyle\int \frac{x}{1+x^4}dx\)=\(\frac{tan^{-1}(x^2)}{2}\)
\item
  \(\displaystyle\int \frac{x^2}{1+x^6}dx\)=\(\frac{tan^{-1}(x^3)}{3}\)
\item
  \(\displaystyle\int \frac{x}{x^4-16}dx\)=\(\frac{log(4-x^2)-log(x^2+4)}{16}\)
\item
  \(\displaystyle\int \frac{e^x}{1+e^{2x}}dx\)=\(tan^{-1}(e^x)\)
\item
  \(\displaystyle\int \frac{\sec^2(x)}{1+\tan(x)}dx\)=\(log(sin(x)+cos(x))-log(cos(x))\)
\item
  \(\displaystyle\int \frac{1}{2x\sqrt{x-1}}dx\)=\(tan^{-1}(\sqrt{x-1})\)
\item
  \(\displaystyle\int \frac{\sin(\alpha)}{\sqrt{4-\cos^2(\alpha)}}d\alpha\)=\(-sin^{-1}(\frac{cos(\alpha)}{2})\)
\item
  \(\displaystyle\int \frac{6-2x}{\sqrt{9-x^2}}dx\)=\(2\sqrt{9-x^2}+6sin^{-1}(\frac{x}{3})\)
\item
  \(\displaystyle\int \frac{4x^3+2x}{x^4+1}dx\)=\(log(x^4+1)+tan^{-1}(x^2)\)
\item
  \(\displaystyle\int \tan^4(x) + \tan^2(x)dx\)=\(\frac{tan^3(x)}{3}\)
\item
  \(\displaystyle\int (\cos^2(x))(\sin(x) + 1)dx\)=\(\frac{3x-2cos^3(x)+3sin(x)cos(x)}{6}\)
\item
  \(\displaystyle\int \frac{\sec^2(x)}{\sqrt{9-\tan^2(x)}}dx\)=\(\frac{\sqrt{5cos(2x)+4}\sec(x)tan^{-1}(\frac{sin(x)}{\sqrt{5cos(2x)+4}})}{\sqrt{9-\tan^2(x)}}\)
  **way too hard!!
\item
  \(\displaystyle\int \frac{2}{x\sqrt{25x^4-1}}dx\)=\(tan^{-1}(\sqrt{25x^4-1})\)
\item
  \(\displaystyle\int x\sqrt{x+1}dx\)=\(\frac{2}{15}(x+1)^{3/2}(3x-2)\)
\item
  \(\displaystyle\int \frac{1}{(e^x + e^{-x})}dx\)=\(tan^{-1}(e^x)\)
\item
  \(\displaystyle\int \frac{1}{\sqrt{x}\sqrt{1-4x}}dx\)=\(\frac{\sqrt{1-4x}\sqrt{x}\sin^{-1}(2\sqrt{x})}{\sqrt{(1-4x)x}}\)
\item
  \(\displaystyle\int \frac{1}{x\sqrt{4x-1}}dx\)=\(log(1-\sqrt{1-4x})-log(\sqrt{1-4x}+1)\)
\item
  \(\displaystyle\int \frac{7-\ln(x)}{x(3+\ln(x))}dx\)=\(10ln(ln(x)+3)-ln(x)\)
\item
  \(\displaystyle\int \frac{\tan(x)}{\tan(x) + \sec(x)}dx\)=\(x-\frac{2sin(\frac{x}{2})}{sin(\frac{x}{2})+cos(\frac{x}{2})}\)
\item
  \(\displaystyle\int \ln\sqrt{x^2+1}dx\)=\(\frac{1}{2}x(ln(x^2+1)-2)+tan^{-1}(x)\)
\item
  \(\displaystyle\int \frac{1}{\sqrt{x\sqrt{x}-x^2}}dx\)=\(2tan^{-1}(\frac{(2\sqrt{x}-1)\sqrt{x^{3/2}-x^2}}{2(\sqrt{x}-1)x})\)
\end{itemize}

\hypertarget{integration-by-parts}{%
\subsubsection{Integration by Parts}\label{integration-by-parts}}

\begin{itemize}
\tightlist
\item
  \(\displaystyle\int (\ln(x))^2 dx\) = \(x (2 - 2 log(x) + log^2(x))\)
\item
  \(\displaystyle\int x^2e^x dx\) = \(e^x (2 - 2 x + x^2)\)
\item
  \(\displaystyle\int x\sin(2x) dx\)=\(1/4 (-2 x cos(2 x) + sin(2 x))\)
\item
  \(\displaystyle\int e^x\cos(x) dx\) = \(1/2 e^x (cos(x) + sin(x))\)
\item
  \(\displaystyle\int \sqrt{x}\ln(x) dx\) =
  \(2/9 x^{3/2} (-2 + 3 log(x))\)
\item
  \(\displaystyle\int x\ln(x) dx\) = \(1/4 x^2 (-1 + 2 log(x))\)
\item
  \(\displaystyle\int x^2\ln(x) dx\) = \(1/9 x^3 (-1 + 3 log(x))\)
\item
  \(\displaystyle\int \tan^{-1}(x) dx\) =
  \(x tan^{-1}(x) - 1/2 log(1 + x^2)\)
\item
  \(\displaystyle\int \sec^{-1}(x) dx\) = \({cos(x) + x sin(x)}\)
\item
  \(\displaystyle\int x\sec^{-1}(x) dx\)=\(1/2 x (-\sqrt{1 - 1/x^2} + x sec^{-1}(x))\)
\item
  \(\displaystyle\int (\ln(x))^2 dx\) = \(x (2 - 2 log(x) + log^2(x))\)
\item
  \(\displaystyle\int \sin(\ln(x)) dx\)=\(-1/2 x (cos(log(x)) - sin(log(x)))\)
\item
  \(\displaystyle\int x\tan^{-1}(x) dx\) =
  \(1/2 (-x + (1 + x^2) tan^{-1}(x))\)
\item
  \(\displaystyle\int e^{\sqrt[3]{x}} dx\) =
  \(3e^{x^{1/3}} (2 - 2x^{1/3}+ x^{2/3})\)
\item
  \(\displaystyle\int \frac{x\tan^{-1}(x)}{(x^2+1)^2} dx\) =
  \((x + (-1 + x^2)tan^{-1}(x))/(4 (1 + x^2))\)
\item
  \(\displaystyle\int \frac{\sqrt{1+\ln(x)}}{x\ln(x)} dx\) =
  \(2\sqrt{1 + log(x)} + log(1 - \sqrt{1 + log(x)}) - log(1 + \sqrt{1 + log(x)})\)
\item
  \(\displaystyle\int e^{\sqrt{x}} dx\) =
  \(2e^\sqrt{x} (-1 + \sqrt{x})\)
\item
  \(\displaystyle\int \sec^3(x) dx\) =
  \(1/2 (-log(cos(x/2) - sin(x/2)) + log(cos(x/2) + sin(x/2)) + sec(x) tan(x))\)
\item
  \(\displaystyle\int \sin^{-1}(x) dx\)=\(\sqrt{1 - x^2} + x sin^{-1}(x)\)
\item
  \(\displaystyle\int \frac{2\arctan(e^x)}{e^{2x}}dx\)=\(tan^{-1}(e^{-x}) - e^{-2x}(e^x + tan^{-1}(e^x))\)
\end{itemize}

\hypertarget{powers-of-trigonometric-functions}{%
\subsubsection{Powers of Trigonometric
Functions}\label{powers-of-trigonometric-functions}}

\begin{itemize}
\tightlist
\item
  \(\displaystyle\int \sin^2(2x) + \sec(2x) dx\)=
  \(\frac{1}{2}\left(x-\frac{1}{4}\sin \left(4x\right)\right)+\frac{1}{2}\ln \left|\tan \left(2x\right)+\sec \left(2x\right)\right|+C\)
\item
  \(\displaystyle\int \cos^2 \left (\frac{1}{2}x \right) + \sec \left(\frac{1}{2}x\right)dx\)=
  \(\frac{1}{2}\left(\sin \left(x\right)+x\right)+2\ln \left|\tan \left(\frac{x}{2}\right)+\sec \left(\frac{x}{2}\right)\right|+C\)
\item
  \(\displaystyle\int \sin^3(x) dx\)=
  \(-\cos \left(x\right)+\frac{\cos ^3\left(x\right)}{3}+C\)
\item
  \(\displaystyle\int \cos^3(x) dx\)=
  \(\sin \left(x\right)-\frac{\sin ^3\left(x\right)}{3}+C\)
\item
  \(\displaystyle\int \sec^4(x) dx\) =
  \(\frac{\left(\sqrt{\sec ^2\left(x\right)-1}\right)^3}{3}+\sqrt{\sec ^2\left(x\right)-1}+C\)
\item
  \(\displaystyle\int \sin^3(x)\cos^5(x) dx\) =
  \(-\frac{\cos ^6\left(x\right)}{6}+\frac{\cos ^8\left(x\right)}{8}+C\)
\item
  \(\displaystyle\int (\cos(x)+1)^3 dx\) =
  \(\frac{5}{2}x-\frac{1}{3}\sin ^3\left(x\right)+4\sin \left(x\right)+\frac{3}{4}\sin \left(2x\right)+C\)
\item
  \(\displaystyle\int (\sin(x)+1)^3 dx\) =
  \(\frac{5}{2}x+\frac{1}{3}\cos ^3\left(x\right)-4\cos \left(x\right)-\frac{3}{4}\sin \left(2x\right)+C\)
\item
  \(\displaystyle\int \sin^3(z)\sqrt{\cos(z)} dz\) =
  \(-\frac{2}{3}\cos ^{\frac{3}{2}}\left(z\right)+\frac{2}{7}\cos ^{\frac{7}{2}}\left(z\right)+C\)
\item
  \(\displaystyle\int \sec^3(x)\tan^3(x) dx\) =
  \(-\frac{\sec ^3\left(x\right)}{3}+\frac{\sec ^5\left(x\right)}{5}+C\)
\item
  \(\displaystyle\int \sqrt{1+\cos(8x)} dx\) =
  \(\frac{1}{2\sqrt{2}}\sin \left(4x\right)+C\)
\item
  \(\displaystyle\int \sec^3(x)\tan^3(x) dx\) =
  \(-\frac{\sec ^3\left(x\right)}{3}+\frac{\sec ^5\left(x\right)}{5}+C\)
\item
  \(\displaystyle\int \sin^4(x) + \sin^3(x) dx\) =
  \(-\frac{1}{4}\sin ^3\left(x\right)\cos \left(x\right)+\frac{3}{8}\left(x-\frac{1}{2}\sin \left(2x\right)\right)-\cos \left(x\right)+\frac{\cos ^3\left(x\right)}{3}+C\)
\item
  \(\displaystyle\int \frac{\sin^2(x)}{\cos^3(x)} dx\) =
  \(\frac{\sqrt{1-\cos ^2\left(x\right)}}{2\cos ^2\left(x\right)}-\frac{1}{4}\ln \left|\sqrt{-\cos ^2\left(x\right)+1}+1\right|+\frac{1}{4}\ln \left|\sqrt{-\cos ^2\left(x\right)+1}-1\right|+C\)
\end{itemize}

\hypertarget{trigonometric-substitution}{%
\subsubsection{Trigonometric
Substitution}\label{trigonometric-substitution}}

\begin{itemize}
\tightlist
\item
  \(\displaystyle\int \sqrt{1 -x^2} dx\) =
  \(\frac{1}{2}\left(\arcsin \left(x\right)+\frac{1}{2}\sin \left(2\arcsin \left(x\right)\right)\right)+C\)
\item
  \(\displaystyle\int \frac{1}{\sqrt{4-9x^2}} dx\) =
  \(\frac{1}{3}\arcsin \left(\frac{3}{2}x\right)+C\)
\item
  \(\displaystyle\int \frac{\sqrt{x^2-9}}{x} dx\) =
  \(-3\arcsec \left(\frac{1}{3}x\right)+x\sqrt{1-\frac{9}{x^2}}+C\)
\item
  \(\displaystyle\int \frac{1}{x^2\sqrt{x^2-9}} dx\) =
  \(\frac{1}{9}\sqrt{1-\frac{9}{x^2}}+C\)
\item
  \(\displaystyle\int \frac{1}{2x\sqrt{x^2-16}} dx\) =
  \(\frac{1}{8}\arcsec \left(\frac{1}{4}x\right)+C\)
\item
  \(\displaystyle\int \frac{1}{x^2\sqrt{x^2-4}} dx\) =
  \(\frac{1}{4}\sqrt{1-\frac{4}{x^2}}+C\)
\item
  \(\displaystyle\int \frac{x^3}{\sqrt{25-x^2}} dx\) =
  \(\frac{\left(\sqrt{25-x^2}\right)^3}{3}-25\sqrt{25-x^2}+C\)
\item
  \(\displaystyle\int \frac{\sqrt{x^2-1}}{x} dx\) =
  \(-\arctan \left(\sqrt{x^2-1}\right)+\sqrt{x^2-1}+C\)
\item
  \(\displaystyle\int \sqrt{9+4x^2} dx\)
\item
  \(\displaystyle\int \frac{x}{\sqrt{4-x^4}} dx\) =
  \(\frac{1}{4}\left(2x\sqrt{4x^2+9}+9\ln \left|\frac{2x+\sqrt{4x^2+9}}{3}\right|\right)+C\)
\item
  \(\displaystyle\int \frac{1}{x^2\sqrt{x^2-4}} dx\) =
  \(\frac{1}{4}\sqrt{1-\frac{4}{x^2}}+C\)
\item
  \(\displaystyle\int \frac{x+16}{\sqrt{x^2-4x+8}} dx\) =
  \(\sqrt{x^2-4x+8}+18\ln \left|\frac{1}{2}x-1+\sqrt{\frac{1}{4}x^2-x+2}\right|+C\)
\item
  \(\displaystyle\int \frac{x+1}{\sqrt{4-x^2}} dx\) =
  \(-\sqrt{4-x^2}+\arcsin \left(\frac{1}{2}x\right)+C\)
\end{itemize}

\hypertarget{partial-fraction-decomposition}{%
\subsubsection{Partial Fraction
Decomposition}\label{partial-fraction-decomposition}}

\begin{itemize}
\tightlist
\item
  \(\displaystyle\int \frac{7x + 5}{x^2 + x - 2} dx\) =
  \(7\left(\frac{2}{3}\ln \left|x+2\right|+\frac{1}{3}\ln \left|x-1\right|\right)-\frac{5}{3}\left(\ln \left|\frac{2x+4}{3}\right|-\ln \left|\frac{2x-2}{3}\right|\right)+C\)
\item
  \(\displaystyle \int \frac{4x^2 + x + 7}{x^2 + 1}\) =
  \(4x+3\arctan \left(x\right)+\frac{1}{2}\ln \left|x^2+1\right|+C\)
\item
  \(\displaystyle \int \frac{x^3 + 4x^2}{x^2 + 4x + 3}dx\) =
  \(\frac{1}{2}x^2+\frac{13}{2}\ln \left|x^2+4x+3\right|-11\ln \left|x+3\right|-5\ln \left|x+1\right|-10+C\)
\item
  \(\displaystyle \int \frac{x^3 - x^2 - 3x + 1}{x^2 - x - 6}dx\)
\item
  \(\displaystyle \int \frac{2x^3 + 2x^2 - 9x - 1}{x^2 + x - 6}dx\)
\item
  \(\displaystyle \int \frac{3x^3 - 12x^2 + 15x -5}{x^2 - 4x + 4}dx\) =
  \(-\frac{3x^3-12x^2+15x-5}{x-2}+\frac{9}{2}x^2-6x+3\ln \left|x-2\right|-6+C\)
\item
  \(\displaystyle \int \frac{2x-3}{x^3 + x}dx\) =
  \(2\arctan \left(x\right)-3\left(\ln \left|x\right|-\frac{1}{2}\ln \left|x^2+1\right|\right)+C\)
\item
  \(\displaystyle \int \frac{4x^2 - 11x - 19}{(x-5)(x^2 + 1)}dx\) =
  \(\ln \left|x-5\right|+\frac{3}{2}\ln \left|x^2+1\right|+4\arctan \left(x\right)+C\)
\item
  \(\displaystyle \int \frac{3x^2 + 2x - 7}{x^2(x-5) + (x-5)}dx\) =
\item
  \(\displaystyle \int \frac{x^4 - 2x^3 - 8x^2 + 2x + 10 }{x^2 - 2x - 8}dx\)
\item
  \(\displaystyle \int \frac{3x^2 + x + 3}{x(x^2+1)}dx\) =
  \(\arctan \left(x\right)+3\ln \left|x\right|+C\)
\item
  \(\displaystyle \int \frac{-2x+4}{(x^2+1)(x-1)^2}dx\) =
  \(\ln \left|x^2+1\right|+\arctan \left(x\right)-2\ln \left|x-1\right|-\frac{1}{x-1}+C\)
\item
  \(\displaystyle \int \frac{2x^5 + 4x^3 + 4x}{(x^4+1)}dx\) =
  \(x^2+\arctan \left(x^2\right)+\ln \left|x^4+1\right|+C\)
\item
  \(\displaystyle \int \frac{x^3 + 4x^2}{x^2 + 4x + 3}dx\) =
  \(\frac{1}{2}x^2+\frac{13}{2}\ln \left|x^2+4x+3\right|-11\ln \left|x+3\right|-5\ln \left|x+1\right|-10+C\)
\item
  \(\displaystyle \int \frac{x(x^2 - 3x + 5)}{x^2 - 2x + 1}dx\) =
  \(-\frac{x\left(x^2-3x+5\right)}{x-1}+\frac{3}{2}x^2-3x+\frac{3}{2}+2\ln \left|x-1\right|+C\)
\item
  \(\displaystyle \int \frac{x^4 + 2x^3 + 5x^2 + 4x + 6}{x^3 + x^2 + x + 1}dx\)
\end{itemize}
