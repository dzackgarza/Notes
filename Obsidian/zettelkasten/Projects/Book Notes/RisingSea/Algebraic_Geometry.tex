\RequirePackage[l2tabu, orthodox]{nag}

%\documentclass[]{article}
\documentclass[11pt]{scrartcl}
\usepackage[usename, dvipsnames]{xcolor}
\usepackage[pdfencoding=auto]{hyperref}
\usepackage[msc-links]{amsrefs}
\usepackage{cleveref} % use \cref{}, automatically deduces theorem, proposition, etc
\usepackage[mathletters]{ucs}
\usepackage[utf8]{inputenc}
\usepackage[T1]{fontenc}
\usepackage{datetime}

\usepackage{array}
\usepackage{mathtools}
\usepackage{amsmath, amsthm, amssymb, amsfonts, amsxtra, amscd, thmtools}
\let\proof\relax
\let\endproof\relax

% Boxes around theorem environments.
\usepackage[many]{tcolorbox}

\usepackage{color}
%\usepackage{unicode-math}
\usepackage{newunicodechar}
\newunicodechar{ε}{\varepsilon}
\newunicodechar{δ}{\delta}
\newunicodechar{µ}{\mu}
\newunicodechar{→}{\to}
\newunicodechar{≤}{\leq}
\newunicodechar{∈}{\in}
\newunicodechar{⊆}{\subseteq}
\newunicodechar{Λ}{\Lambda}
\newunicodechar{∞}{\infty}
\newunicodechar{×}{\times}
\everymath{\displaystyle}



\usepackage{microtype}
\usepackage[pdfencoding=auto]{hyperref}
\usepackage{bookmark}
\usepackage{booktabs}
\usepackage{todonotes}
\usepackage[msc-links]{amsrefs}
\usepackage{cleveref} % use \cref{}, automatically deduces theorem, proposition, etc
\usepackage{csquotes}
\usepackage{longtable}
\usepackage{tabularx}
\usepackage{bbm}
% Creating multiple types of index
\usepackage{imakeidx}

% Remove indentation for new paragraphs
\usepackage{parskip}
% But leave space before amsthm environments
\makeatletter
\def\thm@space@setup{%
  \thm@preskip=2em
  \thm@postskip=2em
}
\makeatother


\usepackage{stmaryrd}
\usepackage{adjustbox}
\usepackage{centernot}
% \centernot\whatever


% Better indicator function
\usepackage{bbm}
\newcommand{\indic}[1]{\mathbbm{1} \left[ {#1} \right] }

% Highlight quote
\usepackage{environ}
\definecolor{camel}{rgb}{0.76, 0.6, 0.42}
\definecolor{babyblue}{rgb}{0.54, 0.81, 0.94}
\definecolor{block-gray}{gray}{0.85}
\NewEnviron{myblock}
{\colorbox{block-gray}{%
\parbox{\dimexpr\linewidth-2\fboxsep\relax}{%
\small\addtolength{\leftskip}{10mm}
\addtolength{\rightskip}{10mm}
\BODY}}
}
\renewcommand{\quote}{\myblock}
\renewcommand{\endquote}{\endmyblock}

% Nice math font that journals use
%\usepackage[lite]{mtpro2}
%\usepackage{mathrsfs}
%\usepackage{mathptmx}
\usepackage{lmodern}
%\usepackage[sc]{mathpazo}

% Theorem Styles
\usepackage[framemethod=tikz]{mdframed}

\theoremstyle{definition}
\newtheorem{exercise}{Exercise}[section]
\newtheorem{solution}{Solution}

% Theorem Style
\newtheoremstyle{theorem}% name
  {0em}%         Space above, empty = `usual value'
  {1em}%         Space below
  {\normalfont}% Body font
  {\parindent}%         Indent amount (empty = no indent, \parindent = para indent)
  {\bfseries}% Thm head font
  {.}%        Punctuation after thm head
  {\newline}% Space after thm head: \newline = linebreak
  {\thmname{#1}\thmnumber{ #2}\thmnote{\itshape{(#3)}}}%
\theoremstyle{theorem}
\tcolorboxenvironment{theorem}{
  boxrule=0pt,
  boxsep=0pt,
  breakable,
  enhanced jigsaw,
  fonttitle={\large\bfseries},
  opacityback=0.8,
  colframe=cyan,
  borderline west={4pt}{0pt}{orange},
  attach title to upper={}
}
\newtheorem{theorem}{Theorem}[section]

% Proposition Style
\tcolorboxenvironment{proposition}{
  boxrule=1pt,
  boxsep=0pt,
  breakable,
  enhanced jigsaw,
  opacityback=0.0,
  colframe=cyan
}
\newtheorem{proposition}[theorem]{Proposition}
\tcolorboxenvironment{lemma}{
  boxrule=1pt,
  boxsep=0pt,
  breakable,
  enhanced jigsaw,
  opacityback=0.2,
  colframe=cyan
}
\newtheorem{lemma}[theorem]{Lemma}
% Claim
\tcolorboxenvironment{claim}{
  boxrule=1pt,
  boxsep=0pt,
  breakable,
  enhanced jigsaw,
  opacityback=0.2,
  colframe=cyan
}
\newtheorem{claim}[theorem]{Claim}


% Corollary
\tcolorboxenvironment{corollary}{
  colback=cyan,
  boxrule=1pt,
  boxsep=0pt,
  breakable,
  enhanced jigsaw,
  opacityback=0.1,
  colframe=cyan
}
\newtheorem{corollary}[theorem]{Corollary}

% Proof Style
\newtheoremstyle{proof}% name
  {0em}%         Space above, empty = `usual value'
  {2em}%         Space below
  {\normalfont}% Body font
  {\parindent}%         Indent amount (empty = no indent, \parindent = para indent)
  {\itshape}% Thm head font
  {.}%        Punctuation after thm head
  {\newline}% Space after thm head: \newline = linebreak
  {\thmname{#1} \thmnote{\itshape{(#3)}}}%         Thm head spec
\theoremstyle{proof}
\tcolorboxenvironment{proof}{
  colback=camel,
  opacityfill=0.25,
  boxrule=1pt,
  boxsep=0pt,
  breakable,
  enhanced jigsaw
}
\newtheorem*{pf}{Proof}
\newenvironment{proof}
{\pushQED{$\qed$}\pf}
{\par\popQED\endpf}

% Definition Style
\newtheoremstyle{definition}% name
  {0em}%         Space above, empty = `usual value'
  {2em}%         Space below
  {\normalfont}% Body font
  {\parindent}%         Indent amount (empty = no indent, \parindent = para indent)
  {\bfseries}% Thm head font
  {.}%        Punctuation after thm head
  {\newline}% Space after thm head: \newline = linebreak
  {}%         Thm head spec
\theoremstyle{definition}
\tcolorboxenvironment{definition}{
  colback=babyblue,
  boxrule=0pt,
  boxsep=0pt,
  opacityfill=0.45,
  breakable,
  enhanced jigsaw,
  borderline west={4pt}{0pt}{blue},
  colbacktitle={babyblue},
  coltitle={black},
  fonttitle={\large\bfseries},
  attach title to upper={},
}
\newtheorem{definition}{Definition}[theorem]

% Break Environment
\makeatletter
\newtheoremstyle{break}% name
  {}%         Space above, empty = `usual value'
  {2em}%         Space below
  {
    \addtolength{\@totalleftmargin}{2.5em}
    \addtolength{\linewidth}{-2.5em}
    \parshape 1 2.5em \linewidth
  }% Body font
  {}%         Indent amount (empty = no indent, \parindent = para indent)
  {\bfseries}% Thm head font
  {.}%        Punctuation after thm head
  {\newline}% Space after thm head: \newline = linebreak
  {}%         Thm head spec
\makeatother

\theoremstyle{break}
\newtheorem{example}{Example}[section]

% Problem Style
\newtheoremstyle{problem} % name
  {0em}                   % Space above, empty = `usual value'
  {2em}                   % Space below
  {\normalfont}           % Body font
  {\parindent}            % Indent amount (empty = no indent, \parindent = para indent)
  {\itshape}              % Thm head font
  {}                      % Punctuation after thm head
  {\newline}              % Space after thm head: \newline = linebreak
  {\thmnote{\itshape{(#3)}}}     % Thm head spec
\theoremstyle{problem}
\tcolorboxenvironment{problem}{
  boxrule=1pt,
  boxsep=0pt,
  breakable,
  enhanced jigsaw,
  opacityback=0.0,
  colframe=cyan
}
\newtheorem{problem}{Problem}


%Pagination stuff.
\setlength{\topmargin}{-.3 in}
\setlength{\oddsidemargin}{0in}
\setlength{\evensidemargin}{0in}
\setlength{\textheight}{9.in}
\setlength{\textwidth}{6.5in}
% \pagestyle{empty} %removes page numbers.

% Inkscape figures from Vim
\usepackage{import}
\usepackage{pdfpages}
\usepackage{transparent}

\newcommand{\incfig}[1]{%
    \def\svgwidth{\columnwidth}
    \import{./figures/}{#1.pdf_tex}
}
%\pdfsuppresswarningpagegroup=1

% Pandoc-specific fixes
\providecommand{\tightlist}{%
  \setlength{\itemsep}{0pt}\setlength{\parskip}{0pt}}

% Tikz and Graphics
\usepackage{amscd}
\usepackage{tikz}
\usetikzlibrary{arrows, arrows.meta, cd, fadings, patterns, calc, decorations.markings, matrix, positioning}
\tikzfading[name=fade out, inner color=transparent!0, outer color=transparent!100]
\usepackage{pgfplots}
\pgfplotsset{compat=1.16}
\usepackage[inline]{asymptote}
\usepackage{tikz-layers}

%\usepackage{nath}
%\delimgrowth=1
\DeclarePairedDelimiter\qty{(}{)}

% Major Macros
\usepackage{graphicx}
\usepackage{float}
\DeclareFontFamily{U}{mathx}{\hyphenchar\font45}
\DeclareFontShape{U}{mathx}{m}{n}{
      <5> <6> <7> <8> <9> <10>
      <10.95> <12> <14.4> <17.28> <20.74> <24.88>
      mathx10
      }{}
\DeclareSymbolFont{mathx}{U}{mathx}{m}{n}
\DeclareMathSymbol{\bigtimes}{1}{mathx}{"91}

% Wide tikz equations
\newsavebox{\wideeqbox}
\newenvironment{wideeq}
  {\begin{displaymath}\begin{lrbox}{\wideeqbox}$\displaystyle}
  {$\end{lrbox}\makebox[0pt]{\usebox{\wideeqbox}}\end{displaymath}}



% Fancy chapter headers and footers
\usepackage{fancyhdr}

\pagestyle{fancy}
\fancyhf{}
\fancyhead[LE,RO]{\title}
\fancyhead[RE,LO]{\rightmark}
\fancyfoot[CE,CO]{\leftmark}
\fancyfoot[LE,RO]{\thepage}

\renewcommand{\headrulewidth}{2pt}
\renewcommand{\footrulewidth}{1pt}

% List of Theorems Attempt
\usepackage{etoolbox}
\makeatletter
\patchcmd\thmtlo@chaptervspacehack
  {\addtocontents{loe}{\protect\addvspace{10\p@}}}
  {\addtocontents{loe}{\protect\thmlopatch@endchapter\protect\thmlopatch@chapter{\thechapter}}}
  {}{}
\AtEndDocument{\addtocontents{loe}{\protect\thmlopatch@endchapter}}
\long\def\thmlopatch@chapter#1#2\thmlopatch@endchapter{%
  \setbox\z@=\vbox{#2}%
  \ifdim\ht\z@>\z@
    \hbox{\bfseries\chaptername\ #1}\nobreak
    #2
    \addvspace{10\p@}
  \fi
}
\def\thmlopatch@endchapter{}

\makeatother
\renewcommand{\thmtformatoptarg}[1]{ -- #1}
%\renewcommand{\listtheoremname}{List of definitions}

\newcommand{\ext}{\operatorname{Ext}}
\newcommand{\Ext}{\operatorname{Ext}}
\def\Endo{\operatorname{End}}
\def\Ind{\operatorname{Ind}}
\def\ind{\operatorname{Ind}}
\def\coind{\operatorname{Coind}}
\def\Res{\operatorname{Res}}
\def\Hol{\operatorname{Hol}}
\def\res{\operatorname{Res}}
\def\endo{\operatorname{End}}
\def\ind{\operatorname{Ind}}
\renewcommand{\AA}[0]{{\mathbb{A}}}
\DeclareMathOperator{\Exists}{\exists}
\DeclareMathOperator{\Forall}{\forall}
\newcommand{\Af}[0]{{\mathbb{A}}}
\newcommand{\CC}[0]{{\mathbb{C}}}
\newcommand{\CP}[0]{{\mathbb{CP}}}
\newcommand{\DD}[0]{{\mathbb{D}}}
\newcommand{\FF}[0]{{\mathbb{F}}}
\newcommand{\GF}[0]{{\mathbb{GF}}}
\newcommand{\GG}[0]{{\mathbb{G}}}
\newcommand{\HH}[0]{{\mathbb{H}}}
\newcommand{\HP}[0]{{\mathbb{HP}}}
\newcommand{\KK}[0]{{\mathbb{K}}}
\newcommand{\kk}[0]{{\Bbbk}}
\newcommand{\bbm}[0]{{\mathbb{M}}}
\newcommand{\NN}[0]{{\mathbb{N}}}
\newcommand{\OP}[0]{{\mathbb{OP}}}
\newcommand{\PP}[0]{{\mathbb{P}}}
\newcommand{\QQ}[0]{{\mathbb{Q}}}
\newcommand{\RP}[0]{{\mathbb{RP}}}
\newcommand{\RR}[0]{{\mathbb{R}}}
\newcommand{\SpSp}[0]{{\mathbb{S}}}
\renewcommand{\SS}[0]{{\mathbb{S}}}
\newcommand{\TT}[0]{{\mathbb{T}}}
\newcommand{\ZZ}[0]{{\mathbb{Z}}}
\newcommand{\ZnZ}[0]{\mathbb{Z}/n\mathbb{Z}}
\newcommand{\ZpZ}[0]{\mathbb{Z}/p\mathbb{Z}}
\newcommand{\Qp}[0]{\mathbb{Q}_{(p)}}
\newcommand{\Zp}[0]{\mathbb{Z}_{(p)}}
\newcommand{\Arg}[0]{\mathrm{Arg}}
\newcommand{\PGL}[0]{\mathrm{PGL}}
\newcommand{\GL}[0]{\mathrm{GL}}
\newcommand{\Gl}[0]{\mathrm{GL}}
\newcommand{\gl}[0]{\mathrm{GL}}
\newcommand{\mat}[0]{\mathrm{Mat}}
\newcommand{\Mat}[0]{\mathrm{Mat}}
\newcommand{\Rat}[0]{\mathrm{Rat}}
\newcommand{\Perv}[0]{\mathrm{Perv}}
\newcommand{\Gal}[0]{\mathrm{Gal}}
\newcommand{\Hilb}[0]{\mathrm{Hilb}}
\newcommand{\Quot}[0]{\mathrm{Quot}}
\newcommand{\Art}[0]{\mathrm{Art}}
\newcommand{\red}[0]{\mathrm{red}}
\newcommand{\alg}[0]{\mathrm{alg}}
\newcommand{\Pic}[0]{{\mathrm{Pic}~}}
\newcommand{\lcm}[0]{\mathrm{lcm}}
\newcommand{\maps}[0]{\mathrm{Maps}}
\newcommand{\maxspec}[0]{{\mathrm{maxSpec}~}}
\newcommand{\Tr}[0]{\mathrm{Tr}}
\newcommand{\adj}[0]{\mathrm{adj}}
\newcommand{\ad}[0]{\mathrm{ad}~}
\newcommand{\ann}[0]{\mathrm{Ann}}
\newcommand{\Ann}[0]{\mathrm{Ann}}
\newcommand{\arcsec}[0]{\mathrm{arcsec}}
\newcommand{\ch}[0]{\mathrm{char}~}
\newcommand{\Sp}[0]{{\mathrm{Sp}}}
\newcommand{\syl}[0]{{\mathrm{Syl}}}
\newcommand{\txand}[0]{{\text{ and }}}
\newcommand{\codim}[0]{\mathrm{codim}}
\newcommand{\txor}[0]{{\text{ or }}}
\newcommand{\txt}[1]{{\text{ {#1} }}}
\newcommand{\Gr}[0]{{\text{Gr}}}
\newcommand{\Aut}[0]{{\mathrm{Aut}}}
\newcommand{\aut}[0]{\mathrm{Aut}}
\newcommand{\Inn}[0]{{\mathrm{Inn}}}
\newcommand{\Out}[0]{{\mathrm{Out}}}
\newcommand{\mltext}[1]{\left\{\begin{array}{c}#1\end{array}\right\}}
\newcommand{\Fun}[0]{{\text{Fun}}}
\newcommand{\SL}[0]{{\text{SL}}}
\newcommand{\PSL}[0]{{\text{PSL}}}
\newcommand{\SO}[0]{{\text{SO}}}
\newcommand{\SU}[0]{{\text{SU}}}
\newcommand{\SP}[0]{{\text{SP}}}
\newcommand{\per}[0]{{\text{Per}}}
\newcommand{\loc}[0]{{\text{loc}}}
\newcommand{\Top}[0]{{\text{Top}}}
\newcommand{\Sch}[0]{{\text{Sch}}}
\newcommand{\sch}[0]{{\text{Sch}}}
\newcommand{\Set}[0]{{\text{Set}}}
\newcommand{\Sets}[0]{{\text{Set}}}
\newcommand{\Grp}[0]{{\text{Grp}}}
\newcommand{\Groups}[0]{{\text{Groups}}}
\newcommand{\Homeo}[0]{{\text{Homeo}}}
\newcommand{\Diffeo}[0]{{\text{Diffeo}}}
\newcommand{\MCG}[0]{{\text{MCG}}}
\newcommand{\set}[0]{{\text{Set}}}
\newcommand{\Tor}[0]{\text{Tor}}
\newcommand{\sets}[0]{{\text{Set}}}
\newcommand{\Sm}[0]{{\text{Sm}_k}}
\newcommand{\orr}[0]{{\text{ or }}}
\newcommand{\annd}[0]{{\text{ and }}}
\newcommand{\bung}[0]{\text{Bun}_G}
\newcommand{\const}[0]{{\text{const.}}}
\newcommand{\disc}[0]{{\text{disc}}}
\newcommand{\op}[0]{^\text{op}}
\newcommand{\id}[0]{\text{id}}
\newcommand{\im}[1]{\mathrm{im}({#1})}
\newcommand{\pt}[0]{{\{\text{pt}\}}}
\newcommand{\sep}[0]{^\text{sep}}
% \newcommand{\st}[0]{~{\text{s.t.}}~}
\newcommand{\tors}[0]{{\text{tors}}}
\newcommand{\tor}[0]{\text{Tor}}
\newcommand{\height}[0]{\text{ht}}
\newcommand{\cpt}[0]{\text{compact}}
\newcommand{\abs}[1]{{\left\lvert {#1} \right\rvert}}
\newcommand{\stack}[1]{\mathclap{\substack{ #1 }}} 
\newcommand{\qtext}[1]{{\quad \text{#1} \quad}}
\newcommand{\qst}[0]{{\quad \text{such that} \quad}}
\newcommand{\actsonl}[0]{\curvearrowleft}
\newcommand{\actson}[0]{\curvearrowright}
\newcommand{\bd}[0]{{\del}}
\newcommand{\bigast}[0]{{\mathop{\Large \ast}}}
\newcommand{\coker}[0]{\operatorname{coker}}
\newcommand{\cok}[0]{\operatorname{coker}}
\newcommand{\conjugate}[1]{{\overline{{#1}}}}
\newcommand{\converges}[1]{\overset{#1}}
\newcommand{\correspond}[1]{\theset{\substack{#1}}}
\newcommand{\cross}[0]{\times}
\newcommand{\by}[0]{\times}
\newcommand{\dash}[0]{{\hbox{-}}}
\newcommand{\dd}[2]{{\frac{\partial #1}{\partial #2}\,}}
\newcommand{\definedas}[0]{\coloneqq}
\newcommand{\da}[0]{\coloneqq}
\newcommand{\del}[0]{{\partial}}
\newcommand{\directlim}[0]{\varinjlim}
\newcommand{\disjoint}[0]{{\coprod}}
\newcommand{\divides}[0]{{~\Bigm|~}}
\newcommand{\dual}[0]{^\vee}
\newcommand{\sm}[0]{\setminus}
\newcommand{\smz}[0]{\setminus\theset{0}}
\newcommand{\eps}[0]{\varepsilon}
\newcommand{\equalsbecause}[1] {\stackrel{\mathclap{\scriptscriptstyle{#1}}}{=}}
\newcommand{\floor}[1]{{\left\lfloor #1 \right\rfloor}}
\DeclarePairedDelimiter{\ceil}{\lceil}{\rceil}
\newcommand{\from}[0]{\leftarrow}
\newcommand{\tofrom}[0]{\leftrightarrows}
\newcommand{\up}[0]{\uparrow}
\newcommand{\generators}[1]{\left\langle{#1}\right\rangle}
\newcommand{\gs}[1]{\left\langle{#1}\right\rangle}
\newcommand{\homotopic}[0]{\simeq}
\newcommand{\injectivelim}[0]{\varinjlim}
\newcommand{\injects}[0]{\hookrightarrow}
\newcommand{\inner}[2]{{\left\langle {#1},~{#2} \right\rangle}}
\newcommand{\union}[0]{\cup}
\newcommand{\Union}[0]{\bigcup}
\newcommand{\intersect}[0]{\cap}
\newcommand{\Intersect}[0]{\bigcap}
\newcommand{\into}[0]{\to}
\newcommand{\inverselim}[0]{\varprojlim}
\newcommand{\inv}[0]{^{-1}}
\newcommand{\mfa}[0]{{\mathfrak{a}}}
\newcommand{\mfb}[0]{{\mathfrak{b}}}
\newcommand{\mfc}[0]{{\mathfrak{c}}}
\newcommand{\mff}[0]{{\mathfrak{f}}}
\newcommand{\mfi}[0]{{\mathfrak{I}}}
\newcommand{\mfm}[0]{{\mathfrak{m}}}
\newcommand{\mfn}[0]{{\mathfrak{n}}}
\newcommand{\mfp}[0]{{\mathfrak{p}}}
\newcommand{\mfq}[0]{{\mathfrak{q}}}
\newcommand{\mfr}[0]{{\mathfrak{r}}}
\newcommand{\lieb}[0]{{\mathfrak{b}}}
\newcommand{\liegl}[0]{{\mathfrak{gl}}}
\newcommand{\lieg}[0]{{\mathfrak{g}}}
\newcommand{\lieh}[0]{{\mathfrak{h}}}
\newcommand{\lien}[0]{{\mathfrak{n}}}
\newcommand{\liesl}[0]{{\mathfrak{sl}}}
\newcommand{\lieso}[0]{{\mathfrak{so}}}
\newcommand{\liesp}[0]{{\mathfrak{sp}}}
\newcommand{\lieu}[0]{{\mathfrak{u}}}
\newcommand{\nilrad}[0]{{\mathfrak{N}}}
\newcommand{\jacobsonrad}[0]{{\mathfrak{J}}}
\newcommand{\mm}[0]{{\mathfrak{m}}}
\newcommand{\pr}[0]{{\mathfrak{p}}}
\newcommand{\mapsvia}[1]{\xrightarrow{#1}}
\newcommand{\kx}[1]{k[x_1, \cdots, x_{#1}]}
\newcommand{\MM}[0]{{\mathcal{M}}}
\newcommand{\OO}[0]{{\mathcal{O}}}
\newcommand{\imaginarypart}[1]{{\mathcal{Im}({#1})}}
\newcommand{\mca}[0]{{\mathcal{A}}}
\newcommand{\mcb}[0]{{\mathcal{B}}}
\newcommand{\mcc}[0]{{\mathcal{C}}}
\newcommand{\mcd}[0]{{\mathcal{D}}}
\newcommand{\mce}[0]{{\mathcal{E}}}
\newcommand{\mcf}[0]{{\mathcal{F}}}
\newcommand{\mcg}[0]{{\mathcal{G}}}
\newcommand{\mch}[0]{{\mathcal{H}}}
\newcommand{\mci}[0]{{\mathcal{I}}}
\newcommand{\mcj}[0]{{\mathcal{J}}}
\newcommand{\mck}[0]{{\mathcal{K}}}
\newcommand{\mcl}[0]{{\mathcal{L}}}
\newcommand{\mcm}[0]{{\mathcal{M}}}
\newcommand{\mcp}[0]{{\mathcal{P}}}
\newcommand{\mcs}[0]{{\mathcal{S}}}
\newcommand{\mct}[0]{{\mathcal{T}}}
\newcommand{\mcu}[0]{{\mathcal{U}}}
\newcommand{\mcv}[0]{{\mathcal{V}}}
\newcommand{\mcx}[0]{{\mathcal{X}}}
\newcommand{\mcz}[0]{{\mathcal{Z}}}
\newcommand{\cl}[0]{\mathrm{cl}}
\newcommand{\trdeg}[0]{\mathrm{trdeg}}
\newcommand{\dist}[0]{\mathrm{dist}}
\newcommand{\Dist}[0]{\mathrm{Dist}}
\newcommand{\crit}[0]{\mathrm{crit}}
\newcommand{\diam}[0]{{\mathrm{diam}}}
\newcommand{\gal}[0]{\mathrm{Gal}}
\newcommand{\diff}[0]{\mathrm{Diff}}
\newcommand{\diag}[0]{\mathrm{diag}}
\newcommand{\soc}[0]{\mathrm{Soc}\,}
\newcommand{\hd}[0]{\mathrm{Head}\,}
\newcommand{\grad}[0]{\mathrm{grad}~}
\newcommand{\hilb}[0]{\mathrm{Hilb}}
\newcommand{\minpoly}[0]{{\mathrm{minpoly}}}
\newcommand{\Hom}[0]{{\mathrm{Hom}}}
\newcommand{\Map}[0]{{\mathrm{Map}}}
\newcommand{\multinomial}[1]{\left(\!\!{#1}\!\!\right)}
\newcommand{\nil}[0]{{\mathrm{nil}}}
\newcommand{\normalneq}{\mathrel{\reflectbox{$\trianglerightneq$}}}
\newcommand{\normal}[0]{{~\trianglelefteq~}}
\newcommand{\norm}[1]{{\left\lVert {#1} \right\rVert}}
\newcommand{\pnorm}[2]{{\left\lVert {#1} \right\rVert}_{#2}}
\newcommand{\notdivides}[0]{\nmid}
\newcommand{\onto}[0]{\twoheadhthtarrow}
\newcommand{\ord}[0]{{\mathrm{Ord}}}
\newcommand{\pic}[0]{{\mathrm{Pic}~}}
\newcommand{\projectivelim}[0]{\varprojlim}
\newcommand{\rad}[0]{{\mathrm{rad}~}}
\newcommand{\ralg}[0]{\mathrm{R-alg}}
\newcommand{\kalg}[0]{k\dash\mathrm{alg}}
\newcommand{\rank}[0]{\operatorname{rank}}
\newcommand{\realpart}[1]{{\mathcal{Re}({#1})}}
\newcommand{\Log}[0]{\mathrm{Log}}
\newcommand{\reg}[0]{\mathrm{Reg}}
\newcommand{\restrictionof}[2]{{\left.{#1}\right|_{#2}}}
\newcommand{\ro}[2]{{\left.{#1}\right|_{#2}}}
\newcommand{\rk}[0]{{\mathrm{rank}}}
\newcommand{\evalfrom}[0]{\Big|}
\newcommand{\rmod}[0]{{R\dash\mathrm{mod}}}
\newcommand{\Mod}[0]{{\mathrm{Mod}}}
\newcommand{\rotate}[2]{{\style{display: inline-block; transform: rotate(#1deg)}{#2}}}
\newcommand{\selfmap}[0]{{\circlearrowleft}}
\newcommand{\semidirect}[0]{\rtimes}
\newcommand{\sgn}[0]{\mathrm{sgn}}
\newcommand{\sign}[0]{\mathrm{sign}}
\newcommand{\spanof}[0]{{\mathrm{span}}}
\newcommand{\spec}[0]{\mathrm{Spec}\,}
\newcommand{\mspec}[0]{\mathrm{mSpec}~}
\newcommand{\stab}[0]{{\mathrm{Stab}}}
\newcommand{\stirlingfirst}[2]{\genfrac{[}{]}{0pt}{}{#1}{#2}}
\newcommand{\stirling}[2]{\genfrac\{\}{0pt}{}{#1}{#2}}
\newcommand{\strike}[1]{{\enclose{horizontalstrike}{#1}}}
\newcommand{\suchthat}[0]{{~\mathrel{\Big|}~}}
\newcommand{\st}[0]{{~\mathrel{\Big|}~}}
\newcommand{\supp}[0]{{\mathrm{supp}}}
\newcommand{\surjects}[0]{\twoheadrightarrow}
\newcommand{\sym}[0]{\mathrm{Sym}}
\newcommand{\tensor}[0]{\otimes}
\newcommand{\connectsum}[0]{\mathop{\Large \#}}
\newcommand{\theset}[1]{\left\{{#1}\right\}}
\newcommand{\ts}[1]{\left\{{#1}\right\}}
\newcommand{\gens}[1]{\left\langle{#1}\right\rangle}
\newcommand{\thevector}[1]{{\left[ {#1} \right]}}
\newcommand{\tv}[1]{{\left[ {#1} \right]}}
\newcommand{\too}[1]{{\xrightarrow{#1}}}
\newcommand{\transverse}[0]{\pitchfork}
\newcommand{\trianglerightneq}{\mathrel{\ooalign{\raisebox{-0.5ex}{\reflectbox{\rotatebox{90}{$\nshortmid$}}}\cr$\triangleright$\cr}\mkern-3mu}}
\newcommand{\tr}[0]{\mathrm{Tr}}
\newcommand{\uniformlyconverges}[0]{\rightrightarrows}
\newcommand{\covers}[0]{\rightrightarrows}
\newcommand{\units}[0]{^{\times}}
\newcommand{\nonzero}[0]{^{\bullet}}
\newcommand{\wait}[0]{{\,\cdot\,}}
\newcommand{\wt}[0]{{\mathrm{wt}}}
\renewcommand{\bar}[1]{\mkern 1.5mu\overline{\mkern-1.5mu#1\mkern-1.5mu}\mkern 1.5mu}
\renewcommand{\div}[0]{\mathrm{Div}}
\newcommand{\Div}[0]{\mathrm{Div}}
\renewcommand{\hat}[1]{\widehat{#1}}
\renewcommand{\mid}[0]{\mathrel{\Big|}}
\renewcommand{\qed}[0]{\hfill\blacksquare}
\renewcommand{\too}[0]{\longrightarrow}
\renewcommand{\vector}[1]{\mathbf{#1}}
\let\oldexp\exp
\renewcommand{\exp}[1]{\oldexp\qty{#1}}
\let\oldperp\perp
\renewcommand{\perp}[0]{^\oldperp}
\newcommand*\dif{\mathop{}\!\mathrm{d}}
\newcommand{\ddt}{\tfrac{\dif}{\dif t}}
\newcommand{\ddx}{\tfrac{\dif}{\dif x}}

\DeclareMathOperator{\righttriplearrows} {{\; \tikz{ \foreach \y in {0, 0.1, 0.2} { \draw [-stealth] (0, \y) -- +(0.5, 0);}} \; }}


\let\Begin\begin
\let\End\end
\newcommand\wrapenv[1]{#1}

\makeatletter
\def\ScaleWidthIfNeeded{%
 \ifdim\Gin@nat@width>\linewidth
    \linewidth
  \else
    \Gin@nat@width
  \fi
}
\def\ScaleHeightIfNeeded{%
  \ifdim\Gin@nat@height>0.9\textheight
    0.9\textheight
  \else
    \Gin@nat@width
  \fi
}
\makeatother

\setkeys{Gin}{width=\ScaleWidthIfNeeded,height=\ScaleHeightIfNeeded,keepaspectratio}%
% Create indices up front
\makeindex[title=Concept index]

\title{
\textbf{
    Title
  }
  }

\author{D. Zack Garza}

\date{\today}




\begin{document}
\maketitle
\tableofcontents


\hypertarget{week-1}{%
\section{Week 1}\label{week-1}}

\href{http://math.stanford.edu/~vakil/216blog/FOAGnov1817public.pdf}{Link
to Notes}

\hypertarget{exercise-1.3h-right-exactness-of-tensoring}{%
\subsection{Exercise 1.3H: Right Exactness of
Tensoring}\label{exercise-1.3h-right-exactness-of-tensoring}}

Show that the following endofunctor \begin{align*}
F: \rmod &\to \rmod \\
X &\mapsto X\tensor_R N \\
(X\mapsvia{f} Y) &\mapsto (X\tensor_R N \mapsvia{f \tensor \id_N} Y\tensor_R N)
\end{align*} is exact.

Solution:

\begin{quote}
Note: to make sense of the functor, we may need to show that there is an
isomorphism
\begin{align*}\hom_{\rmod}(X, Y) \tensor_R \hom_{\rmod}(A, B) \to \hom_\rmod(X\tensor_R A, Y\tensor_R B).\end{align*}
This is what makes taking \(f:X\to Y\) and \(g:A\to B\) and forming
\(f\tensor g: X\tensor A \to Y\tensor B\) well-defined?
\end{quote}

Let \(A\mapsvia{f} B \mapsvia{g} C \to 0\) be an exact sequence, so

\begin{itemize}
\tightlist
\item
  \(\im f = \ker g\) by exactness at \(B\)
\item
  \(\im g = C\) by exactness at \(C\).
\end{itemize}

Applying the above \(F\) yields \begin{align*}
A\tensor_R N \mapsvia{f\tensor \id_N} B\tensor_R N \mapsvia{g\tensor \id_N} C\tensor_R N \to 0
.\end{align*}

We thus need to show

\begin{enumerate}
\def\labelenumi{\arabic{enumi}.}
\tightlist
\item
  Exactness as \(C\tensor_R N\): \(\im(g\tensor \id_N) = C\tensor_R N\),
  i.e.~this is surjective.
\item
  Exactness at \(B\tensor_R N\):
  \(\im(f\tensor \id_N) = \ker(g\tensor id_N)\).
\end{enumerate}

We'll use the fact that every element in a tensor product is a finite
sum of elementary tensors.

\begin{itemize}
\tightlist
\item
  Claim: \(\im(g\tensor \id_N) \subseteq C\tensor_R N\).

  \begin{itemize}
  \tightlist
  \item
    Let \(b\tensor n \in B\tensor_R N\) be an elementary tensor
  \item
    Then
    \((g\tensor \id_N)(b\tensor n) \definedas g(b) \tensor \id_N (n) = g(b) \tensor n\)
  \item
    Since \(\im(g) = C\), there exists a \(c\in C\) such that
    \(g(b) = c\), so \(g(b) \tensor n = c \tensor n \in C\tensor_R N\)
  \item
    Extend by linearity:
  \end{itemize}
\end{itemize}

\begin{align*}
\qty{g\tensor_R \id_N}\qty{\sum_{i=1}^m r_i \cdot b_i \tensor n_i} =\sum_{i=1}^m (g\tensor \id_N)(r_i\cdot b_i \tensor n_i) \definedas \sum_{i=1}^m g(r_i\cdot b_i) \tensor \id_N(n_i) =_H \sum_{i=1}^m r_i\cdot c_i \tensor n_i \in C\tensor_R N
\end{align*}

where we've used bilinearity for the first equality, and the equality
marked with \(H\) uses above the proof for elementary tensors, and noted
that we can pull ring scalars \(r_i\in R\) through \(\rmod\) morphisms.
- Claim: \(C \tensor_R N \subseteq \im(g\tensor \id_N)\). - Let
\(c\tensor n \in C\tensor_R N\) be an elementary tensor. - Then
\(c\in C = \im(g)\) implies \(c = g(b)\) for some \(b\in B\). - So
\(c\tensor n = g(b) \tensor n = (g\tensor \id_N)(b\tensor n) \in B\tensor_R N\).
- Extend by linearity: \begin{align*}
  \sum_{i=1}^m r_i\cdot c_i \tensor n_i =_H \sum_{i=1}^m g(r_i\cdot b_i) \tensor n_i = \sum_{i=1}^m (g\tensor \id_N)(r_i\cdot b_i \tensor n_i) = (g\tensor \id_N)\qty{\sum_{i=1}^m r_i\cdot b_i \tensor n_i}
  .\end{align*}

This proves (1).

\begin{itemize}
\item
  Claim: \(\im(f\tensor \id_N) \subseteq \ker(g\tensor \id_N)\).

  \begin{itemize}
  \tightlist
  \item
    Let \(b\tensor n \in \im(f\tensor \id_N)\), we want to show
    \((g\tensor \id_N)(b\tensor n) = 0 \in C\tensor_R N\).
  \item
    Then \(b\tensor n = f(a)\tensor n\) for some \(a\in A\).
  \item
    By exactness of the original sequence, \(\im f \subseteq \ker g\),
    so \(g(f(a)) = 0 \in C\)
  \item
    Then \begin{align*}
    (g\tensor \id_N)\qty{ b \tensor n} = (g\tensor \id_N)(f(a)\tensor n) \definedas g(f(a)) \tensor n = 0\tensor n = 0\in C\tensor_R N
    \end{align*} where we've used the fact that \(0\tensor x = 0\) in
    any tensor product.
  \item
    Extend by linearity.
  \end{itemize}
\item
  Claim (\textbf{nontrivial part}):
  \(\ker(g\tensor \id_N) \subseteq \im(f\tensor \id_N)\).

  \begin{quote}
  Note: the problem is that \begin{align*}
  x\in \ker(g\tensor \id_N) \implies x = \sum_{i=1}^m r_i\cdot b_i \tensor n_i \implies (g\tensor \id_N)\qty{\sum_{i=1}^m r_i\cdot b_i \tensor n_i} = \sum_{i=1}^m r_i\cdot g(b_i) \tensor n_i = 0\in C\tensor_R N
  \end{align*} \textbf{but} this does not imply that \(g(b_i) = 0\in C\)
  for all \(i\), which is what you would need to use \(\im f = \ker g\)
  to write \(g(b_i) = 0\implies \exists a_i, f(a_i) = b_i\) and pull
  everything back to \(A\tensor_R N\).
  \end{quote}

  \begin{itemize}
  \tightlist
  \item
    Strategy: use the first claim and the first isomorphism theorem to
    obtain this situation:

    \begin{center}
    \begin{tikzcd}
    {B\tensor_R N \over \im(g\tensor_R \id_N)} \ar[r, hook, "i"]\ar[rrr, bend left, dotted, "\alpha"] & {B\tensor_R N \over \ker(g \tensor_R \id_N)} \ar[r, "\cong"] & \im(g\tensor_R \id_N) \ar[equal]{r} & C\tensor_R N
    \end{tikzcd}
    \end{center}
  \item
    The first injection \(i\) will exist because
    \(\im(g\tensor_R \id_N) \subseteq \ker(g\tensor_R \id_N)\) by the
    first claim.
  \item
    The middle isomorphism is the first isomorphism theorem.
  \item
    The RHS equality follows from surjectivity of \(g\tensor_R \id_N\)
  \item
    We then apply a strengthened version of the 1st isomorphism theorem
    for modules:
  \end{itemize}

  \begin{quote}
  Hungerford Ch.4 Thm 1.7: If \(f:A\to B\) is a \(R\dash\)module
  morphism and \(C\leq \ker f\) then there is a unique map
  \(\tilde f: A/C\to B\) which is an isomorphism iff \(f\) is an
  epimorphism and \(C = \ker f\).

  Following Hungerford Ch.4 Prop. 5.4, p.210.
  \end{quote}

  \begin{itemize}
  \item
    Since \(\im(g\tensor_R \id_N)\subseteq \ker(g\tensor_R \id_N)\), by
    the theorem the map \(\alpha\) exists and satisfies the same
    formula, i.e.~\(\alpha = \tilde g \tensor \tilde \id_N\) where the
    tilde denotes the induced map on quotients, so
    \(\alpha([b\tensor n]) = g(b)\tensor n\).

    \begin{itemize}
    \tightlist
    \item
      We will show it is an isomorphism, which forces
      \(\im(g\tensor_R \id_N) \cong \ker(g\tensor_R \id_N)\) by the
      above theorem.
    \end{itemize}
  \item
    Constructing the inverse map: define \begin{align*}
    \tilde \alpha\inv: C\times N &\to {B\tensor_R N \over \im(g\tensor_R \id_N) } \\
    (c, n) &\mapsto (b \tensor n)  \mod \im(g\tensor_R \id_N) \qtext{where} b \in g\inv(c)
    ,\end{align*} which we will show well-defined (i.e.~independent of
    choice of \(b\)) and \(R\dash\)linear, lifting to a map
    \(\alpha\inv\) out of the tensor product by the universal property
    which is a two-sided inverse for \(\alpha\).
  \item
    Well-defined:

    \begin{itemize}
    \tightlist
    \item
      \(g\inv(b)\) exists because \(g\) is surjective.
    \item
      If \(b\neq b'\) and \(g(b') = 0\), then
      \(0 = g(b) - g(b') = g(b-b')\) so \(b-b' \in \ker g\).
    \item
      By the original exactness, \(b-b' \in \im f\) so \(b-b' = f(a)\)
      for some \(a\in A\).
    \item
      Then \(f(a) \tensor n \in \im(f\tensor \id)\) implies
      \(f(a)\tensor n \equiv 0 \mod \im(f\tensor \id)\).
    \item
      Then noting that \(b-b' = f(a) \implies b = f(a) + b'\), working
      mod \(\im(g\tensor_R \id_N)\) we have \begin{align*}
      b \tensor n \equiv (f(a) + b') \tensor n \equiv \qty{f(a) \tensor n} + \qty{b' \tensor n} \equiv b'\tensor n 
      .\end{align*}
    \end{itemize}
  \item
    \(R\dash\)linear:

    \begin{itemize}
    \tightlist
    \item
      ?
    \end{itemize}
  \item
    Two-sided identity:

    \begin{itemize}
    \tightlist
    \item
      \((\alpha \circ \alpha\inv)(c\tensor n) = \alpha(b\tensor n) = g(b)\tensor n = c\tensor n\),
      so \(\alpha\circ \alpha\inv = \id\).
    \item
      \((\alpha\inv \circ \alpha)([b\tensor n]) = \alpha\inv(g(b) \tensor n) = [b'\tensor n]\)
      where \(b'\in g\inv(g(b))\) implies \(b'=b\), so
      \(\alpha\circ\alpha\inv = \id\).
    \end{itemize}
  \end{itemize}
\end{itemize}

\(\qed\)

\hypertarget{more-exercises}{%
\section{More Exercises}\label{more-exercises}}

\hypertarget{k}{%
\subsection{1.3.K}\label{k}}

\begin{quote}
Note: I think this is an exercise about base change.
\end{quote}

\textbf{Part a}: For \(M\) an \(A\dash\)module and \(\phi: A\to B\) a
morphism of rings, give \(B\tensor_A M\) the structure of a
\(B\dash\)module and show that it describes a functor
\(B\dash\text{Mod}\to A\dash\text{Mod}\).

Solution

\begin{itemize}
\item
  \(B\tensor_A M\) makes sense: \(B\) is a \((B, A)\dash\)bimodule with
  the usual multiplication on the left and the right action
  \begin{align*}
  A &\to \endo(B) \\
  a &\mapsto (b\mapsto b\cdot \phi(a))
  .\end{align*}
\item
  \(B\tensor_A M\) is a left \(B\dash\)module via the following action:
  \begin{align*}
  B &\to \endo(B\tensor_A M) \\
  b_0 &\mapsto (b\tensor m \mapsto b_0 b \tensor m)
  .\end{align*}
\item
  This describes a functor: \begin{align*}
  F: A\dash\text{Mod} &\to B\dash\text{Mod} \\
  X &\mapsto B\tensor_A X \\
  (X\mapsvia{f} Y) &\mapsto (B\tensor_A X \mapsvia{\id_B \tensor f} B\tensor_A Y)
  .\end{align*}

  \begin{itemize}
  \tightlist
  \item
    Need to check:

    \begin{itemize}
    \tightlist
    \item
      Preserves identity morphism, i.e.~\(X\in A\dash\)Mod implies
      \(F(\id_X) = \id_{F(X)}\) in \(B\dash\)Mod.
    \item
      Preserves composition: \(F(f\circ g) = F(f) \circ F(g)\).
    \end{itemize}
  \end{itemize}
\item
  Preserving identity morphisms:

  \begin{itemize}
  \tightlist
  \item
    By construction \(X\selfmap_{\id_X}\) maps to
    \(B\tensor_A X \mapsvia{\id_B \tensor \id_X} B\tensor_A X\), can
    argue that this is the identity map for \(B\dash\)modules.
  \end{itemize}
\item
  Preserving composition: \begin{align*}
  (X\mapsvia{f} Y \mapsvia{g} Z) \mapsto (B\tensor_A X \mapsvia{ \id_B\tensor f} B\tensor_A Y \mapsvia{\id_B \tensor g} B\tensor_A Z) = (B\tensor_A X \mapsvia{\id_B \tensor (g\circ f)} B\tensor_A Z )
  .\end{align*}
\end{itemize}

\begin{quote}
Note: not sure if there's anything to show here.
\end{quote}

\textbf{Part b}: If \(\psi: A\to C\) is another ring morphism, show that
\(B\tensor_A C\) has a ring structure.

Solution:

\begin{itemize}
\item
  Note \(B\tensor_A C\) makes sense, since \(C\) is a left
  \(A\dash\)module via \(a\mapsto (c\mapsto \psi(a)c)\).
\item
  Need to define \((B\tensor_A C, P, M)\) such that it's an abelian
  group under \(P\) (plus), a monoid under \(M\) (multiplication), and
  left/right distributivity.
\item
  Start by defining on cartesian products: \begin{align*}
  P: \qty{B\tensor_A C}^{\times 2} &\to B\tensor_A C \\
  P\qty{ (b_1 \tensor c_1), (b_2\tensor c_2)} &= (b_1 +_B b_2) \tensor (c_1 +_C c_2)
  ,\end{align*} \begin{align*}
  M: \qty{B\tensor_A C}^{\times 2} &\to B\tensor_A C \\
  M\qty{ (b_1 \tensor c_1),  (b_2\tensor c_2)} &= (b_1 \cdot_B b_2) \tensor (c_1 \cdot_C c_2)
  .\end{align*}
\item
  Check \(A\dash\)bilinearity: \begin{align*}
  P(a\cdot (b_1\tensor c_1),\, (b_2\tensor c_2)) 
  &\definedas \qty{ a \cdot (b_1 + b_2)} \tensor (c_1 + c_2)  \\
  &= \qty{ (b_1 + b_2)} \tensor a\cdot (c_1 + c_2) \quad\text{since $C$ is a left $A\dash$module} \\
  &\definedas P((b_1\tensor c_1),\, a\cdot (b_2\tensor c_2)) 
  .\end{align*} \begin{align*}
  M(a\cdot (b_1\tensor c_1),\, (b_2\tensor c_2)) 
  &\definedas \qty{a\cdot (b_1 \cdot b_2)} \tensor (c_1 \cdot c_2) \\
  &= (b_1 \cdot b_2) \tensor \qty{ a\cdot (c_1 \cdot c_2) } \quad\text{since $C$ is a left $A\dash$module} \\
  &\definedas M((b_1\tensor c_1),\, a\cdot (b_2\tensor c_2)) 
  .\end{align*}
\item
  So these lift to maps out of \((B\tensor_A C)^{\tensor 2}\).
\item
  \(P\) forms an abelian group: clear because \(+_B, +_C\) do, and
  commuting is just done within each factor.
\item
  \(M\) forms a monoid: clear for some reason.
\item
  Checking distributivity, claim: it suffices to check on elementary
  tensors and extend by linearity? \begin{align*}
  (b_0 \tensor c_0) \cdot \qty{(b_1 \tensor c_1) + (b_2\tensor c_2) } 
  &= (b_0\tensor c_0) \cdot \qty{ (b_1 + b_2) \tensor (c_1 + c_2) } \\
  &= (b_0(b_1 + b_2)) \tensor ( c_0(c_1 + c_2)) \\
  &= (b_0 b_1 + b_0 b_2) \tensor (c_0 c_1 + c_0 c_2) \\
  &= \cdots
   .\end{align*}
\end{itemize}

\hypertarget{l}{%
\subsection{1.3.L}\label{l}}

If \(S\subseteq A\) is multiplicative and \(M\in A\dash\)Mod, describe a
natural isomorphism \begin{align*}
\eta: (S\inv A)\tensor_A M \to (S\inv M)
\end{align*} as both \(S\inv A\dash\)modules and \(A\dash\)modules.

Solution

\begin{itemize}
\tightlist
\item
  Recall the definition \begin{align*}
  S\inv A\definedas \theset{ {a\over s} \suchthat a\in A, s\in S} / \sim \\
  {a_1 \over s_1} \sim {a_2 \over s_2} \iff \exists s\in S \text{ such that } s\qty{ s_2 a_1 - s_1 a_2 } = 0_A
  .\end{align*}
\item
  Similarly \(S\inv M = \theset{{m\over s}}/\sim\).
\end{itemize}

The universal property: in \(A\dash\)Mod, \(M\to S\inv M\) is initial
among all morphisms \(\alpha: M\to N\) such that
\(\alpha(S) \subseteq N\units\):

\begin{center}
\begin{tikzcd}
& S\inv M \ar[d, dotted, "{\exists ! \tilde\alpha}"] \\
M\ar[ru, "S\inv \cdot"]\ar[r, "\alpha"] & N 
\end{tikzcd}
\end{center}

\begin{quote}
Strategy: define a map \(M\to S\inv A \tensor_A M\) such that \(S\) is
invertible in the image to obtain a map? Show they satisfy the same
universal property?
\end{quote}

\begin{itemize}
\item
  Since \(M \in A\dash\)Mod, we have an action \(a\cdot m\), so define
  \begin{align*}
  \eta: (S\inv A)\times M &\to (S\inv M) \\
  \qty{ {a\over s}, m } &\mapsto {a\cdot m \over s }
  .\end{align*}
\item
  The tensor product \(S\inv A \tensor_A M\) makes sense.

  \begin{itemize}
  \tightlist
  \item
    \(S\inv A\) is a right \(A\dash\)module by
    \(a_0 \mapsto \qty{ {a\over s} \mapsto {a_0 a \over s}}\).
  \item
    \(S\inv M\) is a left \(A\dash\)module by
    \(a_0 \mapsto (m \mapsto a_0 \cdot m)\) where the action comes from
    the \(A\dash\)module structure of \(M\).
  \end{itemize}
\item
  The map makes sense as an \(A\dash\)module morphism

  \begin{itemize}
  \tightlist
  \item
    \(S\inv A \tensor_A M\) is a left \(A\dash\)module by
    \(a_0 \mapsto \qty{{a\over s}\tensor m \mapsto {a_0 a \over s} \tensor m}\)
  \item
    \(S\inv M\) is a left \(A\dash\)module by
    \(a_0 \mapsto \qty{ {m\over s } \mapsto {a_0 \cdot m \over s}}\)
    using the \(A\dash\)module structure on \(M\).
  \end{itemize}
\item
  The map makes sense as an \(S\inv A\dash\)module morphism

  \begin{itemize}
  \tightlist
  \item
    \(S\inv A \tensor_A M\) is a left \(S\inv A\dash\)module by
    \({a_0\over s_0} \mapsto \qty{ {a\over s}\tensor m \mapsto {a_0 a \over s_0 s} \tensor m }\)
  \item
    \(S\inv M\) is a left \(S\inv A\dash\)module by
    \({a_0\over s_0} \mapsto \qty{{m \over s} \mapsto {a_0 \cdot m \over s_0 s} }\)
    by the \(A\dash\)module structure on \(M\).
  \end{itemize}
\item
  Well-defined: ?
\item
  \(A\dash\)bilinear: let \(r\in A\), then \begin{align*}
  \eta\qty{r \cdot {a\over s}, m} 
  &\definedas \eta\qty{{r\cdot a\over s}, m}  \\
  &\definedas {\psi(r\cdot a)(m) \over s} \\
  &= {r\cdot \psi(a)(m) \over s} \quad\text{since $\psi$ is a ring morphism} \\
  &= {\psi(a)(r\cdot m) \over s} \quad\text{since $\psi(a)$ is a ring morphism} \\
  &\definedas \eta\qty{ {a\over s}, r\cdot m}
   .\end{align*} So this lifts to a map out of the tensor product.
\item
  \(S\inv A\dash\)bilinear?
\end{itemize}

\hypertarget{p}{%
\subsection{1.3.P}\label{p}}

Show that the fiber product over the terminal object is the cartesian
product.

Solution:

\begin{itemize}
\tightlist
\item
  Recall definition: \(T\) is terminal iff every object \(X\) admits a
  morphism \(X\to T\).
\item
  Strategy: use both universal products to produce an isomorphism
\item
  Let \(\pr_X, \pr_Y\) by the cartesian product projections, and
  \(\pr_X^T, \pr_Y^T\) be the fiber product projections
\item
  Let \(T_X, T_Y\) be the maps \(X\to T, Y\to T\).
\item
  Since \(X\cross Y\) is an object in this category, it admits one
  unique map to \(T\)
\end{itemize}

\begin{center}
\begin{tikzcd}
X\cross Y\ar[r, "\pr_Y"]\ar[d, "\pr_X"']\ar[dr, "T_{X\cross Y}"] & Y\ar[d, "T_Y"] \\
X\ar[r, "T_X"'] & T 
\end{tikzcd}
\end{center}

\begin{itemize}
\tightlist
\item
  But now \(T_Y \circ \pr_Y: X\cross Y \to T\) is another such map, so
  it must equal \(T_{X\cross Y}\).
\item
  Similarly \(T_X \circ \pr_X\) is equal to \(T_{X\cross Y}\).
\item
  Thus \(T_Y \circ \pr_Y = T_X \circ \pr_Y\), which is part of the
  universal property for \(X\cross_T Y\).
\item
  By the universal property of \(X\cross Y\), for every \(W\) admitting
  maps to \(X, Y\) we get the following \(h_0\):

  \begin{center}
  \begin{tikzcd}
  W \ar[drr, bend left]\ar[rdd, bend right]\ar[dr, dotted, "\exists ! h_0"] & & \\
  & X\cross Y\ar[r, "\pr_Y"]\ar[d, "\pr_X"] & Y\ar[d, "T_Y"] \\
  & X\ar[r, "T_X"] & T 
  \end{tikzcd}
  \end{center}

  \begin{quote}
  Note that \(T\) doesn't matter in this particular diagram.
  \end{quote}
\item
  This gives us the LHS diagram, the RHS comes from the universal
  property of \(X\cross Y\):

  \begin{center}
  \begin{tikzcd}
  X\cross_T Y\ar[drr, "\pr_Y^T", bend left]\ar[rdd, "\pr_X^T", bend right]\ar[dr, dotted, "\exists ! h_0"] & & \\
  & X\cross Y\ar[r, "\pr_Y"]\ar[d, "\pr_X"] & Y\ar[d, "T_Y"] \\
  & X\ar[r, "T_X"] & T 
  \end{tikzcd}
  \begin{tikzcd}
  X\cross Y\ar[drr, "\pr_Y", bend left]\ar[rdd, "\pr_X", bend right]\ar[dr, dotted, "\exists ! h_1"] & & \\
  & X\cross_T Y\ar[r, "\pr_Y^T"]\ar[d, "\pr_X^T"] & Y\ar[d, "T_Y"] \\
  & X\ar[r, "T_X"] & T 
  \end{tikzcd}
  \end{center}
\item
  By commutativity, \(h_0 \circ h_1 = \id_{X\cross Y}\) and vice-versa?
\end{itemize}

\hypertarget{q}{%
\subsection{1.3.Q}\label{q}}

Show that if the two squares in this diagram are cartesian, then then
outer square is also cartesian:

\begin{center}
\begin{tikzcd}
U \ar[r]\ar[d] & V\ar[d] \\
W \ar[r]\ar[d] & X\ar[d] \\
Y \ar[r] & Z
\end{tikzcd}
\end{center}

Solution:

\begin{itemize}
\tightlist
\item
  Need to show that given two maps \(R\to V\) and \(R\to Y\) such that
  \((V\to Z) \circ (U\to V) = (Y\to Z) \circ (R\to Y)\), then there is a
  unique map \(R\to U\) giving a commuting diagram:

  \begin{center}
  \begin{tikzcd}
  U\ar[drr, bend left] \ar[rdd, bend right] & & \\
  & R\ar[ul, dotted, "\exists !\, ?"] \ar[r]\ar[d] & V\ar[d] \\
   & Y \ar[r] & Z
  \end{tikzcd}
  \end{center}
\item
  Applying the bottom square:

  \begin{itemize}
  \tightlist
  \item
    Need to produce maps \(R\to X\) and \(R\to Y\)
  \item
    We're given a map \(R\to Y\) by assumption.
  \item
    We can build a map \(R\to X\) by taking \((V\to X) \circ (R\to V)\).
  \item
    We then get a map \(R\to W\):

    \begin{center}
    \begin{tikzcd}
    W\ar[drr, bend left] \ar[rdd, bend right] & & \\
    & R\ar[ul, dotted, "\exists !"] \ar[r]\ar[d] & X\ar[d] \\
     & Y \ar[r] & Z
    \end{tikzcd}
    \end{center}
  \end{itemize}
\item
  Applying the top square:

  \begin{itemize}
  \tightlist
  \item
    We have a map \(R\to V\) by assumption
  \item
    We have a map \(R\to W\) from step 1
  \item
    We have maps \(V\to X\) and \(W\to X\) from the top square
  \item
    We thus obtain

    \begin{center}
    \begin{tikzcd}
    U\ar[drr, bend left] \ar[rdd, bend right] & & \\
    & R\ar[ul, dotted, "\exists !"] \ar[r]\ar[d] & V\ar[d] \\
     & W \ar[r] & X
    \end{tikzcd}
    \end{center}
  \end{itemize}
\end{itemize}


\bibliography{/home/zack/Notes/library.bib}

\end{document}
